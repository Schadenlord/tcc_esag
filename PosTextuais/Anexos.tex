
% ----------------------------------------------------------
% Anexos
% ----------------------------------------------------------
%
% ---
% Inicia os anexos
% ---
\begin{anexosenv}

	% Imprime uma página indicando o início dos anexos
	%\partanexos

	% ---
% \chapter{Variáveis de controle analisadas por Caplan}

% \begin{longtable}{|>{\raggedright\arraybackslash}p{4cm}
%         |>{\raggedright\arraybackslash}p{8cm}
%         |>{\raggedright\arraybackslash}p{4cm}|}
% \caption{Variáveis de Controle e Codificação (Caplan, 2002)}
% \label{tab:caplan_controls} \\
% \hline
% \textbf{Variável} & \textbf{Pergunta} & \textbf{Codificação} \\
% \hline
% \endfirsthead

% \hline
% \textbf{Variável} & \textbf{Pergunta} & \textbf{Codificação} \\
% \hline
% \endhead

% \hline
% \endfoot

% \hline
% \endlastfoot

% \textbf{Econ} & – & 1 se economista, 0 caso contrário \\
% \hline
% \textbf{Black} & Qual é a sua raça? Branco, negro, asiático ou outra? & 1 se negro, 0 caso contrário \\
% \hline
% \textbf{Asian} & Idem & 1 se asiático, 0 caso contrário \\
% \hline
% \textbf{Othrace} & Idem & 1 se outra raça, 0 caso contrário \\
% \hline
% \textbf{Age} & – & 1996 - ano de nascimento \\
% \hline
% \textbf{Male} & – & 1 se homem, 0 caso contrário \\
% \hline
% \textbf{Jobsecurity} & Qual o seu nível de preocupação com a possibilidade de perder o emprego no próximo ano? &
% 3 = “Nada preocupado”; 2 = “Pouco preocupado”; 1 = “Um pouco preocupado”; 0 = “Muito preocupado” \\
% \hline
% \textbf{Yourlast5} & Nos últimos 5 anos, a renda da sua família cresceu mais rápido, igual ou mais devagar que o custo de vida? &
% 2 = “Cresceu”; 1 = “Igual”; 0 = “Caiu” \\
% \hline
% \textbf{Yournext5} & Nos próximos 5 anos, sua renda crescerá mais rápido, igual ou mais devagar que o custo de vida? &
% 2 = “Crescerá mais rápido”; 1 = “Igual”; 0 = “Mais devagar” \\
% \hline
% \textbf{Income} & Qual a renda total anual da sua família (antes dos impostos)? &
% 1 = Até \$10.000; 2 = \$10.001–\$19.999; 3 = \$20.000–\$24.999; 4 = \$25.000–\$29.999; 5 = \$30.000–\$39.999;
% 6 = \$40.000–\$49.999; 7 = \$50.000–\$74.999; 8 = \$75.000–\$99.999; 9 = \$100.000 ou mais \\
% \hline
% \textbf{Dem} & Você se considera democrata? & 1 se sim, 0 caso contrário \\
% \hline
% \textbf{Rep} & Você se considera republicano? & 1 se sim, 0 caso contrário \\
% \hline
% \textbf{Indep} & Você se considera independente? & 1 se sim, 0 caso contrário \\
% \hline
% \textbf{Othparty} & Você pertence a outro partido? & 1 se sim, 0 caso contrário \\
% \hline
% \textbf{Ideology} & Como você se classifica ideologicamente? &
% -2 = “Muito liberal”; -1 = “Liberal”; 0 = “Moderado”; 1 = “Conservador”; 2 = “Muito conservador”; 3 = “Não pensa nesses termos” \\
% \hline
% \textbf{Othideol} & Dummy para ideologia indefinida & 1 se Ideology = 3, 0 caso contrário \\
% \hline
% \textbf{Education} & Qual o maior nível de escolaridade que você concluiu? &
% 1 = Fundamental (1ª–8ª); 2 = Ensino médio incompleto; 3 = Ensino médio completo;
% 4 = Técnico/profissionalizante; 5 = Superior incompleto; 6 = Superior completo; 7 = Pós-graduação \\
% \end{longtable}

% \vspace{2mm}
% \noindent\textbf{Fonte:} Elaborado pelo autor com base em \citeonline{Systematically_Biased_Beliefs_about_Economics}.
% 	% ---



\end{anexosenv}
