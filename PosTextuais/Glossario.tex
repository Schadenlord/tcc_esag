% ----------------------------------------------------------
% Glossário
% ----------------------------------------------------------

%Consulte o manual da classe abntex2 para orientações sobre o glossário.

%\glossary




% ----------------------------------------------------------
% Glossário (Formatado Manualmente)
% ----------------------------------------------------------

\chapter*{GLOSSÁRIO}
\addcontentsline{toc}{chapter}{GLOSSÁRIO}

{ \setlength{\parindent}{0pt} % ambiente sem indentação

\textbf{Amostragem não probabilística}: Procedimento de seleção de respondentes sem sorteio aleatório; adequado a testes de hipótese sob severidade, mas com generalização populacional limitada. 

\textbf{Análise contrafactual}: Estimativa de como seria a resposta de um indivíduo sob uma condição hipotética (mantidas as demais características constantes), usando parâmetros de um modelo ajustado.

\textbf{Bola de neve (amostragem)}: Técnica de recrutamento em que participantes indicam novos respondentes, útil para ampliar o alcance da coleta.

\textbf{Coeficiente $\boldsymbol\beta$}: Parâmetro de regressão que quantifica a associação entre uma covariável e a probabilidade (ou categoria) de resposta prevista pelo modelo.

\textbf{Dummy (variável indicadora)}: Variável binária ($0/1$) que representa a presença/ausência de uma característica (por ex., formação em Economia), incluída como regressora no modelo.

\textbf{Escala Likert}: Escala ordinal de resposta (ex.: 3 a 5 pontos) usada para medir concordância/discordância com afirmações.

\textbf{Erros-padrão robustos}: Estimativas de incerteza ajustadas para heteroscedasticidade, tornando a inferência menos sensível a violações de variância constante.

\textbf{Hipótese de chances proporcionais (paralelismo)}: Suposição do logit ordenado de que o efeito das covariáveis é constante entre limiares/categorias; quando plausível, facilita comparabilidade entre itens.

\textbf{Instrumento de pesquisa (questionário)}: Conjunto estruturado de questões aplicado digitalmente, revisado por especialistas e pré-testado para clareza, tempo de resposta e validade de conteúdo.

\textbf{Limiares ($\boldsymbol{\tau_j}$)}: Parâmetros do logit ordenado que definem as fronteiras entre categorias da variável dependente.

\textbf{Logit (modelo)}: Regressão para variáveis dependentes binárias, que modela a probabilidade de um evento via função logística.

\textbf{Logit ordenado}: Extensão do logit para variáveis dependentes ordinais (ex.: Likert), com estrutura de limiares e, usualmente, hipótese de chances proporcionais.

\textbf{Princípios-ponte}: Conjunto de regras que conecta teoria e dados no desenho empírico (operacionalização, codificação/direção esperada e sinal teórico dos efeitos).

\textbf{Público esclarecido (contrafactual)}: Referência simulada que prevê como economistas responderiam “como se fossem leigos” (sem formação econômica), mantendo constantes as demais variáveis — usada para isolar o papel do conhecimento técnico.

\textbf{Racionalidade limitada}: Ideia de que decisões são tomadas sob informação imperfeita e capacidade cognitiva restrita; no contexto político, ajuda a explicar crenças persistentes e vieses.

\textbf{Validade externa}: Alcance da generalização dos resultados além da amostra estudada; com amostragem não probabilística, a inferência é positiva/explicativa e não descritiva da população.

\textbf{Vieses cognitivos (quatro vieses)}: Padrões sistemáticos de julgamento identificados na literatura (antimercado, antiestrangeiro, antitrabalho e pessimista) que afetam percepções econômicas e preferências por políticas.

} % fim ambiente sem indentação



