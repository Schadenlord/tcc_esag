
% ----------------------------------------------------------
% Apêndices
% ----------------------------------------------------------

% ---
% Inicia os apêndices
% ---
\begin{apendicesenv}

% Imprime uma página indicando o início dos apêndices
%\partapendices



% ----------------------------------------------------------
\chapter{Quadro Síntese das Abordagens sobre (I)Racionalidade Humana e Política}
% ----------------------------------------------------------
\label{apendice:quadro_iracionalidade}

% no preâmbulo:
% \usepackage{longtable,booktabs,caption}

{\renewcommand{\LTcaptype}{quadro}% <- chave
 \captionsetup{type=quadro}%       <- opcional, mantém o rótulo "Quadro"
 \begin{longtable}{p{0.15\textwidth} p{0.23\textwidth} p{0.34\textwidth} p{0.20\textwidth}}
 \caption{Quadro Síntese – (I)Racionalidade Humana e Política: Principais Abordagens}
 \\

 \toprule
 \textbf{Autor/Obra (Ano)} & \textbf{Tópico/Conceito Central} & \textbf{Principais Contribuições e Evidências} & \textbf{Relação com o TCC} \\
 \midrule
 \endfirsthead

 \multicolumn{4}{c}{\small\textbf{Quadro \thequadro\ (continuação)}}\\
 \toprule
 \textbf{Autor/Obra (Ano)} & \textbf{Tópico/Conceito Central} & \textbf{Principais Contribuições e Evidências} & \textbf{Relação com o TCC} \\
 \midrule
 \endhead

 \midrule
 \multicolumn{4}{r}{\small\itshape Continua na próxima página}\\
 \endfoot

 \bottomrule
 \multicolumn{4}{l}{\footnotesize Fonte: elaboração própria.}
 \endlastfoot

 Adam Smith (1759; 1776) & Moralidade e racionalidade econômica & Integra interesse próprio e normas morais; reconhece limites do modelo racional estrito. & Critica o \textit{homo economicus} e destaca a ética na decisão pública. \\
 Herbert Simon (1955) & Racionalidade Limitada & Mostra que decisões são feitas com informação e tempo limitados; “satisficing”. & Base para analisar limites cognitivos em escolhas políticas. \\
 Kahneman \& Tversky (1974, 2011) & Heurísticas e Vieses Cognitivos & Representatividade, ancoragem etc. & Explica previsibilidade dos desvios do eleitor médio. \\
 Anthony Downs (1957) & Ignorância Racional & Custo de informar-se > benefício do voto. & Fundamenta debate sobre baixa informação e democracia. \\
 Bryan Caplan (2007) & Irracionalidade Racional & Crenças falsas por motivações emocionais. & Viés sistemático com impacto em políticas. \\
 Hayek (1945) & Limitação do conhecimento & Conhecimento disperso; centralização inviável. & Crítica à ilusão de controle racional. \\
 Bhagwati, Sowell, Bastiat & Viés antimercado/antiestrangeiro & Protecionismo recorrente e argumentos populares. & Vieses e políticas populistas. \\
 Nyhan, Kahan, Sunstein, Rossini et al. & Resistência à revisão & Confirmação e bolhas informacionais. & Persistência da desinformação política. \\
 \end{longtable}
}

\chapter{Variáveis analisadas}
\label{apencice:variaveis_analisadas}

\renewcommand{\arraystretch}{1.25}

\begin{longtable}{@{}%
  >{\raggedright\arraybackslash}p{4cm}%
  >{\raggedright\arraybackslash}p{4cm}%
  >{\raggedright\arraybackslash}p{7cm}@{}}
\caption{Variáveis de controle e codificação (modelos logit binário/ordenado)}
\label{tab:control_variables}\\

\toprule
\textbf{Variável (no modelo)} & \textbf{Pergunta no questionário} & \textbf{Codificação adotada} \\
\midrule
\endfirsthead

\toprule
\textbf{Variável (no modelo)} & \textbf{Pergunta no questionário} & \textbf{Codificação adotada} \\
\midrule
\endhead

\midrule
\multicolumn{3}{r}{\small Continuação na próxima página} \\
\endfoot

\bottomrule
\multicolumn{3}{@{}l}{\footnotesize \textit{Fonte: Elaboração própria.}}\\
\endlastfoot

    % ------------------ ECON / FORMAÇÃO ------------------
    \texttt{econ} & Qual seu nível de formação em Ciências Econômicas? & 
    Variável derivada (não dummificada). \newline
    \emph{Regra}: $1$ se \textit{Mestrado em Economia} ou \textit{Doutorado em Economia}; $0$ para demais respostas (inclui \textit{Graduação em Economia}, \textit{Não tenho formação} e cursos lato sensu/MBAs). \newline
    \emph{Obs.}: limiar severo para capturar \emph{conhecimento técnico} (ver \autoref{sec:hipoteses}). \\ \hline

    % ------------------ SEXO ------------------
    \texttt{Você é homem?{\_}Sim} & Você é homem? &
    Dummy binária gerada por \texttt{get\_dummies} (\texttt{drop\_first=True}). \newline
    \emph{Regra}: $1=\text{Sim}$; $0=\text{Não}$. \\ \hline

    % --- RAÇA (dummies): acrescentar a base explícita
    \texttt{Com qual grupo racial/étnico você mais se identifica?{\_}\{Branco, Negro, Outro\}} &
    Com qual grupo racial/étnico você mais se identifica? &
    Conjunto de dummies com \texttt{drop\_first=True}. \newline
    \emph{Regra}: colunas no modelo: \texttt{\_Branco}, \texttt{\_Negro}, \texttt{\_Outro}. \newline
    \emph{Categoria de referência (omitida)}: \textit{Asiático}. \\ \hline

    % --- ESCOLARIDADE (dummies): acrescentar a base explícita
    \texttt{Qual seu nível de escolaridade?{\_}
    \{EF~Incompleto, EM~Completo, EM~Incompleto, ES~Completo, ES~Incompleto, Pós-graduação\}} &
    Qual seu nível de escolaridade? &
    Conjunto de dummies com \texttt{drop\_first=True}. \newline
    \emph{Regra}: colunas no modelo: \texttt{EF Incompleto}, \texttt{EM Completo}, \texttt{EM Incompleto}, \texttt{ES Completo}, \texttt{ES Incompleto}, \texttt{Pós-graduação}. \newline
    \emph{Categoria de referência (omitida)}: \textit{Ensino Fundamental Completo}. \\ \hline

    % --- FAIXA ETÁRIA (dummies): acrescentar a base explícita
    \texttt{Qual é a sua faixa etária?{\_}\{Até 18, 26–35, 36–45, 46–55, 56–65, 66+\}} &
    Qual é a sua faixa etária? &
    Conjunto de dummies com \texttt{drop\_first=True}. \newline
    \emph{Regra}: colunas no modelo: \texttt{\_Até 18 anos}, \texttt{\_26 a 35 anos}, \texttt{\_36 a 45 anos}, \texttt{\_46 a 55 anos}, \texttt{\_56 a 65 anos}, \texttt{\_66 anos ou mais}. \newline
    \emph{Categoria de referência (omitida)}: \textit{19 a 25 anos}. \\ \hline

    % ------------------ ENGJTO POLÍTICO ------------------
    \texttt{Você se considera uma pessoa politicamente engajada?{\_}Sim} &
    Você se considera uma pessoa politicamente engajada? &
    Dummy binária com \texttt{drop\_first=True}. \newline
    \emph{Regra}: $1=\text{Sim}$; $0=\text{Não}$. \\ \hline

    % --- IDEOLOGIA (escala): explicitar a colisão Centro/Sem opinião
    \texttt{Com qual espectro político você mais se identifica?} &
    Com qual espectro político você mais se identifica? &
    Variável ordinal/numerizada: \newline
    \quad $-2=$ Extrema-esquerda; $-1=$ Esquerda; $0=$ Centro; $1=$ Direita; $2=$ Extrema-direita; $3=$ Independente; $0=$ Sem opinião. \newline
    \emph{Obs.:} ``Sem opinião'' foi recodificada como $0$ (mesmo valor de \textit{Centro}). \\ \hline

    % ------------------ VÍNCULO EMPREGATÍCIO (escala) ------------------
    \texttt{Qual é o seu vínculo empregatício?}\footnote{A variável de vinculo empregatício foi codificada em escala de dependência estatal, de servidores públicos (maior dependência) a empresários (maior autonomia econômica). A literatura mostra que servidores públicos tendem a posições mais à esquerda e favoráveis à expansão do Estado por razões ideológicas e de interesse próprio \cite{jensen2009political}; já a autonomia e a estrutura das tarefas moldam preferências mais pró-mercado e individualistas, especialmente entre empresários e autônomos \cite{kitschelt2014occupations}. Por fim, a heterogeneidade dos autônomos (precários vs. estáveis) também afeta a orientação política, justificando distinções internas na categoria \cite{jansen2016self}.} &
    Qual é o seu vínculo empregatício? &
    Variável ordinal/numerizada (mapeamento fixo): \newline
    $-4=$ Servidor público;\quad $-3=$ Aposentado;\quad $-2=$ Estudante;\newline
    $-1=$ Desempregado;\quad $0=$ Empregado sem carteira;\quad $1=$ CLT;\newline
    $2=$ Autônomo;\quad $3=$ Empresário;\quad $4=$ Outro. \newline
    (Usada no modelo como numérica.) \\ \hline

\end{longtable}

\begin{landscape}

\chapter{Tabela-Síntese Mestra dos Resultados Empíricos}
\label{apendice:tabela_sintese}

\begingroup
\scriptsize
\setlength{\tabcolsep}{3.5pt}

\begin{ThreePartTable}
\begin{TableNotes}[flushleft]\footnotesize
\item \textit{\textbf{Notas gerais:}} Tabela elaborada pelo autor com base nos modelos da \autoref{sec:modelo-estatistico} (contrafactual da \autoref{sec:publico_esclarecido}). \newline Valores de $\beta$ e $p$ entre parênteses. Coeficientes em \textbf{negrito} indicam $p<0{,}05$. ``Econ'' refere-se à dummy de formação em Economia. Em ``Controles'', listam-se apenas efeitos consistentes ($p<0{,}10$ e sinal estável entre especificações). Médias: escala normalizada da pesquisa.
\item \textbf{``n/a''} indica que o modelo não foi reportado: \textit{n/a (não estimável)} quando a estimação falhou (p.\,ex., falta de variação na DV, separação quase-perfeita, colinearidade perfeita ou não convergência); \textit{n/a (não estável)} quando os resultados variaram de forma sensível a perturbações razoáveis (p.\,ex., limiares mal ordenados, erros-padrão inflados), não atendendo ao critério mínimo de estabilidade.
\item \textbf{``(nenhum)''} na coluna \emph{Controles} significa que, embora os controles tenham sido incluídos no modelo, nenhum apresentou associação estatisticamente significativa a $10\%$ \emph{com sinal estável}; por isso nenhum controle é destacado na síntese.
\item \textbf{Sinal ``$+$''/``$-$'':} refere-se ao \emph{sinal do coeficiente} estimado. Em logit ordenado, com a escala ordenada de menor para maior, $\beta>0$ (``$+$'') desloca a probabilidade para categorias mais altas da DV; $\beta<0$ (``$-$'') desloca para categorias mais baixas (interpretação: maior/menor concordância quando a escala é crescente).
\end{TableNotes}

\begin{longtable}{%
  >{\centering\arraybackslash}p{0.5cm}
  >{\RaggedRight\arraybackslash}p{3.0cm}
  >{\centering\arraybackslash}p{0.6cm}
  >{\RaggedRight\arraybackslash}p{2cm}
  >{\RaggedRight\arraybackslash}p{3.5cm}
  >{\RaggedRight\arraybackslash}p{3.2cm}
  >{\RaggedRight\arraybackslash}p{3.2cm}
  >{\centering\arraybackslash}p{1.2cm}
  >{\centering\arraybackslash}p{1.2cm}
  >{\centering\arraybackslash}p{1.4cm}
  >{\RaggedRight\arraybackslash}p{4.5cm}}
\caption{Tabela-Síntese Mestra — Resultados dos Modelos Ordenados (52 DVs)}\label{tab:resultados_ordenados}\\
\toprule
\textbf{No.} & \textbf{DV (Pergunta)} & \textbf{N} & \textbf{Modelo} & \textbf{Espectro político} & \textbf{Formação em Economia} & \textbf{Controles ($p<0{,}10$; sinal)} & \textbf{Média (econ=0)} & \textbf{Média (econ=1)} & \textbf{Média contraf.} & \textbf{Leitura}\\
\midrule
\endfirsthead

\multicolumn{11}{l}{\footnotesize \textit{(continuação)}}\\
\toprule
\textbf{No.} & \textbf{DV (Pergunta)} & \textbf{N} & \textbf{Modelo} & \textbf{Espectro político} & \textbf{Formação em Economia} & \textbf{Controles ($p<0{,}10$; sinal)} & \textbf{Média (econ=0)} & \textbf{Média (econ=1)} & \textbf{Média contraf.} & \textbf{Leitura}\\
\midrule
\endhead

\midrule
\multicolumn{10}{r}{\footnotesize \textit{continua na próxima página}}\\
\endfoot

\bottomrule
\endlastfoot

1 & Os impostos são muito altos & 183 & Logit ord.\ (limiares ok) & $+$ ($\beta = \mathbf{0{,}686}$; $p = 0{,}000$) & $-1{,}008$; $p = 0{,}150$ & (nenhum) & 1{,}470 & 1{,}235 & 1{,}488 & Percepção amplamente compartilhada; economistas concordam quase tanto; direita intensifica levemente.\\

2 & O déficit federal é grande demais (Dívida pública) & 182 & Logit ord.\ (limiares ok) & $+$ ($\beta = \mathbf{0{,}708}$; $p = 0{,}000$) & $\mathbf{-1{,}461}$; $p = 0{,}026$ & (nenhum) & 1{,}521 & 1{,}118 & 1{,}503 & Público vê excesso; economistas menos alarmistas; formação técnica atenua preocupação.\\

3 & O gasto com ajuda externa é alto demais & 183 & Logit ord.\ (limiares ok) & $+$ ($\beta = \mathbf{0{,}246}$; $p = 0{,}034$) & $\mathbf{-1{,}443}$; $p = 0{,}018$ & Você é homem? Sim: $-$; Qual é a sua faixa etária? 46 a 55 anos: $+$ & 0{,}861 & 0{,}412 & 0{,}901 & Grande viés no público; economistas discordam amplamente; formação corrige.\\

4 & Temos imigrantes demais & 183 & Logit ord.\ (limiares ok) & $+$ ($\beta = 0{,}047$; $p = 0{,}762$) & $-0{,}515$; $p = 0{,}473$ & Qual é a sua faixa etária? 46 a 55 anos: $+$; Qual é a sua faixa etária? 66 anos ou mais: $+$ & 0{,}241 & 0{,}235 & 0{,}363 & Sem diferença ideológica/ formação; ausência de viés sistemático.\\

5 & Há deduções demais para as empresas (Impostos) & 184 & Logit ord.\ (limiares ok) & $+$ ($\beta = \mathbf{0{,}284}$; $p = 0{,}015$) & $0{,}642$; $p = 0{,}247$ & Qual é a sua faixa etária? Até 18 anos: $-$ & 1{,}096 & 1{,}294 & 1{,}038 & Leigos e economistas veem deduções elevadas; sem diferença robusta por formação.\\

6 & A educação e a qualificação profissional são inadequadas & 183 & Logit ord.\ (limiares ok) & $+$ ($\beta = 0{,}096$; $p = 0{,}431$) & $0{,}730$; $p = 0{,}248$ & Você se considera uma pessoa politicamente engajada? Sim: $-$ & 1{,}482 & 1{,}706 & 1{,}487 & Convergência: ambos veem insuficiência; sem viés a corrigir.\\

7 & A seguridade social (Previdência) atende pessoas demais & 184 & Logit ord.\ (limiares ok) & $+$ ($\beta = \mathbf{0{,}523}$; $p = 0{,}000$) & $-1{,}024$; $p = 0{,}086$ & Qual é a sua faixa etária? 26 a 35 anos: $-$ & 0{,}832 & 0{,}647 & 0{,}943 & Público acha que atende “demais”; direita acentua fortemente essa visão.\\

8 & Mulheres e minorias têm vantagens demais por causa das ações afirmativas (cotas) & 183 & Logit ord.\ (limiares ok) & $+$ ($\beta = \mathbf{0{,}464}$; $p = 0{,}002$) & $-0{,}446$; $p = 0{,}567$ & Qual é a sua faixa etária? 56 a 65 anos: $+$ & 0{,}349 & 0{,}176 & 0{,}271 & Viés ideológico marcante: direita aumenta crença contrária às cotas.\\

9 & As pessoas não dão valor ao trabalho duro & 184 & Logit ord.\ (limiares ok) & $+$ ($\beta = \mathbf{0{,}288}$; $p = 0{,}018$) & $-0{,}277$; $p = 0{,}638$ & Qual é o seu vínculo empregatício? : $-$; Qual é a sua faixa etária? 46 a 55 anos: $+$ & 0{,}677 & 0{,}588 & 0{,}715 & Direita tende a concordar mais; economistas não diferem significativamente.\\

10 & O governo regulamenta muito os negócios & 184 & Logit ord.\ (limiares ok) & $+$ ($\beta = \mathbf{0{,}994}$; $p = 0{,}000$) & $-1{,}067$; $p = 0{,}087$ & Você é homem? Sim: $+$; Qual seu nível de escolaridade? Ensino Médio Completo: $+$; Qual seu nível de escolaridade? Pós-graduação: $+$; Qual é a sua faixa etária? 66 anos ou mais: $-$ & 1{,}192 & 0{,}941 & 1{,}213 & Viés anti-regulação em ambos; homens mais propensos; direita reforça.\\

11 & As pessoas não poupam o bastante & 184 & Logit ord.\ (limiares ok) & $+$ ($\beta = 0{,}214$; $p = 0{,}065$) & $-0{,}506$; $p = 0{,}378$ & Você é homem? Sim: $+$; Qual é a sua faixa etária? 46 a 55 anos: $+$; Qual é a sua faixa etária? 66 anos ou mais: $+$ & 1{,}006 & 0{,}941 & 1{,}101 & Consenso de insuficiência de poupança; direita atribui mais culpa.\\

12 & As empresas lucram demais & 183 & Logit ord.\ (limiares ok) & $-$ ($\beta = \mathbf{-0{,}788}$; $p = 0{,}000$) & $-0{,}422$; $p = 0{,}517$ & Qual é o seu vínculo empregatício? : $-$; Qual seu nível de escolaridade? Ensino Médio Completo: $-$; Qual seu nível de escolaridade? Ensino Médio Incompleto: $-$; Qual seu nível de escolaridade? Ensino Superior Incompleto: $-$; Qual seu nível de escolaridade? Pós-graduação: $-$; Qual é a sua faixa etária? 46 a 55 anos: $+$; Você se considera uma pessoa politicamente engajada? Sim: $-$ & 0{,}512 & 0{,}706 & 0{,}884 & Viés anti-lucro (esquerda) contraposto por economistas; correção técnica.\\

13 & Altos executivos ganham demais & 183 & Logit ord.\ (limiares ok) & $-$ ($\beta = \mathbf{-0{,}634}$; $p = 0{,}000$) & $-0{,}623$; $p = 0{,}283$ & Qual é o seu vínculo empregatício? : $-$; Você é homem? Sim: $-$ & 0{,}705 & 0{,}765 & 0{,}992 & Público mais crítico; economistas discordam; homens menos críticos.\\

14 & A produtividade está aumentando devagar demais & 183 & Logit ord.\ (limiares ok) & $+$ ($\beta = 0{,}009$; $p = 0{,}937$) & $\mathbf{1{,}722}$; $p = 0{,}010$ & Você é homem? Sim: $+$; Qual é a sua faixa etária? 26 a 35 anos: $-$ & 1{,}114 & 1{,}765 & 1{,}239 & Convergência; economistas mais preocupados (efeito de formação).\\

15 & A tecnologia causa demissões & 183 & Logit ord.\ (limiares ok) & $-$ ($\beta = -0{,}148$; $p = 0{,}261$) & $-0{,}231$; $p = 0{,}697$ & Qual é o seu vínculo empregatício? : $-$; Qual é a sua faixa etária? 46 a 55 anos: $+$ & 0{,}434 & 0{,}588 & 0{,}680 & Leve discordância; economistas discordam mais; sem viés ludista forte.\\

16 & As empresas estão enviando funcionários ao estrangeiro & 184 & Logit ord.\ (limiares ok) & $-$ ($\beta = -0{,}151$; $p = 0{,}293$) & $1{,}166$; $p = 0{,}081$ & Qual é o seu vínculo empregatício? : $-$; Qual seu nível de escolaridade? Ensino Médio Completo: $-$; Qual seu nível de escolaridade? Ensino Superior Incompleto: $-$; Qual seu nível de escolaridade? Pós-graduação: $-$; Qual é a sua faixa etária? 26 a 35 anos: $-$; Você se considera uma pessoa politicamente engajada? Sim: $-$ & 0{,}299 & 0{,}412 & 0{,}188 & Leigos pouco veem fuga; economistas notam mais (atenção técnica ao fenômeno).\\

17 & As empresas estão reduzindo os postos de trabalho & 184 & Logit ord.\ (limiares ok) & $-$ ($\beta = \mathbf{-0{,}263}$; $p = 0{,}028$) & $0{,}128$; $p = 0{,}821$ & Qual é o seu vínculo empregatício? : $-$ & 0{,}659 & 0{,}941 & 0{,}903 & Ambos percebem cortes; esquerda enfatiza mais (espectro negativo).\\

18 & As empresas não investem o suficiente em educação e qualificação profissional & 183 & Logit ord.\ (limiares ok) & $-$ ($\beta = \mathbf{-0{,}398}$; $p = 0{,}001$) & $-0{,}016$; $p = 0{,}978$ & Qual é o seu vínculo empregatício? : $-$; Você se considera uma pessoa politicamente engajada? Sim: $-$ & 1{,}084 & 1{,}294 & 1{,}287 & Crítica compartilhada, mais forte à esquerda; formação não mitiga.\\

19 & Corte de impostos & 184 & Logit ord.\ (limiares ok) & $+$ ($\beta = \mathbf{0{,}938}$; $p = 0{,}000$) & $\mathbf{-0{,}348}$; $p = 0{,}000$ & (nenhum) & 1{,}659 & 1{,}353 & 1{,}496 & Forte apoio geral; direita apoia mais; sem diferença robusta por formação.\\

20 & Mais mulheres na força de trabalho & 184 & Logit ord.\ (limiares ok) & $-$ ($\beta = \mathbf{-0{,}710}$; $p = 0{,}000$) & $0{,}511$; $p = 0{,}510$ & Você é homem? Sim: $-$; Qual é a sua faixa etária? 26 a 35 anos: $+$; Qual é a sua faixa etária? 36 a 45 anos: $+$; Qual é a sua faixa etária? 46 a 55 anos: $+$ & 1{,}407 & 1{,}765 & 1{,}681 & Apoio geral; homens menos favoráveis; esquerda apoia mais.\\

21 & Aumento do uso de tecnologia no trabalho & 184 & n/a (não estimável) & n/a & n/a (modelo não convergiu) & (nenhum) & 1{,}922 & 2{,}000 & 1{,}890 & Quase consenso pró-tecnologia; falta de variação impediu estimação.\\

22 & Acordos comerciais com outros países & 184 & n/a (não estimável) & n/a & n/a (modelo não convergiu) & (nenhum) & 1{,}916 & 1{,}882 & 1{,}963 & Abertura bem vista por ambos; sem efeitos identificáveis.\\

23 & Redução recente dos postos de trabalho em grandes empresas & 183 & Logit ord.\ (limiares ok) & $+$ ($\beta = \mathbf{0{,}458}$; $p = 0{,}000$) & $-0{,}932$; $p = 0{,}170$ & (nenhum) & 0{,}524 & 0{,}235 & 0{,}454 & Leigos percebem mais demissões; direita eleva essa percepção.\\

24 & Algumas pessoas dizem que, para ter uma vida confortável, a família média deve ter dois assalariados em tempo integral. Você concorda com isso, ou acha que a família média pode viver confortavelmente com apenas um assalariado em tempo integral? & 183 & Logit ord.\ (limiares ok) & $+$ ($\beta = \mathbf{0{,}324}$; $p = 0{,}007$) & $0{,}146$; $p = 0{,}789$ & Qual é o seu vínculo empregatício? : $+$; Qual é a sua faixa etária? 46 a 55 anos: $-$; Você se considera uma pessoa politicamente engajada? Sim: $+$ & 1{,}289 & 1{,}176 & 1{,}099 & Maioria concorda que duas rendas são necessárias; direita reforça essa necessidade; formação não altera percepção.\\

25 & Você acha que os acordos comerciais entre o Brasil e outros países, por seu impacto no emprego, são bons, ruins ou indiferentes para a economia? & 182 & n/a (não estável) & n/a & n/a (modelo não estável) & (nenhum) & 1{,}693 & 1{,}812 & 1{,}783 & Maioria acha bons; sem influência consistente de ideologia/formação.\\

26 & Quem você considera o maior responsável pelo aumento dos preços dos combustíveis? & 182 & Logit ord.\ (limiares ok) & $-$ ($\beta = \mathbf{-0{,}319}$; $p = 0{,}005$) & $0{,}328$; $p = 0{,}558$ & Qual é a sua faixa etária? 26 a 35 anos: $+$; Qual é a sua faixa etária? 36 a 45 anos: $+$; Qual é a sua faixa etária? Até 18 anos: $+$ & 0{,}885 & 1{,}118 & 0{,}961 & Público culpa mais governo; economistas distribuem causas (mercado externo etc.).\\

27 & Você acha que os preços dos combustíveis são altos demais, baixos demais ou no nível certo? & 184 & n/a (não estável) & n/a & n/a (modelo não estável) & (nenhum) & 1{,}868 & 1{,}765 & 1{,}852 & Quase unanimidade: combustíveis altos; falta variação.\\

28 & Você acha que o presidente pode fazer muito, pouco ou que está além da capacidade de ele melhorar a economia? & 180 & Logit ord.\ (limiares ok) & $+$ ($\beta = \mathbf{0{,}264}$; $p = 0{,}042$) & $0{,}119$; $p = 0{,}854$ & (nenhum) & 1{,}434 & 1{,}600 & 1{,}544 & Ambos atribuem algum poder; direita atribui mais.\\

29 & Você acha que os novos postos de trabalho do país pagam bem ou mal? & 181 & Logit ord.\ (limiares ok) & $+$ ($\beta = \mathbf{0{,}669}$; $p = 0{,}000$) & $0{,}102$; $p = 0{,}877$ & Escolaridade (EM compl., Pós-grad.: $+$) & 0{,}526 & 0{,}500 & 0{,}483 & Consenso: pagam mal; direita um pouco mais otimista; mais escolarizados menos pessimistas.\\

30 & A desigualdade entre ricos e pobres, comparada a 20 anos atrás, está: & 183 & Logit ord.\ (limiares ok) & $-$ ($\beta = \mathbf{-0{,}288}$; $p = 0{,}020$) & $-0{,}318$; $p = 0{,}594$ & (nenhum) & 1{,}318 & 1{,}062 & 1{,}180 & Público crê que aumentou; esquerda acentua; economistas menos pessimistas.\\

31 & Nos últimos 20 anos, você acha que, em geral, as rendas familiares dos brasileiros médios têm aumentado mais rapidamente do que o custo de vida, têm se mantido aproximadamente no nível do custo de vida, ou têm ficado atrás do custo de vida? & 182 & Logit ord.\ (limiares ok) & $+$ ($\beta = 0{,}221$; $p = 0{,}097$) & $1{,}111$; $p = 0{,}096$ & Faixa etária (46--55: $-$) & 0{,}536 & 0{,}750 & 0{,}388 & Público mais pessimista; formação reduz pessimismo sobre renda real.\\

32 & Pensando apenas nos salários do trabalhador brasileiro médio, você acha que, nos últimos 20 anos, eles têm aumentado mais rapidamente do que o custo de vida, têm se mantido aproximadamente no mesmo nível do custo de vida, ou têm ficado atrás do custo de vida? & 181 & n/a (não estável) & n/a & n/a (modelo não estável) & (nenhum) & 0{,}390 & 0{,}588 & 0{,}219 & Ambos veem salários atrás; economistas um pouco menos pessimistas.\\

33 & Nos próximos cinco anos, você acha que o padrão de vida do cidadão médio vai subir, cair ou se manter aproximadamente o mesmo? & 178 & Logit ord.\ (limiares ok) & $-$ ($\beta = -0{,}172$; $p = 0{,}157$) & $0{,}427$; $p = 0{,}485$ & Faixa etária (46--55: $-$) & 0{,}671 & 0{,}562 & 0{,}448 & Pessimismo moderado compartilhado; economistas levemente menos pessimistas.\\

34 & Você espera que a geração de seus filhos desfrute de um padrão de vida mais alto ou mais baixo do que a sua geração, ou acha que será aproximadamente o mesmo? & 182 & Logit ord.\ (limiares ok) & $+$ ($\beta = 0{,}229$; $p = 0{,}082$) & $0{,}268$; $p = 0{,}647$ & Faixa etária (36--45: $-$) & 1{,}425 & 1{,}250 & 1{,}154 & Divisão equilibrada; sem viés extremo; economistas um pouco menos otimistas.\\

35 & (Se você tem filhos com menos de 30 anos) Quando eles atingirem a sua idade, você espera que eles desfrutem de um padrão de vida mais alto ou mais baixo do que o seu agora, ou espera que seja aproximadamente o mesmo? & 138 & n/a (não estável) & n/a & n/a (N menor) & (nenhum) & 1{,}553 & 1{,}357 & 1{,}331 & Leve otimismo; economistas menos otimistas; sem significância robusta.\\

36 & A reforma da previdência é necessária & 169 & Logit ord.\ (limiares ok) & $+$ ($\beta = \mathbf{0{,}542}$; $p = 0{,}000$) & $-0{,}843$; $p = 0{,}186$ & Escolaridade (Sup. Incompl., Pós: $+$) & 1{,}556 & 1{,}188 & 1{,}460 & Apoio geral; direita e maior escolaridade elevam apoio; economistas majoritariamente favoráveis.\\

37 & A reforma trabalhista é necessária & 169 & Logit ord.\ (limiares ok) & $+$ ($\beta = \mathbf{0{,}657}$; $p = 0{,}000$) & $-0{,}722$; $p = 0{,}252$ & Escolaridade (EM compl., Sup. compl./incompl., Pós: $+$) & 1{,}503 & 1{,}188 & 1{,}426 & Apoio em ambos; direita e escolaridade elevam apoio.\\

38 & A reforma tributária é necessária & 171 & Logit ord.\ (limiares ok) & $+$ ($\beta = 0{,}212$; $p = 0{,}332$) & $0{,}751$; $p = 0{,}540$ & Escolaridade (EM compl., Sup. compl.: $+$) & 1{,}810 & 1{,}938 & 1{,}877 & Quase unanimidade pró-reforma; sem diferenças ideológicas/ formação significativas.\\

39 & A privatização de estatais é benéfica & 172 & Logit ord.\ (limiares ok) & $+$ ($\beta = \mathbf{0{,}945}$; $p = 0{,}000$) & $-0{,}995$; $p = 0{,}142$ & Qual é o seu vínculo empregatício? : $+$; Qual é a sua faixa etária? 46 a 55 anos: $+$ & 1{,}039 & 0{,}882 & 1{,}075 & Apoio generalizado à privatização; direita apoia muito mais; economistas concordam em grau semelhante.\\

40 & Produtos importados são benéficos & 171 & Logit ord.\ (limiares ok) & $+$ ($\beta = \mathbf{0{,}455}$; $p = 0{,}001$) & $\mathbf{1{,}627}$; $p = 0{,}018$ & Qual seu nível de escolaridade? Ensino Médio Incompleto: $-$; Qual é a sua faixa etária? 36 a 45 anos: $-$ & 1{,}279 & 1{,}588 & 1{,}224 & Consenso de que importações são benéficas; economistas concordam ainda mais; direita apoia mais o livre-comércio.\\

41 & A corrupção é a principal causa do subdesenvolvimento & 171 & Logit ord.\ (limiares ok) & $+$ ($\beta = \mathbf{0{,}725}$; $p = 0{,}000$) & $-0{,}991$; $p = 0{,}110$ & Você é homem? Sim: $-$; Com qual grupo racial/étnico você mais se identifica? Outro: $+$; Qual é a sua faixa etária? 46 a 55 anos: $+$; Qual é a sua faixa etária? 66 anos ou mais: $+$; Você se considera uma pessoa politicamente engajada? Sim: $-$ & 1{,}357 & 0{,}941 & 1{,}205 & Público atribui problemas à corrupção; direita acentua esse foco; economistas relativizam (outros fatores).\\

42 & A taxa de juros (SELIC) é alta demais & 170 & Logit ord.\ (limiares ok) & $-$ ($\beta = \mathbf{-0{,}458}$; $p = 0{,}001$) & $1{,}141$; $p = 0{,}094$ & Qual seu nível de escolaridade? Ensino Superior Completo: $-$; Qual seu nível de escolaridade? Ensino Superior Incompleto: $-$; Qual seu nível de escolaridade? Pós-graduação: $-$; Qual é a sua faixa etária? 46 a 55 anos: $+$; Qual é a sua faixa etária? 66 anos ou mais: $+$; Qual é a sua faixa etária? Até 18 anos: $-$; Você se considera uma pessoa politicamente engajada? Sim: $-$ & 1{,}327 & 1{,}765 & 1{,}472 & Quase todos concordam que juros estão altos; economistas endossam essa visão; esquerda se preocupa mais.\\

43 & O governo atual sabe lidar com crises econômicas & 171 & Logit ord.\ (limiares ok) & $-$ ($\beta = \mathbf{-1{,}328}$; $p = 0{,}000$) & $1{,}064$; $p = 0{,}126$ & Qual é o seu vínculo empregatício? : $-$; Qual seu nível de escolaridade? Ensino Médio Completo: $-$; Qual seu nível de escolaridade? Ensino Superior Completo: $-$; Qual seu nível de escolaridade? Ensino Superior Incompleto: $-$; Qual seu nível de escolaridade? Pós-graduação: $-$ & 0{,}494 & 0{,}941 & 0{,}641 & Público descrente da gestão atual; direita muito mais cética; economistas um pouco menos críticos.\\

44 & O governo deve intervir para controlar preços & 171 & Logit ord.\ (limiares ok) & $-$ ($\beta = \mathbf{-0{,}625}$; $p = 0{,}000$) & $-0{,}071$; $p = 0{,}906$ & Você é homem? Sim: $-$; Qual seu nível de escolaridade? Ensino Médio Completo: $-$; Qual seu nível de escolaridade? Ensino Superior Completo: $-$; Qual seu nível de escolaridade? Ensino Superior Incompleto: $-$; Qual seu nível de escolaridade? Pós-graduação: $-$ & 0{,}695 & 0{,}647 & 0{,}709 & Divisão ideológica clara: esquerda defende maior intervenção; direita rejeita; formação não afeta.\\

45 & A entrada de estrangeiros no mercado de trabalho é benéfica & 171 & Logit ord.\ (limiares ok) & $-$ ($\beta = -0{,}106$; $p = 0{,}386$) & $0{,}675$; $p = 0{,}266$ & Você é homem? Sim: $+$; Qual seu nível de escolaridade? Ensino Médio Incompleto: $-$; Qual é a sua faixa etária? Até 18 anos: $+$ & 1{,}318 & 1{,}529 & 1{,}346 & Consenso: participação estrangeira é positiva; sem divergências ideológicas ou de formação significativas.\\

46 & A indústria nacional deve ser protegida & 171 & Logit ord.\ (limiares ok) & $-$ ($\beta = \mathbf{-0{,}603}$; $p = 0{,}000$) & $-0{,}639$; $p = 0{,}274$ & Qual é o seu vínculo empregatício? : $-$; Você é homem? Sim: $-$; Você se considera uma pessoa politicamente engajada? Sim: $-$ & 1{,}214 & 1{,}000 & 1{,}214 & Público tende ao protecionismo; esquerda muito mais favorável; economistas em geral não endossam.\\

47 & O Brasil tem chance de virar potência econômica & 172 & Logit ord.\ (limiares ok) & $-$ ($\beta = \mathbf{-0{,}316}$; $p = 0{,}010$) & $-0{,}212$; $p = 0{,}704$ & Qual seu nível de escolaridade? Ensino Médio Completo: $-$ & 1{,}406 & 1{,}176 & 1{,}251 & Otimismo moderado; esquerda mais otimista; economistas mais céticos.\\

48 & Lucros empresariais ocorrem às custas dos trabalhadores & 171 & Logit ord.\ (limiares ok) & $-$ ($\beta = \mathbf{-0{,}830}$; $p = 0{,}000$) & $0{,}527$; $p = 0{,}411$ & Qual é o seu vínculo empregatício? : $-$; Você é homem? Sim: $-$; Qual seu nível de escolaridade? Ensino Superior Incompleto: $-$; Qual seu nível de escolaridade? Pós-graduação: $-$ & 0{,}916 & 1{,}176 & 0{,}973 & Público dividido; forte clivagem ideológica; economistas não apresentam correção unívoca do viés.\\

49 & A competição entre empresas beneficia consumidores & 171 & Logit ord.\ (limiares ok) & $+$ ($\beta = \mathbf{0{,}619}$; $p = 0{,}000$) & $0{,}758$; $p = 0{,}437$ & (nenhum) & 1{,}799 & 1{,}882 & 1{,}781 & Consenso pró-competição; direita reforça ainda mais; formação não altera percepções.\\

50 & O Brasil deveria priorizar a produção interna mesmo com preços maiores & 170 & Logit ord.\ (limiares ok) & $-$ ($\beta = \mathbf{-0{,}615}$; $p = 0{,}000$) & $-0{,}490$; $p = 0{,}425$ & (nenhum) & 1{,}214 & 1{,}000 & 1{,}154 & Público tende a priorizar preços baixos; esquerda enfatiza produção interna; economistas não apresentam diferença significativa.\\

51 & A automação prejudica o mercado de trabalho & 172 & Logit ord.\ (limiares ok) & $-$ ($\beta = \mathbf{-0{,}337}$; $p = 0{,}029$) & $-0{,}505$; $p = 0{,}529$ & Você é homme? Sim: $+$; Qual é a sua faixa etária? 46 a 55 anos: $+$; Qual é a sua faixa etária? 66 anos ou mais: $+$ & 0{,}368 & 0{,}235 & 0{,}318 & Ideia de prejuízo da automação amplamente rejeitada; esquerda um pouco mais apreensiva; economistas igualmente despreocupados.\\

52 & O Brasil deveria adotar política de livre comércio com outros países & 171 & Logit ord.\ (limiares ok) & $+$ ($\beta = \mathbf{0{,}646}$; $p = 0{,}000$) & $0{,}224$; $p = 0{,}713$ & (nenhum) & 1{,}429 & 1{,}471 & 1{,}398 & Apoio majoritário ao livre-comércio; direita apoia mais; economistas acompanham esse apoio.\\

\bottomrule 
\insertTableNotes 
\end{longtable} 
\end{ThreePartTable}


\endgroup
\end{landscape}
% ---------- FIM DA TABELA ----------

\chapter{Repositório de Códigos e Dados}\label{ap:repo}

\section{Visão geral}
Este apêndice documenta os materiais de \textbf{reprodutibilidade} do estudo: adaptação da SAEE ao contexto brasileiro, \textit{pipelines} de tratamento, modelos (\textit{logit} binário e \textit{logit} ordenado), geração de tabelas/figuras e o pacote de dados \textbf{anonimizados} utilizado nos resultados. As rotinas são implementadas em \textbf{Python} (e.g., \texttt{pandas}, \texttt{statsmodels}) e acompanham instruções para execução controlada.

\section{URL oficial e versão}
Repositório (GitHub): \url{https://github.com/Schadenlord/tcc_esag}\\
Versão analisada: \texttt{v1.0-thesis} (tag)\footnote{A tag fixa garante que os resultados descritos no TCC possam ser reproduzidos \emph{exatamente} no futuro, mesmo que o repositório continue evoluindo. Recomenda-se registrar também o \texttt{commit} específico no \texttt{README}.}

\section{Metadados essenciais (para auditoria)}
\begin{itemize}
  \item Projeto e ética: ``Econometria Comportamental na Política Pública: Análise de Vieses Cognitivos Políticos''. \textbf{CAAE: 89374225.9.0000.0118}. \textbf{Parecer: 7.719.326} (Res.~CNS 510/2016).
  \item Janela de coleta: \textbf{ago–out/2025} (questionário estruturado em plataforma digital).
  \item Amostra: \textbf{$n=183$} respostas válidas (com $n$ efetivo por item reportado nas tabelas).
  \item Modelagem: \textit{logit} binário e \textit{logit} ordenado com controles sociodemográficos/ideológicos; estimação em \texttt{Python}.
\end{itemize}

\section{Como replicar}\label{ap:repo:execucao}

\subsection*{Opção A — Notebook (recomendado)}
\begin{enumerate}
  \item Tenha \textbf{Python~$\geq$~3.10}. (Opcional) Crie um ambiente virtual:
\begin{verbatim}
python -m venv .venv
# Windows PowerShell:
.venv\Scripts\Activate.ps1
# Linux/macOS:
source .venv/bin/activate
\end{verbatim}
  \item Instale as dependências. Se existir \texttt{requirements.txt} na raiz, execute:
\begin{verbatim}
pip install -U pip wheel
pip install -r requirements.txt
\end{verbatim}
  \noindent Caso contrário, instale o mínimo necessário:
\begin{verbatim}
pip install -U pip wheel
pip install pandas statsmodels numpy matplotlib jupyter
\end{verbatim}
  \item Abra o notebook principal:
\begin{verbatim}
jupyter notebook Textuais/analise/analise.ipynb
\end{verbatim}
  \item Execute todas as células. As saídas serão salvas nas pastas indicadas no próprio notebook (tabelas e figuras).
\end{enumerate}

\subsection*{Opção B — Execução não interativa (script, se disponível)}
Algumas versões podem fornecer um \texttt{runner} em \texttt{src/}. Quando presente:
\begin{enumerate}
  \item Configure o ambiente conforme a Opção A (dependências).
  \item Execute:
\begin{verbatim}
python src/run_analysis.py
\end{verbatim}
  \item Verifique as pastas de \texttt{outputs} (tabelas/figuras/logs).
\end{enumerate}

\section{Estrutura do repositório (visão prática)}
\begin{verbatim}
tcc_esag/
|-- PreTextuais/                 # capa, folha de rosto, resumo(s), sumário etc.
|-- Textuais/
|   |-- ...                      # capítulos (introdução, teoria, método, resultados, conclusão)
|   `-- analise/
|       |-- analise.ipynb        # notebook principal de replicação
|       |-- data/                # dados ANONIMIZADOS e dicionário de variáveis (quando aplicável)
|       |-- outputs/
|       |   |-- tables/          # tabelas reproducíveis (modelos, descritas, etc.)
|       |   `-- figures/         # figuras reproducíveis (gráficos)
|       `-- logs/                # registros de execução (opcional)
|-- PosTextuais/                 # apêndices, anexos e materiais adicionais
|-- referencias.bib              # base bibliográfica (ABNT/abntex2)
|-- abntex2-alf_revNBR2023.bst   # estilo bibliográfico utilizado (BibTeX)
|-- tcc_pronto.tex               # arquivo principal LaTeX (entry point)
|-- PacotesBasicos.tex           # configuração de pacotes/estilo
|-- README.md                    # guia de compilação e reprodutibilidade
`-- requirements.txt             # dependências Python (se presente)
\end{verbatim}

\section{Saídas e correspondência com o texto}
As tabelas e figuras geradas no diretório \texttt{Textuais/analise/outputs/} reproduzem os resultados descritos nos capítulos de \textbf{Resultados} e \textbf{Discussão}. Cada artefato traz nomes e/ou comentários que permitem mapear o objeto no PDF (e.g., ``\texttt{tables/}'' para quadros de coeficientes; ``\texttt{figures/}'' para gráficos). Quando aplicável, o notebook inclui células com referências cruzadas (``Section--Figure/Table'') para facilitar a checagem.

\section{Observações éticas e LGPD}
Os dados disponibilizados são \textbf{estritamente anonimizados}, sem identificação direta ou quase-identificação, em conformidade com o parecer do CEP (CAAE e Parecer acima) e com a LGPD. Dados brutos sensíveis \textbf{não} são publicados. Qualquer compartilhamento adicional seguirá termo específico e aprovação ética, quando cabível.

\section{Licenças e citação}
Salvo indicação em contrário no \texttt{README.md}:
\begin{itemize}
  \item \textbf{Código}: licença recomendada \textit{MIT}.
  \item \textbf{Texto/figuras do TCC}: recomendada \textit{CC BY-NC 4.0}.
  \item \textbf{Dados anonimizados}: uso acadêmico, com citação obrigatória deste repositório e do TCC.
\end{itemize}

\noindent Exemplo de citação (Bib\TeX):
\begin{verbatim}
@misc{schaden2025_tcc_repo,
  author       = {Schaden, Bruno Francisco},
  title        = {Racionalidade Coletiva em Xeque:
                  Uma Investigação Comportamental sobre Percepções
                  Econômicas no Brasil},
  year         = {2025},
  howpublished = {Repositório GitHub},
  note         = {Arquivos fonte LaTeX, código e saídas do TCC (UDESC/ESAG).
                  Orientação: Marianne Zwiling Stampe.
                  CAAE 89374225.9.0000.0118.},
  url          = {https://github.com/Schadenlord/tcc_esag},
  urldate      = {2025-10-26}
}
\end{verbatim}

\section{Reprodutibilidade do PDF}
A compilação do manuscrito utiliza \texttt{abntex2} com \texttt{BibTeX} e o estilo \texttt{abntex2-alf\_revNBR2023.bst}. Recomenda-se \texttt{latexmk} para \emph{build} automático do PDF (ver instruções no \texttt{README.md}). Todas as referências citadas no texto estão em \texttt{referencias.bib}.

\bigskip
\noindent\emph{Nota:} Em caso de divergência futura entre versões, prevalece a tag \texttt{v1.0-thesis} indicada neste apêndice.

\end{apendicesenv}
% ---