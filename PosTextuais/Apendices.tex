
% ----------------------------------------------------------
% Apêndices
% ----------------------------------------------------------

% ---
% Inicia os apêndices
% ---
\begin{apendicesenv}

% Imprime uma página indicando o início dos apêndices
%\partapendices



% ----------------------------------------------------------
\chapter{Quadro Síntese das Abordagens sobre (I)Racionalidade Humana e Política}
% ----------------------------------------------------------

\label{apendice:quadro_iracionalidade}

\begin{quadro}[htbp]
\caption{Quadro Síntese – (I)Racionalidade Humana e Política: Principais Abordagens}

\begin{longtable}{p{0.15\textwidth} p{0.23\textwidth} p{0.34\textwidth} p{0.20\textwidth}}
\toprule
\textbf{Autor/Obra (Ano)} & \textbf{Tópico/Conceito Central} & \textbf{Principais Contribuições e Evidências} & \textbf{Relação com o TCC} \\
\midrule
\endfirsthead
\multicolumn{4}{c}{\textbf{Quadro – Continuação}} \\
\toprule
\textbf{Autor/Obra (Ano)} & \textbf{Tópico/Conceito Central} & \textbf{Principais Contribuições e Evidências} & \textbf{Relação com o TCC} \\
\midrule
\endhead

Adam Smith (1759; 1776) & Moralidade e racionalidade econômica & Integra interesse próprio e normas morais; reconhece limites do modelo racional estrito. & Critica o homo economicus e destaca a ética na decisão pública. \\
Herbert Simon (1955) & Racionalidade Limitada & Mostra que decisões são feitas com informação e tempo limitados; conceito de “satisficing”. & Base para analisar limites cognitivos em escolhas políticas. \\
Kahneman \& Tversky (1974, 2011) & Heurísticas e Vieses Cognitivos & Demonstram vieses como representatividade e ancoragem, fundamentais para economia comportamental. & Explica previsibilidade dos desvios do eleitor médio. \\
Anthony Downs (1957) & Ignorância Racional & Eleitor não busca informação pois custo supera o benefício do voto. & Fundamenta debate sobre baixa informação e democracia. \\
Bryan Caplan (2007) & Irracionalidade Racional & Eleitores mantêm crenças falsas por motivações emocionais, não apenas ignorância. & Mostra padrões sistemáticos de viés com impacto em políticas. \\
Hayek (1945) & Limitação do conhecimento & Defende dispersão do conhecimento, inviabilizando decisões centralizadas. & Sustenta crítica à ilusão de controle racional em políticas. \\
Bhagwati, Sowell, Bastiat & Viés antimercado e antiestrangeiro & Abordam recorrência de protecionismo, desconfiança do livre comércio e argumentos populares falhos. & Ampliam análise sobre vieses e políticas populistas. \\
Nyhan, Kahan, Sunstein, Rossini et al. & Resistência à revisão de crenças & Mostram que vieses de confirmação e bolhas informacionais dificultam revisão de crenças e promovem polarização. & Explicam a persistência da desinformação política. \\
\bottomrule
\end{longtable}
\end{quadro}

\chapter{Tabela-Síntese Mestra dos Resultados Empíricos}
\label{apendice:tabela_sintese}

\begin{landscape}
\begingroup
\scriptsize
\setlength{\tabcolsep}{3.5pt}

\begin{ThreePartTable}
\begin{TableNotes}[flushleft]\footnotesize
\item \textit{\textbf{Notas gerais:}} Valores de $\beta$ e $p$ entre parênteses. Coeficientes em \textbf{negrito} indicam $p<0{,}05$. ``Econ'' refere-se à dummy de formação em Economia. Em ``Controles'', listam-se apenas efeitos consistentes ($p<0{,}10$ e sinal estável entre especificações). Médias: escala normalizada da pesquisa.
\item \textbf{``n/a''} indica que o modelo não foi reportado: \textit{n/a (não estimável)} quando a estimação falhou (p.\,ex., falta de variação na DV, separação quase-perfeita, colinearidade perfeita ou não convergência); \textit{n/a (não estável)} quando os resultados variaram de forma sensível a perturbações razoáveis (p.\,ex., limiares mal ordenados, erros-padrão inflados), não atendendo ao critério mínimo de estabilidade.
\item \textbf{``(nenhum)''} na coluna \emph{Controles} significa que, embora os controles tenham sido incluídos no modelo, nenhum apresentou associação estatisticamente significativa a $10\%$ \emph{com sinal estável}; por isso nenhum controle é destacado na síntese.
\item \textbf{Sinal ``$+$''/``$-$'':} refere-se ao \emph{sinal do coeficiente} estimado. Em logit ordenado, com a escala ordenada de menor para maior, $\beta>0$ (``$+$'') desloca a probabilidade para categorias mais altas da DV; $\beta<0$ (``$-$'') desloca para categorias mais baixas (interpretação: maior/menor concordância quando a escala é crescente).
\end{TableNotes}

\begin{longtable}{%
  >{\RaggedRight\arraybackslash}p{3.0cm}
  >{\centering\arraybackslash}p{0.6cm}
  >{\RaggedRight\arraybackslash}p{2cm}
  >{\RaggedRight\arraybackslash}p{3.5cm}
  >{\RaggedRight\arraybackslash}p{3.2cm}
  >{\RaggedRight\arraybackslash}p{3.2cm}
  >{\centering\arraybackslash}p{1.2cm}
  >{\centering\arraybackslash}p{1.2cm}
  >{\centering\arraybackslash}p{1.4cm}
  >{\RaggedRight\arraybackslash}p{4.5cm}}
\caption{Tabela-Síntese Mestra — Resultados dos Modelos Ordenados (37 DVs)}\label{tab:resultados_ordenados}\\
\toprule
\textbf{DV (Pergunta)} & \textbf{N} & \textbf{Modelo} & \textbf{Espectro político} & \textbf{Formação em Economia} & \textbf{Controles ($p<0{,}10$; sinal)} & \textbf{Média (econ=0)} & \textbf{Média (econ=1)} & \textbf{Média contraf.} & \textbf{Leitura}\\
\midrule
\endfirsthead

\multicolumn{10}{l}{\footnotesize \textit{(continuação)}}\\
\toprule
\textbf{DV (Pergunta)} & \textbf{N} & \textbf{Modelo} & \textbf{Espectro político} & \textbf{Formação em Economia} & \textbf{Controles ($p<0{,}10$; sinal)} & \textbf{Média (econ=0)} & \textbf{Média (econ=1)} & \textbf{Média contraf.} & \textbf{Leitura}\\
\midrule
\endhead

\midrule
\multicolumn{10}{r}{\footnotesize \textit{continua na próxima página}}\\
\endfoot

\bottomrule
\endlastfoot

1. Os impostos são muito altos & 170 & Logit ord. (limiares ok) & $+\;(\beta=0{,}668;\;p=0{,}000)$ & $-0{,}743;\;p=0{,}246$ & Faixa etária (26--35, $\leq$18: $-$) & 1{,}448 & 1{,}312 & 1{,}492 & Percepção amplamente compartilhada; economistas concordam quase tanto; direita intensifica levemente. \\
2. O déficit federal é grande demais (dívida pública) & 169 & Logit ord. (limiares ok) & $+\;(\beta=0{,}737;\;p=0{,}000)$ & $\mathbf{-1{,}297;\;p=0{,}054}$ & (nenhum) & 1{,}516 & 1{,}188 & 1{,}522 & Público vê excesso; economistas menos alarmistas; formação técnica atenua preocupação. \\
3. O gasto com ajuda externa é alto demais & 170 & Logit ord. (limiares ok) & $\mathbf{+\;(\beta=0{,}260;\;p=0{,}030)}$ & $\mathbf{-1{,}283;\;p=0{,}042}$ & Sexo (masc. $-$; $p=0{,}018$) & 0{,}838 & 0{,}438 & 0{,}871 & Grande viés no público; economistas discordam amplamente; formação corrige. \\
4. Temos imigrantes demais & 171 & Logit ord. (limiares ok) & $+\;(\beta=0{,}005;\;p=0{,}976)$ & $-0{,}310;\;p=0{,}676$ & Faixa etária (46--55, 66+: $+$) & 0{,}240 & 0{,}250 & 0{,}337 & Sem diferença ideológica/ formação; ausência de viés sistemático. \\
5. Há deduções demais para as empresas (Impostos) & 171 & Logit ord. (limiares ok) & $+\;(\beta=0{,}232;\;p=0{,}052)$ & $1{,}013;\;p=0{,}080$ & (nenhum) & 1{,}052 & 1{,}375 & 0{,}981 & Leigos e economistas veem deduções elevadas; sem diferença robusta por formação. \\
6. A educação e a qualificação profissional são inadequadas & 171 & Logit ord. (limiares ok) & $+\;(\beta=0{,}106;\;p=0{,}408)$ & $0{,}628;\;p=0{,}338$ & (nenhum) & 1{,}474 & 1{,}688 & 1{,}501 & Convergência: ambos veem insuficiência; sem viés a corrigir. \\
7. A seguridade social atende pessoas demais & 171 & Logit ord. (limiares ok) & $\mathbf{+\;(\beta=0{,}486;\;p=0{,}000)}$ & $-0{,}930;\;p=0{,}133$ & (nenhum) & 0{,}806 & 0{,}688 & 0{,}949 & Público acha que atende ``demais''; direita acentua fortemente essa visão. \\
8. Ações afirmativas dão vantagens demais & 170 & Logit ord. (limiares ok) & $\mathbf{+\;(\beta=0{,}480;\;p=0{,}002)}$ & $-0{,}367;\;p=0{,}644$ & (nenhum) & 0{,}318 & 0{,}188 & 0{,}276 & Viés ideológico marcante: direita aumenta crença contrária às cotas. \\
9. O governo regulamenta muito os negócios & 171 & Logit ord. (limiares ok) & $\mathbf{+\;(\beta=0{,}297;\;p=0{,}020)}$ & $-0{,}402;\;p=0{,}523$ & Sexo (masc. $+$; $p=0{,}001$) & 1{,}142 & 1{,}000 & 1{,}240 & Viés anti-regulação em ambos; homens mais propensos; direita reforça. \\
10. As pessoas não poupam o bastante & 171 & Logit ord. (limiares ok) & $\mathbf{+\;(\beta=1{,}003;\;p=0{,}000)}$ & $-0{,}940;\;p=0{,}139$ & Sexo (masc. $+$; $p=0{,}083$) & 0{,}994 & 1{,}000 & 1{,}062 & Consenso de insuficiência de poupança; direita atribui mais culpa. \\
11. As empresas lucram demais & 170 & Logit ord. (limiares ok) & $\mathbf{-\;(\beta=-0{,}817;\;p=0{,}000)}$ & $-0{,}239;\;p=0{,}692$ & (nenhum) & 0{,}562 & 1{,}625 & 0{,}813 & Viés anti-lucro (esquerda) contraposto por economistas; correção técnica. \\
12. Altos executivos ganham demais & 170 & Logit ord. (limiares ok) & $\mathbf{-\;(\beta=-0{,}661;\;p=0{,}000)}$ & $-0{,}473;\;p=0{,}493$ & Sexo (masc. $-$; $p=0{,}025$) & 0{,}714 & 0{,}750 & 0{,}878 & Público mais crítico; economistas discordam; homens menos críticos. \\
13. Produtividade aumenta devagar demais & 170 & Logit ord. (limiares ok) & $+\;(\beta=0{,}006;\;p=0{,}961)$ & $\mathbf{+\;(\beta=1{,}552;\;p=0{,}023)}$ & Sexo (masc. $+$; $p=0{,}056$) & 1{,}110 & 1{,}750 & 1{,}279 & Convergência; economistas mais preocupados (efeito de formação). \\
14. A tecnologia causa demissões & 169 & Logit ord. (limiares ok) & $-\;(\beta=-0{,}182;\;p=0{,}193)$ & $-0{,}259;\;p=0{,}681$ & (nenhum) & 0{,}422 & 0{,}562 & 0{,}658 & Leve discordância; economistas discordam mais; sem viés ludista forte. \\
15. Empresas estão enviando funcionários ao exterior & 170 & Logit ord. (limiares ok) & $-\;(\beta=-0{,}139;\;p=0{,}352)$ & $\mathbf{+\;(\beta=1{,}394;\;p=0{,}047)}$ & (nenhum) & 0{,}297 & 0{,}438 & 0{,}171 & Leigos pouco veem fuga; economistas notam mais (atenção técnica ao fenômeno). \\
16. Empresas estão reduzindo postos de trabalho & 171 & Logit ord. (limiares ok) & $\mathbf{-\;(\beta=-0{,}265;\;p=0{,}030)}$ & $0{,}095;\;p=0{,}872$ & (nenhum) & 0{,}658 & 0{,}875 & 0{,}852 & Ambos percebem cortes; esquerda enfatiza mais (espectro negativo). \\
17. Empresas investem pouco em qualificação & 170 & Logit ord. (limiares ok) & $\mathbf{-\;(\beta=-0{,}394;\;p=0{,}001)}$ & $0{,}262;\;p=0{,}654$ & (nenhum) & 1{,}071 & 1{,}312 & 1{,}218 & Crítica compartilhada, mais forte à esquerda; formação não mitiga. \\
18. Apoio a corte de impostos & 171 & Logit ord. (limiares ok) & $\mathbf{+\;(\beta=0{,}982;\;p=0{,}000)}$ & $-0{,}498;\;p=0{,}520$ & (nenhum) & 1{,}658 & 1{,}375 & 1{,}540 & Forte apoio geral; direita apoia mais; sem diferença robusta por formação. \\
19. Mais mulheres na força de trabalho (positivo) & 171 & Logit ord. (limiares ok) & $\mathbf{-\;(\beta=-0{,}698;\;p=0{,}000)}$ & $0{,}627;\;p=0{,}416$ & Sexo (masc. $-$; $p=0{,}015$) & 1{,}419 & 1{,}750 & 1{,}638 & Apoio geral; homens menos favoráveis; esquerda apoia mais. \\
20. Aumento do uso de tecnologia (positivo) & 171 & \textit{n/a} (não estimável) & \textit{n/a} & \textit{n/a} & (modelo não convergiu) & 1{,}923 & 2{,}000 & 1{,}878 & Quase consenso pró-tecnologia; falta de variação impediu estimação. \\
21. Acordos comerciais (positivo) & 171 & \textit{n/a} (não estimável) & \textit{n/a} & \textit{n/a} & (modelo não convergiu) & 1{,}916 & 1{,}875 & 1{,}950 & Abertura bem vista por ambos; sem efeitos identificáveis. \\
22. Redução recente de postos em grandes empresas & 170 & Logit ord. (limiares ok) & $\mathbf{+\;(\beta=0{,}520;\;p=0{,}000)}$ & $-0{,}992;\;p=0{,}153$ & (nenhum) & 0{,}526 & 0{,}250 & 0{,}491 & Leigos percebem mais demissões; direita eleva essa percepção. \\
23. Acordos comerciais: bons/ruins p/ economia & 169 & \textit{n/a} (não estável) & \textit{n/a} & \textit{n/a} & (modelo não estável) & 1{,}688 & 1{,}800 & 1{,}772 & Maioria acha bons; sem influência consistente de ideologia/formação. \\
24. Maior responsável por alta dos combustíveis & 169 & Logit ord. (limiares ok) & $\mathbf{-\;(\beta=-0{,}304;\;p=0{,}010)}$ & $0{,}343;\;p=0{,}559$ & (nenhum) & 0{,}908 & 1{,}125 & 0{,}972 & Público culpa mais governo; economistas distribuem causas (mercado externo etc.). \\
25. Nível dos preços dos combustíveis & 171 & \textit{n/a} (não estável) & \textit{n/a} & \textit{n/a} & (modelo não estável) & 1{,}858 & 1{,}750 & 1{,}848 & Quase unanimidade: combustíveis altos; falta variação. \\
26. Quanto o presidente pode melhorar a economia & 167 & Logit ord. (limiares ok) & $\mathbf{+\;(\beta=0{,}264;\;p=0{,}042)}$ & $0{,}119;\;p=0{,}854$ & (nenhum) & 1{,}434 & 1{,}600 & 1{,}544 & Ambos atribuem algum poder; direita atribui mais. \\
27. Novos postos de trabalho pagam bem/mal & 168 & Logit ord. (limiares ok) & $\mathbf{+\;(\beta=0{,}669;\;p=0{,}000)}$ & $0{,}102;\;p=0{,}877$ & Escolaridade (EM compl., Pós-grad.: $+$) & 0{,}526 & 0{,}500 & 0{,}483 & Consenso: pagam mal; direita um pouco mais otimista; mais escolarizados menos pessimistas. \\
28. Desigualdade hoje vs. 20 anos & 170 & Logit ord. (limiares ok) & $\mathbf{-\;(\beta=-0{,}288;\;p=0{,}020)}$ & $-0{,}318;\;p=0{,}594$ & (nenhum) & 1{,}318 & 1{,}062 & 1{,}180 & Público crê que aumentou; esquerda acentua; economistas menos pessimistas. \\
29. Rendas familiares vs. custo de vida (20 anos) & 169 & Logit ord. (limiares ok) & $+\;(\beta=0{,}221;\;p=0{,}097)$ & $+\;(\beta=1{,}111;\;p=0{,}096)$ & Faixa etária (46--55: $-$; $p=0{,}018$) & 0{,}536 & 0{,}750 & 0{,}388 & Público mais pessimista; formação reduz pessimismo sobre renda real. \\
30. Salários vs. custo de vida (20 anos) & 168 & \textit{n/a} (não estável) & \textit{n/a} & \textit{n/a} & (modelo não estável) & 0{,}421 & 0{,}625 & 0{,}243 & Ambos veem salários atrás; economistas um pouco menos pessimistas. \\
31. Padrão de vida nos próximos 5 anos & 165 & Logit ord. (limiares ok) & $-\;(\beta=-0{,}171;\;p=0{,}157)$ & $0{,}427;\;p=0{,}485$ & Faixa etária (46--55: $-$; $p=0{,}016$) & 0{,}671 & 0{,}562 & 0{,}448 & Pessimismo moderado compartilhado; economistas levemente menos pessimistas. \\
32. Padrão de vida dos filhos vs. dos pais & 169 & Logit ord. (limiares ok) & $+\;(\beta=0{,}229;\;p=0{,}082)$ & $0{,}268;\;p=0{,}647$ & Faixa etária (36--45: $-$; $p=0{,}084$) & 1{,}425 & 1{,}250 & 1{,}154 & Divisão equilibrada; sem viés extremo; economistas um pouco menos otimistas. \\
33. Filhos < 30: padrão de vida futuro & 128 & \textit{n/a} (não estável) & \textit{n/a} & \textit{n/a} & (N menor) & 1{,}553 & 1{,}357 & 1{,}331 & Leve otimismo; economistas menos otimistas; sem significância robusta. \\
34. Reforma da Previdência é necessária & 169 & Logit ord. (limiares ok) & $\mathbf{+\;(\beta=0{,}541;\;p=0{,}000)}$ & $-0{,}843;\;p=0{,}185$ & Escolaridade (Sup. Incompl., Pós: $+$) & 1{,}556 & 1{,}188 & 1{,}460 & Apoio geral; direita e maior escolaridade elevam apoio; economistas majoritariamente favoráveis. \\
35. Reforma trabalhista é necessária & 169 & Logit ord. (limiares ok) & $\mathbf{+\;(\beta=0{,}657;\;p=0{,}000)}$ & $-0{,}722;\;p=0{,}252$ & Escolaridade (EM compl., Sup. compl./incompl., Pós: $+$) & 1{,}503 & 1{,}188 & 1{,}426 & Apoio em ambos; direita e escolaridade elevam apoio. \\
36. Reforma tributária é necessária & 169 & Logit ord. (limiares ok) & $+\;(\beta=0{,}211;\;p=0{,}332)$ & $0{,}751;\;p=0{,}540$ & Escolaridade (EM compl., Sup. compl.: $+$) & 1{,}810 & 1{,}938 & 1{,}877 & Quase unanimidade pró-reforma; sem diferenças ideológicas/ formação significativas. \\

\bottomrule
\insertTableNotes
\end{longtable}
\end{ThreePartTable}

\endgroup
\end{landscape}
% ---------- FIM DA TABELA ----------


\chapter{Epílogo Metafísico: Entre a Evidência e a Esperança \\ \smallskip \textit{(Reflexão Filosófica para Além do Empírico)}}

\noindent
\textbf{Nota de método.} \textit{A partir deste ponto, movo-me do domínio do empiricamente testável para a esfera da reflexão filosófica e institucional. Como Popper recomendaria, assumo aqui apenas conjecturas abertas ao debate, inspiradas — e não determinadas — pelos achados empíricos deste trabalho.}

\vspace{1em}

A política moderna consagrou na democracia a liberdade das massas como princípio supremo. Mas que liberdade é essa? A de fazer tudo o que se quer, mesmo sem saber por quê? Ou a de resistir ao próprio desejo, quando tudo grita por rendição? Talvez liberdade seja, afinal, a capacidade de fazer não o que se quer, mas o que se deve — mesmo (ou sobretudo) quando não se deseja. E o dever, aqui, não nasce de comando externo, mas da gravidade interna de quem aprendeu a discernir.

A democracia, em sua forma atual, opera como o teatro onde desejos transitam com legitimidade garantida. A cada eleição, reencena-se a soberania do querer sobre o dever, do imediato sobre o necessário, do impulso sobre o juízo. O sistema, tal como está, recompensa quem responde rápido, não quem pensa devagar. E há algo de estrutural nisso: o calendário democrático segue o tempo do relógio — mas a justiça segue o tempo da consciência. Como ensinou Santo Agostinho, “há um tempo exterior, que corre; e há um tempo interior, que pondera”. A política deveria aspirar a este último.

É nesse palco que se manifesta o paradoxo já desvelado por Caplan: mesmo eleitores bem-intencionados erram. Erram não por ignorância ocasional, mas por uma racionalidade limitada, sistematicamente enviesada, que prefere o conforto da ilusão ao rigor da verdade. E se, como advertiu Hayek, o conhecimento social está sempre disperso e fragmentado, todo sistema que aposta na onisciência da vontade coletiva naufraga entre o caos e a tirania.

Assim, quando um eleitor escolhe, não apenas escolhe: julga. E esse juízo, por mais livre que pareça, carrega uma cadeia invisível de valores, memórias, convicções e desejos. A liberdade, portanto, não é um ponto de partida; é o último degrau de uma escada moral. Só é verdadeiramente livre quem sabe por que deve escolher o que escolhe — e está disposto a não querer o que não deve.

Se a liberdade for reduzida à soma dos desejos, o resultado será sempre um sistema vulnerável à manipulação, à irracionalidade e à moral líquida. Não há mercado perfeito de ideias, nem há “engenharia institucional” capaz de garantir, por decreto, que a liberdade coincida com o bem comum. Como Acemoglu e Robinson demonstram, a liberdade nasce no estreito corredor entre o excesso de poder e a ausência de ordem: não é dádiva, mas conquista precária, resultado de uma tensão viva entre Estado e sociedade.

Votar é fácil; escolher bem é difícil. Não basta, portanto, melhorar os eleitores. É preciso repensar as engrenagens do sistema que transforma preferências em leis. Um sistema político que se submete inteiramente às oscilações das paixões populares perde sua capacidade de guardar aquilo que não muda: a justiça, o bem, a verdade. Precisamos, pois, de instituições que operem não apenas no tempo do poder, mas no tempo do juízo. E esse tempo não é democrático nem autoritário — é simplesmente humano.

Talvez o caminho esteja em resgatar uma arquitetura política onde a liberdade não seja apenas protegida, mas formada; onde a decisão coletiva seja amparada por raízes mais fundas que as modas da hora; onde o bem comum não seja o resultado do grito da maioria, mas o fruto do silêncio que pensa. Pois, se algo há — e há — é a necessidade de juízo. Porque o nada, isto é, o puro desejo, não institui ordem. Só o ser — aquilo que permanece, que resiste ao fluxo — pode fundar liberdade duradoura.

Este trabalho não pretende esgotar essas possibilidades. Limita-se a constatar, com base empírica e fundamento teórico, que o problema não está apenas nos votos, mas naquilo que os votos ignoram, no que, como diria Bastiat, “o que se vê e o que não se vê”. É preciso olhar para o que não se vê, analisando as evidências com cuidado. E que, para que a verdade tenha alguma chance, será preciso mais do que pedagogia: será preciso estrutura, limite e tempo.

Há decisões que não cabem em ciclos de quatro anos. Há verdades que não se revelam em sondagens. Há justiças que não se alcançam por aclamação. E há liberdades que só florescem quando já não estamos mais olhando para nós mesmos — mas para o bem que nos excede.

A política, quando é séria, não é desejo. É juízo. E o juízo não se apressa. Espera o tempo certo — o tempo do discernimento — para decidir com liberdade verdadeira. Talvez esse tempo ainda venha. Ou talvez já esteja vindo, silencioso, em meio às ruínas de um mundo que confundiu liberdade com vontade e nos deixou uma moral relativa com a qual lidar. Seja como for, uma coisa é certa: não será a pedagogia que nos salvará, mas a arquitetura moral que soubermos legar — e que quisermos deixar — para as próximas gerações.

Por fim, não pretendo impor juízos, mas apenas propor reflexão. Que o dissenso, longe de nos afastar, seja ocasião de busca conjunta por aquilo que resiste ao tempo e nos chama para além de nós mesmos. Não encerro com respostas definitivas, mas com a confissão do limite: a verdade é como a água pura que tentamos reter entre as mãos — ela escorre por entre os dedos, mas ainda assim refresca, orienta e purifica. Talvez, como intuiu Jung, só possamos nos aproximar do valor mais alto reconhecendo aquilo que, em nosso íntimo, ocupa o lugar supremo, fonte de todo sentido. E se toda busca sincera se inclina, por fim, diante desse Bem maior que nos transcende e nos funda, resta-nos apenas o papel de aprendizes: atentos ao mistério, pacientes com o tempo, humildes diante do real.

\begin{flushright}
\textit{
“Ninguém alcança a verdade senão aquele\
que ousa caminhar entre dúvidas,\
mas não abandona a esperança.\
Pois só na humildade diante do Mistério\
a razão encontra repouso e o coração, paz.”\
— inspirado em Santo Agostinho, Confissões
}
\end{flushright}

\end{apendicesenv}
% ---