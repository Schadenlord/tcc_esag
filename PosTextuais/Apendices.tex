
% ----------------------------------------------------------
% Apêndices
% ----------------------------------------------------------

% ---
% Inicia os apêndices
% ---
\begin{apendicesenv}

% Imprime uma página indicando o início dos apêndices
%\partapendices



% ----------------------------------------------------------
\chapter{Quadro Síntese das Abordagens sobre (I)Racionalidade Humana e Política}
% ----------------------------------------------------------

% no preâmbulo:
% \usepackage{longtable,booktabs,caption}

{\renewcommand{\LTcaptype}{quadro}% <- chave
 \captionsetup{type=quadro}%       <- opcional, mantém o rótulo "Quadro"
 \begin{longtable}{p{0.15\textwidth} p{0.23\textwidth} p{0.34\textwidth} p{0.20\textwidth}}
 \caption{Quadro Síntese – (I)Racionalidade Humana e Política: Principais Abordagens}
 \label{apendice:quadro_iracionalidade}\\

 \toprule
 \textbf{Autor/Obra (Ano)} & \textbf{Tópico/Conceito Central} & \textbf{Principais Contribuições e Evidências} & \textbf{Relação com o TCC} \\
 \midrule
 \endfirsthead

 \multicolumn{4}{c}{\small\textbf{Quadro \thequadro\ (continuação)}}\\
 \toprule
 \textbf{Autor/Obra (Ano)} & \textbf{Tópico/Conceito Central} & \textbf{Principais Contribuições e Evidências} & \textbf{Relação com o TCC} \\
 \midrule
 \endhead

 \midrule
 \multicolumn{4}{r}{\small\itshape Continua na próxima página}\\
 \endfoot

 \bottomrule
 \multicolumn{4}{l}{\footnotesize Fonte: elaboração própria.}
 \endlastfoot

 Adam Smith (1759; 1776) & Moralidade e racionalidade econômica & Integra interesse próprio e normas morais; reconhece limites do modelo racional estrito. & Critica o \textit{homo economicus} e destaca a ética na decisão pública. \\
 Herbert Simon (1955) & Racionalidade Limitada & Mostra que decisões são feitas com informação e tempo limitados; “satisficing”. & Base para analisar limites cognitivos em escolhas políticas. \\
 Kahneman \& Tversky (1974, 2011) & Heurísticas e Vieses Cognitivos & Representatividade, ancoragem etc. & Explica previsibilidade dos desvios do eleitor médio. \\
 Anthony Downs (1957) & Ignorância Racional & Custo de informar-se > benefício do voto. & Fundamenta debate sobre baixa informação e democracia. \\
 Bryan Caplan (2007) & Irracionalidade Racional & Crenças falsas por motivações emocionais. & Viés sistemático com impacto em políticas. \\
 Hayek (1945) & Limitação do conhecimento & Conhecimento disperso; centralização inviável. & Crítica à ilusão de controle racional. \\
 Bhagwati, Sowell, Bastiat & Viés antimercado/antiestrangeiro & Protecionismo recorrente e argumentos populares. & Vieses e políticas populistas. \\
 Nyhan, Kahan, Sunstein, Rossini et al. & Resistência à revisão & Confirmação e bolhas informacionais. & Persistência da desinformação política. \\
 \end{longtable}
}



\begin{landscape}

\chapter{Tabela-Síntese Mestra dos Resultados Empíricos}
\label{apendice:tabela_sintese}

\begingroup
\scriptsize
\setlength{\tabcolsep}{3.5pt}

\begin{ThreePartTable}
\begin{TableNotes}[flushleft]\footnotesize
\item \textit{\textbf{Notas gerais:}} Valores de $\beta$ e $p$ entre parênteses. Coeficientes em \textbf{negrito} indicam $p<0{,}05$. ``Econ'' refere-se à dummy de formação em Economia. Em ``Controles'', listam-se apenas efeitos consistentes ($p<0{,}10$ e sinal estável entre especificações). Médias: escala normalizada da pesquisa.
\item \textbf{``n/a''} indica que o modelo não foi reportado: \textit{n/a (não estimável)} quando a estimação falhou (p.\,ex., falta de variação na DV, separação quase-perfeita, colinearidade perfeita ou não convergência); \textit{n/a (não estável)} quando os resultados variaram de forma sensível a perturbações razoáveis (p.\,ex., limiares mal ordenados, erros-padrão inflados), não atendendo ao critério mínimo de estabilidade.
\item \textbf{``(nenhum)''} na coluna \emph{Controles} significa que, embora os controles tenham sido incluídos no modelo, nenhum apresentou associação estatisticamente significativa a $10\%$ \emph{com sinal estável}; por isso nenhum controle é destacado na síntese.
\item \textbf{Sinal ``$+$''/``$-$'':} refere-se ao \emph{sinal do coeficiente} estimado. Em logit ordenado, com a escala ordenada de menor para maior, $\beta>0$ (``$+$'') desloca a probabilidade para categorias mais altas da DV; $\beta<0$ (``$-$'') desloca para categorias mais baixas (interpretação: maior/menor concordância quando a escala é crescente).
\end{TableNotes}

\begin{longtable}{%
  >{\RaggedRight\arraybackslash}p{3.0cm}
  >{\centering\arraybackslash}p{0.6cm}
  >{\RaggedRight\arraybackslash}p{2cm}
  >{\RaggedRight\arraybackslash}p{3.5cm}
  >{\RaggedRight\arraybackslash}p{3.2cm}
  >{\RaggedRight\arraybackslash}p{3.2cm}
  >{\centering\arraybackslash}p{1.2cm}
  >{\centering\arraybackslash}p{1.2cm}
  >{\centering\arraybackslash}p{1.4cm}
  >{\RaggedRight\arraybackslash}p{4.5cm}}
\caption{Tabela-Síntese Mestra — Resultados dos Modelos Ordenados (37 DVs)}\label{tab:resultados_ordenados}\\
\toprule
\textbf{DV (Pergunta)} & \textbf{N} & \textbf{Modelo} & \textbf{Espectro político} & \textbf{Formação em Economia} & \textbf{Controles ($p<0{,}10$; sinal)} & \textbf{Média (econ=0)} & \textbf{Média (econ=1)} & \textbf{Média contraf.} & \textbf{Leitura}\\
\midrule
\endfirsthead

\multicolumn{10}{l}{\footnotesize \textit{(continuação)}}\\
\toprule
\textbf{DV (Pergunta)} & \textbf{N} & \textbf{Modelo} & \textbf{Espectro político} & \textbf{Formação em Economia} & \textbf{Controles ($p<0{,}10$; sinal)} & \textbf{Média (econ=0)} & \textbf{Média (econ=1)} & \textbf{Média contraf.} & \textbf{Leitura}\\
\midrule
\endhead

\midrule
\multicolumn{9}{r}{\footnotesize \textit{continua na próxima página}}\\
\endfoot

\bottomrule
\endlastfoot

Os impostos são muito altos & 172 & Logit ord.\ (limiares ok) & + ($\beta = \mathbf{0,674}$; $p = 0{,}000$) & $-0{,}893$; $p = 0{,}150$ & Qual é a sua faixa etária? 26 a 35 anos: $-$; Qual é a sua faixa etária? 36 a 45 anos: $-$; Qual é a sua faixa etária? Até 18 anos: $-$ & 1{,}445 & 1{,}235 & 1{,}459 & Percepção amplamente compartilhada; economistas concordam quase tanto; direita intensifica levemente.\\

O déficit federal é grande demais (dívida pública) & 171 & Logit ord.\ (limiares ok) & + ($\beta = \mathbf{0{,}731}$; $p = 0{,}000$) & $-1{,}449$; $p = \mathbf{0{,}026}$ & (nenhum) & 1{,}513 & 1{,}118 & 1{,}501 & Público vê excesso; economistas menos alarmistas; formação técnica atenua preocupação.\\

O gasto com ajuda externa é alto demais & 172 & Logit ord.\ (limiares ok) & + ($\beta = \mathbf{0{,}256}$; $p = 0{,}031$) & $-1{,}332$; $p = \mathbf{0{,}031}$ & Você é homem? Sim: $-$; Qual é a sua faixa etária? 46 a 55 anos: + & 0{,}832 & 0{,}412 & 0{,}864 & Grande viés no público; economistas discordam amplamente; formação corrige.\\

Temos imigrantes demais & 172 & Logit ord.\ (limiares ok) & + ($\beta = 0{,}009$; $p = 0{,}958$) & $-0{,}351$; $p = 0{,}633$ & Qual é a sua faixa etária? 46 a 55 anos: +; Qual é a sua faixa etária? 66 anos ou mais: + & 0{,}239 & 0{,}235 & 0{,}327 & Sem diferença ideológica/ formação; ausência de viés sistemático.\\

Há deduções demais para as empresas (Impostos) & 173 & Logit ord.\ (limiares ok) & + ($\beta = \mathbf{0{,}259}$; $p = 0{,}028$) & $0{,}789$; $p = 0{,}161$ & Qual é a sua faixa etária? Até 18 anos: $-$ & 1{,}058 & 1{,}294 & 0{,}987 & Leigos e economistas veem deduções elevadas; sem diferença robusta por formação.\\

A educação e a qualificação profissional são inadequadas & 172 & Logit ord.\ (limiares ok) & + ($\beta = 0{,}075$; $p = 0{,}553$) & $0{,}765$; $p = 0{,}234$ & Você se considera uma pessoa politicamente engajada? Sim: $-$ & 1{,}471 & 1{,}706 & 1{,}485 & Convergência: ambos veem insuficiência; sem viés a corrigir.\\

A seguridade social atende pessoas demais & 173 & Logit ord.\ (limiares ok) & + ($\beta = \mathbf{0{,}477}$; $p = 0{,}000$) & $-0{,}976$; $p = 0{,}108$ & Qual seu nível de escolaridade? Ensino Médio Completo: $-$; Qual é a sua faixa etária? 26 a 35 anos: $-$ & 0{,}801 & 0{,}647 & 0{,}920 & Público acha que atende “demais”; direita acentua fortemente essa visão.\\

Ações afirmativas dão vantagens demais & 172 & Logit ord.\ (limiares ok) & + ($\beta = \mathbf{0{,}479}$; $p = 0{,}002$) & $-0{,}362$; $p = 0{,}644$ & (nenhum) & 0{,}316 & 0{,}176 & 0{,}259 & Viés ideológico marcante: direita aumenta crença contrária às cotas.\\

O governo regulamenta muito os negócios & 173 & Logit ord.\ (limiares ok) & + ($\beta = \mathbf{1{,}015}$; $p = 0{,}000$) & $-0{,}978$; $p = 0{,}120$ & Você é homem? Sim: +; Qual seu nível de escolaridade? Ensino Médio Completo: +; Qual seu nível de escolaridade? Pós-graduação: +; Qual é a sua faixa etária? 36 a 45 anos: $-$ & 1{,}147 & 0{,}941 & 1{,}183 & Viés anti-regulação em ambos; homens mais propensos; direita reforça.\\

As pessoas não poupam o bastante & 173 & Logit ord.\ (limiares ok) & + ($\beta = \mathbf{0{,}248}$; $p = 0{,}041$) & $-0{,}340$; $p = 0{,}563$ & Você é homem? Sim: +; Qual é a sua faixa etária? 46 a 55 anos: +; Qual é a sua faixa etária? 66 anos ou mais: +; Qual é a sua faixa etária? Até 18 anos: + & 0{,}987 & 0{,}941 & 1{,}033 & Consenso de insuficiência de poupança; direita atribui mais culpa.\\

As empresas lucram demais & 172 & Logit ord.\ (limiares ok) & $-$ ($\beta = \mathbf{-0{,}852}$; $p = 0{,}000$) & $-0{,}298$; $p = 0{,}653$ & Qual é o seu vínculo empregatício?: $-$; Qual seu nível de escolaridade? Ensino Médio Completo: $-$; Qual seu nível de escolaridade? Ensino Médio Incompleto: $-$; Qual seu nível de escolaridade? Ensino Superior Incompleto: $-$; Qual seu nível de escolaridade? Pós-graduação: $-$; Qual é a sua faixa etária? 46 a 55 anos: +; Você se considera uma pessoa politicamente engajada? Sim: $-$ & 0{,}523 & 0{,}706 & 0{,}848 & Viés anti-lucro (esquerda) contraposto por economistas; correção técnica.\\

Altos executivos ganham demais & 172 & Logit ord.\ (limiares ok) & $-$ ($\beta = \mathbf{-0{,}662}$; $p = 0{,}000$) & $-0{,}469$; $p = 0{,}421$ & Qual é o seu vínculo empregatício?: $-$; Você é homem? Sim: $-$; Com qual grupo racial/étnico você mais se identifica? Negro: +; Qual seu nível de escolaridade? Ensino Superior Incompleto: $-$ & 0{,}714 & 0{,}750 & 0{,}878 & Público mais crítico; economistas discordam; homens menos críticos.\\

A produtividade aumenta devagar demais & 172 & Logit ord.\ (limiares ok) & $-$ ($\beta = -0{,}032$; $p = 0{,}797$) & $\mathbf{1{,}753}$; $p = 0{,}010$ & Você é homem? Sim: +; Qual é a sua faixa etária? 26 a 35 anos: $-$ & 1{,}110 & 1{,}750 & 1{,}279 & Convergência; economistas mais preocupados (efeito de formação).\\

A tecnologia causa demissões & 169 & Logit ord.\ (limiares ok) & $-$ ($\beta = -0{,}182$; $p = 0{,}193$) & $-0{,}259$; $p = 0{,}681$ & (nenhum) & 0{,}422 & 0{,}562 & 0{,}658 & Leve discordância; economistas discordam mais; sem viés ludista forte.\\

Empresas estão enviando funcionários ao exterior & 170 & Logit ord.\ (limiares ok) & $-$ ($\beta = -0{,}139$; $p = 0{,}352$) & $1{,}394$; $p = 0{,}047$ & (nenhum) & 0{,}297 & 0{,}438 & 0{,}171 & Leigos pouco veem fuga; economistas notam mais (atenção técnica ao fenômeno).\\

Empresas estão reduzindo postos de trabalho & 171 & Logit ord.\ (limiares ok) & $-$ ($\beta = \mathbf{-0{,}265}$; $p = 0{,}030$) & $0{,}095$; $p = 0{,}872$ & (nenhum) & 0{,}658 & 0{,}875 & 0{,}852 & Ambos percebem cortes; esquerda enfatiza mais (espectro negativo).\\

Empresas investem pouco em qualificação & 170 & Logit ord.\ (limiares ok) & $-$ ($\beta = \mathbf{-0{,}394}$; $p = 0{,}001$) & $0{,}262$; $p = 0{,}654$ & (nenhum) & 1{,}071 & 1{,}312 & 1{,}218 & Crítica compartilhada, mais forte à esquerda; formação não mitiga.\\

Apoio a corte de impostos & 173 & Logit ord.\ (limiares ok) & + ($\beta = \mathbf{0{,}968}$; $p = 0{,}000$) & $-0{,}249$; $p = 0{,}520$ & (nenhum) & 1{,}660 & 1{,}353 & 1{,}469 & Forte apoio geral; direita apoia mais; sem diferença robusta por formação.\\

Mais mulheres na força de trabalho (positivo) & 173 & Logit ord.\ (limiares ok) & $-$ ($\beta = \mathbf{-0{,}688}$; $p = 0{,}000$) & $0{,}612$; $p = 0{,}425$ & Você é homem? Sim: $-$; Qual é a sua faixa etária? 26 a 35 anos: +; Qual é a sua faixa etária? 36 a 45 anos: + & 1{,}423 & 1{,}765 & 1{,}661 & Apoio geral; homens menos favoráveis; esquerda apoia mais.\\

Aumento do uso de tecnologia (positivo) & 171 & n/a (não estimável) & n/a & n/a (modelo não convergiu) & (nenhum) & 1{,}923 & 2{,}000 & 1{,}878 & Quase consenso pró-tecnologia; falta de variação impediu estimação.\\

Acordos comerciais (positivo) & 171 & n/a (não estimável) & n/a & n/a (modelo não convergiu) & (nenhum) & 1{,}916 & 1{,}875 & 1{,}950 & Abertura bem vista por ambos; sem efeitos identificáveis.\\

Redução recente de postos em grandes empresas & 170 & Logit ord.\ (limiares ok) & + ($\beta = \mathbf{0{,}520}$; $p = 0{,}000$) & $-1{,}008$; $p = 0{,}154$ & (nenhum) & 0{,}526 & 0{,}250 & 0{,}491 & Leigos percebem mais demissões; direita eleva essa percepção.\\

Acordos comerciais: bons/ruins para a economia & 169 & n/a (não estável) & n/a & n/a (modelo não estável) & (nenhum) & 1{,}688 & 1{,}800 & 1{,}772 & Maioria acha bons; sem influência consistente de ideologia/formação.\\

Maior responsável por alta dos combustíveis & 169 & Logit ord.\ (limiares ok) & $-$ ($\beta = \mathbf{-0{,}304}$; $p = 0{,}010$) & $0{,}343$; $p = 0{,}559$ & (nenhum) & 0{,}908 & 1{,}125 & 0{,}972 & Público culpa mais governo; economistas distribuem causas (mercado externo etc.).\\

Nível dos preços dos combustíveis & 171 & n/a (não estável) & n/a & n/a (modelo não estável) & (nenhum) & 1{,}858 & 1{,}750 & 1{,}848 & Quase unanimidade: combustíveis altos; falta variação.\\

Quanto o presidente pode melhorar a economia & 167 & Logit ord.\ (limiares ok) & + ($\beta = \mathbf{0{,}264}$; $p = 0{,}042$) & $0{,}119$; $p = 0{,}854$ & (nenhum) & 1{,}434 & 1{,}600 & 1{,}544 & Ambos atribuem algum poder; direita atribui mais.\\

Novos postos de trabalho pagam bem/mal & 168 & Logit ord.\ (limiares ok) & + ($\beta = \mathbf{0{,}669}$; $p = 0{,}000$) & $0{,}102$; $p = 0{,}877$ & Escolaridade (EM compl., Pós-grad.: +) & 0{,}526 & 0{,}500 & 0{,}483 & Consenso: pagam mal; direita um pouco mais otimista; mais escolarizados menos pessimistas.\\

Desigualdade hoje vs. 20 anos atrás & 170 & Logit ord.\ (limiares ok) & $-$ ($\beta = \mathbf{-0{,}288}$; $p = 0{,}020$) & $-0{,}318$; $p = 0{,}594$ & (nenhum) & 1{,}318 & 1{,}062 & 1{,}180 & Público crê que aumentou; esquerda acentua; economistas menos pessimistas.\\

Rendas familiares vs. custo de vida (20 anos) & 169 & Logit ord.\ (limiares ok) & + ($\beta = 0{,}221$; $p = 0{,}097$) & $1{,}111$; $p = 0{,}096$ & Faixa etária (46--55: $-$; $p = 0{,}018$) & 0{,}536 & 0{,}750 & 0{,}388 & Público mais pessimista; formação reduz pessimismo sobre renda real.\\

Salários vs. custo de vida (20 anos) & 168 & n/a (não estável) & n/a & n/a (modelo não estável) & (nenhum) & 0{,}421 & 0{,}625 & 0{,}243 & Ambos veem salários atrás; economistas um pouco menos pessimistas.\\

Padrão de vida nos próximos 5 anos & 165 & Logit ord.\ (limiares ok) & $-$ ($\beta = -0{,}172$; $p = 0{,}157$) & $0{,}427$; $p = 0{,}485$ & Faixa etária (46--55: $-$; $p = 0{,}016$) & 0{,}671 & 0{,}562 & 0{,}448 & Pessimismo moderado compartilhado; economistas levemente menos pessimistas.\\

Padrão de vida dos filhos vs. dos pais & 169 & Logit ord.\ (limiares ok) & + ($\beta = 0{,}229$; $p = 0{,}082$) & $0{,}268$; $p = 0{,}647$ & Faixa etária (36--45: $-$; $p = 0{,}084$) & 1{,}425 & 1{,}250 & 1{,}154 & Divisão equilibrada; sem viés extremo; economistas um pouco menos otimistas.\\

Filhos $<$ 30: padrão de vida futuro & 128 & n/a (não estável) & n/a & n/a (N menor) & (nenhum) & 1{,}553 & 1{,}357 & 1{,}331 & Leve otimismo; economistas menos otimistas; sem significância robusta.\\

Reforma da Previdência é necessária & 169 & Logit ord.\ (limiares ok) & + ($\beta = \mathbf{0{,}542}$; $p = 0{,}000$) & $-0{,}843$; $p = 0{,}186$ & Escolaridade (Sup. Incompl., Pós: +) & 1{,}556 & 1{,}188 & 1{,}460 & Apoio geral; direita e maior escolaridade elevam apoio; economistas majoritariamente favoráveis.\\

Reforma trabalhista é necessária & 169 & Logit ord.\ (limiares ok) & + ($\beta = \mathbf{0{,}657}$; $p = 0{,}000$) & $-0{,}722$; $p = 0{,}252$ & Escolaridade (EM compl., Sup. compl./incompl., Pós: +) & 1{,}503 & 1{,}188 & 1{,}426 & Apoio em ambos; direita e escolaridade elevam apoio.\\

Reforma tributária é necessária & 171 & Logit ord.\ (limiares ok) & + ($\beta = 0{,}212$; $p = 0{,}332$) & $0{,}751$; $p = 0{,}540$ & Escolaridade (EM compl., Sup. compl.: +) & 1{,}810 & 1{,}938 & 1{,}877 & Quase unanimidade pró-reforma; sem diferenças ideológicas/ formação significativas.\\

A privatização de estatais é benéfica & 172 & Logit ord.\ (limiares ok) & + ($\beta = \mathbf{0{,}945}$; $p = 0{,}000$) & $-0{,}995$; $p = 0{,}142$ & Qual é o seu vínculo empregatício?: +; Qual é a sua faixa etária? 46 a 55 anos: + & 1{,}039 & 0{,}882 & 1{,}075 & Apoio generalizado à privatização; direita apoia muito mais; economistas concordam em grau semelhante.\\

Produtos importados são benéficos para a economia & 171 & Logit ord.\ (limiares ok) & + ($\beta = \mathbf{0{,}455}$; $p = 0{,}001$) & $\mathbf{1{,}627}$; $p = 0{,}018$ & Qual seu nível de escolaridade? Ensino Médio Incompleto: $-$; Qual é a sua faixa etária? 36 a 45 anos: $-$ & 1{,}279 & 1{,}588 & 1{,}224 & Consenso de que importações são benéficas; economistas concordam ainda mais; direita apoia mais o livre-comércio.\\

A corrupção é a principal causa dos problemas econômicos & 171 & Logit ord.\ (limiares ok) & + ($\beta = \mathbf{0{,}725}$; $p = 0{,}000$) & $-0{,}991$; $p = 0{,}110$ & Você é homem? Sim: $-$; Com qual grupo racial/étnico você mais se identifica? Outro: +; Qual é a sua faixa etária? 46 a 55 anos: +; Qual é a sua faixa etária? 66 anos ou mais: +; Você se considera uma pessoa politicamente engajada? Sim: $-$ & 1{,}357 & 0{,}941 & 1{,}205 & Público atribui problemas à corrupção; direita acentua esse foco; economistas relativizam (outros fatores).\\

A taxa básica de juros (Selic) está alta demais & 170 & Logit ord.\ (limiares ok) & $-$ ($\beta = \mathbf{-0{,}458}$; $p = 0{,}001$) & $1{,}141$; $p = 0{,}094$ & Qual seu nível de escolaridade? Ensino Superior Completo: $-$; Qual seu nível de escolaridade? Ensino Superior Incompleto: $-$; Qual seu nível de escolaridade? Pós-graduação: $-$; Qual é a sua faixa etária? 46 a 55 anos: +; Qual é a sua faixa etária? 66 anos ou mais: +; Qual é a sua faixa etária? Até 18 anos: $-$; Você se considera uma pessoa politicamente engajada? Sim: $-$ & 1{,}327 & 1{,}765 & 1{,}472 & Quase todos concordam que juros estão altos; economistas endossam essa visão; esquerda se preocupa mais.\\

O governo atual sabe o que faz na economia & 171 & Logit ord.\ (limiares ok) & $-$ ($\beta = \mathbf{-1{,}328}$; $p = 0{,}000$) & $1{,}064$; $p = 0{,}126$ & Qual é o seu vínculo empregatício?: $-$; Qual seu nível de escolaridade? Ensino Médio Completo: $-$; Qual seu nível de escolaridade? Ensino Superior Completo: $-$; Qual seu nível de escolaridade? Ensino Superior Incompleto: $-$; Qual seu nível de escolaridade? Pós-graduação: $-$ & 0{,}494 & 0{,}941 & 0{,}641 & Público descrente da gestão atual; direita muito mais cética; economistas um pouco menos críticos.\\

O governo deve intervir mais na economia & 171 & Logit ord.\ (limiares ok) & $-$ ($\beta = \mathbf{-0{,}625}$; $p = 0{,}000$) & $-0{,}071$; $p = 0{,}906$ & Você é homem? Sim: $-$; Qual seu nível de escolaridade? Ensino Médio Completo: $-$; Qual seu nível de escolaridade? Ensino Superior Completo: $-$; Qual seu nível de escolaridade? Ensino Superior Incompleto: $-$; Qual seu nível de escolaridade? Pós-graduação: $-$ & 0{,}695 & 0{,}647 & 0{,}709 & Divisão ideológica clara: esquerda defende maior intervenção; direita rejeita; formação não afeta.\\

A entrada de estrangeiros no mercado é benéfica para a economia & 171 & Logit ord.\ (limiares ok) & $-$ ($\beta = -0{,}106$; $p = 0{,}386$) & $0{,}675$; $p = 0{,}266$ & Você é homem? Sim: +; Qual seu nível de escolaridade? Ensino Médio Incompleto: $-$; Qual é a sua faixa etária? Até 18 anos: + & 1{,}318 & 1{,}529 & 1{,}346 & Consenso: participação estrangeira é positiva; sem divergências ideológicas ou de formação significativas.\\

A indústria nacional deve ser protegida da concorrência estrangeira & 171 & Logit ord.\ (limiares ok) & $-$ ($\beta = \mathbf{-0{,}603}$; $p = 0{,}000$) & $-0{,}639$; $p = 0{,}274$ & Qual é o seu vínculo empregatício?: $-$; Você é homem? Sim: $-$; Você se considera uma pessoa politicamente engajada? Sim: $-$ & 1{,}214 & 1{,}000 & 1{,}214 & Público tende ao protecionismo; esquerda muito mais favorável; economistas em geral não endossam.\\

O Brasil tem chance de virar um país desenvolvido & 172 & Logit ord.\ (limiares ok) & $-$ ($\beta = \mathbf{-0{,}316}$; $p = 0{,}010$) & $-0{,}212$; $p = 0{,}704$ & Qual seu nível de escolaridade? Ensino Médio Completo: $-$ & 1{,}406 & 1{,}176 & 1{,}251 & Otimismo moderado; esquerda mais otimista; economistas mais céticos.\\

Os lucros empresariais ocorrem à custa dos trabalhadores & 171 & Logit ord.\ (limiares ok) & $-$ ($\beta = \mathbf{-0{,}830}$; $p = 0{,}000$) & $0{,}527$; $p = 0{,}411$ & Qual é o seu vínculo empregatício?: $-$; Você é homem? Sim: $-$; Qual seu nível de escolaridade? Ensino Superior Incompleto: $-$; Qual seu nível de escolaridade? Pós-graduação: $-$ & 0{,}916 & 1{,}176 & 0{,}973 & Público dividido; forte clivagem ideológica; economistas não apresentam correção unívoca do viés.\\

A competição entre empresas é benéfica para a economia & 171 & Logit ord.\ (limiares ok) & + ($\beta = \mathbf{0{,}618}$; $p = 0{,}003$) & $0{,}727$; $p = 0{,}437$ & (nenhum) & 1{,}799 & 1{,}882 & 1{,}781 & Consenso pró-competição; direita reforça ainda mais; formação não altera percepções.\\

O Brasil deveria priorizar a redução da desigualdade em vez do crescimento econômico & 170 & Logit ord.\ (limiares ok) & $-$ ($\beta = \mathbf{-0{,}594}$; $p = 0{,}000$) & $-0{,}409$; $p = 0{,}508$ & Qual seu nível de escolaridade? Ensino Superior Completo: $-$; Qual seu nível de escolaridade? Ensino Superior Incompleto: $-$; Qual seu nível de escolaridade? Pós-graduação: $-$; Qual é a sua faixa etária? 46 a 55 anos: +; Qual é a sua faixa etária? 66 anos ou mais: +; Qual é a sua faixa etária? Até 18 anos: $-$ & 0{,}876 & 0{,}706 & 0{,}810 & Público tende a priorizar crescimento; esquerda foca mais na igualdade; economistas pró-crescimento.\\

A automação prejudica o mercado de trabalho & 172 & Logit ord.\ (limiares ok) & $-$ ($\beta = \mathbf{-0{,}339}$; $p = 0{,}029$) & $-0{,}505$; $p = 0{,}529$ & Você é homem? Sim: +; Qual é a sua faixa etária? 46 a 55 anos: +; Qual é a sua faixa etária? 66 anos ou mais: + & 0{,}368 & 0{,}235 & 0{,}318 & Ideia de prejuízo da automação amplamente rejeitada; esquerda um pouco mais apreensiva; economistas igualmente despreocupados.\\

O Brasil deveria adotar um modelo econômico mais liberal & 171 & Logit ord.\ (limiares ok) & + ($\beta = \mathbf{0{,}625}$; $p = 0{,}000$) & $0{,}199$; $p = 0{,}744$ & (nenhum) & 1{,}429 & 1{,}471 & 1{,}398 & Apoio majoritário a um modelo mais liberal; direita apoia mais; economistas acompanham esse apoio.\\


\bottomrule
\insertTableNotes
\end{longtable}
\end{ThreePartTable}


\endgroup
\end{landscape}
% ---------- FIM DA TABELA ----------

\end{apendicesenv}
% ---