

\chapter{Introdução} % O Labirinto das Decisões: Como Julgamos e Escolhemos?
% Adicionar aqui uma parte principal sobre qual a contribuição principal do tcc, que é sobre o desenvolvimento do campo da "economia politica comportamental", que é nova e ainda não foi amplamente explorada. 
% A ideia é que a economia comportamental, que já é um campo estabelecido, pode ser aplicada à política, e que isso pode ser uma nova forma de entender a política, que é mais realista e menos idealista do que a economia política tradicional.

Aqui vamos apresentar o problema de pesquisa, sua relevância e contribuição para o campo da economia política comportamental.




\section{A Racionalidade Coletiva} %(Tema e Problema de Pesquisa)

Introduzir o tema do trabalho e a questão central: os vieses de julgamento dos eleitores e suas implicações políticas e econômicas.

\section{Por que a Realidade Econômica É Distorcida pelo Eleitor?} %(Justificativa e Relevância do Estudo)
% deve enfatizar que o problema não é ignorância aleatória, mas crenças sistematicamente erradas, o que gera políticas ruins mesmo quando os eleitores estão bem informados.

Destacar que o problema não é apenas ignorância, mas crenças sistematicamente erradas que influenciam a formulação de políticas públicas.

\section{O Que Precisamos Descobrir} % Hipóteses

Formular as hipóteses da pesquisa, como a persistência de crenças enviesadas mesmo diante de informações contraditórias.

\section{Os Filtros da Percepção Econômica} % Objetivos

Definir os objetivos da pesquisa

\subsection{Objetivo Geral} % Objetivo Geral

\textbf{Analisar como os vieses cognitivos dos eleitores impactam a formulação de políticas econômicas no Brasil e nos EUA, e avaliar as implicações para o desenvolvimento econômico e a qualidade das instituições democráticas.}

\subsection{Objetivos Específicos} % Objetivos Específicos

\begin{itemize}
    \item Identificar os principais vieses cognitivos que afetam a percepção econômica dos eleitores no Brasil e nos EUA, com base nos dados da SAEE e sua replicação no Brasil.
    \item Investigar como esses vieses influenciam a aceitação de determinadas políticas econômicas e regulatórias.
    \item Avaliar a relação entre a irracionalidade sistemática dos eleitores e a ineficiência das políticas públicas adotadas.
    \item Comparar as diferenças institucionais entre Brasil e EUA que podem amplificar ou mitigar os efeitos desses vieses.
    \item Propor formas de reduzir a influência dos vieses cognitivos na formulação de políticas públicas e melhorar a qualidade do debate econômico.
\end{itemize}

\chapter{Teorias e Evidências Sobre a (I)Racionalidade Humana} % Revisão de Literatura

Revisar a literatura sobre economia comportamental e sua aplicação à política.

\section{Entre Adam Smith e Kahneman} % Economia Comportamental vs. Escolha Racional 

Contrastar a visão da escolha racional com a abordagem da economia comportamental.

\section{Como os Vieses Moldeiam as Escolhas Políticas} % (Vieses Cognitivos e Política)

Explicar os principais vieses cognitivos identificados na literatura e sua relação com decisões políticas (viés antimercado, antiestrangeiro, antitrabalho, pessimista, entre outros).

\section{O Custo da Ignorância} % (Economia da Informação e Racionalidade Limitada)

Discutir os impactos da desinformação e da racionalidade limitada no processo eleitoral e na formulação de políticas públicas.

\section{Do Sofista ao Populista} %  Como as Ideias Econômicas se Espalham (História do Pensamento Econômico e Propagação de Crenças)

Explorar como ideias econômicas se propagam, influenciadas por interesses políticos, mídia e cultura.

\section{Preferência por Crenças e Resistência ao Conhecimento}

Argumentar que muitas crenças são mantidas por fatores emocionais e sociais, e não por evidências racionais.
