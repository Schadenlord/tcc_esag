

\chapter{Introdução} % O Labirinto das Decisões: Como Julgamos e Escolhemos?
% Adicionar aqui uma parte principal sobre qual a contribuição principal do tcc, que é sobre o desenvolvimento do campo da "economia politica comportamental", que é nova e ainda não foi amplamente explorada. 
% A ideia é que a economia comportamental, que já é um campo estabelecido, pode ser aplicada à política, e que isso pode ser uma nova forma de entender a política, que é mais realista e menos idealista do que a economia política tradicional.



\section{A Racionalidade Coletiva} %(Tema e Problema de Pesquisa)




\section{Por que a Realidade Econômica É Distorcida pelo Eleitor?} %(Justificativa e Relevância do Estudo)
% deve enfatizar que o problema não é ignorância aleatória, mas crenças sistematicamente erradas, o que gera políticas ruins mesmo quando os eleitores estão bem informados.


\section{O Que Precisamos Descobrir} % Hipóteses





\section{Os Filtros da Percepção Econômica} % Objetivos




\subsection{Objetivo Geral} % Objetivo Geral




\subsection{Objetivos Específicos} % Objetivos Específicos





\chapter{Teorias e Evidências Sobre a (I)Racionalidade Humana} % Revisão de Literatura




\section{Entre Adam Smith e Kahneman} % Economia Comportamental vs. Escolha Racional 





\section{Como os Vieses Moldeiam as Escolhas Políticas} % (Vieses Cognitivos e Política)

\section{O Custo da Ignorância} % (Economia da Informação e Racionalidade Limitada)


\section{Do Sofista ao Populista} %  Como as Ideias Econômicas se Espalham (História do Pensamento Econômico e Propagação de Crenças)



\section{Preferência por Crenças e Resistência ao Conhecimento}



