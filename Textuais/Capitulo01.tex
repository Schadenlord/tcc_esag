

\chapter{Introdução} % O Labirinto das Decisões: Como Julgamos e Escolhemos?
% Adicionar aqui uma parte principal sobre qual a contribuição principal do tcc, que é sobre o desenvolvimento do campo da "economia politica comportamental", que é nova e ainda não foi amplamente explorada. 
% A ideia é que a economia comportamental, que já é um campo estabelecido, pode ser aplicada à política, e que isso pode ser uma nova forma de entender a política, que é mais realista e menos idealista do que a economia política tradicional.

%Aqui vamos apresentar o problema de pesquisa, sua relevância e contribuição para o campo da economia política comportamental.

A democracia moderna parte do pressuposto de que os eleitores são agentes racionais, capazes de avaliar as consequências econômicas de suas escolhas políticas e apoiar medidas que maximizem o bem-estar social \cite{downs1957economic}. No entanto, a realidade demonstra que esse ideal frequentemente se desvia devido à influência de vieses cognitivos e heurísticas que distorcem a percepção econômica da população \cite{The_Myth_of_the_Rational_Voter,kahneman2011thinking}. Como resultado, políticas públicas são frequentemente moldadas por crenças equivocadas, levando a decisões sub-ótimas que comprometem o desenvolvimento econômico e social \cite{acemoglu2012nations}.

O fenômeno central deste estudo é o impacto da irracionalidade eleitoral sobre a formulação de políticas econômicas. Eleitores bem-intencionados, mas cognitivamente limitados, acabam apoiando medidas que contradizem princípios econômicos fundamentais, como a vantagem comparativa, a eficiência dos mercados e os benefícios da inovação. A pesquisa de Caplan (\citeyear{Systematically_Biased_Beliefs_about_Economics,The_Myth_of_the_Rational_Voter}) demonstra que o público frequentemente rejeita consensos econômicos básicos, sustentando crenças que favorecem protecionismo, intervencionismo excessivo e desconfiança do setor produtivo. Essas distorções, denominadas por \citeauthoronline{The_Myth_of_the_Rational_Voter} (\citeyear{The_Myth_of_the_Rational_Voter}) como os vieses antimercado, antiestrangeiro, antitrabalho e pessimista, levam à implementação de políticas que reduzem o crescimento e a prosperidade.

A dificuldade de corrigir essas distorções não está apenas na falta de informação, mas também na resistência psicológica dos eleitores em abandonar crenças que reforçam sua visão de mundo. Como \citeauthoronline{hayek_knowledge_use} (\citeyear{hayek_knowledge_use}) já alertava, a dispersão do conhecimento e a complexidade econômica criam barreiras para que a população compreenda os efeitos reais das políticas públicas. \citeauthoronline{downs1957economic} (\citeyear{downs1957economic}) complementa essa visão ao sugerir que, para a maioria dos eleitores, o custo de se informar sobre economia supera os benefícios individuais, levando a uma ignorância racional que perpetua escolhas equivocadas.

Essa discrepância entre conhecimento econômico e opinião pública tem implicações diretas para a qualidade das políticas adotadas. Como demonstrado por evidências empíricas, como a \textit{Survey of Americans and Economists on the Economy} (SAEE) \cite{saee1996}, há uma diferença sistemática entre a visão dos especialistas em economia e a do público geral, evidenciando que a percepção popular frequentemente se afasta da realidade econômica.

Para analisar empiricamente esse fenômeno no Brasil, este estudo recria a metodologia da SAEE, originalmente aplicada nos Estados Unidos, adaptando-a ao contexto brasileiro. A pesquisa coletará dados primários sobre as crenças econômicas dos eleitores brasileiros e os comparará com os dados da SAEE americana. O objetivo é investigar se os vieses observados nos EUA também estão presentes no Brasil, identificar possíveis divergências e avaliar fatores institucionais e culturais que possam influenciar essas percepções. Essa comparação se justifica porque ambos os países possuem democracias consolidadas, mas apresentam diferenças significativas em termos de escolaridade média, acesso à informação econômica e nível de desconfiança nas instituições. Enquanto os Estados Unidos possuem um longo histórico de pesquisas sobre a percepção pública da economia e sua relação com as políticas públicas, o Brasil ainda carece de estudos que explorem sistematicamente como o eleitorado interpreta questões econômicas e como isso se reflete no cenário político. Comparar esses dois contextos permite entender se os vieses do eleitorado são fenômenos universais ou se há particularidades ligadas ao ambiente institucional e ao desenvolvimento econômico.

No entanto, algumas limitações devem ser reconhecidas. Apesar da adaptação da metodologia da SAEE ao contexto brasileiro, diferenças institucionais e culturais entre os dois países podem afetar a comparabilidade dos resultados. Além disso, a pesquisa se concentra na percepção econômica dos eleitores, não abrangendo outros fatores que também influenciam a formulação de políticas públicas, como o papel da mídia, o impacto de campanhas eleitorais e a disseminação de desinformação.

É relevante destacar que esta pesquisa não pretende fornecer uma solução definitiva para o problema da irracionalidade eleitoral, nem testar empiricamente as estratégias de mitigação sugeridas. O foco está na análise dos vieses cognitivos e suas consequências para a formulação de políticas, buscando oferecer um panorama teórico e empírico sobre o tema. Questões mais amplas relacionadas à desinformação deliberada, ao papel da mídia e a outros fatores externos não serão o foco central deste estudo, ainda que possam ser tangencialmente mencionadas.

Embora a literatura internacional tenha avançado significativamente na análise dos vieses cognitivos na política, esse debate ainda é incipiente no Brasil. Há poucos estudos que exploram de forma sistemática como a percepção econômica dos eleitores brasileiros se distancia dos consensos acadêmicos e como isso afeta a formulação de políticas públicas. Dado o impacto de decisões econômicas equivocadas sobre o desenvolvimento do país — incluindo políticas protecionistas ineficientes, subsídios distorcidos e resistência a reformas estruturais —, compreender esses vieses se torna essencial para o aprimoramento do panorama político e econômico nacional.

Diante desse cenário, este estudo busca compreender como os vieses cognitivos moldam a percepção econômica do eleitorado brasileiro e de que forma essas distorções influenciam a formulação de políticas públicas. Ao adaptar a metodologia da \textit{Survey of Americans and Economists on the Economy} (SAEE) para o contexto brasileiro, pretende-se avaliar se os vieses observados nos Estados Unidos também se manifestam no Brasil e em que medida fatores institucionais e culturais moldam essas crenças. Com base nessa análise comparativa, este trabalho tem como objetivo central investigar por que a soma das decisões individuais frequentemente resulta em escolhas coletivas que prejudicam o desenvolvimento econômico. Para isso, o estudo examinará a interação entre cognição, incentivos políticos e disseminação do conhecimento econômico, buscando oferecer evidências empíricas e teóricas sobre os desafios e as oportunidades para tornar o processo democrático mais alinhado com princípios econômicos sólidos.

\section{A Racionalidade Coletiva} %(Tema e Problema de Pesquisa)

% Introduzir o tema do trabalho e a questão central: os vieses de julgamento dos eleitores e suas implicações políticas e econômicas.

O ideal democrático parte do pressuposto de que a soma das decisões individuais resulta em escolhas coletivas racionais e benéficas para a sociedade. No entanto, a realidade política demonstra que essa expectativa nem sempre se confirma. Mesmo indivíduos que tomam decisões racionais em suas vidas privadas frequentemente apoiam políticas públicas que contrariam princípios econômicos fundamentais, resultando em prejuízos para o bem-estar geral \cite{downs1957economic,The_Myth_of_the_Rational_Voter}.  

Esse paradoxo pode ser explicado pela diferença entre racionalidade individual e racionalidade coletiva. No mercado, os indivíduos possuem incentivos diretos para tomar boas decisões, uma vez que as consequências de suas escolhas recaem diretamente sobre eles. Já no ambiente eleitoral, esse incentivo é diluído: um único voto tem impacto insignificante no resultado final, tornando o custo de se informar maior do que o benefício individual de uma escolha mais embasada. Assim, ao contrário do que ocorre no mercado, onde decisões ruins trazem consequências imediatas, no processo democrático não há penalização direta para crenças equivocadas. Isso favorece a persistência de visões distorcidas sobre economia e política \cite{bastiat1859sofismas,downs1957economic,The_Myth_of_the_Rational_Voter}.

A pesquisa de \citeauthoronline{The_Myth_of_the_Rational_Voter} (\citeyear{The_Myth_of_the_Rational_Voter}) reforça essa tese ao argumentar que o maior obstáculo para a formulação de boas políticas econômicas não são apenas os interesses de grupos organizados ou o lobby político, mas sim as concepções errôneas generalizadas entre os eleitores. Esses vieses coletivos resultam na eleição de políticos que compartilham – ou fingem compartilhar – essas crenças, consolidando um ciclo de formulação de políticas ineficazes. Caplan identifica quatro vieses principais que distorcem a percepção econômica popular: o viés antimercado, que leva à supervalorização do papel do Estado e à desconfiança da economia de mercado; o viés antiestrangeiro, que alimenta políticas protecionistas e rejeição à globalização; o viés antitrabalho, que considera a geração de empregos um fim em si mesmo, ignorando a importância da produtividade; e o viés pessimista, que tende a exagerar crises econômicas e minimizar avanços.

A democracia, portanto, não falha por ignorar os desejos dos eleitores, mas justamente porque os realiza. Como aponta Caplan, ao contrário da crença de que o sistema político corrige equívocos através do debate e da informação, ele muitas vezes institucionaliza esses erros, produzindo políticas que respondem mais às percepções enviesadas do público do que às realidades econômicas subjacentes. Isso gera incentivos perversos para políticos, que, em vez de esclarecer a população, reforçam suas crenças preexistentes para maximizar apoio eleitoral.

Diante desse cenário, este estudo busca responder à seguinte questão: por que a soma das decisões individuais frequentemente resulta em escolhas coletivas que prejudicam o desenvolvimento econômico? Para isso, será necessário examinar as interações entre cognição, incentivos políticos e percepção econômica, além de explorar de que maneira a disseminação de conhecimento econômico pode mitigar esses vieses \cite{positive_economics_friedman,Judgment_under_Uncertainty,kahneman2011thinking}.

Compreender esse fenômeno é essencial para aprimorar a qualidade do debate político e reduzir o impacto de crenças distorcidas na formulação de políticas públicas. Ao identificar os mecanismos que levam eleitores bem-intencionados a apoiar medidas ineficazes, este estudo abre espaço para o desenvolvimento de estratégias que tornem o processo político mais alinhado com princípios econômicos sólidos e baseados em evidências.  

\section{Hipóteses a Serem Testadas}

Com base nos dados coletados e na metodologia aplicada, este estudo propõe as seguintes hipóteses, estruturadas para serem empiricamente testáveis e refutáveis:

\begin{itemize}
    
    \item \textbf{H1 - Eleitores apresentam vieses sistemáticos que distorcem sua percepção econômica.}  
    Se os vieses cognitivos afetam a percepção econômica dos eleitores, então os resultados da pesquisa devem indicar uma discrepância estatisticamente significativa entre a opinião dos eleitores e a dos economistas sobre fenômenos econômicos objetivos. Caso essa discrepância não seja estatisticamente significativa, a hipótese será refutada \cite{The_Myth_of_the_Rational_Voter}.  
    % Caplan (2007) argumenta que os eleitores mantêm crenças enviesadas sobre economia, diferindo sistematicamente dos economistas. Esta hipótese testa empiricamente essa divergência no contexto brasileiro.

    \item \textbf{H2 - O nível de conhecimento econômico reduz a incidência de vieses cognitivos.}  
    Se o conhecimento econômico reduz vieses, então eleitores com maior escolaridade em economia devem apresentar uma menor taxa de discordância em relação ao consenso dos economistas sobre questões econômicas. Se os indivíduos mais instruídos demonstrarem vieses semelhantes aos menos instruídos, a hipótese será refutada \cite{downs1957economic}.  
    % Downs (1957) argumenta que a ignorância racional leva os eleitores a não se informarem sobre economia. Testamos se esse efeito é mitigado por mais educação.

    \item \textbf{H3 - O viés antimercado está presente no eleitorado brasileiro e influencia seu apoio a políticas protecionistas.}  
    Se o viés antimercado for significativo no Brasil, então eleitores que expressam desconfiança no mercado devem demonstrar maior apoio a políticas protecionistas e intervencionistas. Se não houver correlação entre essas atitudes ou se o apoio a tais políticas for uniforme entre os grupos analisados, a hipótese será refutada \cite{The_Myth_of_the_Rational_Voter}.  
    % Caplan (2007) define o viés antimercado como a crença de que o lucro privado prejudica a sociedade. Essa hipótese testa sua manifestação empírica.

    \item \textbf{H4 - O viés antiestrangeiro está associado ao apoio a políticas restritivas de comércio e imigração.}  
    Se o viés antiestrangeiro for um fator relevante, então eleitores que expressam preocupações com a concorrência estrangeira devem demonstrar maior probabilidade de apoiar restrições comerciais e políticas anti-imigração. Caso essa correlação não seja encontrada, a hipótese será refutada \cite{The_Myth_of_the_Rational_Voter}.  
    % O viés antiestrangeiro, segundo Caplan (2007), leva os eleitores a superestimar os danos da imigração e do comércio internacional. Testamos se esse efeito se manifesta estatisticamente.

    \item \textbf{H5 - O viés antitrabalho leva ao apoio a políticas que favorecem a criação artificial de empregos.}  
    Se os eleitores tendem a superestimar os benefícios da geração de empregos sem ganhos de produtividade, então aqueles que priorizam a criação de empregos devem apresentar maior apoio a políticas que restringem inovações tecnológicas e automação. Caso essa associação não seja encontrada, a hipótese será refutada \cite{The_Myth_of_the_Rational_Voter}.  
    % Segundo Caplan (2007), eleitores frequentemente veem o trabalho como um fim em si mesmo, apoiando políticas que mantêm empregos mesmo quando ineficientes.

    \item \textbf{H6 - O viés pessimista distorce a percepção sobre a economia, levando a uma avaliação mais negativa do crescimento econômico.}  
    Se o viés pessimista for um fator relevante, então os eleitores devem apresentar uma visão mais negativa do desempenho econômico passado e presente em comparação aos dados objetivos. Se não houver diferença estatística entre a percepção popular e os dados históricos, a hipótese será refutada \cite{The_Myth_of_the_Rational_Voter}.  
    % O viés pessimista, conforme identificado por Caplan (2007), leva os eleitores a subestimar o progresso econômico e social, resultando em avaliações excessivamente negativas.

    \item \textbf{H7 - A filiação ideológica interfere na aceitação de evidências econômicas.}  
    Se a ideologia for um fator determinante na interpretação de dados econômicos, então eleitores alinhados a diferentes espectros políticos devem demonstrar discrepâncias estatisticamente significativas na aceitação de evidências econômicas objetivas. Se essa diferença não for encontrada ou se a aceitação de evidências for uniforme, a hipótese será refutada \cite{The_Myth_of_the_Rational_Voter}.  
    % Caplan (2007) argumenta que os eleitores interpretam seletivamente informações econômicas para reforçar crenças ideológicas preexistentes. Testamos esse efeito empiricamente.

\end{itemize}

A validação dessas hipóteses será conduzida utilizando modelos econométricos Logit, replicando a metodologia da \textit{Survey of Americans and Economists on the Economy} (SAEE) no Brasil.


\section{Objetivo Geral}

Analisar por que a soma das decisões individuais frequentemente resulta em escolhas coletivas que prejudicam o desenvolvimento econômico, investigando o papel dos vieses cognitivos na percepção econômica dos eleitores e seu impacto na formulação de políticas públicas. Para isso, o estudo adotará uma abordagem teórica abrangente, fundamentada na história do pensamento econômico, em trabalhos históricos e na literatura contemporânea sobre economia política comportamental. Busca-se compreender como esses vieses emergem, como se perpetuam no processo democrático e quais estratégias institucionais poderiam mitigar seus efeitos, reduzindo a discrepância entre o conhecimento econômico especializado e as crenças do eleitorado.

\section{Objetivos Específicos}

\begin{enumerate}[label=\alph*)]
    \item Identificar os principais vieses econômicos presentes no eleitorado e suas origens psicológicas e históricas, utilizando uma abordagem que combine história do pensamento econômico, literatura acadêmica clássica e estudos contemporâneos sobre economia política comportamental.
    \item Analisar o impacto desses vieses na formulação de políticas públicas e no funcionamento da democracia, considerando evidências empíricas e teóricas.
    \item Comparar a percepção econômica da população com a dos especialistas, utilizando pesquisas como a \textit{Survey of Americans and Economists on the Economy} (SAEE) e sua replicação no Brasil, além de contextualizar as diferenças com base em elementos institucionais e históricos.
    \item Avaliar possíveis estratégias para mitigar os efeitos desses vieses, explorando propostas teóricas e institucionais que reduzam o impacto da irracionalidade eleitoral na formulação de políticas públicas.
    \item Investigar como diferentes arranjos institucionais podem influenciar a qualidade das decisões políticas e econômicas, analisando modelos que descentralizam a tomada de decisão e introduzem mecanismos de filtragem política, à luz da economia política comportamental e da teoria das instituições.
\end{enumerate}

\section{Por que a Realidade Econômica É Distorcida pelo Eleitor?} %(Justificativa e Relevância do Estudo)
% deve enfatizar que o problema não é ignorância aleatória, mas crenças sistematicamente erradas, o que gera políticas ruins mesmo quando os eleitores estão bem informados.

A democracia parte do pressuposto de que os eleitores escolhem representantes e políticas que maximizam o bem-estar coletivo. No entanto, a realidade política demonstra que decisões econômicas frequentemente são guiadas por crenças sistematicamente erradas, e não apenas por ignorância ou falta de informação. O problema central não é a mera desinformação, mas a presença de vieses cognitivos estruturais, que levam os eleitores a interpretar equivocadamente fenômenos econômicos e a apoiar políticas públicas ineficazes, muitas vezes prejudiciais ao crescimento econômico e ao desenvolvimento social.

A literatura sobre economia política comportamental mostra que esses vieses seguem padrões previsíveis. \citeauthoronline{The_Myth_of_the_Rational_Voter} (\citeyear{The_Myth_of_the_Rational_Voter}) argumenta que os eleitores mantêm crenças enviesadas sobre temas econômicos fundamentais, como comércio internacional, concorrência de mercado e avanços tecnológicos. Essas crenças distorcidas favorecem o protecionismo, o intervencionismo excessivo e a resistência à inovação, independentemente do nível de escolaridade ou acesso à informação dos eleitores. O próprio funcionamento da democracia intensifica esse fenômeno: como o custo de estar errado é difuso e diluído entre milhões de eleitores, não há incentivos diretos para revisar crenças erradas \cite{downs1957economic}.

Além disso, a teoria do conhecimento disperso de \citeauthoronline{hayek_knowledge_use} (\citeyear{hayek_knowledge_use}) destaca que a economia é uma ciência contraintuitiva, onde efeitos de segunda ordem frequentemente contradizem percepções imediatas. Por exemplo, a crença de que "baixar preços reduz salários" ou que "importações eliminam empregos" ignora os mecanismos de compensação que os economistas compreendem, mas que são difíceis de comunicar ao público geral. Como resultado, políticos têm mais incentivos para reforçar crenças populares do que para corrigi-las, pois a aceitação de políticas impopulares, ainda que economicamente racionais, pode significar uma derrota eleitoral.

A relevância deste estudo está em investigar essa discrepância entre conhecimento econômico e opinião pública, explorando suas consequências para a formulação de políticas públicas e para o funcionamento da democracia. Ao analisar os vieses cognitivos e seus impactos, este trabalho contribui para o avanço da economia política comportamental, propondo estratégias para mitigar os efeitos dessas distorções – seja por meio da educação econômica, seja pelo aprimoramento de incentivos institucionais.

Sem um entendimento claro das causas dessa dissonância entre economia e democracia, ciclos de políticas ineficazes continuarão a se repetir, dificultando o desenvolvimento sustentável. O desafio, portanto, não é apenas combater a ignorância, mas desenvolver formas eficazes de corrigir crenças sistematicamente equivocadas que influenciam as escolhas políticas e moldam o futuro das sociedades.


\section{Metodologia} % Metodologia

Este estudo combina uma abordagem teórica e empírica para investigar os vieses cognitivos na formulação de políticas públicas. A metodologia empregada baseia-se em três pilares principais: 
(i) a análise da evolução da percepção econômica ao longo da história do pensamento econômico e sua influência nas crenças populares; 
(ii) a replicação da \textit{Survey of Americans and Economists on the Economy} (SAEE) no Brasil; e 
(iii) a aplicação de métodos econométricos para testar empiricamente a presença e o impacto desses vieses na população brasileira.

\subsection{Definição das Variáveis e Modelagem} % Variáveis e Modelagem

A modelagem empírica segue a metodologia da SAEE, adaptada ao contexto brasileiro. As variáveis foram operacionalizadas conforme descrito abaixo:

\renewcommand{\arraystretch}{1.3}

\begin{longtable}{|>{\raggedright\arraybackslash}p{4cm} 
                  |>{\raggedright\arraybackslash}p{8cm} 
                  |>{\raggedright\arraybackslash}p{4cm}|}
    \caption{Variáveis de Controle e Codificação}
    \label{tab:control_variables} \\
    \hline
    \textbf{Variável} & \textbf{Pergunta} & \textbf{Codificação} \\
    \hline
    \endfirsthead

    \hline
    \textbf{Variável} & \textbf{Pergunta} & \textbf{Codificação} \\
    \hline
    \endhead

    \hline
    \endfoot

    \hline
    \endlastfoot

    \textbf{Econ} & Qual seu nível de formação em Ciências Econômicas? & 
    1 = Doutorado ou mais; 0 = Não economista \\ 
    \hline
    \textbf{Sex} & Qual seu Gênero & 
    1 se Homem, 0 caso contrário \\ 
    \hline
    \textbf{Black} & Qual é a sua raça? Branco, negro, asiático ou outra? & 1 se negro, 0 caso contrário \\ 
    \hline
    \textbf{Asian} & Idem & 1 se asiático, 0 caso contrário \\ 
    \hline
    \textbf{Othrace} & Idem & 1 se outra raça, 0 caso contrário \\   \hline
    \textbf{Age} & Qual é o seu ano de nascimento? & 
    2025 - resposta \\
    \hline
    \textbf{School} & Qual o seu nível de escolaridade? & 
    1 = Fundamental incompleto; 
    2 = Fundamental completo; 
    3 = Médio incompleto; 
    4 = Médio completo; 
    5 = Superior incompleto; 
    6 = Superior completo; 
    7 = Pós-graduação \\ 
    \hline
    \textbf{Brregion} & Em qual região do Brasil você reside? & 
    1 = Norte; 
    2 = Nordeste; 
    3 = Centro--Oeste; 
    4 = Sudeste; 
    5 = Sul \\ 
    \hline
    \textbf{Clt}\footnote{A variável ``Clt'' foi codificada conforme uma escala de dependência estatal, variando de servidores públicos (maior dependência) a empresários (maior autonomia econômica). A literatura mostra que servidores públicos tendem a adotar posições mais à esquerda e favoráveis à expansão do Estado por razões ideológicas e de interesse próprio \cite{jensen2009political}. Por outro lado, a autonomia e a estrutura de tarefas no trabalho moldam preferências políticas mais pró-mercado e individualistas, sobretudo entre empresários e profissionais autônomos \cite{kitschelt2014occupations}. Além disso, a heterogeneidade entre os autônomos — especialmente entre os precários e os mais estáveis — também influencia a orientação política, justificando distinções internas nessa categoria \cite{jansen2016self}.}  & Qual o seu vínculo empregatício? & 
    -4 = Servidor Público;  
    -3 = Aposentado;        
    -2 = Estudante;         
    -1 = Desempregado;      
    0 = Sem carteira;      
    1 = CLT;               
    2 = Autônomo;          
    3 = Empresário;        
    4 = Outro             
    \\ 
    \hline
    \textbf{Financearea} & Você trabalha com economia, finanças, contabilidade ou áreas correlatas? & 
    1 = Sim; 0 = Não \\ 
    \hline
    \textbf{Politicalarea} & Você trabalha com política ou áreas correlatas? & 
    1 = Sim; 0 = Não \\ 
    \hline
    \textbf{Politicallife} & Você se considera uma pessoa politicamente engajada? & 
    1 = Sim; 0 = Não \\ 
    \hline
    \textbf{Politicalnews} & Você costuma acompanhar notícias sobre economia e política? & 
    1 = Sim; 0 = Não \\ 
    \hline
    \textbf{Ideol} & Com qual espectro político você mais se identifica? & 
    -2 = Extrema-esquerda; 
    -1 = Esquerda; 
    0 = Centro; 
    1 = Direita; 
    2 = Extrema-direita; 
    3 = Independente; 
    0 = Sem opinião \\ 
\end{longtable}


\subsection{Replicação da Pesquisa no Brasil} % Adaptação ao contexto brasileiro

A adaptação da SAEE ao contexto brasileiro foi realizada para capturar especificidades institucionais e culturais que influenciam a percepção econômica. Os principais ajustes metodológicos incluem:

\begin{itemize}
    \item Adaptação das perguntas: Algumas questões foram modificadas para refletir o ambiente econômico e político do Brasil. Por exemplo, perguntas sobre filiação partidária foram ajustadas para incluir o sistema multipartidário brasileiro.
    \item Inclusão de questões adicionais: Foram adicionadas perguntas específicas sobre temas como reforma tributária, previdenciária e trabalhista, bem como a percepção sobre o papel do governo na economia.
    \item Amostragem estratificada: Para minimizar vieses de seleção, a pesquisa utiliza três estratégias de recrutamento:
    \begin{itemize}
        \item Amostragem em cadeia (snowball sampling) para ampliar a disseminação da pesquisa.
        \item Parcerias institucionais com o Conselho Regional de Economia (Corecon) e o Conselho Federal de Economia (Cofecon) para garantir a participação de economistas de diferentes perfis.
        \item Seleção probabilística de respondentes em bases de dados públicas e listas institucionais, reduzindo viés de autoseleção.
    \end{itemize}
    \item Distribuição e coleta de dados: O questionário foi aplicado online, via Google Forms, com tempo médio de resposta de 15 minutos.
    \item Tamanho da amostra: A pesquisa busca entre 1.500 e 2.500 respondentes, assegurando robustez estatística. O mínimo necessário para análises confiáveis foi definido em 500 respondentes.
\end{itemize}

A comparação entre os resultados obtidos no Brasil e os da SAEE original permitirá avaliar a replicabilidade dos achados e detectar possíveis diferenças institucionais e culturais na formação dos vieses econômicos.

\subsection{Técnicas de Análise Empírica}

A análise econométrica foi conduzida utilizando modelos de regressão \textit{logit}, sendo escolhida a versão binária ou ordenada conforme a natureza da variável dependente em cada especificação. Quando a variável de interesse assume apenas dois valores possíveis (\( y_i \in \{0,1\} \)), como no caso de perguntas com resposta dicotômica (ex: ``concorda'' vs. ``não concorda''), aplica-se o modelo \textit{logit} binário. Já quando a variável apresenta três categorias ordinais (\( y_i \in \{0,1,2\} \)), como nas escalas de grau de concordância ou percepção, emprega-se o modelo \textit{logit} ordenado, o qual estima os pontos de corte (\( \tau_j \)) entre os níveis. As especificações dos dois modelos são apresentadas a seguir:


\begin{equation}
    P(y_i = 1 \mid X_i) = \frac{e^{X_i \beta}}{1 + e^{X_i \beta}}, \quad y_i \in \{0, 1\}
\end{equation}

\begin{equation}
    P(y_i \leq j \mid X_i) = \frac{1}{1 + e^{-(\tau_j - X_i \beta)}}, \quad j = 0, 1; \quad y_i \in \{0, 1, 2\}
\end{equation}




Nas equações acima, \( y_i \) representa a resposta do indivíduo às perguntas do questionário, enquanto \( X_i \) denota o vetor de variáveis explicativas. A estimação dos modelos foi realizada por meio do método L-BFGS (Limited-memory Broyden–Fletcher–Goldfarb–Shanno), uma técnica de otimização baseada em gradiente, eficiente para modelos com grande número de parâmetros. Cada modelo foi ajustado separadamente para as diferentes perguntas do questionário, permitindo identificar quais fatores influenciam as variações nas percepções econômicas dos respondentes.


\chapter{Teorias e Evidências Sobre a (I)Racionalidade Humana} % Revisão de Literatura

% Revisar a literatura sobre economia comportamental e sua aplicação à política.

A hipótese da racionalidade plena dos agentes econômicos, consagrada pela teoria da escolha racional, tem sido um dos pilares analíticos da economia neoclássica e de modelos normativos de democracia. No entanto, ao confrontar-se com a realidade política, especialmente no que tange ao comportamento do eleitor médio, tal suposição revela-se cada vez menos plausível. A crescente literatura da economia comportamental, em diálogo com a ciência política e a psicologia cognitiva, oferece uma alternativa teórica mais realista: os indivíduos, em contextos de incerteza e baixa responsabilização, tomam decisões sistematicamente enviesadas.

Nesse sentido, autores como \citeonline{kahneman2011thinking} e \citeonline{Judgment_under_Uncertainty} demonstram empiricamente que os julgamentos humanos são fortemente influenciados por heurísticas cognitivas — atalhos mentais que, embora úteis, produzem erros previsíveis. Esses desvios da racionalidade instrumental se intensificam no campo da política, onde o custo da desinformação é socialmente disperso e o benefício individual de votar corretamente é estatisticamente nulo.

A economia política comportamental surge, assim, como um campo híbrido, cuja proposta é investigar as falhas sistemáticas do processo democrático à luz da racionalidade limitada dos eleitores. Bryan Caplan, em \textit{The Myth of the Rational Voter} \citeyear{The_Myth_of_the_Rational_Voter}, articula essa perspectiva ao argumentar que os eleitores não apenas carecem de informação, mas mantêm ativamente crenças econômicas falsas com convicção, um fenômeno que denomina ``irracionalidade racional''. Para Caplan, a democracia falha precisamente porque responde às preferências dos eleitores — e essas preferências, por sua vez, são moldadas por vieses cognitivos persistentes e emocionalmente gratificantes.

O desafio proposto neste capítulo, portanto, é examinar criticamente a suposição de que os eleitores agem como agentes racionais e bem informados. Em vez disso, buscaremos sistematizar evidências empíricas e abordagens teóricas que apontam para um padrão recorrente de distorções cognitivas nas decisões políticas. A orientação metodológica adotada segue os princípios do rigor científico: partimos de hipóteses refutáveis, sustentadas por dados observáveis e modelos teóricos claros, rejeitando interpretações tautológicas ou não falsificáveis.

Esta revisão será estruturada em cinco seções principais. Na primeira, contrastaremos a tradição da escolha racional com os avanços da economia comportamental, do pensamento de Adam Smith à psicologia de Kahneman. Em seguida, abordaremos os vieses cognitivos mais relevantes para o comportamento político, como o viés antimercado, antiestrangeiro, antitrabalho e o pessimismo econômico, conforme identificados por \cite{Systematically_Biased_Beliefs_about_Economics} e operacionalizados em pesquisas como a SAEE\footnote{Survey of Americans and Economists on the Economy, de 1996, replicado posteriormente com adaptações em diversos contextos.}. Na terceira seção, exploraremos os limites da informação no processo político, discutindo a racionalidade limitada e os custos da ignorância deliberada.

A quarta seção amplia a discussão para o plano da difusão de ideias econômicas, enfatizando o papel da mídia, da cultura e dos incentivos políticos na propagação de crenças disfuncionais. Por fim, argumentaremos que muitas crenças persistem não apesar das evidências, mas por causa de estruturas emocionais, sociais e identitárias que as tornam desejáveis. A política, nesse contexto, torna-se menos uma arena de deliberação racional e mais um campo de validação simbólica.

Ao longo das próximas seções, o objetivo será, portanto, compreender como e por que o ``homo politicus'' se afasta do ``homo economicus'' — e quais as consequências dessa ruptura para o funcionamento da democracia.

\section{Entre Adam Smith e Kahneman} % Economia Comportamental vs. Escolha Racional

A teoria da escolha racional pressupõe que os indivíduos possuem preferências bem definidas, transitivas, estáveis no tempo e plenamente informadas, sendo capazes de maximizar sua utilidade sob restrições orçamentárias ou institucionais. Esse arcabouço, predominante na teoria econômica neoclássica e em grande parte da teoria política normativa, fundamenta-se na suposição de que os agentes tomam decisões com base em cálculo racional e objetivo.

Entretanto, a própria tradição clássica já revelava indícios de uma racionalidade mais complexa. Adam Smith, em sua obra \textit{The Theory of Moral Sentiments} \cite{smith1759-theory-of-moral-sentiments}, reconhecia a importância das emoções, da empatia e dos julgamentos morais na ação humana, elementos frequentemente negligenciados na leitura reducionista de sua obra posterior, \textit{The Wealth of Nations} \cite{smith1776inquiry}. Smith argumentava que o ser humano age não apenas por interesse próprio, mas também por um ``espectador imparcial'' interior, que guia seus comportamentos com base em um senso de propriedade moral.

Ao longo do século XX, contudo, consolidou-se uma abordagem que privilegiava o modelo do \textit{homo economicus} — agente maximizador, autocentrado e plenamente racional. Essa perspectiva é defendida por autores como \citeonline{becker1976}, que buscam generalizar o modelo racional para além da esfera econômica, incorporando-o ao estudo de comportamentos familiares, criminosos e políticos. Para Becker, desvios comportamentais não são falhas da racionalidade, mas reflexos de diferentes restrições ou preferências.

Em contraposição, a economia comportamental, liderada por autores como Daniel Kahneman e Amos Tversky, introduz evidências robustas de que os indivíduos recorrem a heurísticas para lidar com situações de incerteza e complexidade cognitiva \cite{Judgment_under_Uncertainty}. Tais heurísticas — como representatividade, disponibilidade e ancoragem — são mentalmente econômicas, mas produzem vieses sistemáticos de julgamento, especialmente em decisões com baixa responsabilização ou com forte carga emocional \cite{kahneman2011thinking}. A crítica central à escolha racional não reside na presença de erros pontuais, mas na recorrência e previsibilidade desses erros.

Ainda no campo da economia política, Anthony Downs formulou a teoria da ignorância racional, segundo a qual os eleitores têm pouco incentivo para buscar informações políticas de qualidade, dado que o custo de adquirir conhecimento é elevado e a probabilidade de influenciar o resultado eleitoral é virtualmente nula \cite{downs1957economic}. Embora essa formulação mantenha a lógica da racionalidade instrumental — onde os indivíduos escolhem ser ignorantes de forma racional — o próprio Downs reconhece que esse comportamento frequentemente escapa aos modelos estritamente racionais e demanda investigação empírica. Ao fazer isso, ele antecipa a crítica posterior da economia comportamental ao reconhecer limitações cognitivas e traços de irracionalidade no comportamento político.

Essa divergência epistemológica implica consequências distintas para a formulação de políticas públicas. Enquanto o modelo racional fornece fundamentos para intervenções baseadas em incentivos marginais, a abordagem comportamental exige o reconhecimento das limitações cognitivas e emocionais dos indivíduos, bem como dos contextos institucionais em que as decisões são tomadas. \citeonline{Hausman_McPherson_Satz_2016}, por exemplo, defendem que os pressupostos normativos da racionalidade devem ser avaliados também à luz de critérios éticos, considerando os efeitos distributivos e as estruturas de poder subjacentes aos mecanismos de escolha.

Além disso, autores como \citeonline{hayek_knowledge_use} e \citeonline{positive_economics_friedman} enfatizam a natureza dispersa e tácita do conhecimento na sociedade. Para Hayek, a impossibilidade de centralizar todas as informações relevantes torna o sistema de preços uma ferramenta insubstituível de coordenação — o que exige, necessariamente, respeito às limitações cognitivas dos planejadores centrais. Friedman, por sua vez, defende que o mérito de uma teoria econômica não está na veracidade de suas premissas, mas na capacidade de gerar previsões consistentes com a realidade empírica.

Portanto, o contraste entre a visão clássica da racionalidade plena e a abordagem comportamental da racionalidade limitada não é apenas uma divergência de escopo analítico, mas um debate metodológico e filosófico sobre a natureza da ação humana. Enquanto a escolha racional pressupõe agentes isolados e informados, a economia comportamental propõe um agente falho, influenciado por emoções, contextos sociais e atalhos cognitivos — uma figura que se aproxima mais do cidadão comum do que do idealizado \textit{decision-maker} dos modelos normativos.

\section{Como os Vieses Moldeiam as Escolhas Políticas} % (Vieses Cognitivos e Política)

% Explicar os principais vieses cognitivos identificados na literatura e sua relação com decisões políticas (viés antimercado, antiestrangeiro, antitrabalho, pessimista, entre outros).

Ao analisar o comportamento do eleitor médio, Bryan Caplan propõe um deslocamento da hipótese da ignorância racional — centrada na ausência de informação — para a noção de \textit{irracionalidade racional}, na qual o eleitor sustenta ativamente crenças falsas, mesmo diante de evidências contrárias \cite{The_Myth_of_the_Rational_Voter}. Segundo o autor, os erros de julgamento político não ocorrem de forma aleatória ou esporádica, mas seguem padrões previsíveis, sistemáticos e estáveis ao longo do tempo. Tais distorções não derivam apenas da dificuldade de acesso à informação, mas sobretudo de motivações emocionais e ideológicas que incentivam os indivíduos a rejeitar proposições verdadeiras em favor de crenças confortáveis ou moralmente atraentes.

Caplan identifica quatro vieses cognitivos centrais que moldam negativamente o julgamento político dos eleitores: o viés antimercado, o viés antiestrangeiro, o viés antitrabalho e o viés pessimista \cite{Systematically_Biased_Beliefs_about_Economics}. Cada um desses vieses opera como uma lente interpretativa que distorce o entendimento sobre fenômenos econômicos complexos, gerando apoio a políticas públicas ineficientes ou contraproducentes.

O \textit{viés antimercado} refere-se à tendência de subestimar os benefícios dos mecanismos de mercado e superestimar os efeitos negativos do lucro, da concorrência e da desregulação. Essa visão, embora difundida, colide com fundamentos consolidados da teoria econômica. Autores como \citeonline{mankiw2020introducao} e \citeonline{sowell2000basic} enfatizam que o mercado, embora imperfeito, tende a alocar recursos de maneira mais eficiente do que alternativas centralizadas. A crítica liberal clássica, expressa já no século XIX por \citeonline{bastiat1859sofismas}, denunciava a miopia das políticas protecionistas e intervencionistas, sintetizada em sua célebre distinção entre ``o que se vê e o que não se vê''. O eleitor médio, no entanto, tende a interpretar lucros como exploração e concorrência como caos, ignorando os efeitos sistêmicos da coordenação via preços.

O \textit{viés antiestrangeiro} manifesta-se na desconfiança em relação ao comércio internacional, à imigração e à cooperação econômica global. O senso comum político frequentemente interpreta a importação de bens como ameaça à produção nacional e a presença de imigrantes como competição desleal no mercado de trabalho. Essa percepção, entretanto, ignora os ganhos de eficiência associados à especialização e à vantagem comparativa, conforme discutido por autores como \citeonline{bhagwati2003free} e \citeonline{landsburg2012armchair}. Historicamente, \citeonline{schumpeter1976capitalism} já advertia que o nacionalismo econômico tende a ressurgir em contextos de crise, alimentado por argumentos emocionais em detrimento da análise racional.

O \textit{viés antitrabalho} (ou \textit{make-work bias}, na formulação original) consiste na crença de que o objetivo central da economia deve ser preservar empregos, e não maximizar a produção ou a eficiência. Tal viés conduz ao apoio a políticas de proteção de setores ineficientes, de subsídios à manutenção artificial do emprego e de resistência à automação. Para \citeonline{sowell2004applied}, esse tipo de raciocínio falha ao ignorar que o progresso econômico consiste, justamente, em produzir mais com menos trabalho. Schumpeter, por sua vez, via a destruição criadora como essência do dinamismo capitalista — ideia que contraria frontalmente a aversão popular à substituição de empregos obsoletos por novas formas de produção.

O \textit{viés pessimista}, por fim, refere-se à percepção generalizada de que a economia está em constante deterioração, independentemente dos dados objetivos. Esse sentimento é frequentemente explorado por discursos populistas e reformistas que prometem ``salvar o país'' de uma crise iminente. Autores como \citeonline{Myths-of-Rich-and-Poor} e \citeonline{easterbrook2004progress} demonstram que, apesar de oscilações conjunturais, os indicadores de bem-estar material apresentaram melhora significativa nas últimas décadas. Ainda assim, a percepção negativa persiste, alimentada por vieses de confirmação e pela seletividade das narrativas midiáticas.

Esses vieses não apenas distorcem a compreensão dos fenômenos econômicos, mas também moldam as preferências políticas dos eleitores, influenciando diretamente a formulação de políticas públicas. Conforme sintetiza Caplan, ``na visão ingênua do interesse público, a democracia funciona porque faz o que os eleitores querem. Para a maioria dos céticos da democracia, ela falha porque não faz o que os eleitores querem. Na minha visão, a democracia falha justamente porque faz o que os eleitores querem'' \cite[p.~3]{The_Myth_of_the_Rational_Voter}. O custo dessa irracionalidade é amplamente socializado, o que desincentiva a revisão crítica de crenças e reforça a estabilidade dos erros cognitivos.

De modo geral, observa-se que o eleitor médio tende a interpretar os mecanismos de mercado com desconfiança, atribuindo-lhes motivações obscuras ou ilegítimas, ainda que sem fundamentação analítica ou respaldo empírico. Tal percepção, embora compreensível sob uma ótica emocional ou intuitiva, reflete padrões cognitivos enviesados que comprometem a avaliação racional de políticas econômicas. Compreender os vieses subjacentes a esse tipo de raciocínio é fundamental para diagnosticar as disfunções do processo democrático e para delimitar os alcances e limitações da deliberação pública em contextos marcados por racionalidade limitada.

\section{O Custo da Ignorância} % (Economia da Informação e Racionalidade Limitada)

% Discutir os impactos da desinformação e da racionalidade limitada no processo eleitoral e na formulação de políticas públicas.

A qualidade das decisões políticas em uma democracia depende, ao menos em parte, do nível de informação dos eleitores. No entanto, a literatura em economia política aponta que os cidadãos, em geral, demonstram baixo engajamento com a busca por conhecimento relevante sobre política e economia, o que compromete a racionalidade esperada de seu comportamento eleitoral. Esse fenômeno é abordado por \citeonline{downs1957economic} por meio do conceito de ignorância racional: dado que o custo de se informar adequadamente é alto e o impacto de um único voto é estatisticamente insignificante, os eleitores têm poucos incentivos para adquirir conhecimento político de qualidade.

A decisão de permanecer desinformado, nesse contexto, não é um erro de julgamento, mas uma escolha racional diante de incentivos inadequados. Essa formulação, embora coerente com a lógica econômica tradicional, revela limitações quando confrontada com evidências empíricas mais recentes. Estudos como os de \citeonline{The_Myth_of_the_Rational_Voter} indicam que os eleitores não apenas ignoram informações relevantes, mas também sustentam ativamente crenças falsas, mesmo quando confrontados com dados em contrário. A irracionalidade, nesse caso, não é um subproduto da ignorância passiva, mas um comportamento ativo de resistência à revisão de crenças.

Além disso, como argumenta \citeonline{hayek_knowledge_use}, a informação necessária para tomar decisões políticas eficazes é dispersa, tácita e situada nos agentes que vivenciam realidades econômicas específicas. A tentativa de centralizar tal conhecimento por meio de instituições políticas enfrenta limites intransponíveis, o que compromete a eficácia de políticas públicas baseadas em modelos excessivamente agregados ou descontextualizados. A ordem espontânea gerada pelo mercado — guiada por sinais de preços que condensam informação — é, nesse sentido, superior à coordenação política em muitos aspectos, justamente porque respeita a dispersão irredutível do conhecimento.

\citeonline{von1949human} enfatiza que, embora a racionalidade absoluta seja inatingível, ela deve ser preservada como um ideal regulativo. A ciência econômica, nesse sentido, deve se orientar pela busca contínua de maior coerência lógica e empírica, mesmo reconhecendo os limites cognitivos dos agentes. O problema central, portanto, não reside apenas na ausência de informação, mas na estrutura institucional que desincentiva a busca por verdade e promove a indulgência em crenças infundadas. Conforme argumenta Caplan, o problema da democracia não é apenas ignorância; para ele, ``os eleitores são piores que ignorantes: são, em uma palavra, irracionais — e votam de acordo'' \cite[p.~2]{The_Myth_of_the_Rational_Voter}.

O custo dessa ignorância deliberada transcende o plano individual e se manifesta em escolhas coletivas disfuncionais. Políticas ineficazes, protecionistas ou fiscalmente irresponsáveis são frequentemente resultado de crenças equivocadas, amplificadas por incentivos políticos de curto prazo. Quando o custo do erro é difuso e suas consequências são socializadas, torna-se racional para o indivíduo permanecer irracional.

Em última instância, o problema não é que o eleitor esteja mal informado; é que ele está racionalmente mal informado. Esse paradoxo, situado na interseção entre psicologia, teoria econômica e ciência política, desafia os modelos normativos de democracia e impõe limites práticos à qualidade das decisões públicas em contextos de racionalidade limitada.

\section{Do Sofista ao Populista} %  Como as Ideias Econômicas se Espalham (História do Pensamento Econômico e Propagação de Crenças)

% Explorar como ideias econômicas se propagam, influenciadas por interesses políticos, mídia e cultura.

A persistência de ideias econômicas equivocadas, mesmo diante de abundantes evidências contrárias, exige uma análise que transcenda os modelos tradicionais de aprendizagem racional. Mais do que simples desconhecimento, muitos equívocos econômicos decorrem da forma como determinadas crenças são formadas, disseminadas e preservadas ao longo do tempo. A história do pensamento econômico, quando devidamente resgatada, permite compreender como certos erros conceituais adquiriram legitimidade social e passaram a fundamentar políticas públicas, muitas vezes com base em argumentos intuitivos, morais ou puramente retóricos — ainda que desprovidos de fundamentação teórica rigorosa.

Esta seção adota uma abordagem analítica, valendo-se de paralelos conceituais entre formas históricas de persuasão política. O título ``Do Sofista ao Populista'' não deve ser interpretado como um juízo moral ou normativo, mas como uma construção heurística que busca ilustrar, a partir de uma analogia funcional, como estruturas argumentativas emocionalmente sedutoras, mas logicamente frágeis operam na arena pública. Tal escolha estilística se ancora na tradição crítica da história das ideias econômicas, com o objetivo de expor padrões recorrentes de difusão de crenças econômicas infundadas.

Já no século XIX, Frédéric Bastiat alertava para o problema da miopia econômica induzida pela retórica política, distinguindo entre ``o que se vê e o que não se vê'' nas políticas públicas \cite{bastiat1859sofismas}. Enquanto os efeitos imediatos e visíveis de uma medida — como a proteção a uma indústria nacional — são politicamente valorizados, seus custos difusos e de longo prazo tendem a ser ignorados ou deliberadamente ocultados. Bastiat identificava nesse comportamento uma forma de sofisma econômico, ou seja, um raciocínio enganoso que, embora plausível à primeira vista, oculta seus próprios pressupostos falaciosos.

A recorrência desse padrão ao longo da história motivou autores como \citeonline{mark_history} e \citeonline{hart2019bastiat} a defenderem a relevância da história das ideias econômicas como ferramenta crítica para desnaturalizar concepções equivocadas sedimentadas no senso comum. Segundo Blaug, a economia — ao se afastar da história do pensamento — perde sua capacidade de autocrítica e se torna vulnerável à repetição de erros já superados em contextos anteriores. A análise histórica, nesse sentido, não é apenas um exercício erudito, mas um recurso metodológico necessário para compreender como ideias se transformam em crenças e como essas crenças moldam políticas.

Essa transformação é frequentemente mediada por interesses políticos, estruturas de poder e mecanismos culturais de legitimação. \citeonline{schumpeter1976capitalism}, ao teorizar sobre a destruição criadora como essência do dinamismo capitalista, reconhecia o paradoxo de que a própria lógica do progresso econômico — baseada na inovação e na substituição constante — gera ressentimento social e apelos nostálgicos por proteção, estabilidade e ``retorno ao passado''. Esse ressentimento alimenta discursos populistas que rejeitam os fundamentos econômicos da concorrência e da globalização, prometendo uma reconstrução ilusória de uma ordem econômica perdida.

Nesse contexto, o populismo moderno pode ser compreendido como uma reedição contemporânea dos antigos sofistas: ambos manipulam a linguagem, simplificam problemas complexos e oferecem soluções intuitivas, porém equivocadas. O populista de hoje é, em muitos aspectos, o sofista de ontem com acesso a redes sociais. Enquanto o primeiro utiliza algoritmos para propagar narrativas emocionalmente sedutoras, o segundo operava nos ágoras com retórica persuasiva. Em comum, ambos priorizam a adesão simbólica e afetiva à coerência lógica ou empírica.

A difusão de crenças econômicas infundadas é também impulsionada por mecanismos culturais e midiáticos que reforçam certos discursos e marginalizam outros. Como observa \citeonline{franco2022cartas}, o economista que pretende comunicar ideias impopulares — ainda que tecnicamente sólidas — enfrenta não apenas a resistência cognitiva do público, mas também um ambiente institucional e simbólico hostil à complexidade. O senso comum econômico, construído por camadas de intuição, moralismo e ideologia, forma um campo de forças que resiste à crítica técnica, mesmo quando esta é rigorosamente fundamentada.

Robert Higgs acrescenta, nesse debate, a noção de ``consentimento ideológico tácito'' para explicar como certas ideias econômicas se perpetuam institucionalmente mesmo quando são sabidamente ineficientes. Para Higgs, o Estado moderno opera por meio de uma legitimação difusa que transforma crenças populares em política institucional, independentemente de sua adequação técnica \cite{higgs1987crisis}. Esse fenômeno evidencia que a análise da disseminação de ideias econômicas exige não apenas o exame das preferências individuais, mas também a compreensão das estruturas institucionais e culturais que mediam a relação entre crença e política.

Portanto, o estudo da propagação de ideias econômicas não pode prescindir de uma abordagem histórica, cultural e institucional. A racionalidade do eleitor, já comprometida por vieses cognitivos e limitações informacionais, encontra na linguagem política e na cultura midiática um ambiente fértil para a reprodução de erros econômicos. A democracia, nesse cenário, não apenas reflete as crenças populares, mas também as amplifica — transformando intuições falaciosas em consensos legislativos.

\section{Preferência por Crenças e Resistência ao Conhecimento}

A persistência de crenças econômicas disfuncionais, mesmo diante de evidências amplamente disponíveis, não pode ser explicada apenas pela ignorância racional \cite{downs1957economic}. Diversos estudos sugerem que os indivíduos frequentemente adotam e mantêm crenças políticas com base em fatores emocionais, identitários e sociais, descolando-se de critérios epistêmicos objetivos. Em outras palavras, trata-se de uma forma de resistência ativa ao conhecimento — um comportamento que pode ser modelado como racional no curto prazo individual, mas irracional no longo prazo coletivo.

Caplan propõe que as crenças políticas funcionam como bens de consumo simbólico: os indivíduos extraem utilidade subjetiva ao manter convicções agradáveis, independentemente de sua veracidade \cite{The_Myth_of_the_Rational_Voter}. Como o custo dessa crença é diluído socialmente, e os benefícios emocionais são internalizados, a estrutura de incentivos torna-se enviesada em favor de ideias falsas, porém confortáveis.

Complementando esse raciocínio, Kahneman e Tversky identificam uma série de vieses cognitivos sistemáticos, como o viés de confirmação e a ancoragem, que explicam por que os indivíduos resistem à atualização de crenças mesmo quando confrontados com dados contrários \cite{kahneman2011thinking}. A esse fenômeno soma-se o chamado \textit{backfire effect}, pelo qual a apresentação de evidência contrária pode reforçar a crença original, especialmente em contextos politicamente polarizados \cite{nyhan2010corrections}.

Essa resistência é amplificada por fatores sociocognitivos descritos na teoria da cognição cultural \cite{kahan2012polarization}. Segundo esse modelo, os indivíduos tendem a filtrar informações de acordo com os valores predominantes em seu grupo de pertencimento, favorecendo a coesão social em detrimento da precisão factual.

Autores clássicos já advertiam para essa dissociação entre certeza subjetiva e evidência objetiva. John Locke, no \textit{Ensaio sobre o Entendimento Humano}, denuncia a ``exuberância da certeza'' como um indício de crença baseada em afeto e não em provas \cite{locke2014ensaio}. Ayn Rand, ao descrever o fenômeno do \textit{blanking out}, identifica um mecanismo pelo qual o indivíduo ignora deliberadamente informações que contradizem suas convicções ideológicas. Segundo a autora, trata-se de ``não ser cego, mas se recusar a ver; não ser ignorante, mas se recusar a saber [...] o não pensar é um ato de aniquilamento, um desejo de negar a existência'' \cite[p.~869]{rand2012revolta}. Esse apagamento voluntário da realidade serve como um escudo cognitivo para preservar a coerência interna da visão de mundo, mesmo diante de evidências acessíveis e verificáveis.

Além dos mecanismos individuais, há dinâmicas institucionais que reforçam a prevalência de crenças não falseáveis. A seletividade algorítmica das redes sociais e o tribalismo político dificultam a exposição a fontes discordantes, criando ambientes epistêmicos homogêneos. Como nota Zaller, a recepção de argumentos depende da disposição prévia do eleitor em aceitar novas informações, tornando a deliberação pública um processo de reforço, e não de revisão crítica \cite{zaller1992nature}.

Em síntese, a resistência ao conhecimento pode ser compreendida como um comportamento racionalmente motivado em nível individual, mas que compromete a racionalidade coletiva. Esse fenômeno limita a eficácia de políticas públicas baseadas em evidências, pois o eleitor médio não responde à informação como um cientista, mas como um consumidor de narrativas simbólicas. Para que essas crenças sejam tratadas como hipóteses científicas legítimas, elas deveriam estar sujeitas à refutação empírica — o que raramente ocorre na prática política.