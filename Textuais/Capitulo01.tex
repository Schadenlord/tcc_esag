

\chapter{Introdução} % O Labirinto das Decisões: Como Julgamos e Escolhemos?
% Adicionar aqui uma parte principal sobre qual a contribuição principal do tcc, que é sobre o desenvolvimento do campo da "economia politica comportamental", que é nova e ainda não foi amplamente explorada. 
% A ideia é que a economia comportamental, que já é um campo estabelecido, pode ser aplicada à política, e que isso pode ser uma nova forma de entender a política, que é mais realista e menos idealista do que a economia política tradicional.

%Aqui vamos apresentar o problema de pesquisa, sua relevância e contribuição para o campo da economia política comportamental.

% A democracia moderna parte do pressuposto de que os eleitores são agentes racionais, capazes de avaliar as consequências econômicas de suas escolhas políticas e apoiar medidas que maximizem o bem-estar social \cite{downs1957economic}. No entanto, a realidade demonstra que esse ideal frequentemente se desvia, devido à influência de vieses cognitivos e heurísticas que distorcem a percepção econômica da população \cite{The_Myth_of_the_Rational_Voter,kahneman2011thinking}. Como resultado, políticas públicas são frequentemente moldadas por crenças equivocadas, levando a decisões sub-ótimas que comprometem o desenvolvimento econômico e social \cite{acemoglu2012nations}.

% Nesse sentido, \citeauthoronline{The_Myth_of_the_Rational_Voter} (\citeyear{The_Myth_of_the_Rational_Voter}) observa que ``a democracia falha não porque ignora o povo, mas porque o ouve demais''. Essa afirmação sintetiza a tese central deste estudo: os equívocos na formulação de políticas públicas derivam, muitas vezes, do próprio atendimento às preferências populares — preferências estas que frequentemente se baseiam em crenças distorcidas e emocionalmente reconfortantes, mas economicamente disfuncionais.

% O fenômeno central deste trabalho é o impacto das crenças sistematicamente enviesadas dos eleitores — resultantes de uma racionalidade limitada — sobre a formulação de políticas econômicas. Eleitores bem-intencionados, mas cognitivamente limitados, acabam apoiando medidas que contradizem princípios econômicos fundamentais, como a vantagem comparativa, a eficiência dos mercados e os benefícios da inovação. A literatura, especialmente a pesquisa de Caplan \citeyear{Systematically_Biased_Beliefs_about_Economics,The_Myth_of_the_Rational_Voter}, demonstra que o público frequentemente rejeita consensos econômicos básicos, sustentando crenças que favorecem protecionismo, intervencionismo excessivo e desconfiança do setor produtivo. Esses padrões de distorção, identificados como vieses antimercado, antiestrangeiro, antitrabalho e pessimista, contribuem para a adoção de políticas que comprometem o crescimento econômico e o bem-estar social.

% A dificuldade de mitigar essas distorções vai além da simples falta de informação: envolve uma resistência psicológica profunda em abandonar crenças que reforçam visões de mundo preexistentes. Como apontam Hayek \citeyear{hayek_knowledge_use} e Downs \citeyear{downs1957economic}, a complexidade econômica e o custo elevado de se informar adequadamente dificultam o acesso a decisões racionais no ambiente político. Nesse contexto, consolida-se uma discrepância persistente entre o conhecimento técnico da economia e a opinião pública — um fator que influencia diretamente a qualidade das políticas adotadas.

% Embora a economia comportamental já seja um campo consolidado, seu diálogo com a política — particularmente por meio da chamada economia política comportamental — ainda é incipiente. Este trabalho busca contribuir para o desenvolvimento dessa vertente analítica, que considera as limitações cognitivas dos eleitores no processo democrático. Em vez de partir de uma visão idealizada da racionalidade política, propõe-se uma abordagem mais realista e empiricamente fundamentada.

% A relevância deste estudo se justifica pelo fato de que, apesar do avanço da literatura internacional, o Brasil ainda carece de investigações sistemáticas sobre como os eleitores percebem fenômenos econômicos e como essas percepções moldam decisões políticas. A ausência desse tipo de análise empírica em contextos latino-americanos representa uma lacuna teórica importante, especialmente em democracias consolidadas, mas marcadas por baixos níveis de educação econômica e alta desconfiança institucional.

A democracia moderna parte do pressuposto de que os eleitores são agentes racionais, capazes de avaliar as consequências econômicas de suas escolhas políticas e apoiar medidas que promovam o bem-estar coletivo \cite{downs1957economic}. No entanto, esse ideal frequentemente se choca com a realidade: vieses cognitivos e heurísticas distorcem a percepção econômica da população \cite{The_Myth_of_the_Rational_Voter,kahneman2011thinking}, influenciando a forma como ideias econômicas são compreendidas e apoiadas no debate público.

Como observa \citeonline[p.~3]{The_Myth_of_the_Rational_Voter}, a democracia não falha por ignorar o povo, mas justamente por ouvi-lo demais. Essa inversão irônica ilustra a tese central deste estudo: os equívocos políticos não decorrem da exclusão popular, mas da incorporação acrítica de preferências coletivas — frequentemente baseadas em percepções distorcidas e reconfortantes, ainda que incompatíveis com o conhecimento econômico técnico.

A literatura aponta que eleitores bem-intencionados, mas cognitivamente limitados, tendem a rejeitar consensos econômicos básicos e a endossar ideias que conflitam com noções fundamentais da teoria econômica, como a vantagem comparativa, a produtividade e os mecanismos de preços \cite{Systematically_Biased_Beliefs_about_Economics,The_Myth_of_the_Rational_Voter}. O problema vai além da falta de informação: há uma resistência ativa à revisão de crenças que reforçam identidades políticas e visões de mundo consolidadas.

Essa desconexão entre conhecimento técnico e opinião pública não é nova. \citeonline{hayek_knowledge_use} já advertia que o conhecimento econômico é contraintuitivo e disperso; \citeonline{downs1957economic} sugeria que o custo de se informar é alto demais para o retorno marginal de um voto. O resultado é um eleitor racionalmente mal informado — e, portanto, um sistema político sensível à popularidade de ideias economicamente frágeis, independentemente de seus méritos técnicos.

% Neste contexto, este trabalho insere-se no esforço emergente de desenvolver a economia política comportamental: um campo ainda incipiente que aplica os avanços da economia comportamental à compreensão das escolhas coletivas. Ao invés de idealizar a racionalidade democrática, propõe-se aqui encarar suas limitações cognitivas como ponto de partida analítico.

% A relevância do estudo reside na lacuna empírica existente no Brasil e em outras democracias latino-americanas: há escassez de investigações sistemáticas sobre como os eleitores percebem fenômenos econômicos e como essas percepções moldam suas preferências políticas. Ao trazer essa abordagem para o contexto brasileiro, este trabalho busca contribuir com evidências, conceitos e provocações para um debate ainda em construção.

% % A pesquisa SAEE é um documento institucional em meio eletrônico, produzida por entidades coletivas (The Washington Post, Harvard, Kaiser Foundation) e disponibilizada online. Assim, segue o mesmo modelo de referência utilizado para: Relatórios; Dados de pesquisa; Documentos técnicos de instituições; Pesquisas de opinião online.

% Para analisar empiricamente esse fenômeno no Brasil, este estudo recria a metodologia da SAEE, originalmente aplicada nos Estados Unidos, adaptando-a ao contexto brasileiro. A pesquisa coletará dados primários sobre as crenças econômicas dos eleitores brasileiros e os comparará com os dados da SAEE americana. O objetivo é investigar se os vieses observados nos EUA também estão presentes no Brasil, identificar possíveis divergências e avaliar fatores institucionais e culturais que possam influenciar essas percepções. Essa comparação se justifica porque ambos os países possuem democracias consolidadas, mas apresentam diferenças significativas em termos de escolaridade média, acesso à informação econômica e nível de desconfiança nas instituições. Enquanto os Estados Unidos possuem um longo histórico de pesquisas sobre a percepção pública da economia e sua relação com as políticas públicas \cite{blendon1997,Systematically_Biased_Beliefs_about_Economics, page1992}, o Brasil ainda carece de estudos que explorem sistematicamente como o eleitorado interpreta questões econômicas e como isso se reflete no cenário político. Comparar esses dois contextos permite entender se os vieses do eleitorado são fenômenos universais ou se há particularidades ligadas ao ambiente institucional e ao desenvolvimento econômico.

% No entanto, algumas limitações devem ser reconhecidas. Apesar da adaptação da metodologia da SAEE ao contexto brasileiro, diferenças institucionais e culturais entre os dois países podem afetar a comparabilidade dos resultados \cite{laporta1999quality,north1990institutions,acemoglu2012nations}. Além disso, a pesquisa se concentra na percepção econômica dos eleitores, não abrangendo outros fatores que também influenciam a formulação de políticas públicas, como o papel da mídia, o impacto de campanhas eleitorais e a disseminação de desinformação.

% É relevante destacar que esta pesquisa não pretende fornecer uma solução definitiva para o problema das crenças sistematicamente enviesadas no eleitorado, nem testar empiricamente as estratégias de mitigação sugeridas. O foco está na análise dos vieses cognitivos e suas consequências para a formulação de políticas, buscando oferecer um panorama teórico e empírico sobre o tema. Questões mais amplas relacionadas à desinformação deliberada, ao papel da mídia e a outros fatores externos não serão o foco central deste estudo, ainda que possam ser tangencialmente mencionadas.

% Embora a literatura internacional tenha avançado significativamente na análise dos vieses cognitivos na política, esse debate ainda é incipiente no Brasil. Há poucos estudos que exploram de forma sistemática como a percepção econômica dos eleitores brasileiros se distancia dos consensos acadêmicos e como isso afeta a formulação de políticas públicas. Dado o impacto de decisões econômicas equivocadas sobre o desenvolvimento do país — incluindo políticas protecionistas ineficientes, subsídios distorcidos e resistência a reformas estruturais —, compreender esses vieses se torna essencial para o aprimoramento do panorama político e econômico nacional.

% Diante desse cenário, este estudo busca analisar como os vieses cognitivos moldam a percepção econômica dos eleitores brasileiros e influenciam a formulação de políticas públicas. Para isso, será realizado um experimento de survey, inspirado na \textit{Survey of Americans and Economists on the Economy} (SAEE) adaptado ao contexto institucional e cultural brasileiro. Essa investigação comparativa servirá de base para a análise empírica desenvolvida ao longo do trabalho.

\section{Formulação do problema de pesquisa: a racionalidade coletiva} %(Tema e Problema de Pesquisa)

% Introduzir o tema do trabalho e a questão central: os vieses de julgamento dos eleitores e suas implicações políticas e econômicas.

O ideal democrático parte do pressuposto de que a soma das decisões individuais resulta em escolhas coletivas racionais e benéficas para a sociedade. No entanto, a prática política mostra que essa expectativa nem sempre se confirma. Mesmo indivíduos que tomam decisões prudentes em sua vida privada frequentemente apoiam propostas políticas que contrariam princípios econômicos fundamentais, contribuindo para a adoção de medidas desconectadas do conhecimento técnico acumulado na disciplina econômica \cite{downs1957economic,The_Myth_of_the_Rational_Voter}.

Esse paradoxo pode ser explicado pela diferença estrutural entre os contextos de escolha no mercado e na política. Enquanto no mercado os indivíduos arcam diretamente com os custos de suas decisões, no processo eleitoral o impacto de um único voto é praticamente nulo, o que reduz o incentivo à busca por informações qualificadas. Isso leva à chamada “ignorância racional” \cite{downs1957economic}, em que o custo de se informar supera os ganhos esperados de uma decisão fundamentada.

\citeauthoronline{The_Myth_of_the_Rational_Voter} (\citeyear{The_Myth_of_the_Rational_Voter}) aprofunda essa análise ao argumentar que o problema não é apenas de ignorância, mas de vieses sistemáticos nas crenças dos eleitores. Muitos não apenas desconhecem conceitos econômicos básicos, como também resistem a evidências que contradizem suas intuições e convicções ideológicas. Essa racionalidade limitada gera um ciclo no qual políticos — por afinidade ou conveniência — respondem a percepções distorcidas da opinião pública, reforçando ideias populares, ainda que desalinhadas com o consenso técnico.

A consequência é que democracias modernas, em vez de corrigirem equívocos cognitivos por meio do debate público, frequentemente os amplificam. Assim, preferências enviesadas passam a moldar o apoio eleitoral a determinadas políticas econômicas, criando uma assimetria persistente entre opinião popular e conhecimento especializado.

% Diante desse cenário, este trabalho busca investigar como os vieses cognitivos dos eleitores — especialmente em relação à economia — influenciam suas percepções e preferências políticas no Brasil. Parte-se da hipótese de que o ambiente democrático, ao amplificar essas crenças distorcidas, contribui para a consolidação de padrões de opinião pública que favorecem propostas inconsistentes com os fundamentos econômicos, independentemente de sua eficácia prática.

\section{Hipóteses a Serem Testadas}

Com base nos dados coletados e na metodologia aplicada, este estudo propõe as seguintes hipóteses, estruturadas para serem empiricamente testáveis e refutáveis:

\begin{enumerate}[label=\alph*)]

    \item \textbf{H1 - Eleitores brasileiros apresentam vieses sistemáticos que distorcem sua percepção sobre fenômenos econômicos.}  
    Se os vieses cognitivos forem relevantes, os dados coletados devem revelar padrões recorrentes de julgamento econômico que divergem de princípios amplamente consensuais na literatura econômica. Caso não haja tais padrões, a hipótese será refutada \cite{The_Myth_of_the_Rational_Voter, blendon1997};
    % Caplan (2007) e Blendon et al. (1997) mostram que os eleitores tendem a divergir sistematicamente dos economistas em temas centrais de política econômica, revelando padrões consistentes de julgamento enviesado.

    \item \textbf{H2 - O conhecimento econômico reduz a probabilidade de manifestação de vieses cognitivos.}  
    Se o conhecimento econômico atua como mitigador de vieses, então eleitores com maior escolaridade ou maior familiaridade com conceitos econômicos devem apresentar menor adesão a padrões enviesados. Caso não haja correlação significativa, a hipótese será refutada \cite{downs1957economic, Judgment_under_Uncertainty};
    % Downs (1957) sustenta que eleitores optam racionalmente por permanecer desinformados; Kahneman e Tversky (1974) argumentam que educação pode mitigar os atalhos heurísticos que distorcem julgamentos econômicos. Tomara que aqui seja refutado.

    \item \textbf{H3 - O viés antimercado está associado ao apoio a políticas intervencionistas.}  
    Se o viés antimercado estiver presente, eleitores que expressam desconfiança em relação ao livre mercado devem mostrar maior apoio a políticas de regulação, controle ou proteção estatal. Caso não haja correlação, a hipótese será refutada \cite{The_Myth_of_the_Rational_Voter, sowell2004applied};
    % Caplan (2007) identifica o viés antimercado como central nas crenças populares; Sowell (2004) mostra como o desconhecimento da lógica de mercado leva à demanda por controles ineficientes.

    \item \textbf{H4 - O viés antiestrangeiro está associado ao apoio a restrições comerciais e migratórias.}  
    Se esse viés for relevante, eleitores que expressam preocupação com a concorrência estrangeira devem mostrar maior apoio a barreiras ao comércio exterior e à imigração. Se não houver correlação, a hipótese será refutada \cite{The_Myth_of_the_Rational_Voter, bhagwati2003free};
    % Caplan (2007) destaca a tendência de superestimar o dano causado por estrangeiros; Bhagwati (2003) demonstra os ganhos do comércio internacional e critica o protecionismo como resposta emocional.

    \item \textbf{H5 - O viés antitrabalho está associado ao apoio a políticas que priorizam a criação direta de empregos.}  
    Se os eleitores supervalorizam a criação de empregos como um fim em si, eles devem tender a apoiar propostas que desincentivam automação, realocação produtiva ou reestruturação de setores. Se não houver esse padrão, a hipótese será refutada \cite{The_Myth_of_the_Rational_Voter, landsburg2012armchair};
    % Segundo Caplan (2007), esse viés decorre da visão sentimental do trabalho como valor intrínseco; Landsburg (2012) observa que o público muitas vezes ignora os ganhos de produtividade associados à destruição criativa.

    \item \textbf{H6 - O viés pessimista leva a avaliações econômicas mais negativas do que os dados indicam.}  
    Se o viés pessimista estiver presente, os eleitores tenderão a avaliar negativamente o desempenho econômico nacional mesmo em contextos de indicadores positivos. Se a percepção estiver alinhada com os dados, a hipótese será refutada \cite{The_Myth_of_the_Rational_Voter, easterbrook2004progress};
    % Caplan (2007) argumenta que os eleitores tendem a subestimar o progresso econômico; Easterbrook (2004) mostra que, apesar de avanços reais, a percepção popular frequentemente é de estagnação ou retrocesso.

    \item \textbf{H7 - A filiação ideológica influencia a aceitação de evidências econômicas.}  
    Se a ideologia interfere na interpretação de dados econômicos, eleitores de diferentes espectros políticos tenderão a responder de forma distinta a evidências empíricas, mesmo quando essas forem tecnicamente robustas. Se não houver esse padrão, a hipótese será refutada \cite{The_Myth_of_the_Rational_Voter, kahan2012polarization}.
    % Caplan (2007) menciona o viés de confirmação ideológica; Kahan (2012) mostra que mesmo indivíduos com alta capacidade analítica distorcem evidências para se manterem consistentes com sua ideologia.

    % \item \textbf{H8 - Economistas também podem ser influenciados por vieses ideológicos ou autoindulgentes.}  ver se realmente coloco aqui ou deixo somente esses vieses na metodologia 
    % Se os economistas não forem epistemicamente neutros, então suas recomendações de política pública devem refletir, ao menos em parte, preferências normativas, alinhamentos ideológicos ou pressões institucionais. Caso os dados indiquem que a formação técnica é o único determinante das posições econômicas, a hipótese será refutada \cite{The_Myth_of_the_Rational_Voter, Hausman_McPherson_Satz_2016}.  
    % Embora Caplan (2007) reconheça que os economistas são menos propensos a certos vieses, ele admite que ideologia também afeta esse grupo; Hausman et al. (2016) destacam que o juízo técnico frequentemente incorpora valores morais e pressupostos normativos.

\end{enumerate}


\section{Objetivo Geral}

Investigar os principais vieses cognitivos na percepção econômica dos eleitores brasileiros, por meio da aplicação de um \textit{survey} experimental adaptado ao contexto nacional, com base na literatura de economia política comportamental.

\section{Objetivos Específicos}

\begin{enumerate}[label=\alph*)]
    \item identificar, com base na literatura sobre economia comportamental e economia política, os principais vieses cognitivos relacionados à percepção econômica, adaptando-os ao contexto brasileiro;

    \item construir e aplicar um survey experimental junto a eleitores brasileiros, inspirado na metodologia da \textit{Survey of Americans and Economists on the Economy} (SAEE), com questionário adaptado à realidade institucional e cultural do país;

    \item mapear a presença e frequência dos vieses econômicos entre os eleitores, com base nos dados coletados, utilizando análises estatísticas e modelos empíricos apropriados;

    \item identificar padrões de discrepância entre as percepções econômicas dos eleitores brasileiros e os consensos identificados na literatura econômica, investigando possíveis determinantes contextuais e cognitivos dessas diferenças;

    \item explorar a associação entre a presença de vieses econômicos e variáveis sociodemográficas, educacionais e institucionais, buscando compreender fatores relacionados à formação dessas percepções;

    \item contribuir para o avanço da economia política comportamental por meio da produção de evidências empíricas sobre como vieses cognitivos influenciam a percepção econômica dos eleitores em democracias como a brasileira.

\end{enumerate}

\section{Por que a Democracia Premia a Ignorância Econômica?
% (Justificativa teórica e relevância empírica)
} %(Justificativa e Relevância do Estudo)
% % deve enfatizar que o problema não é ignorância aleatória, mas crenças sistematicamente erradas, o que gera políticas ruins mesmo quando os eleitores estão bem informados.

% A democracia parte do pressuposto de que os eleitores escolhem representantes e políticas que maximizam o bem-estar coletivo. No entanto, a realidade demonstra que decisões econômicas são frequentemente moldadas por crenças sistematicamente equivocadas — e não apenas por desinformação aleatória. O problema central está na presença de \textit{vieses cognitivos estruturais}, que distorcem a compreensão dos fenômenos econômicos e sustentam políticas públicas ineficazes.

% A literatura em economia política comportamental mostra que esses vieses seguem padrões previsíveis. \citeonline{The_Myth_of_the_Rational_Voter} argumenta que os eleitores tendem a manter crenças enviesadas sobre comércio, mercados e tecnologia, mesmo com acesso à informação. Essas crenças favorecem o protecionismo, o intervencionismo e a tecnofobia — independentemente do nível de escolaridade. Como o custo de estar errado é diluído entre milhões de votos, não há incentivo individual para revisar crenças equivocadas \cite{downs1957economic}.

% A teoria do conhecimento disperso de \citeauthoronline{hayek_knowledge_use} (\citeyear{hayek_knowledge_use}) reforça que a economia é contraintuitiva: seus efeitos de segunda ordem muitas vezes contradizem percepções imediatas. Crenças como ``importações eliminam empregos'' ou ``preços baixos reduzem salários'' ignoram mecanismos compensatórios difíceis de comunicar ao público. Diante disso, políticos tendem a validar as crenças populares — ainda que incorretas — pois confrontá-las pode significar perder eleições.

% A relevância deste estudo está em investigar a dissonância entre opinião pública e conhecimento técnico, e suas implicações para a formulação de políticas. Ao analisar como vieses cognitivos moldam o debate econômico, esta pesquisa busca contribuir para o avanço da economia política comportamental, apontando caminhos possíveis de mitigação — seja por meio da educação econômica, seja pela reforma de incentivos institucionais.

% Sem compreender as raízes dessa distorção entre democracia e racionalidade econômica, políticas ineficazes continuarão a se repetir. O desafio não está apenas em combater a ignorância, mas em corrigir crenças sistematicamente erradas que moldam decisões políticas e o futuro das sociedades.

% A ideia democrática repousa sobre o princípio da autodeterminação informada. No entanto, o processo político moderno tem revelado uma dissonância persistente entre o conhecimento técnico acumulado pela ciência econômica e as crenças amplamente difundidas entre o eleitorado. Não se trata de ignorância aleatória, mas da presença sistemática de padrões cognitivos que distorcem a percepção da realidade econômica e conduzem à formulação de políticas públicas ineficazes \cite{The_Myth_of_the_Rational_Voter,Judgment_under_Uncertainty}. Esse descompasso tem implicações diretas sobre a qualidade das decisões coletivas e, por consequência, sobre a eficiência e a resiliência das instituições democráticas \cite{downs1957economic}.

% A economia, como disciplina, apresenta uma lógica contraintuitiva: os efeitos mais relevantes de uma política frequentemente não estão nos seus resultados imediatos, mas em suas consequências de segunda ordem — uma característica destacada desde a crítica clássica de “o que se vê e o que não se vê” \cite{bastiat1859sofismas}. Quando o julgamento popular se orienta por intuições enviesadas — como associar importações à perda de empregos ou crescimento ao aumento de desigualdade — o risco não é apenas o erro individual, mas a cristalização desses erros em decisões estatais de amplo alcance \cite{landsburg2012armchair,bhagwati2003free}. Em democracias de massa, onde o custo da má escolha é socializado e o benefício do erro é politicamente rentável, as crenças econômicas equivocadas tendem não apenas a persistir, mas a ser premiadas eleitoralmente \cite{The_Myth_of_the_Rational_Voter}.

% Compreender os mecanismos por meio dos quais essas distorções se consolidam é, portanto, uma exigência metodológica. A proposta deste trabalho é superar abordagens impressionistas ou normativas, testando empiricamente se determinados vieses de julgamento aparecem de forma consistente na população brasileira e como eles se associam à adesão a determinadas políticas públicas. Isso permite distinguir entre percepções legítimas e distorções cognitivas recorrentes, fornecendo evidências sobre os limites da racionalidade democrática em temas econômicos \cite{kahneman2011thinking,kahan2012polarization}.

% A relevância do tema também se manifesta em termos práticos. Em um cenário global marcado por instabilidade, pressão fiscal e polarização ideológica, a adoção de políticas fundamentadas em percepções incorretas não é apenas ineficiente, mas potencialmente desastrosa. O apoio popular a soluções simples para problemas complexos — como protecionismo, subsídios insustentáveis ou expansão arbitrária do papel estatal — revela um cenário de fragilidade cognitiva institucionalizada \cite{schumpeter1976capitalism,taleb2014antifragile}. Ao investigar como essas crenças se formam, se disseminam e se legitimam, este estudo contribui para a construção de mecanismos de resiliência democrática e para o desenho de estratégias de educação econômica mais eficazes \cite{franco2022cartas, zaller1992nature}.

% Em síntese, a relevância deste trabalho reside na sua capacidade de iluminar um paradoxo central da vida política contemporânea: o fato de que o voto, instrumento máximo da soberania popular, pode, quando orientado por percepções sistematicamente equivocadas, comprometer a própria racionalidade das decisões coletivas. Investigar esse fenômeno com rigor metodológico e sensibilidade interdisciplinar é condição indispensável para aperfeiçoar a governança democrática em contextos de crescente complexidade econômica e social.

A ideia democrática repousa sobre o princípio da autodeterminação informada. No entanto, observa-se uma dissonância crescente entre o conhecimento técnico acumulado pela ciência econômica e as crenças amplamente difundidas entre o eleitorado. Trata-se não apenas de ignorância, mas da presença sistemática de padrões cognitivos que distorcem a percepção da realidade econômica e influenciam a formulação de políticas públicas \cite{The_Myth_of_the_Rational_Voter,Judgment_under_Uncertainty}.

Esse descompasso desafia a capacidade deliberativa das democracias contemporâneas, afetando a qualidade das decisões coletivas \cite{downs1957economic}. Ao contrário do que pressupõem modelos normativos de democracia, o julgamento político do eleitor tende a ser moldado por fatores emocionais, ideológicos e cognitivos que resistem à revisão crítica \cite{kahneman2011thinking,kahan2012polarization}.

Diante disso, compreender como essas crenças econômicas se formam, se consolidam e se legitimam na arena política torna-se um desafio relevante para a ciência econômica e política. A presente pesquisa busca contribuir com esse debate ao investigar, empiricamente, se determinados vieses de julgamento aparecem de forma sistemática na população brasileira e como eles se associam ao apoio a diferentes políticas públicas.

A relevância do estudo é tanto teórica quanto prática. Teórica, por abordar as limitações da racionalidade democrática à luz da literatura comportamental e institucional contemporânea. Prática, por lançar luz sobre a aderência de determinadas crenças a projetos de governo, mesmo quando incompatíveis com diagnósticos técnicos. Em um cenário de crescente polarização, pressão fiscal e instabilidade institucional, a adoção de políticas fundamentadas em percepções incorretas não é apenas ineficiente, mas potencialmente desastrosa \cite{schumpeter1976capitalism,taleb2014antifragile}. Investigar esse processo contribui para o desenvolvimento de estratégias mais eficazes de resiliência democrática e educação econômica \cite{franco2022cartas,zaller1992nature}.

Em síntese, este trabalho parte do paradoxo central da vida política moderna: o fato de que o voto, expressão máxima da soberania popular, pode ser guiado por percepções sistematicamente equivocadas. A superação desse paradoxo exige não apenas a denúncia dos erros, mas a construção de uma abordagem interdisciplinar rigorosa — e empiricamente fundamentada — sobre os limites e possibilidades da racionalidade democrática.

\chapter{Teorias e Evidências Sobre a (I)Racionalidade Humana} % Revisão de Literatura

% Revisar a literatura sobre economia comportamental e sua aplicação à política.

A hipótese da racionalidade plena dos agentes econômicos, consagrada pela teoria da escolha racional, tem sido um dos pilares analíticos da economia neoclássica e de modelos normativos de democracia. No entanto, ao confrontar-se com a realidade política, especialmente no que tange ao comportamento do eleitor médio, tal suposição revela-se cada vez menos plausível. A crescente literatura da economia comportamental, em diálogo com a ciência política e a psicologia cognitiva, oferece uma alternativa teórica mais realista: os indivíduos, em contextos de incerteza e baixa responsabilização, tomam decisões sistematicamente enviesadas.

Nesse sentido, autores como \citeonline{kahneman2011thinking} e \citeonline{Judgment_under_Uncertainty} demonstram empiricamente que os julgamentos humanos são fortemente influenciados por heurísticas cognitivas — atalhos mentais que, embora úteis, produzem erros previsíveis. Esses desvios da racionalidade instrumental se intensificam no campo da política, onde o custo da desinformação é socialmente disperso e o benefício individual de votar corretamente é estatisticamente nulo.

A economia política comportamental surge, assim, como um campo híbrido, cuja proposta é investigar as falhas sistemáticas do processo democrático à luz da racionalidade limitada dos eleitores. Bryan Caplan, em \textit{The Myth of the Rational Voter} \citeyear{The_Myth_of_the_Rational_Voter}, articula essa perspectiva ao argumentar que os eleitores não apenas carecem de informação, mas mantêm ativamente crenças econômicas falsas com convicção, um fenômeno que denomina ``irracionalidade racional''. Para Caplan, a democracia falha precisamente porque responde às preferências dos eleitores — e essas preferências, por sua vez, são moldadas por vieses cognitivos persistentes e emocionalmente gratificantes.

O desafio proposto neste capítulo, portanto, é examinar criticamente a suposição de que os eleitores agem como agentes racionais e bem informados. Em vez disso, buscaremos sistematizar evidências empíricas e abordagens teóricas que apontam para um padrão recorrente de distorções cognitivas nas decisões políticas. A orientação metodológica adotada segue os princípios do rigor científico: partimos de hipóteses refutáveis, sustentadas por dados observáveis e modelos teóricos claros, rejeitando interpretações tautológicas ou não falsificáveis.

Esta revisão será estruturada em cinco seções principais. Na primeira, contrastaremos a tradição da escolha racional com os avanços da economia comportamental, do pensamento de Adam Smith à psicologia de Kahneman. Em seguida, abordaremos os vieses cognitivos mais relevantes para o comportamento político, como o viés antimercado, antiestrangeiro, antitrabalho e o pessimismo econômico, conforme identificados por \cite{Systematically_Biased_Beliefs_about_Economics} e operacionalizados em pesquisas como a SAEE\footnote{Survey of Americans and Economists on the Economy, de 1996, replicado posteriormente com adaptações em diversos contextos.}. Na terceira seção, exploraremos os limites da informação no processo político, discutindo a racionalidade limitada e os custos da ignorância deliberada.

A quarta seção amplia a discussão para o plano da difusão de ideias econômicas, enfatizando o papel da mídia, da cultura e dos incentivos políticos na propagação de crenças disfuncionais. Por fim, argumentaremos que muitas crenças persistem não apesar das evidências, mas por causa de estruturas emocionais, sociais e identitárias que as tornam desejáveis. A política, nesse contexto, torna-se menos uma arena de deliberação racional e mais um campo de validação simbólica.

Ao longo das próximas seções, o objetivo será, portanto, compreender como e por que o ``homo politicus'' se afasta do ``homo economicus'' — e quais as consequências dessa ruptura para o funcionamento da democracia.

\section{Entre Adam Smith e Kahneman} % Economia Comportamental vs. Escolha Racional

A teoria da escolha racional parte do pressuposto de que os indivíduos possuem preferências bem definidas, estáveis e informadas, sendo capazes de maximizar sua utilidade sob restrições orçamentárias. Predominante na economia neoclássica e na teoria política normativa, esse modelo assume decisões guiadas por cálculo objetivo e racional.

Contudo, a própria tradição clássica já apontava para uma racionalidade mais complexa. Em \textit{The Theory of Moral Sentiments}, Adam Smith reconhecia o papel das emoções, da empatia e do julgamento moral na ação humana \cite{smith1759-theory-of-moral-sentiments}. Para ele, o comportamento é guiado não apenas pelo interesse próprio, mas por um ``espectador imparcial'' interior, sensível ao senso de justiça.

Essa visão, longe de contradizer a ênfase no interesse próprio apresentada em \textit{The Wealth of Nations}, é parte de um sistema coerente, como argumenta \citeonline{fitzgibbons1995adam}. Segundo o autor, Smith articula uma teoria unificada da ação humana, em que a busca por interesse pessoal é moderada por normas morais internalizadas, derivadas da convivência social e da atuação do ``espectador imparcial''. Reduzir o agente smithiano ao egoísmo individual, como fazem algumas leituras contemporâneas, constitui, para Fitzgibbons, um erro interpretativo — ao ignorar a dimensão normativa essencial ao funcionamento da liberdade econômica. O resultado é um modelo de agente que não é puramente utilitarista, mas situado em um arcabouço ético e institucional — uma formulação que antecipa os debates atuais da economia comportamental ao reconhecer a centralidade das motivações morais e contextuais na tomada de decisão.

Ao longo do século XX, consolidou-se a figura do \textit{homo economicus}: autocentrado, maximizador e plenamente racional. Autores como \citeonline{becker1976} expandiram esse modelo para o comportamento social, criminal e político. Becker sustenta que desvios de conduta não indicam irracionalidade, mas sim diferentes restrições e preferências.

Em contraponto, a economia comportamental — com Daniel Kahneman e Amos Tversky — demonstrou empiricamente que indivíduos recorrem a heurísticas cognitivas para lidar com incertezas \cite{Judgment_under_Uncertainty}. Atalhos como representatividade, disponibilidade e ancoragem são eficientes do ponto de vista mental, mas produzem vieses sistemáticos, especialmente sob baixa responsabilização ou forte carga emocional \cite{kahneman2011thinking}.

Ainda no campo da economia política, Anthony Downs formulou a teoria da ignorância racional, segundo a qual os eleitores têm pouco incentivo para buscar informações políticas de qualidade, dado que o custo de adquirir conhecimento é elevado e a probabilidade de influenciar o resultado eleitoral é virtualmente nula \cite{downs1957economic}. Embora essa formulação mantenha a lógica da racionalidade instrumental — onde os indivíduos escolhem ser ignorantes de forma racional — o próprio Downs reconhece os limites desse modelo. Segundo o autor, a irracionalidade política é um fenômeno empírico que escapa à lógica dedutiva pura e requer investigação própria, o que o leva a não explorá-la diretamente em sua análise \cite[p.~10]{downs1957economic}. Ao fazer essa ressalva, Downs não propõe uma abordagem comportamental, mas admite que seu modelo não esgota o comportamento político real, abrindo espaço para interpretações futuras que incorporam fatores não racionais.

Essa divergência teórica tem implicações práticas. Enquanto o modelo racional favorece políticas baseadas em incentivos marginais, a abordagem comportamental exige considerar limites cognitivos e contextos institucionais. \citeonline{Hausman_McPherson_Satz_2016} argumentam que modelos de racionalidade também devem ser avaliados por seus efeitos éticos e distributivos.

Além disso, autores como \citeonline{hayek_knowledge_use} e \citeonline{positive_economics_friedman} destacam que o conhecimento é disperso e imperfeito. Para Hayek, o sistema de preços coordena ações melhor do que qualquer planejador central. Já Friedman sustenta que o valor de uma teoria está na qualidade de suas previsões, e não na veracidade de suas premissas.

Portanto, o contraste entre racionalidade plena e limitada não é apenas analítico, mas epistemológico. A escolha racional idealiza um agente isolado e infalível; a economia comportamental descreve um agente real: falho, emocional e influenciado por contexto — mais próximo do cidadão comum do que do \textit{decision-maker} normativo.

\section{Como os Vieses Moldeiam as Escolhas Políticas} % (Vieses Cognitivos e Política)

% Explicar os principais vieses cognitivos identificados na literatura e sua relação com decisões políticas (viés antimercado, antiestrangeiro, antitrabalho, pessimista, entre outros).

Ao analisar o comportamento do eleitor médio, Bryan Caplan propõe um deslocamento da hipótese da ignorância racional — centrada na ausência de informação — para a noção de \textit{irracionalidade racional}, na qual o eleitor sustenta ativamente crenças falsas, mesmo diante de evidências contrárias \cite{The_Myth_of_the_Rational_Voter}. Segundo o autor, os erros de julgamento político não ocorrem de forma aleatória ou esporádica, mas seguem padrões previsíveis, sistemáticos e estáveis ao longo do tempo. Tais distorções não derivam apenas da dificuldade de acesso à informação, mas sobretudo de motivações emocionais e ideológicas que incentivam os indivíduos a rejeitar proposições verdadeiras em favor de crenças confortáveis ou moralmente atraentes.

Caplan identifica quatro vieses cognitivos centrais que moldam negativamente o julgamento político dos eleitores: o viés antimercado, o viés antiestrangeiro, o viés antitrabalho e o viés pessimista \cite{Systematically_Biased_Beliefs_about_Economics}. Cada um desses vieses opera como uma lente interpretativa que distorce o entendimento sobre fenômenos econômicos complexos, gerando apoio a políticas públicas ineficientes ou contraproducentes.

O \textit{viés antimercado} refere-se à tendência de subestimar os benefícios dos mecanismos de mercado e superestimar os efeitos negativos do lucro, da concorrência e da desregulação. Essa visão, embora difundida, colide com fundamentos consolidados da teoria econômica. Autores como \citeonline{mankiw2020introducao} e \citeonline{sowell2000basic} enfatizam que o mercado, embora imperfeito, tende a alocar recursos de maneira mais eficiente do que alternativas centralizadas. A crítica liberal clássica, expressa já no século XIX por \citeonline{bastiat1859sofismas}, denunciava a miopia das políticas protecionistas e intervencionistas, sintetizada em sua célebre distinção entre ``o que se vê e o que não se vê''. O eleitor médio, no entanto, tende a interpretar lucros como exploração e concorrência como caos, ignorando os efeitos sistêmicos da coordenação via preços.

O \textit{viés antiestrangeiro} manifesta-se na desconfiança em relação ao comércio internacional, à imigração e à cooperação econômica global. O senso comum político frequentemente interpreta a importação de bens como ameaça à produção nacional e a presença de imigrantes como competição desleal no mercado de trabalho. Essa percepção, entretanto, ignora os ganhos de eficiência associados à especialização e à vantagem comparativa, conforme discutido por autores como \citeonline{bhagwati2003free} e \citeonline{landsburg2012armchair}. Historicamente, \citeonline{schumpeter1976capitalism} já advertia que o nacionalismo econômico tende a ressurgir em contextos de crise, alimentado por argumentos emocionais em detrimento da análise racional.

O \textit{viés antitrabalho} (ou \textit{make-work bias}, na formulação original) consiste na crença de que o objetivo central da economia deve ser preservar empregos, e não maximizar a produção ou a eficiência. Tal viés conduz ao apoio a políticas de proteção de setores ineficientes, de subsídios à manutenção artificial do emprego e de resistência à automação. Para \citeonline{sowell2004applied}, esse tipo de raciocínio falha ao ignorar que o progresso econômico consiste, justamente, em produzir mais com menos trabalho. Schumpeter, por sua vez, via a destruição criadora como essência do dinamismo capitalista — ideia que contraria frontalmente a aversão popular à substituição de empregos obsoletos por novas formas de produção.

O \textit{viés pessimista}, por fim, refere-se à percepção generalizada de que a economia está em constante deterioração, independentemente dos dados objetivos. Esse sentimento é frequentemente explorado por discursos populistas e reformistas que prometem ``salvar o país'' de uma crise iminente. Autores como \citeonline{Myths-of-Rich-and-Poor} e \citeonline{easterbrook2004progress} demonstram que, apesar de oscilações conjunturais, os indicadores de bem-estar material apresentaram melhora significativa nas últimas décadas. Ainda assim, a percepção negativa persiste, alimentada por vieses de confirmação e pela seletividade das narrativas midiáticas.

Esses vieses não apenas distorcem a compreensão dos fenômenos econômicos, mas também moldam as preferências políticas dos eleitores, influenciando diretamente a formulação de políticas públicas. Conforme sintetiza Caplan, ``na visão ingênua do interesse público, a democracia funciona porque faz o que os eleitores querem. Para a maioria dos céticos da democracia, ela falha porque não faz o que os eleitores querem. Na minha visão, a democracia falha justamente porque faz o que os eleitores querem'' \cite[p.~3]{The_Myth_of_the_Rational_Voter}. O custo dessa irracionalidade é amplamente socializado, o que desincentiva a revisão crítica de crenças e reforça a estabilidade dos erros cognitivos.

De modo geral, observa-se que o eleitor médio tende a interpretar os mecanismos de mercado com desconfiança, atribuindo-lhes motivações obscuras ou ilegítimas, ainda que sem fundamentação analítica ou respaldo empírico. Tal percepção, embora compreensível sob uma ótica emocional ou intuitiva, reflete padrões cognitivos enviesados que comprometem a avaliação racional de políticas econômicas. Compreender os vieses subjacentes a esse tipo de raciocínio é fundamental para diagnosticar as disfunções do processo democrático e para delimitar os alcances e limitações da deliberação pública em contextos marcados por racionalidade limitada.

\subsection{Vieses Auxiliares: Ideologia, Autoindulgência e o Público Esclarecido}
\label{sec:vieses_auxiliares}

Críticas à hipótese de que os eleitores mantêm crenças sistematicamente equivocadas muitas vezes apontam a suposta falta de neutralidade dos próprios economistas. Uma primeira objeção é que especialistas tenderiam a defender interesses materiais de sua própria classe, reproduzindo inconscientemente as estruturas ideológicas dominantes. Essa crítica parte da ideia de que instituições como a universidade operam como mecanismos de legitimação ideológica, moldando percepções que mantêm o status quo econômico e político \cite{althusser1971ideology, The_Myth_of_the_Rational_Voter}.

Outra objeção frequente alega que a formação econômica tradicional estaria imersa em valores liberais e pressupostos de mercado, comprometendo sua pretensa neutralidade científica. Argumenta-se que, longe de serem análises objetivas, os diagnósticos econômicos carregam elementos retóricos e culturais que favorecem determinadas visões de mundo \cite{mccloskey1998rhetoric, The_Myth_of_the_Rational_Voter}.

Ambas as críticas são relevantes e merecem ser enfrentadas. A melhor maneira de fazê-lo não é apelando à autoridade, mas confrontando as hipóteses com os dados \cite{popperlogic}. Uma estratégia metodológica eficaz consiste em simular um público que possua características socioeconômicas e ideológicas semelhantes às dos economistas, mas sem formação específica em economia. Se esse “público esclarecido” tender a responder como os economistas, isso sugere que a formação técnica exerce papel preponderante na forma como os indivíduos julgam questões econômicas — e não apenas sua renda, ideologia ou status.

A aplicação dessa abordagem mostra que, mesmo controlando para variáveis como renda, escolaridade e posicionamento político, a discordância entre economistas e o público geral persiste — mas desaparece quando se simula um público com o mesmo perfil, exceto pela ausência da formação econômica. Ou seja, as diferenças nas percepções não são explicadas apenas por ideologia ou conveniência, mas sim por conhecimento técnico sistematizado. O público em geral sustenta crenças incompatíveis com o consenso acadêmico não porque é moralmente inferior ou ideologicamente oposto, mas porque toma decisões sob a influência de distorções cognitivas previsíveis — e sem o instrumental técnico para perceber seus próprios erros.

Essa constatação reforça uma das hipóteses centrais deste trabalho: a irracionalidade política não é produto de má-fé ou de má-formação moral, mas de vieses sistemáticos que sobrevivem à escolarização geral e só começam a ser superados por um tipo específico de alfabetização econômica. O contraste entre o público comum, o público esclarecido e os especialistas permite, portanto, isolar o papel da formação econômica como variável de controle, oferecendo um argumento robusto contra explicações reducionistas que atribuem a discordância à ideologia ou ao elitismo.

\section{O Custo da Ignorância} % (Economia da Informação e Racionalidade Limitada)

% Discutir os impactos da desinformação e da racionalidade limitada no processo eleitoral e na formulação de políticas públicas.

A qualidade das decisões políticas em uma democracia depende, ao menos em parte, do nível de informação dos eleitores. No entanto, a literatura em economia política aponta que os cidadãos, em geral, demonstram baixo engajamento com a busca por conhecimento relevante sobre política e economia, o que compromete a racionalidade esperada de seu comportamento eleitoral. Esse fenômeno é abordado por \citeonline{downs1957economic} por meio do conceito de ignorância racional: dado que o custo de se informar adequadamente é alto e o impacto de um único voto é estatisticamente insignificante, os eleitores têm poucos incentivos para adquirir conhecimento político de qualidade.

A decisão de permanecer desinformado, nesse contexto, não é um erro de julgamento, mas uma escolha racional diante de incentivos inadequados. Essa formulação, embora coerente com a lógica econômica tradicional, revela limitações quando confrontada com evidências empíricas mais recentes. Estudos como os de \citeonline{The_Myth_of_the_Rational_Voter} indicam que os eleitores não apenas ignoram informações relevantes, mas também sustentam ativamente crenças falsas, mesmo quando confrontados com dados em contrário. A irracionalidade, nesse caso, não é um subproduto da ignorância passiva, mas um comportamento ativo de resistência à revisão de crenças.

Além disso, como argumenta \citeonline{hayek_knowledge_use}, a informação necessária para tomar decisões políticas eficazes é dispersa, tácita e situada nos agentes que vivenciam realidades econômicas específicas. A tentativa de centralizar tal conhecimento por meio de instituições políticas enfrenta limites intransponíveis, o que compromete a eficácia de políticas públicas baseadas em modelos excessivamente agregados ou descontextualizados. A ordem espontânea gerada pelo mercado — guiada por sinais de preços que condensam informação — é, nesse sentido, superior à coordenação política em muitos aspectos, justamente porque respeita a dispersão irredutível do conhecimento.

\citeonline{von1949human} enfatiza que, embora a racionalidade absoluta seja inatingível, ela deve ser preservada como um ideal regulativo. A ciência econômica, nesse sentido, deve se orientar pela busca contínua de maior coerência lógica e empírica, mesmo reconhecendo os limites cognitivos dos agentes. O problema central, portanto, não reside apenas na ausência de informação, mas na estrutura institucional que desincentiva a busca por verdade e promove a indulgência em crenças infundadas. Conforme argumenta Caplan, o problema da democracia não é apenas ignorância; para ele, ``os eleitores são piores que ignorantes: são, em uma palavra, irracionais — e votam de acordo'' \cite[p.~2]{The_Myth_of_the_Rational_Voter}.

O custo dessa ignorância deliberada transcende o plano individual e se manifesta em escolhas coletivas disfuncionais. Políticas ineficazes, protecionistas ou fiscalmente irresponsáveis são frequentemente resultado de crenças equivocadas, amplificadas por incentivos políticos de curto prazo. Quando o custo do erro é difuso e suas consequências são socializadas, torna-se racional para o indivíduo permanecer irracional.

Em última instância, o problema não é que o eleitor esteja mal informado; é que ele está racionalmente mal informado. Esse paradoxo, situado na interseção entre psicologia, teoria econômica e ciência política, desafia os modelos normativos de democracia e impõe limites práticos à qualidade das decisões públicas em contextos de racionalidade limitada.

\section{Do Sofista ao Populista} %  Como as Ideias Econômicas se Espalham (História do Pensamento Econômico e Propagação de Crenças)

% Explorar como ideias econômicas se propagam, influenciadas por interesses políticos, mídia e cultura.

A persistência de ideias econômicas equivocadas, mesmo diante de abundantes evidências contrárias, exige uma análise que vá além dos modelos tradicionais de aprendizagem racional. Muitos equívocos não decorrem apenas de desconhecimento, mas da forma como certas crenças são formadas, disseminadas e preservadas. A história do pensamento econômico ajuda a entender como erros conceituais ganham legitimidade social e fundamentam políticas públicas baseadas em argumentos intuitivos, morais ou retóricos — ainda que sem sustentação teórica rigorosa.

Esta seção propõe uma analogia entre formas históricas de persuasão política. O título ``Do Sofista ao Populista'' não deve ser lido como juízo moral, mas como recurso heurístico para ilustrar como argumentos emocionalmente sedutores, mas logicamente frágeis, ganham força no debate público. Trata-se de uma tradição crítica que busca evidenciar padrões recorrentes de difusão de crenças econômicas infundadas.

No século XIX, Bastiat já alertava para a ``miopia econômica'' induzida pela retórica política, distinguindo ``o que se vê e o que não se vê'' \cite{bastiat1859sofismas}. Efeitos imediatos de políticas como o protecionismo são politicamente valorizados, enquanto seus custos difusos e de longo prazo são ignorados ou ocultados. Para Bastiat, esse tipo de raciocínio constitui um sofisma: aparentemente plausível, mas embutido em falácias.

Autores como \citeonline{mark_history} e \citeonline{hart2019bastiat} defendem o uso da história das ideias como instrumento crítico. Para Blaug, ao abandonar essa dimensão, a economia perde sua capacidade autocrítica e se torna vulnerável à repetição de erros superados. A análise histórica, portanto, é uma ferramenta para entender como ideias se transformam em crenças e como estas moldam políticas.

Essa transformação é mediada por interesses políticos, cultura e estruturas de poder. \citeonline{schumpeter1976capitalism} observou que o dinamismo do capitalismo — fundado na inovação e substituição constante — gera ressentimento social e apelos nostálgicos por estabilidade. Esse sentimento alimenta discursos populistas que rejeitam os fundamentos da concorrência e da globalização, prometendo uma volta ilusória ao passado.

Nesse sentido, o populismo contemporâneo pode ser visto como uma reedição moderna do sofismo: ambos manipulam a linguagem, simplificam problemas complexos e oferecem soluções intuitivas, porém equivocadas. Se antes o sofista usava a retórica nas ágoras, hoje o populista usa algoritmos e redes sociais. Ambos priorizam apelo simbólico em detrimento da lógica e da evidência.

Mecanismos culturais e midiáticos também reforçam certas narrativas e marginalizam discursos dissonantes. Como nota \citeonline{franco2022cartas}, o economista que busca comunicar ideias impopulares enfrenta não apenas resistência cognitiva, mas também um ambiente institucional hostil à complexidade. O senso comum, moldado por intuição, moralismo e ideologia, opõe-se à crítica técnica, ainda que esta seja sólida.

Robert Higgs contribui com a ideia de ``consentimento ideológico tácito'' \cite{higgs1987crisis}, explicando como crenças populares, mesmo ineficientes, se transformam em políticas institucionais. Compreender esse processo requer atenção às estruturas culturais e institucionais que moldam a relação entre crença e política.

Assim, o estudo da difusão de ideias econômicas exige uma abordagem histórica, cultural e institucional. A racionalidade do eleitor — limitada por vieses e desinformação — encontra na linguagem política e na cultura midiática um terreno fértil para a reprodução de erros. A democracia, nesse quadro, não apenas reflete crenças populares, mas as amplifica — convertendo intuições falaciosas em consenso legislativo.


\section{Preferência por Crenças e Resistência ao Conhecimento}

A manutenção de crenças econômicas disfuncionais, mesmo frente a evidências disponíveis, não pode ser explicada apenas por ignorância racional \cite{downs1957economic}. Diversos estudos sugerem que indivíduos mantêm crenças políticas com base em fatores emocionais, identitários e sociais, distantes de critérios epistêmicos. Trata-se de uma resistência ativa ao conhecimento — racional no curto prazo individual, mas irracional no coletivo.

Caplan argumenta que crenças políticas funcionam como bens de consumo simbólico: geram utilidade subjetiva independentemente de sua veracidade \cite{The_Myth_of_the_Rational_Voter}. Como o custo da crença é socializado e os benefícios emocionais são internos, os incentivos favorecem ideias confortáveis, ainda que falsas.

Kahneman e Tversky descrevem vieses como o da confirmação e ancoragem, que dificultam a revisão de crenças mesmo diante de novas evidências \cite{kahneman2011thinking}. Some-se a isso o \textit{backfire effect}, descrito por Nyhan e Reifler, segundo o qual argumentos contrários reforçam a crença original em contextos polarizados \cite{nyhan2010when}.

A teoria da cognição cultural mostra que indivíduos tendem a filtrar informações de acordo com os valores predominantes em seus grupos sociais, priorizando a coesão identitária em detrimento da precisão factual \cite{kahan2012polarization}. Esse padrão se intensifica em contextos polarizados, nos quais os cidadãos selecionam narrativas que reforçam suas convicções prévias, mesmo quando estas entram em conflito com dados objetivos. A estrutura informacional contemporânea, marcada por algoritmos e bolhas epistêmicas, favorece o isolamento cognitivo e a evasão da dissonância, dificultando o confronto com perspectivas divergentes \cite{sunstein2017republic}.

Esse tipo de resistência ao conhecimento não é exclusividade da era digital. Pensadores clássicos já advertiam contra a rigidez das crenças não fundamentadas. Locke, por exemplo, associava o excesso de certeza a posicionamentos motivados mais por afeto do que por evidência racional \cite{locke2014ensaio}. De forma mais incisiva, a filosofia objetivista descreve o \textit{blanking out} como a recusa deliberada de reconhecer aquilo que contradiz uma crença internalizada — “não ser cego, mas recusar-se a ver” \cite[p.~869]{rand2012revolta}. Essa cegueira voluntária opera como um escudo cognitivo que protege o indivíduo do desconforto gerado pela inconsistência entre suas crenças e a realidade.

Além dos fatores individuais, há elementos institucionais. Algoritmos de redes sociais e tribalismo político criam bolhas epistêmicas que dificultam o contato com visões divergentes. Zaller aponta que a recepção de argumentos depende da disposição prévia do eleitor, tornando o debate público mais reforçador do que crítico \cite{zaller1992nature}.

Em síntese, a resistência ao conhecimento é racional no nível individual, mas prejudica a racionalidade coletiva. Isso limita a eficácia das políticas baseadas em evidência, pois o eleitor médio age não como cientista, mas como consumidor de narrativas. Para que suas crenças políticas sejam tratadas como hipóteses legítimas, elas deveriam estar sujeitas à refutação — o que raramente ocorre no jogo político.

\chapter{Como Medimos a Irracionalidade Política}

Este capítulo apresenta o desenho metodológico da pesquisa, articulando os procedimentos de coleta e análise de dados utilizados para investigar a presença e a magnitude de vieses cognitivos nas percepções econômicas dos eleitores brasileiros. O objetivo é testar, de forma rigorosa, a hipótese de que tais vieses são sistemáticos e afetam diretamente o julgamento político em contextos democráticos.

\section{Desenho da Pesquisa}

A estratégia metodológica adotada combina três eixos principais:
\begin{enumerate}[label=\alph*)]
    \item análise histórica da persistência de percepções econômicas enviesadas, com base na história do pensamento econômico;
    \item adaptação da \textit{Survey of Americans and Economists on the Economy} (SAEE) ao contexto institucional e cultural brasileiro;
    \item aplicação de modelos econométricos (\textit{logit} binário e ordenado) para estimar, sob controle estatístico, o impacto de fatores individuais sobre a propensão a adotar crenças divergentes do consenso técnico.
\end{enumerate}

\section{Coleta e Amostragem}

A amostragem foi do tipo não probabilística, com estratificação em dois grupos principais:
\begin{enumerate}[label=\alph*)]
    \item \textbf{Grupo de controle}: cidadãos sem formação formal em Economia;
    \item \textbf{Grupo de tratamento}: economistas ou estudantes da área econômica.
\end{enumerate}

A coleta de dados ocorreu de forma online, empregando três estratégias complementares: a técnica de bola de neve (\textit{snowball sampling}), parcerias institucionais com conselhos profissionais e divulgação aberta em redes sociais e acadêmicas. A meta amostral mínima foi definida por meio da fórmula de Cochran, com um mínimo de 175 respondentes e alvo superior de 600 no grupo controle, assegurando poder estatístico para as comparações.

\section{Instrumento de Pesquisa}

O questionário aplicado divide-se em duas seções principais:
\begin{itemize}
    \item \textbf{Seção A}: coleta de variáveis demográficas, socioeconômicas e ideológicas;
    \item \textbf{Seção B}: 36 afirmações relacionadas a temas econômicos relevantes (tributação, comércio, tecnologia, previdência, intervenção estatal, entre outros), adaptadas da SAEE ao contexto brasileiro. As respostas foram registradas em escalas do tipo Likert, variando de acordo com a complexidade e nuance de cada questão.
\end{itemize}

\section{Modelo Estatístico}

Utiliza-se regressão \textit{logit} (binária ou ordenada) para modelar a probabilidade de adesão a uma crença tecnicamente correta. A escolha entre as versões depende da natureza da variável dependente: binária para respostas dicotômicas (ex.: concorda vs. discorda) e ordenada para escalas com múltiplos níveis. A formulação geral dos modelos é dada por:

\begin{equation}
P(y_i = 1 \mid X_i) = \frac{e^{X_i \beta}}{1 + e^{X_i \beta}}
\end{equation}

\begin{equation}
P(y_i \leq j \mid X_i) = \frac{1}{1 + e^{-(\tau_j - X_i \beta)}}
\end{equation}

As variáveis explicativas \(X_i\) incluem escolaridade, renda, ideologia, raça, gênero, ocupação, expectativas econômicas e formação em Economia, conforme detalhado na Tabela de Codificação (ver Anexo~\ref{anexo:A}). Os modelos foram ajustados por meio do algoritmo L-BFGS, adequado para estimação de modelos com múltiplos parâmetros.

\section{O Conceito de Público Esclarecido}

Conforme fundamentado na Seção~\ref{sec:vieses_auxiliares}, define-se como \textit{público esclarecido} uma simulação contrafactual baseada nos modelos logit estimados para cada questão do questionário. A ideia central consiste em prever, para cada economista da amostra, qual seria sua resposta esperada caso não possuísse formação em Economia, mantendo-se constantes todas as demais características individuais.

Na prática, os modelos incluem a variável indicadora de formação econômica como uma das regressoras. Após a estimação, calcula-se a probabilidade predita de cada resposta alternativa sob o cenário contrafactual em que todos os indivíduos têm o mesmo nível de conhecimento técnico, sem alterar renda, ideologia, escolaridade ou outras covariáveis. O valor esperado dessas respostas define, para cada item, a média do público esclarecido.

Essa estratégia permite isolar o efeito marginal da formação econômica sobre a percepção dos respondentes, controlando para fatores sociais, ideológicos e cognitivos. Ao comparar as médias preditas entre o público geral, os economistas e o público esclarecido, torna-se possível avaliar empiricamente se o conhecimento técnico efetivamente corrige os vieses cognitivos previamente identificados.

\section{Etapas Analíticas}

A análise empírica foi conduzida em cinco etapas principais:
\begin{enumerate}
    \item Ajuste de modelos logit ordenados para cada variável dependente do questionário;
    \item Cálculo da probabilidade predita de resposta correta para três grupos: público geral, economistas e público esclarecido (simulado);
    \item Comparação das médias entre os grupos e geração de gráficos explicativos por variável;
    \item Estimativas adicionais por tipo de viés, agrupando as variáveis em blocos conceituais (ex.: antimercado, antiestrangeiro, etc.);
    \item Testes de robustez com especificações alternativas e controle para outliers.
\end{enumerate}

\section{Critério de Refutabilidade}

Este estudo adota o princípio epistemológico da falseabilidade como critério de cientificidade, conforme proposto por \citeonline{popperlogic}. Em vez de buscar confirmações para as hipóteses formuladas, parte-se do pressuposto de que toda proposição científica deve ser logicamente testável e empiricamente refutável. As hipóteses centrais serão submetidas a testes estatísticos com base em modelos logit, e consideradas refutadas caso os dados empíricos revelem padrões incompatíveis com suas previsões.

Mais especificamente, as hipóteses serão rejeitadas nos seguintes cenários:

\begin{itemize}
    \item Caso a formação em Economia não apresente impacto estatisticamente significativo sobre a incidência de vieses nas respostas, refuta-se a hipótese H1;
    \item Caso não se observem diferenças sistemáticas entre as respostas do público geral e do público esclarecido simulado, refuta-se H2;
    \item Caso variáveis estruturais como ideologia política, escolaridade ou renda não se associem de forma estatisticamente significativa à ocorrência de vieses cognitivos, refuta-se H7;
    \item Caso as respostas do grupo controle (não economistas) não revelem padrões sistemáticos de viés (antimercado, antiestrangeiro, antitrabalho ou pessimista), refutam-se as hipóteses H3 a H6.
\end{itemize}

Rejeita-se, portanto, qualquer pretensão de verificação definitiva: mesmo que uma hipótese não seja refutada, ela será considerada apenas corroborada de forma provisória pelos dados disponíveis. Os resultados serão interpretados com base na severidade dos testes empíricos enfrentados, e não na frequência de confirmações observadas, em consonância com a lógica da ciência proposta pelo autor.


\section{Contribuições da Metodologia}

A abordagem aqui descrita permite:
\begin{itemize}
    \item Medir objetivamente a distância entre opinião popular e consenso técnico;
    \item Identificar os determinantes estruturais dos vieses cognitivos;
    \item Estimar simulações de política pública com base em maior esclarecimento técnico;
    \item Posicionar o estudo na fronteira da \textit{Economia Política Comportamental}, com diálogo entre econometria, história do pensamento econômico, psicologia cognitiva e teoria política.
\end{itemize}