

\chapter{Introdução} % O Labirinto das Decisões: Como Julgamos e Escolhemos?
% Adicionar aqui uma parte principal sobre qual a contribuição principal do tcc, que é sobre o desenvolvimento do campo da "economia politica comportamental", que é nova e ainda não foi amplamente explorada. 
% A ideia é que a economia comportamental, que já é um campo estabelecido, pode ser aplicada à política, e que isso pode ser uma nova forma de entender a política, que é mais realista e menos idealista do que a economia política tradicional.

%Aqui vamos apresentar o problema de pesquisa, sua relevância e contribuição para o campo da economia política comportamental.

A democracia moderna parte do pressuposto de que os eleitores são agentes racionais, capazes de avaliar as consequências econômicas de suas escolhas políticas e apoiar medidas que maximizem o bem-estar social \cite{downs1957economic}. No entanto, a realidade demonstra que esse ideal frequentemente se desvia devido à influência de vieses cognitivos e heurísticas que distorcem a percepção econômica da população \cite{The_Myth_of_the_Rational_Voter,kahneman2011thinking}. Como resultado, políticas públicas são frequentemente moldadas por crenças equivocadas, levando a decisões sub-ótimas que comprometem o desenvolvimento econômico e social \cite{acemoglu2019narrow}.

O fenômeno central deste estudo é o impacto da irracionalidade eleitoral sobre a formulação de políticas econômicas. Eleitores bem-intencionados, mas cognitivamente limitados, acabam apoiando medidas que contradizem princípios econômicos fundamentais, como a vantagem comparativa, a eficiência dos mercados e os benefícios da inovação. A pesquisa de Caplan (\citeyear{Systematically_Biased_Beliefs_about_Economics,The_Myth_of_the_Rational_Voter}) demonstra que o público frequentemente rejeita consensos econômicos básicos, sustentando crenças que favorecem protecionismo, intervencionismo excessivo e desconfiança do setor produtivo. Essas distorções, denominadas por \citeauthoronline{The_Myth_of_the_Rational_Voter} (\citeyear{The_Myth_of_the_Rational_Voter}) como os vieses antimercado, antiestrangeiro, antitrabalho e pessimista, levam à implementação de políticas que reduzem o crescimento e a prosperidade.

A dificuldade de corrigir essas distorções não está apenas na falta de informação, mas também na resistência psicológica dos eleitores em abandonar crenças que reforçam sua visão de mundo. Como \citeauthoronline{hayek_knowledge_use} (\citeyear{hayek_knowledge_use}) já alertava, a dispersão do conhecimento e a complexidade econômica criam barreiras para que a população compreenda os efeitos reais das políticas públicas. \citeauthoronline{downs1957economic} (\citeyear{downs1957economic}) complementa essa visão ao sugerir que, para a maioria dos eleitores, o custo de se informar sobre economia supera os benefícios individuais, levando a uma ignorância racional que perpetua escolhas equivocadas.

Essa discrepância entre conhecimento econômico e opinião pública tem implicações diretas para a qualidade das políticas adotadas. Como demonstrado por evidências empíricas, como a \textit{Survey of Americans and Economists on the Economy} (SAEE) \cite{saee1996}, há uma diferença sistemática entre a visão dos especialistas em economia e a do público geral, evidenciando que a percepção popular frequentemente se afasta da realidade econômica.

Para analisar empiricamente esse fenômeno no Brasil, este estudo recria a metodologia da SAEE, originalmente aplicada nos Estados Unidos, adaptando-a ao contexto brasileiro. A pesquisa coletará dados primários sobre as crenças econômicas dos eleitores brasileiros e os comparará com os dados da SAEE americana. O objetivo é investigar se os vieses observados nos EUA também estão presentes no Brasil, identificar possíveis divergências e avaliar fatores institucionais e culturais que possam influenciar essas percepções. Essa comparação se justifica porque ambos os países possuem democracias consolidadas, mas apresentam diferenças significativas em termos de escolaridade média, acesso à informação econômica e nível de desconfiança nas instituições. Enquanto os Estados Unidos possuem um longo histórico de pesquisas sobre a percepção pública da economia e sua relação com as políticas públicas, o Brasil ainda carece de estudos que explorem sistematicamente como o eleitorado interpreta questões econômicas e como isso se reflete no cenário político. Comparar esses dois contextos permite entender se os vieses do eleitorado são fenômenos universais ou se há particularidades ligadas ao ambiente institucional e ao desenvolvimento econômico.

No entanto, algumas limitações devem ser reconhecidas. Apesar da adaptação da metodologia da SAEE ao contexto brasileiro, diferenças institucionais e culturais entre os dois países podem afetar a comparabilidade dos resultados. Além disso, a pesquisa se concentra na percepção econômica dos eleitores, não abrangendo outros fatores que também influenciam a formulação de políticas públicas, como o papel da mídia, o impacto de campanhas eleitorais e a disseminação de desinformação.

É relevante destacar que esta pesquisa não pretende fornecer uma solução definitiva para o problema da irracionalidade eleitoral, nem testar empiricamente as estratégias de mitigação sugeridas. O foco está na análise dos vieses cognitivos e suas consequências para a formulação de políticas, buscando oferecer um panorama teórico e empírico sobre o tema. Questões mais amplas relacionadas à desinformação deliberada, ao papel da mídia e a outros fatores externos não serão o foco central deste estudo, ainda que possam ser tangencialmente mencionadas.

Embora a literatura internacional tenha avançado significativamente na análise dos vieses cognitivos na política, esse debate ainda é incipiente no Brasil. Há poucos estudos que exploram de forma sistemática como a percepção econômica dos eleitores brasileiros se distancia dos consensos acadêmicos e como isso afeta a formulação de políticas públicas. Dado o impacto de decisões econômicas equivocadas sobre o desenvolvimento do país — incluindo políticas protecionistas ineficientes, subsídios distorcidos e resistência a reformas estruturais —, compreender esses vieses se torna essencial para o aprimoramento do panorama político e econômico nacional.

Diante desse cenário, o presente estudo busca não apenas identificar os vieses econômicos presentes no eleitorado brasileiro, mas também compreender em que medida eles diferem dos padrões observados nos Estados Unidos. Ao trazer evidências empíricas comparativas, este trabalho visa contribuir para o debate sobre a irracionalidade eleitoral e seus efeitos sobre a formulação de políticas públicas.


\section{A Racionalidade Coletiva} %(Tema e Problema de Pesquisa)

% Introduzir o tema do trabalho e a questão central: os vieses de julgamento dos eleitores e suas implicações políticas e econômicas.

O ideal democrático pressupõe que a soma das decisões individuais resulta em escolhas coletivas racionais e benéficas para a sociedade. No entanto, a realidade política revela um paradoxo: mesmo quando indivíduos tomam decisões racionais em suas vidas privadas, frequentemente apoiam políticas públicas que contrariam princípios econômicos básicos e resultam em prejuízos para o bem-estar geral \cite{downs1957economic,The_Myth_of_the_Rational_Voter}.

Esse fenômeno pode ser explicado pela diferença entre racionalidade individual e racionalidade coletiva. Enquanto o indivíduo tem incentivos diretos para tomar boas decisões no mercado — escolhendo bens e serviços que maximizem seu benefício pessoal —, no ambiente eleitoral, esse incentivo se dilui. O custo de um voto mal embasado é praticamente inexistente para o eleitor, pois seu impacto sobre o resultado final é insignificante. Assim, ao contrário do que ocorre no mercado, onde decisões ruins trazem consequências imediatas, no processo eleitoral não há penalização direta para escolhas irracionais, o que favorece a persistência de crenças equivocadas \cite{bastiat1859sofismas,downs1957economic,The_Myth_of_the_Rational_Voter}.

Além disso, a formação das preferências políticas é influenciada por vieses cognitivos que distorcem a percepção da realidade econômica. A pesquisa de \citeauthoronline{The_Myth_of_the_Rational_Voter} (\citeyear{The_Myth_of_the_Rational_Voter}) identifica quatro vieses principais que afetam a racionalidade coletiva dos eleitores: antimercado, antiestrangeiro, antitrabalho e pessimista. Esses vieses levam os indivíduos a subestimar os benefícios do livre mercado, superestimar os impactos negativos da globalização, rejeitar avanços tecnológicos e interpretar o progresso econômico de maneira excessivamente negativa. Como resultado, políticas populistas e protecionistas encontram amplo apoio, apesar de seus efeitos prejudiciais à economia.

A questão central que este estudo busca responder é: por que a soma das decisões individuais frequentemente resulta em escolhas coletivas que prejudicam o desenvolvimento econômico? Para isso, será necessário examinar as interações entre cognição, incentivos políticos e percepção econômica, além de explorar como a disseminação de conhecimento econômico pode atuar como um mecanismo de mitigação desses vieses \cite{positive_economics_friedman,Judgment_under_Uncertainty,kahneman2011thinking}.

O entendimento desse problema é fundamental para aprimorar o real entendimento da democracia e reduzir o impacto de crenças distorcidas na formulação de políticas públicas eficientes. Ao identificar os fatores que levam eleitores bem-intencionados a apoiar políticas ineficazes, abre-se espaço para o desenvolvimento de estratégias que tornem o processo político mais alinhado com princípios econômicos sólidos.

\section{O Que Precisamos Descobrir} % Hipóteses

Aqui serão formuladas as hipóteses da pesquisa, como a persistência de crenças enviesadas mesmo diante de informações contraditórias. Essa formulação vai ficar para a próxima versão do texto.

\section{Os Filtros da Percepção Econômica} % Objetivos

A forma como os eleitores percebem a economia não é neutra, mas filtrada por vieses cognitivos que distorcem a realidade econômica e influenciam suas decisões políticas. Esses filtros resultam de fatores psicológicos, históricos e institucionais que tornam certos conceitos econômicos contraintuitivos ou difíceis de aceitar. Assim, mesmo quando há evidências e consenso entre economistas sobre determinadas políticas, o eleitorado pode rejeitá-las, preferindo narrativas mais alinhadas às suas intuições e crenças prévias.

Este trabalho busca compreender como esses filtros moldam a percepção econômica dos eleitores e de que maneira influenciam a formulação de políticas públicas. A pesquisa de \citeauthoronline{The_Myth_of_the_Rational_Voter} (\citeyear{The_Myth_of_the_Rational_Voter}) identificou quatro vieses principais que afetam a visão econômica popular:

\begin{itemize}
    \item \textbf{Viés antimercado}: Tendência a subestimar os benefícios do livre mercado e supervalorizar o papel do Estado na economia.
    \item \textbf{Viés antiestrangeiro}: Rejeição intuitiva à globalização e ao comércio internacional, levando ao apoio a políticas protecionistas.
    \item \textbf{Viés antitrabalho}: Crença de que a geração de empregos é um objetivo em si, ignorando que a inovação e a produtividade são fundamentais para o crescimento econômico.
    \item \textbf{Viés pessimista}: Visão exageradamente negativa da economia, com tendência a superestimar crises e subestimar progressos.
\end{itemize}

A dificuldade de superar esses filtros não se deve apenas à falta de informação econômica, mas também a fatores estruturais. A teoria da ignorância racional de \citeauthoronline{downs1957economic} (\citeyear{downs1957economic}) sugere que os eleitores têm pouco incentivo para buscar informações complexas sobre economia, pois o custo de se informar é maior do que o impacto individual do voto. Além disso, \citeauthoronline{hayek_knowledge_use} (\citeyear{hayek_knowledge_use}) argumenta que a dispersão do conhecimento na sociedade torna difícil para os cidadãos compreenderem os efeitos reais das políticas econômicas.

Diante desse contexto, os objetivos deste estudo são:

\subsection{Objetivo Geral} % Objetivo Geral

Investigar como os vieses cognitivos moldam a percepção econômica dos eleitores e contribuem para a formulação de políticas públicas ineficazes.

\subsection{Objetivos Específicos} % Objetivos Específicos

\begin{itemize}
    \item Identificar os principais vieses econômicos presentes no eleitorado e suas origens psicológicas e históricas.
    \item Analisar o impacto desses vieses na formulação de políticas públicas e no funcionamento da democracia.
    \item Comparar a percepção econômica da população com a dos especialistas, utilizando pesquisas como a \textit{Survey of Americans and Economists on the Economy} (SAEE) e sua replicação no Brasil.
    \item Avaliar possíveis estratégias para mitigar os efeitos desses vieses, como educação econômica e reformas institucionais.
\end{itemize}

Com isso, este estudo pretende contribuir para o campo da economia política comportamental, oferecendo uma análise mais realista e objetiva sobre como os eleitores formam suas crenças econômicas e como isso afeta o processo democrático.

\section{Por que a Realidade Econômica É Distorcida pelo Eleitor?} %(Justificativa e Relevância do Estudo)
% deve enfatizar que o problema não é ignorância aleatória, mas crenças sistematicamente erradas, o que gera políticas ruins mesmo quando os eleitores estão bem informados.

A democracia pressupõe que os eleitores escolham representantes e políticas que maximizem o bem-estar coletivo. No entanto, a realidade política demonstra que decisões econômicas frequentemente são guiadas por crenças sistematicamente erradas, e não apenas por ignorância ou falta de informação. O problema central não é que os eleitores desconhecem a economia, mas que possuem percepções distorcidas sobre ela, sustentadas por vieses cognitivos que os levam a apoiar políticas públicas ineficazes e, muitas vezes, prejudiciais ao crescimento econômico e ao desenvolvimento social.

A literatura sobre economia política comportamental mostra que esses vieses não são distribuídos aleatoriamente, mas seguem padrões previsíveis. \citeauthoronline{The_Myth_of_the_Rational_Voter} (\citeyear{The_Myth_of_the_Rational_Voter}) argumenta que os eleitores mantêm crenças enviesadas sobre temas econômicos fundamentais, como comércio internacional, concorrência de mercado e avanços tecnológicos, favorecendo protecionismo, intervencionismo excessivo e políticas que desestimulam a inovação e o crescimento. Esses equívocos não são simples erros individuais, mas sim falhas sistemáticas que emergem do próprio funcionamento da democracia: como o custo de estar errado é difuso e diluído entre milhões de eleitores, não há incentivos para revisar crenças erradas \cite{downs1957economic}.

Além disso, a teoria do conhecimento disperso de \citeauthoronline{hayek_knowledge_use} (\citeyear{hayek_knowledge_use}) aponta que a economia é uma ciência contraintuitiva, onde efeitos de segunda ordem frequentemente contradizem percepções imediatas. A crença popular de que “baixar preços reduz salários” ou que “importações eliminam empregos” ignora mecanismos de compensação que os economistas compreendem, mas que são difíceis de comunicar ao público geral. Essas distorções cognitivas são reforçadas por incentivos políticos: políticos têm mais sucesso ao atender preferências populares do que ao corrigir equívocos econômicos. Como resultado, políticas públicas são moldadas mais por crenças populares do que por análises racionais dos seus efeitos reais.

A relevância deste estudo está em compreender esse desalinhamento entre conhecimento econômico e opinião pública, explorando suas consequências para a formulação de políticas públicas e para o funcionamento da democracia. Ao analisar os vieses cognitivos e seus impactos, este trabalho contribui para o campo emergente da economia política comportamental, propondo estratégias para mitigar os efeitos dessas distorções – seja por meio da educação econômica, seja pelo aprimoramento de incentivos institucionais.

Sem um entendimento claro das causas dessa dissonância entre economia e democracia, continuaremos a repetir ciclos de políticas ineficazes e ineficiências econômicas, dificultando o desenvolvimento sustentável. O desafio, portanto, não é apenas combater a ignorância, mas encontrar formas de corrigir crenças sistematicamente equivocadas que influenciam as escolhas políticas e determinam o rumo das sociedades.

\chapter{Teorias e Evidências Sobre a (I)Racionalidade Humana} % Revisão de Literatura

% Revisar a literatura sobre economia comportamental e sua aplicação à política.

\section{Entre Adam Smith e Kahneman} % Economia Comportamental vs. Escolha Racional 

% Contrastar a visão da escolha racional com a abordagem da economia comportamental.

\section{Como os Vieses Moldeiam as Escolhas Políticas} % (Vieses Cognitivos e Política)

% Explicar os principais vieses cognitivos identificados na literatura e sua relação com decisões políticas (viés antimercado, antiestrangeiro, antitrabalho, pessimista, entre outros).

\section{O Custo da Ignorância} % (Economia da Informação e Racionalidade Limitada)

% Discutir os impactos da desinformação e da racionalidade limitada no processo eleitoral e na formulação de políticas públicas.

\section{Do Sofista ao Populista} %  Como as Ideias Econômicas se Espalham (História do Pensamento Econômico e Propagação de Crenças)

% Explorar como ideias econômicas se propagam, influenciadas por interesses políticos, mídia e cultura.

\section{Preferência por Crenças e Resistência ao Conhecimento}

% Argumentar que muitas crenças são mantidas por fatores emocionais e sociais, e não por evidências racionais.
