
\chapter{Entrega do Dia 29 de Março - Referência Bibliográfica}

Instruções:

Seguindo o normatizado no Manual de Trabalhos Acadêmicos da UDESC (BU), elaborar a referência dos diferentes tipos de bibliografias:

\begin{itemize}
    \item Um Livro;
    \item Dois Trabalhos de Conclusão de Curso ou duas Teses ou duas Dissertações;
    \item Dois artigos em periódicos científicos.
\end{itemize}

\section{Livros}

Como principal referência teórica deste trabalho, utilizamos o livro de \citeonline{The_Myth_of_the_Rational_Voter}, publicado em \citeyear{The_Myth_of_the_Rational_Voter}. Nessa obra, Bryan Caplan oferece uma crítica profunda à concepção tradicional do eleitor racional. O autor argumenta que os eleitores não apenas carecem de informação qualificada, mas também sofrem de vieses sistemáticos em relação à economia e à política. Essas crenças equivocadas não são aleatórias, mas seguem padrões previsíveis que acabam distorcendo os resultados democráticos. A obra é fundamental para compreender como fatores cognitivos e comportamentais comprometem o funcionamento das instituições representativas e a formulação de políticas públicas, especialmente em sociedades democráticas de massa.

\section{Teses e Dissertações}

Entre os trabalhos acadêmicos, destaca-se a tese de doutorado de \citeonline{souza2022erosao}, intitulada A Erosão da Democracia na Sociedade Informacional: Diretrizes Regulatórias das Tecnologias da Informação e Comunicação como Mecanismo de Fortalecimento Democrático. Defendida na Universidade Estadual do Norte do Paraná (UENP) em \citeyear{souza2022erosao}, a pesquisa investiga o impacto das Tecnologias da Informação e Comunicação (TIC) sobre a qualidade democrática, abordando os desafios da desinformação e da formação das preferências eleitorais. Embora não trate diretamente da irracionalidade do eleitor, o trabalho se aproxima das preocupações de Caplan ao discutir como o ambiente informacional influencia negativamente as decisões políticas dos cidadãos.

Outra contribuição importante é a dissertação de \citeonline{garcia2007democracia}, defendida na Universidade de Brasília. O autor propõe uma análise crítica da racionalidade democrática a partir da interação entre mercado, comportamento político e a teoria da escolha pública. Ao dialogar com autores clássicos dessa escola, a dissertação oferece uma leitura instigante dos limites da democracia quando esta é confrontada com imperfeições cognitivas e incentivos políticos desalinhados. Trata-se de uma abordagem complementar à de Caplan, ao reforçar os riscos de decisões coletivas baseadas em percepções distorcidas da realidade econômica.

\section{Artigos em Periódicos Científicos}

No campo dos periódicos científicos, destaca-se o artigo de \citeonline{kitschelt2014occupations}, publicado na \textit{Comparative Political Studies}. Nele, os autores analisam como diferentes ocupações influenciam a formação de preferências políticas, demonstrando que o ambiente de trabalho atua como um espaço relevante de socialização ideológica. A pesquisa mostra que a experiência ocupacional molda sistematicamente as visões dos indivíduos sobre economia e política, oferecendo uma perspectiva valiosa para entender como fatores sociais e profissionais moldam os posicionamentos dos eleitores — um ponto de contato com as ideias de Caplan sobre a origem dos vieses persistentes nas preferências políticas.

Complementando essa discussão, o artigo de \citeonline{jensen2009political}, publicado no \textit{Journal of Public Administration Research and Theory}, realiza um estudo comparativo em 18 países para investigar se servidores públicos possuem orientações políticas distintas em relação à população geral. Os autores identificam uma tendência à inclinação ideológica à esquerda entre esses profissionais, ainda que isso nem sempre se reflita diretamente no comportamento eleitoral. O estudo é relevante por evidenciar como grupos específicos dentro da burocracia pública podem apresentar vieses ideológicos sistemáticos, o que tem implicações importantes para a formulação de políticas — especialmente em contextos democráticos em que as decisões públicas refletem tais preferências.