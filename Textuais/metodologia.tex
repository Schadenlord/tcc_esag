\chapter{Como Medimos a Irracionalidade Política}
\label{capitulo_3}

Todos os dados analisados neste trabalho foram coletados diretamente pelo autor deste TCC, na condição de assistente de pesquisa, e pela orientadora, como pesquisadora responsável, no âmbito do projeto intitulado \textit{“Econometria Comportamental na Política Pública: Análise de Vieses Cognitivos Políticos”}, aprovado pelo Comitê de Ética em Pesquisa com Seres Humanos da Universidade do Estado de Santa Catarina (UDESC). O projeto foi submetido e apreciado pela Plataforma Brasil, sob o número de processo \textbf{CAAE: 89374225.9.0000.0118}, tendo recebido o Parecer Consubstanciado nº 7.719.326, em conformidade com a Resolução nº 510/2016 do Conselho Nacional de Saúde e a legislação vigente sobre pesquisa com seres humanos.

Os dados primários utilizados neste TCC são de autoria exclusiva dos pesquisadores, tendo sido obtidos por meio de survey aplicado a participantes voluntários, todos maiores de 18 anos e residentes no Brasil, mediante aceitação do Termo de Consentimento Livre e Esclarecido (TCLE). Todo o processo garantiu anonimato, sigilo das informações, ausência de coleta de dados sensíveis e a possibilidade de desistência a qualquer momento, sem prejuízo aos participantes. O armazenamento e a manipulação dos dados respeitaram integralmente as exigências da Lei Geral de Proteção de Dados Pessoais (LGPD), com acesso restrito ao autor e à orientadora da pesquisa.

\section{Desenho da Pesquisa}

A estratégia metodológica adotada combina três eixos principais:
\begin{enumerate}[label=\alph*)]
    \item uma análise histórico-teórica da persistência de crenças econômicas enviesadas, com ênfase na literatura de economia comportamental, história do pensamento econômico e racionalidade limitada;
    \item adaptação da \textit{Survey of Americans and Economists on the Economy} (SAEE) para o contexto institucional e cultural brasileiro, aplicada por meio de questionário estruturado em plataforma digital (Google Forms). O instrumento passou por revisão de especialistas em economia política e políticas públicas e foi submetido a um pré-teste com respondentes voluntários, que serviu para avaliar a clareza das perguntas, o tempo de resposta (em média 7 minutos) e a adequação ao contexto brasileiro. Os ajustes realizados após o pré-teste garantiram a validade de conteúdo e a aplicabilidade do questionário;
    \item aplicação de modelos econométricos (\textit{logit} binário e multinomial/ordenado) para estimar, sob controle de variáveis demográficas, socioeconômicas e ideológicas, o impacto marginal de fatores individuais na propensão a adotar crenças divergentes do consenso técnico. As análises serão conduzidas em \texttt{Python}, utilizando pacotes como \texttt{pandas}, \texttt{statsmodels} e \texttt{scikit-learn}, assegurando reprodutibilidade dos resultados.
\end{enumerate}

\subsection{Escopo inferencial e princípios-ponte}\label{sec:escopo-inferencial}

Este estudo tem propósito analítico (testar relações condicionais) e não visa estimativas descritivas de prevalência populacional. Interpretamos os coeficientes como \textit{associações condicionais} sob hipóteses explícitas de modelo, com validade externa limitada ao quadro amostral; trata-se de inferência positiva/explicativa, não causal nem normativa \cite{hausman2008}.

Para ligar teoria e dados, explicitamos \textit{princípios-ponte}:

\begin{enumerate}[label=\roman*)]
    \item como cada proposição técnica é operacionalizada em variável observável;
    \item a regra de codificação e direção esperada;
    \item a expectativa de sinal. Mantemos a mesma família de modelos (logit binário/ordenado) em todos os itens para garantir \textit{comparabilidade} e reduzir arbitrariedade de especificação, em linha com a ideia de confronto teoria–dados e congruência de modelos \cite{stigum2003}.
\end{enumerate}

\section{Coleta e Amostragem}\label{sec:coleta-amostragem}

A amostragem foi do tipo não probabilística, com estratificação em dois grupos principais:
\begin{enumerate}[label=\alph*)]
    \item \textbf{Grupo de controle}: cidadãos sem formação formal em Economia;
    \item \textbf{Grupo de tratamento}: indivíduos com formação ou em formação superior em Economia.
\end{enumerate}

A coleta de dados ocorreu entre agosto e outubro de 2025, por meio de questionário estruturado em plataforma digital (\textit{Google Forms}). Três estratégias de recrutamento foram empregadas:
\begin{enumerate}[label=\alph*)]
    \item amostragem em bola de neve, iniciada por respondentes-semente com diversidade regional e ideológica;
    \item parcerias institucionais, com apoio de docentes e redes acadêmicas de Economia;
    \item divulgação em redes sociais e e-mails institucionais, com segmentação por área de atuação e convite padrão explicativo.
\end{enumerate}

Embora a meta amostral mínima tenha sido definida com base em parâmetros estatísticos (pelo menos 175 respondentes), a pesquisa buscou ampliar o número de observações especialmente no grupo controle, visando maior robustez estatística para comparações com 95\% de confiança e 80\% de poder.

\subsection{Tamanho amostral e composição}
No total, o survey obteve respostas válidas de \(n = 183\) participantes.\footnote{O \(n\) efetivo por item/modelo varia ligeiramente devido a não resposta item a item; a Tabela-Síntese Mestra (\autoref{apendice:tabela_sintese}) indica média de 181{,}8 observações por questão (mediana 183).}
Desses, \(90{,}7\%\) (\(\approx 166\)) compuseram o grupo de controle (não economistas) e \(9{,}3\%\) (\(\approx 17\)) o grupo de economistas, definido estritamente como formação avançada em Economia com pós-graduação stricto sensu concluída (mestrado ou doutorado).
O subgrupo de economistas é menor em relação ao controle, o que recomenda cautela na leitura de efeitos específicos desse termo, como discutido na seção de robustez e limitações (\autoref{sec:limitacao_robustez}). Em termos de transparência, os modelos ordenados consolidados no \autoref{apendice:tabela_sintese} mantêm lista idêntica de covariáveis e reportam o \(n\) por item.

\subsection{Âmbito de validade externa}
A amostragem não probabilística é apropriada ao objetivo de teste de hipóteses sob severidade; os resultados são corroborações provisórias de relações condicionais no recorte estudado, não estimativas representativas da população. Declaramos explicitamente os limites de generalização e separamos descrição positiva de juízos normativos \cite{hausman2008}.


\section{Instrumento de Pesquisa}

O questionário aplicado dividiu-se em três blocos principais:
\begin{enumerate}[label=\alph*)]
    \item \textbf{Seção A — Características pessoais}: coleta de variáveis demográficas, socioeconômicas e ideológicas (sexo, idade, escolaridade, região, vínculo empregatício, renda familiar e autodeclaração ideológica);
    \item \textbf{Seção B — Percepções econômicas}: proposições adaptadas da \textit{Survey of Americans and Economists on the Economy} (SAEE), contemplando os quatro vieses cognitivos principais descritos por Caplan (2007): antimercado, antiestrangeiro, antitrabalho e pessimista;
    \item \textbf{Seção C — Políticas públicas}: questões relacionadas ao contexto brasileiro, incluindo tributação, previdência, corrupção e taxa de juros.
\end{enumerate}

As respostas foram registradas em escalas do tipo Likert (0 a 2 pontos), adaptadas conforme a nuance de cada questão. O instrumento foi submetido à revisão semântica por dois especialistas em políticas públicas e economia política e passou por um pré-teste com 15 participantes voluntários, que avaliaram clareza, aplicabilidade e tempo médio de resposta (cerca de 7 minutos), resultando em ajustes pontuais de redação e formato.

\section{Modelo Estatístico}\label{sec:modelo-estatistico}

Utiliza-se regressão \textit{logit} (binária ou ordenada) para modelar a probabilidade de adesão a uma crença tecnicamente correta. A escolha entre as versões depende da natureza da variável dependente: binária para respostas dicotômicas (ex.: concorda vs. discorda) e ordenada para escalas com múltiplos níveis. A formulação geral dos modelos é dada por:

\begin{equation}
P(y_i = 1 \mid X_i) = \frac{e^{X_i \beta}}{1 + e^{X_i \beta}}
\end{equation}

\begin{equation}
P(y_i \leq j \mid X_i) = \frac{1}{1 + e^{-(\tau_j - X_i \beta)}}
\end{equation}



As variáveis explicativas \(X_i\) incluem escolaridade, renda, ideologia, raça, gênero, ocupação, expectativas econômicas e formação em Economia, conforme detalhado na Tabela de Codificação (ver \autoref{apencice:variaveis_analisadas}). Os modelos foram ajustados por meio do algoritmo L-BFGS, adequado para estimação de modelos com múltiplos parâmetros.

\subsection{Pressupostos, comparabilidade e inferência}\label{sec:pressupostos-inferencia}
A opção por especificações paralelas entre itens cria uma “régua comum” de mensuração, permitindo atribuir diferenças de resultados ao conteúdo substantivo das questões e não à troca de modelo. 

Nos modelos de \textit{logit ordenado}, assumimos a hipótese de \textit{chances proporcionais}. Não implementamos teste formal por item (p.\,ex., Brant); em seu lugar, realizamos uma \textit{avaliação heurística} da plausibilidade do pressuposto, verificando a \textit{ordenação dos cutpoints} retornados pelo ajuste e a \textit{estabilidade/convergência} mediante \textit{fallback} de otimizadores (\textit{lbfgs} $\rightarrow$ \textit{bfgs} $\rightarrow$ \textit{powell}). Itens com inversão de limiares ou falha de convergência foram rotulados como \textit{não estimáveis/instáveis} e excluídos de interpretação substantiva.

Quanto à \textit{multicolinearidade}, adotou-se mitigação estrutural (remoção de colunas constantes/raras e monitoramento da estabilidade de sinais e erros-padrão); medidas formais (p.\,ex., VIF) não foram computadas nesta versão. Os erros-padrão reportados são os \textit{padrão do estimador}. 

Essas escolhas foram realizadas para privilegiar parcimônia, comparabilidade e \textit{congruência} --- no sentido de alinhar teoria e dados por meio de \textit{princípios-ponte} e modelos empiricamente admissíveis/encompassantes, conforme a metodologia proposta por \citeonline{stigum2003}.

\section{O Conceito de Público Esclarecido}
\label{sec:publico_esclarecido}

Conforme fundamentado na Seção~\ref{sec:vieses_auxiliares}, define-se como \textit{público esclarecido} uma simulação contrafactual baseada nos modelos logit estimados para cada questão do questionário. A ideia central consiste em prever, para cada economista da amostra, qual seria sua resposta esperada caso não possuísse formação em Economia, mantendo-se constantes todas as demais características individuais.

Na prática, os modelos econométricos incluem uma variável indicadora de formação econômica como uma das regressoras explicativas. Após a estimação dos parâmetros do modelo, constrói-se um cenário contrafactual no qual todos os indivíduos que têm formação em Economia (isto é, o grupo de economistas) são simulados como se não a tivessem — mantendo-se constantes variáveis como renda, ideologia, escolaridade, gênero, idade, entre outras. A resposta média predita nesse cenário define o comportamento esperado do público esclarecido para cada item.

A formulação matemática varia conforme a natureza da variável dependente.

\subsection{Modelo Logit Binário}

Para variáveis dicotômicas (ex.: concorda vs. discorda), a probabilidade contrafactual de resposta tecnicamente correta é dada por:

\[
\hat{Y}_i^{(\text{contrafactual})} = \mathbb{P}(Y_i = 1 \mid X_i, \text{Econ}_i = 0) = \frac{e^{X_i^\prime \beta - \beta_{\text{econ}}}}{1 + e^{X_i^\prime \beta - \beta_{\text{econ}}}}
\]

\noindent onde $X_i$ representa o vetor de covariáveis observadas do indivíduo $i$; $\beta$ é o vetor de coeficientes estimados no modelo logit; e $\beta_{\text{econ}}$ é o coeficiente estimado da variável binária de formação econômica.

\subsection{Modelo Logit Ordenado}

Para variáveis com múltiplos níveis ordenados (ex.: escalas tipo Likert), calcula-se a probabilidade predita de cada categoria e, com base nela, a média esperada:

\[
\hat{Y}_i^{(\text{contrafactual})} = \sum_{j=1}^J j \cdot \mathbb{P}(Y_i = j \mid X_i, \text{Econ}_i = 0) = \sum_{j=1}^J j \cdot \left[ \Lambda(\tau_j - X_i^\prime \beta^*) - \Lambda(\tau_{j-1} - X_i^\prime \beta^*) \right]
\]

\noindent onde $J$ é o número de categorias da resposta; $\tau_j$ são os limiares (cutpoints) estimados do modelo ordenado ($j=1,\ldots,J-1$); $\Lambda(\cdot)$ é a função logística acumulada; e $\beta^* = \beta - \beta_{\text{econ}}\, d$, com $d=1$ quando se simula a ausência de formação econômica e $d=0$ para o caso observado.


\bigskip

Essa estratégia permite isolar o efeito marginal da formação técnica sobre a percepção dos respondentes, controlando para fatores sociais, ideológicos e cognitivos. Ao comparar as médias preditas entre o público geral, os economistas e o público esclarecido, torna-se possível avaliar empiricamente se o conhecimento técnico efetivamente corrige os vieses cognitivos previamente identificados. Essa abordagem contribui para refutar interpretações reducionistas que atribuem as divergências exclusivamente à ideologia ou à renda, reforçando a hipótese de que a racionalidade política pode ser sistematicamente afetada por déficits de formação econômica específica.

\subsection{Natureza do contrafactual}
O “público esclarecido” é uma construção \textit{model-based} (contrafactual estatístico) para comparação positiva entre perfis; não descreve uma intervenção real nem implica prescrição normativa. Serve para isolar a contribuição marginal da formação econômica mantendo constantes demais características, no âmbito explicativo do modelo \cite{hausman2008}.



\section{Etapas Analíticas}

A análise empírica foi conduzida em seis etapas principais:

\begin{enumerate}[label=\alph*)]
    \item cálculo da média das respostas para cada pergunta, separando os grupos de economistas e público geral;
    
    \item ajuste de modelos \textit{logit} binário ou ordenado, conforme a natureza da variável dependente (respostas dicotômicas ou em escala Likert), com controle para variáveis sociodemográficas e ideológicas; os coeficientes estimados foram avaliados por meio de testes z para verificar sua significância estatística, replicando a estratégia utilizada por \citeonline{Systematically_Biased_Beliefs_about_Economics};
    
    \item simulação contrafactual do \textit{público esclarecido}, por meio da predição estatística de respostas dos economistas sob a condição de não possuírem formação em Economia, mantendo constantes as demais variáveis individuais; essa simulação permite isolar o efeito marginal da variável ``formação econômica'' em diferentes perfis sociodemográficos;
    
    \item cálculo da média das probabilidades preditas para cada grupo e geração de gráficos explicativos, com escalas padronizadas e comparações visuais diretas;
    
    \item agregação das variáveis em blocos conceituais (por tipo de viés: antimercado, antiestrangeiro, antitrabalho e pessimista);
    
    \item interpretação e discussão dos resultados, com base nas diferenças empíricas observadas entre os grupos e suas implicações cognitivas e políticas; foram conduzidas análises complementares com e sem variáveis de controle para testar a robustez das estimativas.

    % \item tratamento de multiplicidade: controle por FDR (Benjamini–Hochberg) como verificação de robustez, com foco em padrões consistentes de sinal e magnitude entre itens;
    
    \item diagnósticos e inferência: checagens de especificação (incluindo paralelismo nos modelos ordenados) e reporte de erros-padrão robustos; justificativa de manutenção de especificações paralelas por comparabilidade entre DVs \cite{stigum2003}.

\end{enumerate}

\section{Critério de Refutabilidade}

Este estudo adota o princípio epistemológico da falseabilidade como critério de cientificidade, conforme proposto por \citeonline{popperlogic}. Em vez de buscar confirmações para as hipóteses formuladas, parte-se do pressuposto de que toda proposição científica deve ser logicamente testável e empiricamente refutável. As hipóteses centrais serão submetidas a testes estatísticos com base em modelos logit, e consideradas refutadas caso os dados empíricos revelem padrões incompatíveis com suas previsões.

Mais especificamente, as hipóteses serão rejeitadas nos seguintes cenários:

\begin{enumerate}[label=\alph*)]

    \item Caso a formação em Economia não apresente impacto estatisticamente significativo sobre a incidência de vieses nas respostas, refuta-se a hipótese H1;
    \item Caso não se observem diferenças sistemáticas entre as respostas do público geral e do público esclarecido simulado, refuta-se H2;
    \item Caso variáveis estruturais como ideologia política, escolaridade ou renda não se associem de forma estatisticamente significativa à ocorrência de vieses cognitivos, refuta-se H7;
    \item Caso as respostas do grupo controle (não economistas) não revelem padrões sistemáticos de viés (antimercado, antiestrangeiro, antitrabalho ou pessimista), refutam-se as hipóteses H3 a H6.
    
\end{enumerate}

\subsection{Testes severos e corroboração provisória}
As hipóteses são formuladas para enfrentar \textit{testes severos}, com resultados interpretados como corroborações provisórias sujeitas a refutação futura. O foco recai menos em confirmacionismo pontual e mais na consistência de padrões sob especificações comparáveis e princípios-ponte explícitos \cite{hausman2008,stigum2003}.

Rejeita-se, portanto, qualquer pretensão de verificação definitiva: mesmo que uma hipótese não seja refutada, ela será considerada apenas corroborada de forma provisória pelos dados disponíveis. Os resultados serão interpretados com base na severidade dos testes empíricos enfrentados, e não na frequência de confirmações observadas, em consonância com a lógica da ciência proposta pelo autor.

% \section*{Nota de Reprodutibilidade}
% Disponibilizamos \textit{scripts} com versões de pacotes, \texttt{seed} fixa e pipeline para replicação integral de tabelas e figuras, reforçando a transparência do confronto teoria–dados \cite{stigum2003}.

\section{Contribuições da Metodologia}

A abordagem aqui descrita permite:
\begin{enumerate}[label=\alph*)]
    \item medir objetivamente a distância entre opinião popular e consenso técnico;
    \item identificar os determinantes estruturais dos vieses cognitivos;
    \item estimar simulações de política pública com base em maior esclarecimento técnico;
    \item posicionar o estudo na fronteira da \textit{Economia Política Comportamental}, com diálogo entre econometria, história do pensamento econômico, psicologia cognitiva e teoria política.
\end{enumerate}