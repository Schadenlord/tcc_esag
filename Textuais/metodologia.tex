\chapter{Como Medimos a Irracionalidade Política}

Todos os dados analisados neste trabalho foram coletados diretamente pelo autor deste TCC, na condição de assistente de pesquisa, e pela orientadora, como pesquisadora responsável, no âmbito do projeto intitulado \textit{“Econometria Comportamental na Política Pública: Análise de Vieses Cognitivos Políticos”}, aprovado pelo Comitê de Ética em Pesquisa com Seres Humanos da Universidade do Estado de Santa Catarina (UDESC). O projeto foi submetido e apreciado pela Plataforma Brasil, sob o número de processo \textbf{CAAE: 89374225.9.0000.0118}, tendo recebido o Parecer Consubstanciado nº 7.719.326, em conformidade com a Resolução nº 510/2016 do Conselho Nacional de Saúde e a legislação vigente sobre pesquisa com seres humanos.

Os dados primários utilizados neste TCC são de autoria exclusiva dos pesquisadores, tendo sido obtidos por meio de survey aplicado a participantes voluntários, todos maiores de 18 anos e residentes no Brasil, mediante aceitação do Termo de Consentimento Livre e Esclarecido (TCLE). Todo o processo garantiu anonimato, sigilo das informações, ausência de coleta de dados sensíveis e a possibilidade de desistência a qualquer momento, sem prejuízo aos participantes. O armazenamento e a manipulação dos dados respeitaram integralmente as exigências da Lei Geral de Proteção de Dados Pessoais (LGPD), com acesso restrito ao autor e à orientadora da pesquisa.

\section{Desenho da Pesquisa}

A estratégia metodológica adotada combina três eixos principais:
\begin{enumerate}[label=\alph*)]
    \item uma análise histórico-teórica da persistência de crenças econômicas enviesadas, com ênfase na literatura de economia comportamental, história do pensamento econômico e racionalidade limitada;
    \item adaptação da \textit{Survey of Americans and Economists on the Economy} (SAEE) para o contexto institucional e cultural brasileiro, com revisão por especialistas e pré-teste de aplicabilidade ["PREENCHA AQUI com dados do pré-teste"];
    \item aplicação de modelos econométricos (\textit{logit} binário e ordenado) para estimar, sob controle de variáveis demográficas, econômicas e ideológicas, o impacto marginal de fatores individuais na propensão a adotar crenças divergentes do consenso técnico.
\end{enumerate}

\section{Coleta e Amostragem}

A amostragem foi do tipo não probabilística, com estratificação em dois grupos principais:
\begin{enumerate}[label=\alph*)]
    \item \textbf{Grupo de controle}: cidadãos sem formação formal em Economia;
    \item \textbf{Grupo de tratamento}: economistas ou estudantes da área econômica.
\end{enumerate}

A coleta de dados ocorreu entre ["PREENCHA AQUI com datas da coleta"], por meio de questionário online. Três estratégias de recrutamento foram empregadas:

\begin{enumerate}[label=\alph*)]
    \item amostragem em bola de neve, iniciada por respondentes-semente com diversidade regional e ideológica;
    \item parcerias institucionais, com conselhos profissionais e universidades;
    \item divulgação em redes sociais e fóruns acadêmicos, com segmentação por área de atuação.
\end{enumerate}

A meta amostral mínima, calculada pela fórmula de Cochran (1977), estabeleceu um mínimo de 175 respondentes, com objetivo superior de 600 no grupo controle, garantindo poder estatístico para comparações com 95\% de confiança e 80\% de poder.

\section{Instrumento de Pesquisa}

O questionário aplicado divide-se em duas seções principais:
\begin{itemize}
    \item \textbf{Seção A}: coleta de variáveis demográficas, socioeconômicas e ideológicas;
    \item \textbf{Seção B}: 36 afirmações relacionadas a temas econômicos relevantes (tributação, comércio, tecnologia, previdência, intervenção estatal, entre outros), adaptadas da SAEE ao contexto brasileiro. As respostas foram registradas em escalas do tipo Likert, variando de acordo com a complexidade e nuance de cada questão.
\end{itemize}

As respostas foram registradas em escalas do tipo Likert (3 a 5 pontos), adaptadas ao grau de nuance necessário a cada questão. O instrumento foi submetido à avaliação semântica por especialistas e validado por ["PREENCHA AQUI com número de participantes do pré-teste ou especialistas"].

\section{Modelo Estatístico}

Utiliza-se regressão \textit{logit} (binária ou ordenada) para modelar a probabilidade de adesão a uma crença tecnicamente correta. A escolha entre as versões depende da natureza da variável dependente: binária para respostas dicotômicas (ex.: concorda vs. discorda) e ordenada para escalas com múltiplos níveis. A formulação geral dos modelos é dada por:

\begin{equation}
P(y_i = 1 \mid X_i) = \frac{e^{X_i \beta}}{1 + e^{X_i \beta}}
\end{equation}

\begin{equation}
P(y_i \leq j \mid X_i) = \frac{1}{1 + e^{-(\tau_j - X_i \beta)}}
\end{equation}

As variáveis explicativas \(X_i\) incluem escolaridade, renda, ideologia, raça, gênero, ocupação, expectativas econômicas e formação em Economia, conforme detalhado na Tabela de Codificação (ver Anexo~\ref{anexo:A}). Os modelos foram ajustados por meio do algoritmo L-BFGS, adequado para estimação de modelos com múltiplos parâmetros.

\section{O Conceito de Público Esclarecido}

Conforme fundamentado na Seção~\ref{sec:vieses_auxiliares}, define-se como \textit{público esclarecido} uma simulação contrafactual baseada nos modelos logit estimados para cada questão do questionário. A ideia central consiste em prever, para cada economista da amostra, qual seria sua resposta esperada caso não possuísse formação em Economia, mantendo-se constantes todas as demais características individuais.

Na prática, os modelos econométricos incluem uma variável indicadora de formação econômica como uma das regressoras explicativas. Após a estimação dos parâmetros do modelo, constrói-se um cenário contrafactual no qual todos os indivíduos que têm formação em Economia (isto é, o grupo de economistas) são simulados como se não a tivessem — mantendo-se constantes variáveis como renda, ideologia, escolaridade, gênero, idade, entre outras. A resposta média predita nesse cenário define o comportamento esperado do público esclarecido para cada item.

A formulação matemática varia conforme a natureza da variável dependente.

\subsection{Modelo Logit Binário}

Para variáveis dicotômicas (ex.: concorda vs. discorda), a probabilidade contrafactual de resposta tecnicamente correta é dada por:

\[
\hat{Y}_i^{(\text{contrafactual})} = \mathbb{P}(Y_i = 1 \mid X_i, \text{Econ}_i = 0) = \frac{e^{X_i^\prime \beta - \beta_{\text{econ}}}}{1 + e^{X_i^\prime \beta - \beta_{\text{econ}}}}
\]

\noindent onde:
\begin{itemize}
  \item \( X_i \) representa o vetor de covariáveis observadas do indivíduo \( i \),
  \item \( \beta \) é o vetor de coeficientes estimados no modelo logit,
  \item \( \beta_{\text{econ}} \) é o coeficiente estimado da variável binária de formação econômica.
\end{itemize}

\subsection{Modelo Logit Ordenado}

Para variáveis com múltiplos níveis ordenados (ex.: escalas tipo Likert), calcula-se a probabilidade predita de cada categoria e, com base nela, a média esperada:

\[
\hat{Y}_i^{(\text{contrafactual})} = \sum_{j=1}^J j \cdot \mathbb{P}(Y_i = j \mid X_i, \text{Econ}_i = 0) = \sum_{j=1}^J j \cdot \left[ \Lambda(\tau_j - X_i^\prime \beta^*) - \Lambda(\tau_{j-1} - X_i^\prime \beta^*) \right]
\]

\noindent onde:
\begin{itemize}
  \item \( J \) é o número de categorias da resposta,
  \item \( \tau_j \) são os limiares estimados do modelo ordenado,
  \item \( \Lambda(\cdot) \) é a função logística acumulada,
  \item \( \beta^* = \beta - \beta_{\text{econ}} \cdot d \), com \( d = 1 \) para simular ausência de formação econômica.
\end{itemize}

\bigskip

Essa estratégia permite isolar o efeito marginal da formação técnica sobre a percepção dos respondentes, controlando para fatores sociais, ideológicos e cognitivos. Ao comparar as médias preditas entre o público geral, os economistas e o público esclarecido, torna-se possível avaliar empiricamente se o conhecimento técnico efetivamente corrige os vieses cognitivos previamente identificados. Essa abordagem contribui para refutar interpretações reducionistas que atribuem as divergências exclusivamente à ideologia ou à renda, reforçando a hipótese de que a racionalidade política pode ser sistematicamente afetada por déficits de formação econômica específica.


\section{Etapas Analíticas}

A análise empírica foi conduzida em seis etapas principais:

\begin{enumerate}[label=\alph*)]
    \item cálculo da média das respostas para cada pergunta, separando os grupos de economistas e público geral;
    
    \item ajuste de modelos \textit{logit} binário ou ordenado, conforme a natureza da variável dependente (respostas dicotômicas ou em escala Likert), com controle para variáveis sociodemográficas e ideológicas; os coeficientes estimados foram avaliados por meio de testes z para verificar sua significância estatística, replicando a estratégia utilizada por \citeonline{Systematically_Biased_Beliefs_about_Economics};
    
    \item simulação contrafactual do \textit{público esclarecido}, por meio da predição estatística de respostas dos economistas sob a condição de não possuírem formação em Economia, mantendo constantes as demais variáveis individuais; essa simulação permite isolar o efeito marginal da variável ``formação econômica'' em diferentes perfis sociodemográficos;
    
    \item cálculo da média das probabilidades preditas para cada grupo e geração de gráficos explicativos, com escalas padronizadas e comparações visuais diretas;
    
    \item agregação das variáveis em blocos conceituais (por tipo de viés: antimercado, antiestrangeiro, antitrabalho e pessimista);
    
    \item interpretação e discussão dos resultados, com base nas diferenças empíricas observadas entre os grupos e suas implicações cognitivas e políticas; foram conduzidas análises complementares com e sem variáveis de controle para testar a robustez das estimativas.
\end{enumerate}

\section{Critério de Refutabilidade}

Este estudo adota o princípio epistemológico da falseabilidade como critério de cientificidade, conforme proposto por \citeonline{popperlogic}. Em vez de buscar confirmações para as hipóteses formuladas, parte-se do pressuposto de que toda proposição científica deve ser logicamente testável e empiricamente refutável. As hipóteses centrais serão submetidas a testes estatísticos com base em modelos logit, e consideradas refutadas caso os dados empíricos revelem padrões incompatíveis com suas previsões.

Mais especificamente, as hipóteses serão rejeitadas nos seguintes cenários:

\begin{itemize}
    \item Caso a formação em Economia não apresente impacto estatisticamente significativo sobre a incidência de vieses nas respostas, refuta-se a hipótese H1;
    \item Caso não se observem diferenças sistemáticas entre as respostas do público geral e do público esclarecido simulado, refuta-se H2;
    \item Caso variáveis estruturais como ideologia política, escolaridade ou renda não se associem de forma estatisticamente significativa à ocorrência de vieses cognitivos, refuta-se H7;
    \item Caso as respostas do grupo controle (não economistas) não revelem padrões sistemáticos de viés (antimercado, antiestrangeiro, antitrabalho ou pessimista), refutam-se as hipóteses H3 a H6.
\end{itemize}

Rejeita-se, portanto, qualquer pretensão de verificação definitiva: mesmo que uma hipótese não seja refutada, ela será considerada apenas corroborada de forma provisória pelos dados disponíveis. Os resultados serão interpretados com base na severidade dos testes empíricos enfrentados, e não na frequência de confirmações observadas, em consonância com a lógica da ciência proposta pelo autor.


\section{Contribuições da Metodologia}

A abordagem aqui descrita permite:
\begin{enumerate}[label=\alph*)]
    \item medir objetivamente a distância entre opinião popular e consenso técnico;
    \item identificar os determinantes estruturais dos vieses cognitivos;
    \item estimar simulações de política pública com base em maior esclarecimento técnico;
    \item posicionar o estudo na fronteira da \textit{Economia Política Comportamental}, com diálogo entre econometria, história do pensamento econômico, psicologia cognitiva e teoria política.
\end{enumerate}