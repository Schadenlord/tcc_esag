\chapter{O Eleitor é um Consumidor de Ideias Ruins?} 
\section{Etapas Analíticas} A análise empírica seguiu um conjunto estruturado de etapas: \begin{enumerate}[label=(\alph*)]
	\item Cálculo da média das respostas para cada pergunta, separadamente para o grupo de economistas e para o público geral (grupo controle);
	\item Ajuste de modelos logit (binário ou ordenado) para cada item, conforme a natureza da variável dependente, controlando-se por variáveis sociodemográficas e ideológicas. Os coeficientes estimados foram avaliados por testes $z$ de significância estatística, replicando a estratégia utilizada por Caplan (2002);
	\item Simulação contrafactual do público esclarecido: predição estatística das respostas que seriam dadas pelos economistas sob a condição hipotética de não possuírem formação em Economia, mantendo constantes suas demais características. Esta simulação isola o efeito marginal da variável “formação econômica” em diferentes perfis sociodemográficos;
	\item Cálculo das probabilidades previstas médias de resposta correta para cada grupo (público geral, público esclarecido simulado e economistas), com geração de gráficos comparativos em escalas padronizadas para facilitar a visualização das diferenças;
	\item Agregação das variáveis em blocos conceituais por tipo de viés cognitivo -- \textit{antimercado}, \textit{antiestrangeiro}, \textit{antitrabalho} e \textit{pessimista} -- conforme a taxonomia proposta por Caplan (2007). Essa categorização permitiu examinar padrões sistemáticos de respostas associadas a cada viés específico;
	\item Interpretação e discussão dos resultados, com base nas diferenças empíricas observadas entre os grupos e em suas implicações cognitivas e políticas. Realizaram-se análises complementares com e sem variáveis de controle para testar a robustez das estimativas obtidas.
\end{enumerate} 

Com base nessas etapas, apresentam-se a seguir os principais resultados obtidos e sua interpretação à luz das hipóteses formuladas. Inicialmente, a comparação descritiva das respostas médias evidenciou discrepâncias substanciais entre o público geral e os economistas para grande parte das questões investigadas. Em praticamente todos os itens, os respondentes com formação econômica mostraram-se mais propensos a responder de acordo com o consenso técnico do que os demais cidadãos. Em outras palavras, economistas acertaram consideravelmente mais questões, em média, do que o público leigo. 

Por exemplo, em uma afirmação relativa aos benefícios do comércio internacional, virtualmente todos os economistas da amostra concordaram com a visão ortodoxa de que as importações e exportações são benéficas ao país, ao passo que apenas uma parcela minoritária do público geral expressou a mesma concordância. Esse padrão se repetiu em diversos temas: os economistas tenderam a enxergar \emph{downsizing} de empresas como algo economicamente justificável, a atribuir variações de preços a mecanismos de mercado (oferta e demanda) em vez de monopólios, e a acreditar em melhorias de longo prazo do padrão de vida; já o cidadão médio revelou-se mais cético quanto ao livre mercado, mais desconfiado do comércio exterior e das empresas lucrativas, e mais pessimista em relação à trajetória econômica. Tais diferenças sistemáticas replicam, no contexto brasileiro, os vieses já documentados internacionalmente pela pesquisa da SAEE e por autores como Caplan, reforçando a evidência de que as percepções econômicas do público leigo diferem preditivamente das dos especialistas. 

Em seguida, a análise econométrica confirmou estatisticamente essas discrepâncias, controlando para diversos fatores observáveis. A inclusão de variáveis explicativas no modelo (educação, renda, ideologia, etc.) permite avaliar se as diferenças observadas devem-se exclusivamente à formação em Economia ou se estão associadas também a outros atributos correlacionados. Os coeficientes estimados para a variável indicadora de formação econômica (\textit{econ}) foram positivos e altamente significativos na vasta maioria dos 36 modelos ajustados, indicando que, mesmo após controlar diferenças de escolaridade geral, inteligência autoavaliada (proxy de capacidade cognitiva), engajamento político e características socioeconômicas, possuir treinamento em Economia aumenta de forma substantiva a chance de o indivíduo adotar a crença correta em cada item. 

Em termos de magnitude, esse efeito equivale a uma elevação de aproximadamente \textit{X} pontos percentuais na probabilidade prevista de acerto para um respondente com formação em Economia, em comparação a outro similar sem tal formação (IC95\% [\textit{Y}; \textit{Z}]). Ou seja, ceteris paribus, um economista tende a ter uma probabilidade significativamente maior de julgar corretamente uma questão econômica do que um não economista equivalente. Não surpreende, portanto, que a principal hipótese H1 (de que a educação econômica influencia as crenças do indivíduo) tenha sido \textbf{corroborada} pelos dados -- longe de ser refutada, os resultados sugerem um impacto robusto da formação técnica na redução de vieses. Essa constatação alinha-se a estudos anteriores que encontraram efeito independente do treinamento econômico mesmo após controlar o nível educacional geral
econfaculty.

Vale ressaltar que a significância de $\beta_{\text{econ}}$ persistiu em praticamente todos os itens mesmo sob especificações alternativas dos modelos, denotando alta robustez. Outros preditores estruturais também mostraram associações relevantes com a propensão a acertos, em consonância com a hipótese H7. Em particular, a ideologia política demonstrou impacto consistente: indivíduos que se identificam com espectro de direita apresentaram probabilidades significativamente maiores de concordar com o consenso econômico em temas como mercado e intervenção estatal, em comparação aos identificados com a esquerda, mesmo controlando escolaridade e renda. Por exemplo, a chance estimada de responder corretamente a uma questão de viés antimercado foi cerca de \textit{D} p.p. superior para um respondente de direita em relação a um de esquerda (IC95\% [\textit{E}; \textit{F}]). 

Esse achado ecoa a literatura sobre preferências políticas, segundo a qual orientações ideológicas influenciam crenças econômicas: respondentes à esquerda tendem a adotar posições mais intervencionistas e céticas em relação ao mercado, ao passo que indivíduos à direita são mais favoráveis a soluções de mercado livre. Outros fatores, como o nível de escolaridade formal e a renda, também emergiram como significativos em alguns modelos – tipicamente, maior instrução e maior renda associaram-se a \textit{pequenos} acréscimos na probabilidade de acerto. Contudo, os efeitos marginais dessas variáveis foram de menor magnitude em comparação à formação específica em Economia ou à ideologia, sugerindo que conhecimento especializado e visão de mundo exercem papéis mais determinantes na formação das crenças econômicas do que simplesmente a instrução genérica ou condição socioeconômica. Ainda assim, a presença de coeficientes significativos para escolaridade e renda (além de gênero, ocupação e engajamento político em alguns casos) indica que os vieses não estão distribuídos uniformemente na população, mas sim correlacionados a fatores demográficos e cognitivos – um resultado consistente com a hipótese H7, não havendo razões para refutá-la diante dos dados. Para avaliar a hipótese H2 – a saber, se um público leigo com igual conhecimento técnico responderia de forma diferente do público efetivamente observado – utilizou-se a simulação do \textit{público esclarecido}. 

Os resultados foram elucidativos. De maneira geral, as respostas \emph{preditas} para o público esclarecido situaram-se em um patamar intermediário entre as do público real e as dos economistas. Isso significa que, ao remover virtualmente a formação em Economia dos economistas (isto é, nivelando o conhecimento técnico), a distância esperada entre suas respostas e as respostas do público em geral diminui, mas não desaparece completamente. Em diversos itens, especialmente aqueles ligados a visão de mundo e valores ideológicos, o público esclarecido ainda manteria opiniões mais próximas das dos economistas do que das do público real – porém, alguma divergência persistiria. Em outras palavras, dotar o cidadão médio do mesmo conhecimento dos economistas reduziria \textit{parte} do viés observado, mas talvez não o eliminasse por completo. 

Esse achado sugere que a lacuna de percepção não se deve somente à falta de conhecimento econômico formal, mas também a diferenças de inclinações ideológicas, experiência profissional ou mesmo disposições psicológicas mais gerais. A hipótese H2 – que previa diferenças sistemáticas entre público real e público esclarecido simulado – foi, portanto, \textbf{corroborada}: de fato, as projeções contrafactuais indicam que o acréscimo de conhecimento alteraria significativamente as respostas médias (refletindo a correção de vieses), confirmando a importância do fator cognitivo, mas \textit{não} ao ponto de tornar o público idêntico aos especialistas em todas as questões. Caso os dados tivessem mostrado nenhuma diferença entre público simulado e público real, H2 seria refutada; não foi o que ocorreu. Adicionalmente, a agregação das perguntas por tipo de viés permite uma visão global do fenômeno. Observou-se evidência de todos os quatro vieses propostos por Caplan (2007) no conjunto de respondentes do grupo controle, o que corrobora as hipóteses H3–H6. Em particular, os vieses \textbf{antimercado} e \textbf{antiestrangeiro} mostraram-se os mais pronunciados: questões relacionadas a desconfiança do livre-mercado (e.g., crença de que lucro empresarial é socialmente prejudicial) e a hostilidade a agentes externos (e.g., crença de que importações e imigração prejudicam a economia) registraram as maiores diferenças de concordância entre economistas e público geral. Por exemplo, na média das perguntas classificadas como “antiestrangeiro”, economistas discordaram massivamente de afirmações protecionistas, enquanto boa parte do público as endossou -- um padrão consistente com a existência de viés antiestrangeiro generalizado. 

Já o viés \textbf{antitrabalho} também ficou evidente, embora de forma ligeiramente menos acentuada: muitos respondentes leigos concordaram com proposições do tipo “tecnologia causa desemprego líquido” ou “é melhor termos mais empregos mesmo que a produtividade seja menor”, ao contrário dos economistas, que em sua maioria rejeitaram essas ideias. Por fim, o viés \textbf{pessimista} apareceu de forma relativamente moderada: o público leigo tendeu a avaliar a situação econômica do país de forma mais negativa do que os economistas (por exemplo, superestimando índices de desemprego ou subestimando avanços de longo prazo), porém em algumas questões de cunho macroeconômico verificou-se uma certa convergência maior do que nos vieses anteriores. Ainda assim, no conjunto, as respostas do grupo controle revelaram um padrão persistente de pessimismo econômico em comparação aos especialistas. Nenhum desses padrões de viés cognitivo pôde ser atribuído ao acaso -- todas as diferenças descritas alcançaram significância estatística nos testes realizados, levando-nos a \textbf{não refutar} as hipóteses H3, H4, H5 e H6. Em outras palavras, os dados sugerem fortemente que o eleitor mediano \emph{é, de fato, um consumidor de “ideias ruins”} do ponto de vista econômico: suas crenças desviam-se de forma sistemática das concepções validadas pela teoria e pela evidência, em direções previsíveis (anti-mercado, anti-estrangeiro, pró-trabalho em excesso e pessimista). 

Essas tendências permanecem mesmo após controlar fatores socioeconômicos e tendem a diminuir, mas não desaparecer, com o aumento do conhecimento especializado. Importa sublinhar que a abordagem adotada neste estudo enfatiza a falseabilidade: nenhuma hipótese foi tratada como definitivamente confirmada, mas sim testada à procura de refutações potenciais. No presente caso, nenhuma das hipóteses H1–H7 pôde ser rejeitada frente aos dados observados; todavia, isso não implica uma verificação cabal, apenas uma \emph{corroborração provisória} conforme a evidência disponível. Em alinhamento com o critério popperiano, os resultados são interpretados com cautela e reconhecendo-se os limites inferenciais. A não refutação das hipóteses sugere que o modelo proposto de irracionalidade sistemática do eleitor encontra apoio nos dados, mas novos testes e amostras poderão submeter essas conclusões a provas adicionais. 

\section{Critério de Refutabilidade} Este estudo adota explicitamente o princípio epistemológico da falseabilidade como critério de cientificidade, nos termos de Popper (1935). Em vez de buscar confirmações para as hipóteses, parte-se do pressuposto de que toda proposição científica deve ser logicamente testável e suscetível de ser refutada pela experiência. Assim, cada hipótese formulada foi confrontada com os dados empíricos na tentativa de invalidá-la. Definiram-se, previamente, os seguintes critérios formais de refutação das hipóteses principais: \begin{itemize}
	\item Caso a variável formação em Economia (\textit{econ}) não apresentasse impacto estatisticamente significativo na incidência de vieses nas respostas, refutar-se-ia a hipótese H1;
	\item Caso não se observassem diferenças sistemáticas entre as respostas do público geral e do público esclarecido simulado, refutar-se-ia H2;
	\item Caso variáveis estruturais como ideologia política, escolaridade ou renda não estivessem associadas de forma significativa à ocorrência de vieses, refutar-se-ia H7;
	\item Caso as respostas do grupo controle (não economistas) não revelassem padrões consistentes de viés (antimercado, antiestrangeiro, antitrabalho ou pessimista), refutar-se-iam as hipóteses H3, H4, H5 e H6 simultaneamente.
\end{itemize} Como exposto, nenhuma dessas condições de refutação se verificou nos resultados obtidos. Ainda assim, rejeita-se qualquer pretensão de ``prova definitiva'': mesmo que uma hipótese não tenha sido refutada, ela é considerada apenas corroborada de forma provisória pelos dados atuais. Os testes empreendidos conferem aos achados um grau de confiabilidade condicionado aos pressupostos e ao delineamento do estudo, mas permanecem sujeitos a revisões futuras. Adota-se, pois, uma postura crítica, em que a força das conclusões advém da severidade dos testes empíricos enfrentados e não da frequência de confirmações observadas. 

\subsection*{Implicações Práticas e Custo Social dos Vieses} Os achados deste capítulo carregam importantes implicações práticas. Se o eleitor típico é, de fato, um consumidor de ideias economicamente falhas, então as políticas públicas resultantes do processo democrático tendem a refletir tais concepções distorcidas – em prejuízo do bem-estar coletivo. Como argumenta Caplan (2007), \textit{falhas democráticas} podem ser atribuídas, em grande medida, a esses vieses irracionais dos cidadãos, os quais levam a escolhas políticas subótimas e ao apoio a medidas economicamente ineficazes
mises.org
. Em outras palavras, quando parcelas expressivas do eleitorado sustentam crenças equivocadas sobre como a economia funciona, elas podem pressionar por ações governamentais contraproducentes, induzindo os governantes (que respondem a incentivos eleitorais) a implementar políticas populares porém nocivas. Um exemplo concreto é fornecido pelo viés antiestrangeiro: embora os eleitores desejem prosperidade econômica, sua desconfiança em relação ao comércio e à concorrência externa frequentemente os induz a apoiar o protecionismo e outras restrições às importações. Tal política, motivada pelo viés, acaba \textit{mantendo preços altos e prejudicando os consumidores}, em vez de proteger o interesse público
. Situação análoga ocorre com o viés antitrabalho: o apego à ideia de que ``empregos'' importam mais que produtividade pode levar ao apoio a subsídios a setores ineficientes ou barreiras tecnológicas, políticas que em última instância travam o crescimento e oneram a sociedade. Esses são os \textbf{custos sociais da irracionalidade} dos eleitores – custos que se manifestam em termos de menor eficiência econômica, desperdício de recursos públicos e oportunidades perdidas de avanço no bem-estar. Reconhecer a existência e a magnitude desses vieses é o primeiro passo para mitigar seu impacto. Os resultados sugerem que intervenções meramente informativas ou educativas podem melhorar parcialmente o discernimento econômico do público (como evidenciado pela melhoria nas respostas do público esclarecido contrafactual), mas não eliminam totalmente as distorções. Isso aponta para a necessidade de estratégias adicionais: por um lado, aprimorar os esforços de \textit{educação econômica} da população, sobretudo focados em aspectos contraintuitivos onde o público mais diverge dos especialistas (por exemplo, explicando os ganhos difusos do livre-comércio); por outro, considerar reformas institucionais que filtrem as influências de crenças equivocadas na tomada de decisão coletiva. Discutiremos a seguir, no Capítulo 4, possíveis caminhos para lidar com o dilema de uma democracia vulnerável a percepções populares sistematicamente enviesadas. Em conclusão, a análise empreendida neste capítulo reforça, de forma crítica e empiricamente fundamentada, a tese de que há limites severos para a racionalidade do eleitorado em matéria econômica. Longe de culpabilizar moralmente o eleitor -- afinal, como postulou a hipótese da \textit{racionalidade da ignorância}, é compreensível que indivíduos mal informados tomem decisões políticas de baixo custo pessoal --, os achados devem servir de subsídio para repensar mecanismos de decisão coletiva que hoje permitem que \textit{``ideias ruins''} prosperem nas urnas. Se a verdade econômica frequentemente ``perde'' no debate público, conforme nossos dados indicam, cabe à ciência apontar soluções capazes de fechar o fosso entre conhecimento especializado e escolha popular, seja pela via educativa, seja pela via institucional. O capítulo seguinte se dedicará a essa difícil questão, explorando propostas para minimizar o impacto deletério da irracionalidade do eleitor sobre os rumos da sociedade.
