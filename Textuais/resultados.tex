


\chapter{O Eleitor é um Consumidor de Ideias Ruins?} 

\section{Quando os Números Discordam do Senso Comum} 

Resultados das Estimações

Apresentar os resultados das estimações, destacando discrepâncias entre as percepções populares e os dados econômicos objetivos.

dividir em 3 subsections de acordo com a divisão da SAEE

(Interpretação dos Resultados e Implicações)

\section{O Custo Social da Irracionalidade do Eleitor} 

Implicações dos Resultados

Pode incluir um subtópico específico sobre o custo social da irracionalidade do eleitor, mostrando que, diferentemente do mercado, onde decisões ruins afetam apenas o indivíduo, no voto as decisões ruins afetam a coletividade, levando a um ciclo vicioso político.

Discutir as implicações dos resultados, enfatizando que, diferente do mercado, onde decisões ruins afetam apenas o indivíduo, no voto as decisões ruins afetam toda a coletividade.
