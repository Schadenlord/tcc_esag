


\chapter{O Eleitor é um Consumidor de Ideias Ruins?} 

A análise das respostas coletadas revela um padrão consistente de desalinhamento entre as percepções do público e os consensos técnicos estabelecidos na literatura econômica. Mais do que simples desinformação ou desconhecimento factual, as respostas evidenciam crenças enviesadas persistentes, ancoradas em julgamentos morais ou identitários que se mantêm mesmo diante de evidências empíricas contrárias. Esse tipo de irracionalidade não se dá por ausência de informação, mas por uma escolha cognitiva de manter convicções confortáveis, ainda que incorretas — como se certas crenças fossem adotadas menos por serem verdadeiras e mais por reforçarem visões de mundo já estabelecidas.

Os itens do questionário foram agrupados em quatro blocos temáticos permitindo uma organização analítica dos dados em torno de núcleos de crença. Em cada bloco, é possível observar traços distintos de irracionalidade, frequentemente relacionados aos vieses sistemáticos: antimercado, antiestrangeiro, antitrabalho e pessimista.

\section{Quando os Números Discordam do Senso Comum} 

\subsection{Por que a economia não está crescendo? – Parte 1}

Tabela

explicações da tabela

\subsection{Por que a economia não está crescendo? – Parte 2}

\subsection{Bom, ruim ou neutro para a economia?}

\subsection{Questões diversas}

\section{O Custo Social da Irracionalidade do Eleitor} 

A análise comparativa entre o público geral, os economistas e o público esclarecido revela que as discrepâncias de percepção econômica não ocorrem ao acaso. Elas seguem padrões sistemáticos de erro, que refletem vieses cognitivos profundamente enraizados na população. A irracionalidade do eleitor não é apenas um problema de desconhecimento, mas uma consequência da preferência por crenças reconfortantes, mesmo quando incompatíveis com os dados.

Ao observar as diferenças de resposta em perguntas específicas do questionário, é possível mapear a atuação de quatro vieses principais: antimercado, antiestrangeiro, antitrabalho e pessimismo econômico. Cada um deles contribui para a construção de narrativas equivocadas que distorcem a formulação de políticas públicas e comprometem a eficiência do processo democrático.

\subsection{Viés Antimercado}

\subsection{Viés Antiestrangeiro}

\subsection{Viés Antitrabalho}

\subsection{Viés Pessimista}