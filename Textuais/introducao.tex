

\chapter{Introdução} % O Labirinto das Decisões: Como Julgamos e Escolhemos?
% Adicionar aqui uma parte principal sobre qual a contribuição principal do tcc, que é sobre o desenvolvimento do campo da "economia politica comportamental", que é nova e ainda não foi amplamente explorada. 
% A ideia é que a economia comportamental, que já é um campo estabelecido, pode ser aplicada à política, e que isso pode ser uma nova forma de entender a política, que é mais realista e menos idealista do que a economia política tradicional.

%Aqui vamos apresentar o problema de pesquisa, sua relevância e contribuição para o campo da economia política comportamental.

% A democracia moderna parte do pressuposto de que os eleitores são agentes racionais, capazes de avaliar as consequências econômicas de suas escolhas políticas e apoiar medidas que maximizem o bem-estar social \cite{downs1957economic}. No entanto, a realidade demonstra que esse ideal frequentemente se desvia, devido à influência de vieses cognitivos e heurísticas que distorcem a percepção econômica da população \cite{The_Myth_of_the_Rational_Voter,kahneman2011thinking}. Como resultado, políticas públicas são frequentemente moldadas por crenças equivocadas, levando a decisões sub-ótimas que comprometem o desenvolvimento econômico e social \cite{acemoglu2012nations}.

% Nesse sentido, \citeauthoronline{The_Myth_of_the_Rational_Voter} (\citeyear{The_Myth_of_the_Rational_Voter}) observa que ``a democracia falha não porque ignora o povo, mas porque o ouve demais''. Essa afirmação sintetiza a tese central deste estudo: os equívocos na formulação de políticas públicas derivam, muitas vezes, do próprio atendimento às preferências populares — preferências estas que frequentemente se baseiam em crenças distorcidas e emocionalmente reconfortantes, mas economicamente disfuncionais.

% O fenômeno central deste trabalho é o impacto das crenças sistematicamente enviesadas dos eleitores — resultantes de uma racionalidade limitada — sobre a formulação de políticas econômicas. Eleitores bem-intencionados, mas cognitivamente limitados, acabam apoiando medidas que contradizem princípios econômicos fundamentais, como a vantagem comparativa, a eficiência dos mercados e os benefícios da inovação. A literatura, especialmente a pesquisa de Caplan \citeyear{Systematically_Biased_Beliefs_about_Economics,The_Myth_of_the_Rational_Voter}, demonstra que o público frequentemente rejeita consensos econômicos básicos, sustentando crenças que favorecem protecionismo, intervencionismo excessivo e desconfiança do setor produtivo. Esses padrões de distorção, identificados como vieses antimercado, antiestrangeiro, antitrabalho e pessimista, contribuem para a adoção de políticas que comprometem o crescimento econômico e o bem-estar social.

% A dificuldade de mitigar essas distorções vai além da simples falta de informação: envolve uma resistência psicológica profunda em abandonar crenças que reforçam visões de mundo preexistentes. Como apontam Hayek \citeyear{hayek_knowledge_use} e Downs \citeyear{downs1957economic}, a complexidade econômica e o custo elevado de se informar adequadamente dificultam o acesso a decisões racionais no ambiente político. Nesse contexto, consolida-se uma discrepância persistente entre o conhecimento técnico da economia e a opinião pública — um fator que influencia diretamente a qualidade das políticas adotadas.

% Embora a economia comportamental já seja um campo consolidado, seu diálogo com a política — particularmente por meio da chamada economia política comportamental — ainda é incipiente. Este trabalho busca contribuir para o desenvolvimento dessa vertente analítica, que considera as limitações cognitivas dos eleitores no processo democrático. Em vez de partir de uma visão idealizada da racionalidade política, propõe-se uma abordagem mais realista e empiricamente fundamentada.

% A relevância deste estudo se justifica pelo fato de que, apesar do avanço da literatura internacional, o Brasil ainda carece de investigações sistemáticas sobre como os eleitores percebem fenômenos econômicos e como essas percepções moldam decisões políticas. A ausência desse tipo de análise empírica em contextos latino-americanos representa uma lacuna teórica importante, especialmente em democracias consolidadas, mas marcadas por baixos níveis de educação econômica e alta desconfiança institucional.

A democracia moderna parte do pressuposto de que os eleitores são agentes racionais, capazes de avaliar as consequências econômicas de suas escolhas políticas e apoiar medidas que promovam o bem-estar coletivo \cite{downs1957economic}. No entanto, esse ideal frequentemente se choca com a realidade: vieses cognitivos e heurísticas distorcem a percepção econômica da população \cite{The_Myth_of_the_Rational_Voter,kahneman2011thinking}, influenciando a forma como ideias econômicas são compreendidas e apoiadas no debate público.

Como observa \citeonline{The_Myth_of_the_Rational_Voter}, a democracia não falha por ignorar o povo, mas justamente por ouvi-lo demais. Essa inversão irônica ilustra a tese central deste estudo: os equívocos políticos não decorrem da exclusão popular, mas da incorporação acrítica de preferências coletivas — frequentemente baseadas em percepções distorcidas e reconfortantes, ainda que incompatíveis com o conhecimento econômico técnico.

A literatura aponta que eleitores bem-intencionados, mas cognitivamente limitados, tendem a rejeitar consensos econômicos básicos e a endossar ideias que conflitam com noções fundamentais da teoria econômica, como a vantagem comparativa, a produtividade e os mecanismos de preços (Caplan, \citeyear{Systematically_Biased_Beliefs_about_Economics,The_Myth_of_the_Rational_Voter}) O problema vai além da falta de informação: há uma resistência ativa à revisão de crenças que reforçam identidades políticas e visões de mundo consolidadas.

Essa desconexão entre conhecimento técnico e opinião pública não é nova. \citeonline{hayek_knowledge_use} já advertia que o conhecimento econômico é contraintuitivo e disperso; \citeonline{downs1957economic} sugeria que o custo de se informar é alto demais para o retorno marginal de um voto. O resultado é um eleitor racionalmente mal informado — e, portanto, um sistema político sensível à popularidade de ideias economicamente frágeis, independentemente de seus méritos técnicos.

% Neste contexto, este trabalho insere-se no esforço emergente de desenvolver a economia política comportamental: um campo ainda incipiente que aplica os avanços da economia comportamental à compreensão das escolhas coletivas. Ao invés de idealizar a racionalidade democrática, propõe-se aqui encarar suas limitações cognitivas como ponto de partida analítico.

% A relevância do estudo reside na lacuna empírica existente no Brasil e em outras democracias latino-americanas: há escassez de investigações sistemáticas sobre como os eleitores percebem fenômenos econômicos e como essas percepções moldam suas preferências políticas. Ao trazer essa abordagem para o contexto brasileiro, este trabalho busca contribuir com evidências, conceitos e provocações para um debate ainda em construção.

% % A pesquisa SAEE é um documento institucional em meio eletrônico, produzida por entidades coletivas (The Washington Post, Harvard, Kaiser Foundation) e disponibilizada online. Assim, segue o mesmo modelo de referência utilizado para: Relatórios; Dados de pesquisa; Documentos técnicos de instituições; Pesquisas de opinião online.

% Para analisar empiricamente esse fenômeno no Brasil, este estudo recria a metodologia da SAEE, originalmente aplicada nos Estados Unidos, adaptando-a ao contexto brasileiro. A pesquisa coletará dados primários sobre as crenças econômicas dos eleitores brasileiros e os comparará com os dados da SAEE americana. O objetivo é investigar se os vieses observados nos EUA também estão presentes no Brasil, identificar possíveis divergências e avaliar fatores institucionais e culturais que possam influenciar essas percepções. Essa comparação se justifica porque ambos os países possuem democracias consolidadas, mas apresentam diferenças significativas em termos de escolaridade média, acesso à informação econômica e nível de desconfiança nas instituições. Enquanto os Estados Unidos possuem um longo histórico de pesquisas sobre a percepção pública da economia e sua relação com as políticas públicas \cite{blendon1997,Systematically_Biased_Beliefs_about_Economics, page1992}, o Brasil ainda carece de estudos que explorem sistematicamente como o eleitorado interpreta questões econômicas e como isso se reflete no cenário político. Comparar esses dois contextos permite entender se os vieses do eleitorado são fenômenos universais ou se há particularidades ligadas ao ambiente institucional e ao desenvolvimento econômico.

% No entanto, algumas limitações devem ser reconhecidas. Apesar da adaptação da metodologia da SAEE ao contexto brasileiro, diferenças institucionais e culturais entre os dois países podem afetar a comparabilidade dos resultados \cite{laporta1999quality,north1990institutions,acemoglu2012nations}. Além disso, a pesquisa se concentra na percepção econômica dos eleitores, não abrangendo outros fatores que também influenciam a formulação de políticas públicas, como o papel da mídia, o impacto de campanhas eleitorais e a disseminação de desinformação.

% É relevante destacar que esta pesquisa não pretende fornecer uma solução definitiva para o problema das crenças sistematicamente enviesadas no eleitorado, nem testar empiricamente as estratégias de mitigação sugeridas. O foco está na análise dos vieses cognitivos e suas consequências para a formulação de políticas, buscando oferecer um panorama teórico e empírico sobre o tema. Questões mais amplas relacionadas à desinformação deliberada, ao papel da mídia e a outros fatores externos não serão o foco central deste estudo, ainda que possam ser tangencialmente mencionadas.

% Embora a literatura internacional tenha avançado significativamente na análise dos vieses cognitivos na política, esse debate ainda é incipiente no Brasil. Há poucos estudos que exploram de forma sistemática como a percepção econômica dos eleitores brasileiros se distancia dos consensos acadêmicos e como isso afeta a formulação de políticas públicas. Dado o impacto de decisões econômicas equivocadas sobre o desenvolvimento do país — incluindo políticas protecionistas ineficientes, subsídios distorcidos e resistência a reformas estruturais —, compreender esses vieses se torna essencial para o aprimoramento do panorama político e econômico nacional.

% Diante desse cenário, este estudo busca analisar como os vieses cognitivos moldam a percepção econômica dos eleitores brasileiros e influenciam a formulação de políticas públicas. Para isso, será realizado um experimento de survey, inspirado na \textit{Survey of Americans and Economists on the Economy} (SAEE) adaptado ao contexto institucional e cultural brasileiro. Essa investigação comparativa servirá de base para a análise empírica desenvolvida ao longo do trabalho.

\section{A racionalidade coletiva} %(Tema e Problema de Pesquisa)

% Introduzir o tema do trabalho e a questão central: os vieses de julgamento dos eleitores e suas implicações políticas e econômicas.

O ideal democrático parte do pressuposto de que a soma das decisões individuais resulta em escolhas coletivas racionais e benéficas para a sociedade. No entanto, a prática política mostra que essa expectativa nem sempre se confirma. Mesmo indivíduos que tomam decisões prudentes em sua vida privada frequentemente apoiam propostas políticas que contrariam princípios econômicos fundamentais, contribuindo para a adoção de medidas desconectadas do conhecimento técnico acumulado na disciplina econômica \cite{downs1957economic,The_Myth_of_the_Rational_Voter}.

Esse paradoxo pode ser explicado pela diferença estrutural entre os contextos de escolha no mercado e na política. Enquanto no mercado os indivíduos arcam diretamente com os custos de suas decisões, no processo eleitoral o impacto de um único voto é praticamente nulo, o que reduz o incentivo à busca por informações qualificadas. Isso leva à chamada “ignorância racional” \cite{downs1957economic}, em que o custo de se informar supera os ganhos esperados de uma decisão fundamentada.

\citeonline{The_Myth_of_the_Rational_Voter} aprofunda essa análise ao argumentar que o problema não é apenas de ignorância, mas de vieses sistemáticos nas crenças dos eleitores. Muitos não apenas desconhecem conceitos econômicos básicos, como também resistem a evidências que contradizem suas intuições e convicções ideológicas. Essa racionalidade limitada gera um ciclo no qual políticos — por afinidade ou conveniência — respondem a percepções distorcidas da opinião pública, reforçando ideias populares, ainda que desalinhadas com o consenso técnico.

A consequência é que democracias modernas, em vez de corrigirem equívocos cognitivos por meio do debate público, frequentemente os amplificam. Assim, preferências enviesadas passam a moldar o apoio eleitoral a determinadas políticas econômicas, criando uma assimetria persistente entre opinião popular e conhecimento especializado.

% Diante desse cenário, este trabalho busca investigar como os vieses cognitivos dos eleitores — especialmente em relação à economia — influenciam suas percepções e preferências políticas no Brasil. Parte-se da hipótese de que o ambiente democrático, ao amplificar essas crenças distorcidas, contribui para a consolidação de padrões de opinião pública que favorecem propostas inconsistentes com os fundamentos econômicos, independentemente de sua eficácia prática.

\section{Hipóteses a Serem Testadas}

Com base nos dados coletados e na metodologia aplicada, este estudo propõe as seguintes hipóteses, estruturadas para serem empiricamente testáveis e refutáveis:

\begin{enumerate}[label=\alph*)]

    \item \textbf{H1 - Eleitores brasileiros apresentam vieses sistemáticos que distorcem sua percepção sobre fenômenos econômicos.}  
    Se os vieses cognitivos forem relevantes, os dados coletados devem revelar padrões recorrentes de julgamento econômico que divergem de princípios amplamente consensuais na literatura econômica. Caso não haja tais padrões, a hipótese será refutada \cite{The_Myth_of_the_Rational_Voter, blendon1997};
    % Caplan (2007) e Blendon et al. (1997) mostram que os eleitores tendem a divergir sistematicamente dos economistas em temas centrais de política econômica, revelando padrões consistentes de julgamento enviesado.

    \item \textbf{H2 - O conhecimento econômico reduz a probabilidade de manifestação de vieses cognitivos.}  
    Se o conhecimento econômico atua como mitigador de vieses, então eleitores com maior escolaridade ou maior familiaridade com conceitos econômicos devem apresentar menor adesão a padrões enviesados. Caso não haja correlação significativa, a hipótese será refutada \cite{downs1957economic, Judgment_under_Uncertainty};
    % Downs (1957) sustenta que eleitores optam racionalmente por permanecer desinformados; Kahneman e Tversky (1974) argumentam que educação pode mitigar os atalhos heurísticos que distorcem julgamentos econômicos. Tomara que aqui seja refutado.

    \item \textbf{H3 - O viés antimercado está associado ao apoio a políticas intervencionistas.}  
    Se o viés antimercado estiver presente, eleitores que expressam desconfiança em relação ao livre mercado devem mostrar maior apoio a políticas de regulação, controle ou proteção estatal. Caso não haja correlação, a hipótese será refutada \cite{The_Myth_of_the_Rational_Voter, sowell2004applied};
    % Caplan (2007) identifica o viés antimercado como central nas crenças populares; Sowell (2004) mostra como o desconhecimento da lógica de mercado leva à demanda por controles ineficientes.

    \item \textbf{H4 - O viés antiestrangeiro está associado ao apoio a restrições comerciais e migratórias.}  
    Se esse viés for relevante, eleitores que expressam preocupação com a concorrência estrangeira devem mostrar maior apoio a barreiras ao comércio exterior e à imigração. Se não houver correlação, a hipótese será refutada \cite{The_Myth_of_the_Rational_Voter, bhagwati2003free};
    % Caplan (2007) destaca a tendência de superestimar o dano causado por estrangeiros; Bhagwati (2003) demonstra os ganhos do comércio internacional e critica o protecionismo como resposta emocional.

    \item \textbf{H5 - O viés antitrabalho está associado ao apoio a políticas que priorizam a criação direta de empregos.}  
    Se os eleitores supervalorizam a criação de empregos como um fim em si, eles devem tender a apoiar propostas que desincentivam automação, realocação produtiva ou reestruturação de setores. Se não houver esse padrão, a hipótese será refutada \cite{The_Myth_of_the_Rational_Voter, landsburg2012armchair};
    % Segundo Caplan (2007), esse viés decorre da visão sentimental do trabalho como valor intrínseco; Landsburg (2012) observa que o público muitas vezes ignora os ganhos de produtividade associados à destruição criativa.

    \item \textbf{H6 - O viés pessimista leva a avaliações econômicas mais negativas do que os dados indicam.}  
    Se o viés pessimista estiver presente, os eleitores tenderão a avaliar negativamente o desempenho econômico nacional mesmo em contextos de indicadores positivos. Se a percepção estiver alinhada com os dados, a hipótese será refutada \cite{The_Myth_of_the_Rational_Voter, easterbrook2004progress};
    % Caplan (2007) argumenta que os eleitores tendem a subestimar o progresso econômico; Easterbrook (2004) mostra que, apesar de avanços reais, a percepção popular frequentemente é de estagnação ou retrocesso.

    \item \textbf{H7 - A filiação ideológica influencia a aceitação de evidências econômicas.}  
    Se a ideologia interfere na interpretação de dados econômicos, eleitores de diferentes espectros políticos tenderão a responder de forma distinta a evidências empíricas, mesmo quando essas forem tecnicamente robustas. Se não houver esse padrão, a hipótese será refutada \cite{The_Myth_of_the_Rational_Voter, kahan2012polarization}.
    % Caplan (2007) menciona o viés de confirmação ideológica; Kahan (2012) mostra que mesmo indivíduos com alta capacidade analítica distorcem evidências para se manterem consistentes com sua ideologia.

    % \item \textbf{H8 - Economistas também podem ser influenciados por vieses ideológicos ou autoindulgentes.}  ver se realmente coloco aqui ou deixo somente esses vieses na metodologia 
    % Se os economistas não forem epistemicamente neutros, então suas recomendações de política pública devem refletir, ao menos em parte, preferências normativas, alinhamentos ideológicos ou pressões institucionais. Caso os dados indiquem que a formação técnica é o único determinante das posições econômicas, a hipótese será refutada \cite{The_Myth_of_the_Rational_Voter, Hausman_McPherson_Satz_2016}.  
    % Embora Caplan (2007) reconheça que os economistas são menos propensos a certos vieses, ele admite que ideologia também afeta esse grupo; Hausman et al. (2016) destacam que o juízo técnico frequentemente incorpora valores morais e pressupostos normativos.

\end{enumerate}


\section{Objetivo Geral}

Investigar os principais vieses cognitivos na percepção econômica dos eleitores brasileiros, por meio da aplicação de um \textit{survey} experimental adaptado ao contexto nacional, com base na literatura de economia política comportamental.

\section{Objetivos Específicos}

% duas frases antes das alineas

Como desdobramento do objetivo geral, temos desdobramentos específicos que visam aprofundar a compreensão dos vieses cognitivos na percepção econômica dos eleitores brasileiros. Esses objetivos devem ser alcançados de forma sistemática e rigorosa, utilizando uma abordagem empírica que permita a coleta e análise de dados relevantes. Os objetivos específicos são:

\begin{enumerate}[label=\alph*)]
    \item identificar, com base na literatura sobre economia comportamental e economia política, os principais vieses cognitivos relacionados à percepção econômica, adaptando-os ao contexto brasileiro;

    \item construir e aplicar um survey experimental junto a eleitores brasileiros, inspirado na metodologia da \textit{Survey of Americans and Economists on the Economy} (SAEE), com questionário adaptado à realidade institucional e cultural do país;

    \item mapear a presença e frequência dos vieses econômicos entre os eleitores, com base nos dados coletados, utilizando análises estatísticas e modelos empíricos apropriados;

    \item identificar padrões de discrepância entre as percepções econômicas dos eleitores brasileiros e os consensos identificados na literatura econômica, investigando possíveis determinantes contextuais e cognitivos dessas diferenças;

    \item explorar a associação entre a presença de vieses econômicos e variáveis sociodemográficas, educacionais e institucionais, buscando compreender fatores relacionados à formação dessas percepções.

    % \item contribuir para o avanço da economia política comportamental por meio da produção de evidências empíricas sobre como vieses cognitivos influenciam a percepção econômica dos eleitores da democracia brasileira.

\end{enumerate}

\section{Justificativa e Relevância do Estudo}
% % deve enfatizar que o problema não é ignorância aleatória, mas crenças sistematicamente erradas, o que gera políticas ruins mesmo quando os eleitores estão bem informados.

% A democracia parte do pressuposto de que os eleitores escolhem representantes e políticas que maximizam o bem-estar coletivo. No entanto, a realidade demonstra que decisões econômicas são frequentemente moldadas por crenças sistematicamente equivocadas — e não apenas por desinformação aleatória. O problema central está na presença de \textit{vieses cognitivos estruturais}, que distorcem a compreensão dos fenômenos econômicos e sustentam políticas públicas ineficazes.

% A literatura em economia política comportamental mostra que esses vieses seguem padrões previsíveis. \citeonline{The_Myth_of_the_Rational_Voter} argumenta que os eleitores tendem a manter crenças enviesadas sobre comércio, mercados e tecnologia, mesmo com acesso à informação. Essas crenças favorecem o protecionismo, o intervencionismo e a tecnofobia — independentemente do nível de escolaridade. Como o custo de estar errado é diluído entre milhões de votos, não há incentivo individual para revisar crenças equivocadas \cite{downs1957economic}.

% A teoria do conhecimento disperso de \citeauthoronline{hayek_knowledge_use} (\citeyear{hayek_knowledge_use}) reforça que a economia é contraintuitiva: seus efeitos de segunda ordem muitas vezes contradizem percepções imediatas. Crenças como ``importações eliminam empregos'' ou ``preços baixos reduzem salários'' ignoram mecanismos compensatórios difíceis de comunicar ao público. Diante disso, políticos tendem a validar as crenças populares — ainda que incorretas — pois confrontá-las pode significar perder eleições.

% A relevância deste estudo está em investigar a dissonância entre opinião pública e conhecimento técnico, e suas implicações para a formulação de políticas. Ao analisar como vieses cognitivos moldam o debate econômico, esta pesquisa busca contribuir para o avanço da economia política comportamental, apontando caminhos possíveis de mitigação — seja por meio da educação econômica, seja pela reforma de incentivos institucionais.

% Sem compreender as raízes dessa distorção entre democracia e racionalidade econômica, políticas ineficazes continuarão a se repetir. O desafio não está apenas em combater a ignorância, mas em corrigir crenças sistematicamente erradas que moldam decisões políticas e o futuro das sociedades.

% A ideia democrática repousa sobre o princípio da autodeterminação informada. No entanto, o processo político moderno tem revelado uma dissonância persistente entre o conhecimento técnico acumulado pela ciência econômica e as crenças amplamente difundidas entre o eleitorado. Não se trata de ignorância aleatória, mas da presença sistemática de padrões cognitivos que distorcem a percepção da realidade econômica e conduzem à formulação de políticas públicas ineficazes \cite{The_Myth_of_the_Rational_Voter,Judgment_under_Uncertainty}. Esse descompasso tem implicações diretas sobre a qualidade das decisões coletivas e, por consequência, sobre a eficiência e a resiliência das instituições democráticas \cite{downs1957economic}.

% A economia, como disciplina, apresenta uma lógica contraintuitiva: os efeitos mais relevantes de uma política frequentemente não estão nos seus resultados imediatos, mas em suas consequências de segunda ordem — uma característica destacada desde a crítica clássica de “o que se vê e o que não se vê” \cite{bastiat1859sofismas}. Quando o julgamento popular se orienta por intuições enviesadas — como associar importações à perda de empregos ou crescimento ao aumento de desigualdade — o risco não é apenas o erro individual, mas a cristalização desses erros em decisões estatais de amplo alcance \cite{landsburg2012armchair,bhagwati2003free}. Em democracias de massa, onde o custo da má escolha é socializado e o benefício do erro é politicamente rentável, as crenças econômicas equivocadas tendem não apenas a persistir, mas a ser premiadas eleitoralmente \cite{The_Myth_of_the_Rational_Voter}.

% Compreender os mecanismos por meio dos quais essas distorções se consolidam é, portanto, uma exigência metodológica. A proposta deste trabalho é superar abordagens impressionistas ou normativas, testando empiricamente se determinados vieses de julgamento aparecem de forma consistente na população brasileira e como eles se associam à adesão a determinadas políticas públicas. Isso permite distinguir entre percepções legítimas e distorções cognitivas recorrentes, fornecendo evidências sobre os limites da racionalidade democrática em temas econômicos \cite{kahneman2011thinking,kahan2012polarization}.

% A relevância do tema também se manifesta em termos práticos. Em um cenário global marcado por instabilidade, pressão fiscal e polarização ideológica, a adoção de políticas fundamentadas em percepções incorretas não é apenas ineficiente, mas potencialmente desastrosa. O apoio popular a soluções simples para problemas complexos — como protecionismo, subsídios insustentáveis ou expansão arbitrária do papel estatal — revela um cenário de fragilidade cognitiva institucionalizada \cite{schumpeter1976capitalism,taleb2014antifragile}. Ao investigar como essas crenças se formam, se disseminam e se legitimam, este estudo contribui para a construção de mecanismos de resiliência democrática e para o desenho de estratégias de educação econômica mais eficazes \cite{franco2022cartas, zaller1992nature}.

% Em síntese, a relevância deste trabalho reside na sua capacidade de iluminar um paradoxo central da vida política contemporânea: o fato de que o voto, instrumento máximo da soberania popular, pode, quando orientado por percepções sistematicamente equivocadas, comprometer a própria racionalidade das decisões coletivas. Investigar esse fenômeno com rigor metodológico e sensibilidade interdisciplinar é condição indispensável para aperfeiçoar a governança democrática em contextos de crescente complexidade econômica e social.

A ideia democrática repousa sobre o princípio da autodeterminação informada. No entanto, observa-se uma dissonância crescente entre o conhecimento técnico acumulado pela ciência econômica e as crenças amplamente difundidas entre o eleitorado. Trata-se não apenas de ignorância, mas da presença sistemática de padrões cognitivos que distorcem a percepção da realidade econômica e influenciam a formulação de políticas públicas \cite{The_Myth_of_the_Rational_Voter,Judgment_under_Uncertainty}.

Esse descompasso desafia a capacidade deliberativa das democracias contemporâneas, afetando a qualidade das decisões coletivas \cite{downs1957economic}. Ao contrário do que pressupõem modelos normativos de democracia, o julgamento político do eleitor tende a ser moldado por fatores emocionais, ideológicos e cognitivos que resistem à revisão crítica \cite{kahneman2011thinking,kahan2012polarization}.

Diante disso, compreender como essas crenças econômicas se formam, se consolidam e se legitimam na arena política torna-se um desafio relevante para a ciência econômica e política. A presente pesquisa busca contribuir com esse debate ao investigar, empiricamente, se determinados vieses de julgamento aparecem de forma sistemática na população brasileira e como eles se associam ao apoio a diferentes políticas públicas.

A relevância do estudo é tanto teórica quanto prática. Teórica, por abordar as limitações da racionalidade democrática à luz da literatura comportamental e institucional contemporânea. Prática, por lançar luz sobre a aderência de determinadas crenças a projetos de governo, mesmo quando incompatíveis com diagnósticos técnicos. Em um cenário de crescente polarização, pressão fiscal e instabilidade institucional, a adoção de políticas fundamentadas em percepções incorretas não é apenas ineficiente, mas potencialmente desastrosa \cite{schumpeter1976capitalism,taleb2014antifragile}. Investigar esse processo contribui para o desenvolvimento de estratégias mais eficazes de resiliência democrática e educação econômica \cite{franco2022cartas,zaller1992nature}.

Em síntese, este trabalho parte do paradoxo central da vida política moderna: o fato de que o voto, expressão máxima da soberania popular, pode ser guiado por percepções sistematicamente equivocadas. A superação desse paradoxo exige não apenas a denúncia dos erros, mas a construção de uma abordagem interdisciplinar rigorosa — e empiricamente fundamentada — sobre os limites e possibilidades da racionalidade democrática.

