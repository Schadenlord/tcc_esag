\chapter{Conclusão}

Este trabalho partiu de uma hipótese simples e exigente: parte das divergências entre as opiniões econômicas do público e o consenso técnico não é ruído, mas sim um padrão. Os exercícios com modelos logit binário e ordenado mostraram que, para um conjunto amplo de afirmações, o público leigo no Brasil apresenta crenças sistematicamente distintas das dos economistas, em direção consistente com vieses documentados na literatura — antimercado, antiestrangeiro, antitrabalho e pessimismo — conforme proposto por \citeonline{The_Myth_of_the_Rational_Voter} e evidenciado no SAEE. A construção do contrafactual de \textit{público contrafactual} reforçou o ponto central: ao controlar o nível de informação econômica, o hiato de percepções encolhe de modo substancial e, em vários itens, praticamente desaparece. Assim, o mecanismo compatível com os dados é parcimonioso: a informação econômica de qualidade desloca o nível médio das crenças, não apenas sua dispersão \cite{downs1957economic,simon1955behavioral,Judgment_under_Uncertainty}.

Do ponto de vista teórico, os achados dialogam diretamente com \citeonline{The_Myth_of_the_Rational_Voter}, para quem a irracionalidade do eleitor é racional porque tem baixo custo individual, e com a tradição da racionalidade limitada e das heurísticas \cite{simon1955behavioral,Judgment_under_Uncertainty}. Em nossa amostra, as associações entre literacia econômica e julgamentos mais alinhados ao consenso técnico persistem mesmo quando condicionadas por marcadores ideológicos, o que sugere que a ideologia importa, mas não esgota a explicação do desacordo. Em termos de narrativa unificada, Caplan indica a direção do viés; a economia comportamental explica os atalhos mentais que o sustentam; e nosso exercício empírico mostra que um choque verossímil de informação econômica desloca crenças para mais perto do conhecimento especializado (Caplan, \citeyear{Systematically_Biased_Beliefs_about_Economics,The_Myth_of_the_Rational_Voter}).

Os resultados autorizam um passo adicional: a democracia não falha porque “não faz a vontade do povo”, mas precisamente porque a realiza quando a vontade é informacionalmente frágil e enviesada. Se o voto é, em parte, um consumo expressivo com baixo custo de erro individual, as políticas públicas podem espelhar crenças populares sistematicamente enviesadas \cite{The_Myth_of_the_Rational_Voter}. A literatura sobre emoção e decisão política sugere que narrativas e identidades moldam o processamento de informação, tornando apelos afetivos mais eficazes do que argumentos técnicos \cite{westen2007political}. Na prática, quando a política maximiza a \textit{utilidade expressiva} do eleitorado, tende a internalizar seus vieses e a externalizar seus custos intertemporais.

Esse diagnóstico recoloca a dimensão formativa e moral das escolhas públicas. Em \textit{A Teoria dos Sentimentos Morais}, Adam Smith descreve o “espectador imparcial” que disciplina paixões e preferências por meio de normas de justiça internalizadas, articulando o interesse próprio com o juízo moral. Reintroduzir Smith no fecho ajuda a esclarecer que “educação econômica” não é mera transmissão técnica, mas cultivo de virtudes epistêmicas — autocontrole, prudência e humildade intelectual — que reduzem a demanda por narrativas confortáveis e falsas. Nessa chave, a literacia econômica tem um papel duplo: melhora o processamento de informações e fortalece o padrão de julgamento do “espectador imparcial”, criando um elo entre a formação individual e o funcionamento institucional.

% Bloco reescrito com transição + storytelling coeso
No ponto em que os resultados indicam que a informação econômica de qualidade desloca as crenças médias em direção ao consenso técnico, a questão deixa de ser apenas \textit{o que} as pessoas pensam e passa a ser \textit{como} as decisões coletivas lidam com erros previsíveis quando a literacia é escassa. Em termos de Caplan, preferências expressivas de baixo custo tendem a vazar para a política \cite{The_Myth_of_the_Rational_Voter}; em termos smithianos, quando o “espectador imparcial” falha no agregado, resta ao arranjo institucional prover freios que disciplinem paixões e miopias. É essa passagem do nível individual (crenças e heurísticas) ao nível sistêmico (regras do jogo) que justifica abordar o tema institucional.

Nesse plano, o foco que emerge não é a comparação entre formas de governo, mas o desenho de mecanismos que alonguem horizontes, elevem o custo de políticas populares, porém inconsistentes ao longo do tempo, e preservem o uso socialmente eficiente do conhecimento. A intuição de Hayek sobre conhecimento disperso ilumina por que regras estáveis e descentralização informacional tendem a produzir decisões mais próximas dos fundamentos \cite{hayek_knowledge_use}. A teoria da consistência temporal (regras versus discricionariedade) é convergente: sem amarras credíveis, prevalecem soluções de curto prazo \cite{prescott1977,barro1983}. Em termos históricos e comparativos, instituições são as “regras do jogo” \cite{north1990institutions}; e a prosperidade depende do equilíbrio entre um Estado capaz e uma sociedade capaz de controlá-lo, ou seja, de manter-se no \textit{corredor estreito} \cite{Acemoglu2019,acemoglu2019narrow}. A inferência relevante é arquitetônica: múltiplos vetos (para reprecificar decisões miopes), separação e escalonamento de mandatos (para evitar alinhamentos de ciclo), âncoras macroeconômicas críveis com cláusulas de escape estreitas, avaliação ex ante/ex post baseada em evidência e independência operacional de órgãos técnicos são dispositivos que aproximam a política do regime de regras recomendado por \citeonline{prescott1977} e \citeonline{barro1983}, explorando conhecimento local (à la Hayek) e reduzindo o espaço para a utilidade expressiva moldar políticas intertemporalmente custosas.

Esse arranjo dialoga com duas intuições institucionais compatíveis com nossa evidência. Primeiro, o princípio da subsidiariedade recomenda decidir no nível mais próximo do usuário quando isso for eficiente, o que também é uma estratégia epistêmica para aproveitar o conhecimento local e reduzir erros agregados \cite{hayek_knowledge_use,pcjp2004compendio,pioxi1931quadragesimo}. Segundo, a difusão ampla de propriedade produtiva — ideia associada ao distributismo\footnote{Em Chesterton e Belloc, “distributismo” significa a dispersão ampla da propriedade produtiva como antídoto à concentração de poder econômico e político; não implica rejeição de mercados ou de empresas, mas enfatiza estruturas proprietárias pulverizadas e responsabilidade local. Ver \citeonline{chesterton1926outline} e \citeonline{belloc1912servile}.} — pode multiplicar freios horizontais, diminuir a captura por coalizões rentistas e tornar mais visíveis os custos de políticas ruins \cite{chesterton1926outline,belloc1912servile}.
 Não se trata de proselitismo, mas de hipóteses de mecanismo coerentes com o padrão observado: quando a informação econômica é localmente densa e os incentivos são descentralizados, as crenças aproximam-se do consenso técnico.

Há, por fim, uma dimensão educacional sem a qual as amarras institucionais são frágeis. \citeonline{newman2020ideia} defende a educação como uma disciplina da mente\footnote{Newman concebe a universidade como um lugar de “conexão e comparação” de saberes, onde a mente aprende a julgar \citeonline{newman2020ideia}. Platão dá a matriz (\textit{paideía} da \textit{República}); e Whitehead, a síntese clássica: \textit{“A caracterização geral mais segura da tradição filosófica europeia é que ela consiste em uma série de notas de rodapé a Platão”} \cite[p.~39]{whitehead1978process}. Nussbaum argumenta que as democracias precisam das humanidades para cultivar a imaginação empática e a razão pública \cite{nussbaum2010notforprofit}; Sen chama isso de “exame crítico” na praça pública \cite{sen1999development}. Em ambos, a educação forma capacidades de julgamento — condição para reduzir a demanda por narrativas econômicas sedutoras e falsas.}
: formar o intelecto para julgar antes de se profissionalizar para atuar. Nossos resultados — que evidenciam o efeito substantivo da literacia econômica sobre o posicionamento das crenças — sugerem que intervenções de educação econômica, cuidadosamente desenhadas para evitar proselitismo, podem reduzir a distância entre o juízo popular e os diagnósticos técnicos. Isso vale não só para o “público em geral”, mas também para produtores profissionais de opinião econômica. No caso brasileiro, os dados indicam um risco: quando a técnica confronta preferências ideológicas, parte dos economistas tende a alinhar as respostas à ideologia, não à \textit{téchne}. \citeonline{westen2007political} lembra que ninguém, nem especialistas, é imune ao entrelaçamento entre cognição e emoção; \citeonline{sowell2007conflict} adverte que visões de mundo diferentes selecionam evidências de modos distintos; e \citeonline{franco2021licoes} observa que ideias econômicas ruins, embaladas por boas histórias, custam caro. A autocrítica da profissão é, portanto, condição de credibilidade: currículos que ensinem lógica de identificação, testes de robustez, \textit{preregistration}, replicação e leitura adversária; avaliações às cegas; e uma cultura de \textit{disconfirmation} coerente com o ideal popperiano \cite{popperlogic}. Em outro registro, \citeonline{franco2022cartas} reforça a necessidade de cultivar a humildade intelectual desde a formação inicial.


\section{Escopo e limitações}

A amostra é não probabilística; os modelos estimam associações, não efeitos causais. Nos modelos ordenados, pressupomos a hipótese de \textit{chances proporcionais}, base da comparabilidade entre itens; essa hipótese foi avaliada (teste de Brant) e, quando houve indícios de violação, estimamos modelos de \textit{chances proporcionais parciais} e/ou adotamos enlaces alternativos, com detalhes no Apêndice. Mantivemos a mesma família de especificações por um princípio de comparabilidade. Para mitigar a multiplicidade, interpretamos os resultados por meio de padrões de sinal e magnitudes (odds ratios com IC95\%), e não por \textit{p-valor} isolado, além de aplicarmos um procedimento de controle da taxa de descobertas falsas (FDR). Como robustez, realizamos reestimativas com pontos de corte alternativos, reamostragens simples e \textit{bootstrap} para avaliar a estabilidade de sinais e magnitudes, além de verificações de convergência e sensibilidade ao otimizador. Tais cautelas não anulam a principal contribuição: a consistência do padrão empírico ao longo de diferentes variáveis dependentes e o poder explicativo da literacia econômica sobre o posicionamento das crenças.

Além disso, alguns padrões empíricos não se alinharam perfeitamente às hipóteses previstas. Houve casos em que nem o viés ideológico nem o grau de conhecimento em economia explicaram integralmente a divergência de opiniões, sugerindo a presença de mecanismos adicionais — como experiências pessoais, narrativas culturais ou fatores emocionais — que transcendem as categorias analisadas. Esse espaço teórico abre margem para novas hipóteses, como a influência de exposição midiática específica, identificação afetiva com certos grupos ou até resistência motivada à evidência. Pesquisas futuras podem abordar essas camadas com metodologias mistas (quantitativas e qualitativas), delineamentos experimentais (testando intervenções informacionais) ou replicações com grupos populacionais distintos.

Por fim, à luz do princípio popperiano, as conclusões são provisórias e abertas à refutação \cite{popperlogic}. Elas apontam um caminho realista: regras que desincentivem erros previsíveis; subsidiariedade que devolva decisões ao nível onde o conhecimento é mais rico; difusão de propriedade que multiplique freios horizontais; e educação que reduza a demanda por más políticas. Pesquisas futuras podem testar intervenções informacionais de baixo custo, explorar heterogeneidades regionais e avaliar o papel do ecossistema midiático na persistência ou correção de vieses. À epígrafe que adverte que talvez precisemos provar que a grama é verde, este estudo acrescenta um roteiro: instituições que ampliam horizontes, comunidades que aprendem perto do chão e universidades que educam a inteligência mantêm a cor verdadeira à vista do público — e a política, dentro do corredor estreito em que a liberdade e a prosperidade conseguem respirar \cite{acemoglu2019narrow,Acemoglu2019}.
