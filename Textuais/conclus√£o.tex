\chapter{Conclusão}

Este trabalho partiu de uma hipótese simples e exigente: parte das divergências entre opiniões econômicas do público e o consenso técnico não é ruído, mas padrão. Os exercícios com modelos logit binário e ordenado mostraram que, para um conjunto amplo de afirmações, o público leigo no Brasil apresenta crenças sistematicamente distintas das de economistas, em direção consistente com vieses documentados na literatura — antimercado, antiestrangeiro, antitrabalho e pessimismo — conforme proposto por \citeonline{The_Myth_of_the_Rational_Voter} e evidenciado no SAEE. A construção do contrafactual de público esclarecido reforçou o ponto central: ao controlar o nível de informação econômica, o hiato de percepções encolhe de modo substantivo e, em vários itens, praticamente desaparece. Assim, o mecanismo compatível com os dados é parcimonioso: informação econômica de qualidade muda o nível médio das crenças, não apenas sua dispersão \cite{downs1957economic,simon1955behavioral,Judgment_under_Uncertainty}.

Do ponto de vista teórico, os achados dialogam diretamente com \citeonline{The_Myth_of_the_Rational_Voter}, para quem a irracionalidade do eleitor é racional porque tem baixo custo individual, e com a tradição de racionalidade limitada e heurísticas \cite{simon1955behavioral,Judgment_under_Uncertainty}. Em nossa amostra, as associações entre literacia econômica e julgamentos mais alinhados ao consenso técnico persistem mesmo condicionando por marcadores ideológicos, o que sugere que ideologia importa, mas não esgota a explicação do desacordo. Em termos de narrativa unificada, Caplan aponta a direção do viés; a economia comportamental explica os atalhos mentais que o sustentam; e nosso exercício empírico mostra que um choque de informação econômica verossímil desloca crenças para mais perto do conhecimento especializado (Caplan, \citeyear{Systematically_Biased_Beliefs_about_Economics,The_Myth_of_the_Rational_Voter}).

Os resultados autorizam um passo adicional: a democracia não falha porque “não faz a vontade do povo”, mas precisamente porque a realiza quando a vontade é informacionalmente frágil e enviesada. Se o voto é, em parte, consumo expressivo com baixo custo de erro individual, políticas públicas podem espelhar crenças populares sistematicamente enviesadas \cite{The_Myth_of_the_Rational_Voter}. A literatura sobre emoção e decisão política sugere que narrativas e identidades moldam o processamento de informação, tornando apelos afetivos mais eficazes que argumentos técnicos \cite{westen2007political}. Na prática, quando a política maximiza a \textit{utilidade expressiva} do eleitorado, tende a internalizar seus vieses e a externalizar seus custos intertemporais.

Esse diagnóstico recoloca o problema institucional. A intuição de Hayek sobre o uso do conhecimento disperso ajuda a interpretar o resultado: quando o conhecimento efetivamente circula, decisões se aproximam do que os fundamentos permitem; quando se deforma por vieses sistemáticos, a política agrega erros \cite{hayek_knowledge_use}. A teoria da consistência temporal (regras versus discricionariedade) é convergente: sem amarras que elevem o custo de políticas populares porém ineficientes, prevalecem soluções de curto prazo \cite{prescott1977,barro1983}. Em chave histórica, \citeonline{north1990institutions} lembram que instituições são as regras do jogo; e \citeonline{Acemoglu2019,acemoglu2019narrow} mostram que prosperidade depende do equilíbrio entre um Estado capaz e uma sociedade capaz de controlá-lo.

Nessa gramática, uma inferência incômoda — mas relevante — emerge: sob certas condições, monarquias constitucionais podem performar melhor que repúblicas ao mitigar a miopia política. Não se trata de nostalgia de formas políticas, mas de incentivos. Regimes que separam chefe de Estado de chefe de governo, introduzem um \textit{ponto focal} de continuidade e alongam o horizonte decisório reduzem a tentação de ciclos populares de curto prazo. Monarquias constitucionais bem amarradas por regras, parlamento forte e accountability público podem funcionar como dispositivos de compromisso intertemporal, aproximando a política do regime de regras recomendado por \citeonline{prescott1977} e \citeonline{barro1983}. Já repúblicas com forte personalização executiva e calendários eleitorais intensos são mais vulneráveis a picos de demanda por políticas inconsistentes ao longo do tempo. Em termos do \textit{corredor estreito}, não é a coroa ou o plebiscito em si que importam, mas a capacidade da forma de governo de manter o Estado dentro do corredor — forte o suficiente para agir, suficientemente constrangido para não capturar a sociedade \cite{acemoglu2019narrow}. Sob esse prisma, uma monarquia constitucional pode, em média e em determinados contextos, oferecer arranjos mais estáveis do que repúblicas personalistas para conter vieses previsíveis do eleitorado. A lição geral vale para qualquer forma: alongar horizontes, multiplicar vetos e blindar o núcleo de regras.

Se as instituições devem desincentivar erros previsíveis, o desenho dos freios importa. O princípio da subsidiariedade — tal como formulado na tradição da doutrina social da Igreja — recomenda que decisões sejam tomadas no nível mais próximo do indivíduo capaz de resolvê-las. Isso conversa diretamente com o problema do conhecimento de Hayek: descentralizar é uma estratégia epistêmica, não apenas moral \cite{hayek_knowledge_use}. Ao reduzir o raio de ação de políticas uniformes e trazer o aprendizado para perto do usuário, subsidiariedade atenua vieses agregados. O distributismo de Chesterton, entendido como difusão ampla de propriedade e meios de produção, pode ser lido como complemento econômico desse desenho: mais proprietários significam mais checagens horizontais, menor captura por coalizões rentistas e maior resiliência de comunidades a narrativas simplistas. Não se trata de proselitismo, mas de uma intuição institucional compatível com nossa evidência: quando a informação econômica é localmente densa, as crenças aproximam-se do consenso técnico e os custos de políticas ruins tornam-se mais visíveis ao eleitor.

As implicações para o ecossistema de políticas são concretas. Âncoras como responsabilidade fiscal, estabilidade de preços e abertura à concorrência preservam o caráter de “regras do jogo” \cite{north1990institutions}. Complementos densos podem incluir cláusulas de \textit{sunset} regulatório, avaliação ex ante de impacto com padrões de evidência, independência operacional de órgãos técnicos com mandatos alongados e vedação a mudanças discricionárias em metas no ciclo eleitoral. Em um plano federativo, a subsidiariedade sugere desenhar competências e receitas de modo a reforçar a accountability local, com \textit{matching grants} condicionados a resultados e transparência. Tudo isso é coerente com o diagnóstico corrido por \citeonline{franco2021licoes}: ideias econômicas ruins, embaladas por boas histórias, custam caro; e só disciplina de regras e aprendizado institucional quebram o ciclo.

Há, porém, uma dimensão formativa sem a qual as amarras institucionais são frágeis. \citeonline{newman2020ideia} defende a educação como disciplina da mente: formar o intelecto para julgar antes de profissionalizar para fazer. Nossos resultados — que mostram o efeito substantivo da literacia econômica sobre o posicionamento das crenças — sugerem que intervenções de educação econômica, cuidadosamente desenhadas para evitar proselitismo, podem reduzir a distância entre juízo popular e diagnósticos técnicos. Isso vale não só para o “público em geral”, mas também para os produtores profissionais de opinião econômica. No caso brasileiro, nossos dados indicam um risco: quando a técnica econômica confronta preferências ideológicas, parte dos economistas tende a alinhar respostas à ideologia, não à \textit{tecnê}. \citeonline{newman2020ideia} ajuda a compreender a falha de fundo: universidades que negligenciam a disciplina da mente — a ordenação dos saberes, a hierarquia dos métodos e a \textit{intellectual humility} — formam menos juízos e mais partidarismos. \citeonline{westen2007political} lembra que ninguém, nem especialistas, é imune ao entrelaçamento entre cognição e emoção. E \citeonline{sowell2007conflict} adverte que visões de mundo diferentes selecionam evidências de modos diferentes. \citeonline{franco2021licoes} e, em outro registro, as \textit{Cartas a um jovem economista}, notam que jovens ingressam na economia com mentalidade ideológica marcada e que muitas vezes a graduação não desfaz esse viés, outorgando diplomas sem consolidar o senso crítico que a boa técnica exige. A autocrítica da profissão, nesse ponto, é condição de credibilidade: currículos que ensinem lógica de identificação, testes de robustez, preregistro, replicação e leitura adversária; avaliações às cegas; e uma cultura de \textit{disconfirmation} coerente com o ideal popperiano \cite{popperlogic}.

Escopo e limitações precisam ser lembrados. A amostra é não probabilística; os modelos estimam associações, não efeitos causais; nos modelos ordenados assumimos a hipótese de chances proporcionais, base da comparabilidade entre itens. Mantivemos a mesma família de especificações por princípio de comparabilidade; robustez adicional pode incluir testes de proporcionalidade (p.\,ex., Brant), reestimativas com pontos de corte alternativos, modelos de chances proporcionais parciais quando pertinente, enlaces alternativos (logit/probit), reamostragens simples e \textit{bootstrap} para avaliar estabilidade de sinais e magnitudes, bem como checagens de convergência e sensibilidade ao otimizador. Tais cautelas não anulam a principal contribuição: a consistência do padrão empírico ao longo de diferentes variáveis dependentes e o poder explicativo da literacia econômica sobre o posicionamento das crenças.

Por fim, à luz do princípio popperiano, as conclusões são provisórias e abertas à refutação \cite{popperlogic}. Elas apontam um caminho realista: regras que desincentivem erros previsíveis; subsidiariedade que devolva decisões ao nível onde o conhecimento é mais rico; difusão de propriedade que multiplique freios horizontais; e educação que reduza a demanda por más políticas. Pesquisas futuras podem testar intervenções informacionais de baixo custo, explorar heterogeneidades regionais e avaliar o papel do ecossistema midiático sobre a persistência ou a correção de vieses. À epígrafe que adverte que talvez precisemos provar que a grama é verde, este estudo acrescenta um roteiro: instituições que alonguem horizontes, comunidades que aprendam perto do chão e universidades que eduquem a inteligência mantêm a cor verdadeira à vista do público — e a política, dentro do corredor estreito em que a liberdade e a prosperidade conseguem respirar \cite{acemoglu2019narrow,Acemoglu2019}.
