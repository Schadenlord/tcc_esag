

\chapter{O Que Fazer Quando a Verdade Perde na Urna?} % Conclusão e discussão

Discutir soluções para mitigar o impacto da irracionalidade dos eleitores.

\section{Limitações das Intervenções Educacionais} % explicando se há evidências empíricas de que a educação econômica pode reduzir vieses no voto. Poderia falar sobre o que o nassim taleb fala no livro "antifrágil" que o principal problema é o excesso de controle e a sobrevalorização da educação formal, onde deveriamos valorizar mais a prática e a experiência.

Examinar se a educação econômica pode reduzir vieses e discutir a visão de Nassim Taleb sobre o excesso de controle e a sobrevalorização da educação formal.

\section{Educação Econômica e Tomada de Decisão} % Educação Econômica e Tomada de Decisão

Explorar como o ensino econômico pode ser mais eficaz para diminuir os vieses cognitivos.

\section{Como Melhorar as Escolhas Coletivas} % Implicações para Políticas Públicas

Analisar possíveis reformas institucionais e medidas para aumentar a qualidade das decisões políticas.

\section{As Perguntas que Ainda Precisamos Responder} % Limitações e Sugestões para Futuras Pesquisas

Identificar as limitações da pesquisa e sugerir direções para estudos futuros.
