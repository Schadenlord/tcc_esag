


\chapter{Metodologia} % Metodologia

Apresentar os métodos usados na pesquisa empírica.

\section{Definição das Variáveis e Modelagem} % Variáveis e Modelagem

Explicar quais variáveis foram analisadas e como foram operacionalizadas.

\section{Técnicas de Análise Empírica} % Análise Empírica

Descrever os métodos estatísticos usados, como a modelagem econométrica (Logit), e a replicação da Survey of Americans and Economists on the Economy (SAEE) no Brasil.

\section{Quando os Números Discordam do Senso Comum} % Resultados das Estimações

Apresentar os resultados das estimações, destacando discrepâncias entre as percepções populares e os dados econômicos objetivos.

\section{O Eleitor é um Consumidor de Ideias Ruins? } %(Interpretação dos Resultados e Implicações)
% Pode incluir um subtópico específico sobre o custo social da irracionalidade do eleitor, mostrando que, diferentemente do mercado, onde decisões ruins afetam apenas o indivíduo, no voto as decisões ruins afetam a coletividade.

Discutir as implicações dos resultados, enfatizando que, diferente do mercado, onde decisões ruins afetam apenas o indivíduo, no voto as decisões ruins afetam toda a coletividade.
