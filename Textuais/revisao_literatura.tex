\chapter{Teorias e Evidências Sobre a (I)Racionalidade Humana} % Revisão de Literatura

% Revisar a literatura sobre economia comportamental e sua aplicação à política.

A hipótese da racionalidade plena dos agentes econômicos, consagrada pela teoria da escolha racional, tem sido um dos pilares analíticos da economia neoclássica e de modelos normativos de democracia. No entanto, ao confrontar-se com a realidade política, especialmente no que tange ao comportamento do eleitor médio, tal suposição revela-se cada vez menos plausível. A crescente literatura da economia comportamental, em diálogo com a ciência política e a psicologia cognitiva, oferece uma alternativa teórica mais realista: os indivíduos, em contextos de incerteza e baixa responsabilização, tomam decisões sistematicamente enviesadas.

Nesse sentido, autores como \citeonline{kahneman2011thinking} e \citeonline{Judgment_under_Uncertainty} demonstram empiricamente que os julgamentos humanos são fortemente influenciados por heurísticas cognitivas — atalhos mentais que, embora úteis, produzem erros previsíveis. Esses desvios da racionalidade instrumental se intensificam no campo da política, onde o custo da desinformação é socialmente disperso e o benefício individual de votar corretamente é estatisticamente nulo.

A economia política comportamental surge, assim, como um campo híbrido, cuja proposta é investigar as falhas sistemáticas do processo democrático à luz da racionalidade limitada dos eleitores. Bryan Caplan, em \textit{The Myth of the Rational Voter} \citeyear{The_Myth_of_the_Rational_Voter}, articula essa perspectiva ao argumentar que os eleitores não apenas carecem de informação, mas mantêm ativamente crenças econômicas falsas com convicção, um fenômeno que denomina ``irracionalidade racional''. Para Caplan, a democracia falha precisamente porque responde às preferências dos eleitores — e essas preferências, por sua vez, são moldadas por vieses cognitivos persistentes e emocionalmente gratificantes.

O desafio proposto neste capítulo, portanto, é examinar criticamente a suposição de que os eleitores agem como agentes racionais e bem informados. Em vez disso, buscaremos sistematizar evidências empíricas e abordagens teóricas que apontam para um padrão recorrente de distorções cognitivas nas decisões políticas. A orientação metodológica adotada segue os princípios do rigor científico: partimos de hipóteses refutáveis, sustentadas por dados observáveis e modelos teóricos claros, rejeitando interpretações tautológicas ou não falsificáveis.

Esta revisão será estruturada em cinco seções principais. Na primeira, contrastaremos a tradição da escolha racional com os avanços da economia comportamental, do pensamento de Adam Smith à psicologia de Kahneman. Em seguida, abordaremos os vieses cognitivos mais relevantes para o comportamento político, como o viés antimercado, antiestrangeiro, antitrabalho e o pessimismo econômico, conforme identificados por \cite{Systematically_Biased_Beliefs_about_Economics} e operacionalizados em pesquisas como a SAEE\footnote{Survey of Americans and Economists on the Economy, de 1996, replicado posteriormente com adaptações em diversos contextos.}. Na terceira seção, exploraremos os limites da informação no processo político, discutindo a racionalidade limitada e os custos da ignorância deliberada.

A quarta seção amplia a discussão para o plano da difusão de ideias econômicas, enfatizando o papel da mídia, da cultura e dos incentivos políticos na propagação de crenças disfuncionais. Por fim, argumentaremos que muitas crenças persistem não apesar das evidências, mas por causa de estruturas emocionais, sociais e identitárias que as tornam desejáveis. A política, nesse contexto, torna-se menos uma arena de deliberação racional e mais um campo de validação simbólica.

Ao longo das próximas seções, o objetivo será, portanto, compreender como e por que o ``homo politicus'' se afasta do ``homo economicus'' — e quais as consequências dessa ruptura para o funcionamento da democracia.

\section{Entre Adam Smith e Kahneman} % Economia Comportamental vs. Escolha Racional

A teoria da escolha racional parte do pressuposto de que os indivíduos possuem preferências bem definidas, estáveis e informadas, sendo capazes de maximizar sua utilidade sob restrições orçamentárias. Predominante na economia neoclássica e na teoria política normativa, esse modelo assume decisões guiadas por cálculo objetivo e racional.

Contudo, a própria tradição clássica já apontava para uma racionalidade mais complexa. Em \textit{The Theory of Moral Sentiments}, Adam Smith reconhecia o papel das emoções, da empatia e do julgamento moral na ação humana \cite{smith1759-theory-of-moral-sentiments}. Para ele, o comportamento é guiado não apenas pelo interesse próprio, mas por um ``espectador imparcial'' interior, sensível ao senso de justiça.

Essa visão, longe de contradizer a ênfase no interesse próprio apresentada em \textit{The Wealth of Nations}, é parte de um sistema coerente, como argumenta \citeonline{fitzgibbons1995adam}. Segundo o autor, Smith articula uma teoria unificada da ação humana, em que a busca por interesse pessoal é moderada por normas morais internalizadas, derivadas da convivência social e da atuação do ``espectador imparcial''. Reduzir o agente smithiano ao egoísmo individual, como fazem algumas leituras contemporâneas, constitui, para Fitzgibbons, um erro interpretativo — ao ignorar a dimensão normativa essencial ao funcionamento da liberdade econômica. O resultado é um modelo de agente que não é puramente utilitarista, mas situado em um arcabouço ético e institucional — uma formulação que antecipa os debates atuais da economia comportamental ao reconhecer a centralidade das motivações morais e contextuais na tomada de decisão.

Ao longo do século XX, consolidou-se a figura do \textit{homo economicus}: autocentrado, maximizador e plenamente racional. Autores como \citeonline{becker1976} expandiram esse modelo para o comportamento social, criminal e político. Becker sustenta que desvios de conduta não indicam irracionalidade, mas sim diferentes restrições e preferências.

Em contraponto, a economia comportamental — com Daniel Kahneman e Amos Tversky — demonstrou empiricamente que indivíduos recorrem a heurísticas cognitivas para lidar com incertezas \cite{Judgment_under_Uncertainty}. Atalhos como representatividade, disponibilidade e ancoragem são eficientes do ponto de vista mental, mas produzem vieses sistemáticos, especialmente sob baixa responsabilização ou forte carga emocional \cite{kahneman2011thinking}.

Ainda no campo da economia política, Anthony Downs formulou a teoria da ignorância racional, segundo a qual os eleitores têm pouco incentivo para buscar informações políticas de qualidade, dado que o custo de adquirir conhecimento é elevado e a probabilidade de influenciar o resultado eleitoral é virtualmente nula \cite{downs1957economic}. Embora essa formulação mantenha a lógica da racionalidade instrumental — onde os indivíduos escolhem ser ignorantes de forma racional — o próprio Downs reconhece os limites desse modelo. Segundo o autor, a irracionalidade política é um fenômeno empírico que escapa à lógica dedutiva pura e requer investigação própria, o que o leva a não explorá-la diretamente em sua análise \cite[p.~10]{downs1957economic}. Ao fazer essa ressalva, Downs não propõe uma abordagem comportamental, mas admite que seu modelo não esgota o comportamento político real, abrindo espaço para interpretações futuras que incorporam fatores não racionais.

Essa divergência teórica tem implicações práticas. Enquanto o modelo racional favorece políticas baseadas em incentivos marginais, a abordagem comportamental exige considerar limites cognitivos e contextos institucionais. \citeonline{Hausman_McPherson_Satz_2016} argumentam que modelos de racionalidade também devem ser avaliados por seus efeitos éticos e distributivos.

Além disso, autores como \citeonline{hayek_knowledge_use} e \citeonline{positive_economics_friedman} destacam que o conhecimento é disperso e imperfeito. Para Hayek, o sistema de preços coordena ações melhor do que qualquer planejador central. Já Friedman sustenta que o valor de uma teoria está na qualidade de suas previsões, e não na veracidade de suas premissas.

Portanto, o contraste entre racionalidade plena e limitada não é apenas analítico, mas epistemológico. A escolha racional idealiza um agente isolado e infalível; a economia comportamental descreve um agente real: falho, emocional e influenciado por contexto — mais próximo do cidadão comum do que do \textit{decision-maker} normativo.

\section{Como os Vieses Moldeiam as Escolhas Políticas} % (Vieses Cognitivos e Política)

% Explicar os principais vieses cognitivos identificados na literatura e sua relação com decisões políticas (viés antimercado, antiestrangeiro, antitrabalho, pessimista, entre outros).

Ao analisar o comportamento do eleitor médio, Bryan Caplan propõe um deslocamento da hipótese da ignorância racional — centrada na ausência de informação — para a noção de \textit{irracionalidade racional}, na qual o eleitor sustenta ativamente crenças falsas, mesmo diante de evidências contrárias \cite{The_Myth_of_the_Rational_Voter}. Segundo o autor, os erros de julgamento político não ocorrem de forma aleatória ou esporádica, mas seguem padrões previsíveis, sistemáticos e estáveis ao longo do tempo. Tais distorções não derivam apenas da dificuldade de acesso à informação, mas sobretudo de motivações emocionais e ideológicas que incentivam os indivíduos a rejeitar proposições verdadeiras em favor de crenças confortáveis ou moralmente atraentes.

Caplan identifica quatro vieses cognitivos centrais que moldam negativamente o julgamento político dos eleitores: o viés antimercado, o viés antiestrangeiro, o viés antitrabalho e o viés pessimista \cite{Systematically_Biased_Beliefs_about_Economics}. Cada um desses vieses opera como uma lente interpretativa que distorce o entendimento sobre fenômenos econômicos complexos, gerando apoio a políticas públicas ineficientes ou contraproducentes.

O \textit{viés antimercado} refere-se à tendência de subestimar os benefícios dos mecanismos de mercado e superestimar os efeitos negativos do lucro, da concorrência e da desregulação. Essa visão, embora difundida, colide com fundamentos consolidados da teoria econômica. Autores como \citeonline{mankiw2020introducao} e \citeonline{sowell2000basic} enfatizam que o mercado, embora imperfeito, tende a alocar recursos de maneira mais eficiente do que alternativas centralizadas. A crítica liberal clássica, expressa já no século XIX por \citeonline{bastiat1859sofismas}, denunciava a miopia das políticas protecionistas e intervencionistas, sintetizada em sua célebre distinção entre ``o que se vê e o que não se vê''. O eleitor médio, no entanto, tende a interpretar lucros como exploração e concorrência como caos, ignorando os efeitos sistêmicos da coordenação via preços.

O \textit{viés antiestrangeiro} manifesta-se na desconfiança em relação ao comércio internacional, à imigração e à cooperação econômica global. O senso comum político frequentemente interpreta a importação de bens como ameaça à produção nacional e a presença de imigrantes como competição desleal no mercado de trabalho. Essa percepção, entretanto, ignora os ganhos de eficiência associados à especialização e à vantagem comparativa, conforme discutido por autores como \citeonline{bhagwati2003free} e \citeonline{landsburg2012armchair}. Historicamente, \citeonline{schumpeter1976capitalism} já advertia que o nacionalismo econômico tende a ressurgir em contextos de crise, alimentado por argumentos emocionais em detrimento da análise racional.

O \textit{viés antitrabalho} (ou \textit{make-work bias}, na formulação original) consiste na crença de que o objetivo central da economia deve ser preservar empregos, e não maximizar a produção ou a eficiência. Tal viés conduz ao apoio a políticas de proteção de setores ineficientes, de subsídios à manutenção artificial do emprego e de resistência à automação. Para \citeonline{sowell2004applied}, esse tipo de raciocínio falha ao ignorar que o progresso econômico consiste, justamente, em produzir mais com menos trabalho. Schumpeter, por sua vez, via a destruição criadora como essência do dinamismo capitalista — ideia que contraria frontalmente a aversão popular à substituição de empregos obsoletos por novas formas de produção.

O \textit{viés pessimista}, por fim, refere-se à percepção generalizada de que a economia está em constante deterioração, independentemente dos dados objetivos. Esse sentimento é frequentemente explorado por discursos populistas e reformistas que prometem ``salvar o país'' de uma crise iminente. Autores como \citeonline{Myths-of-Rich-and-Poor} e \citeonline{easterbrook2004progress} demonstram que, apesar de oscilações conjunturais, os indicadores de bem-estar material apresentaram melhora significativa nas últimas décadas. Ainda assim, a percepção negativa persiste, alimentada por vieses de confirmação e pela seletividade das narrativas midiáticas.

Esses vieses não apenas distorcem a compreensão dos fenômenos econômicos, mas também moldam as preferências políticas dos eleitores, influenciando diretamente a formulação de políticas públicas. Conforme sintetiza Caplan, ``na visão ingênua do interesse público, a democracia funciona porque faz o que os eleitores querem. Para a maioria dos céticos da democracia, ela falha porque não faz o que os eleitores querem. Na minha visão, a democracia falha justamente porque faz o que os eleitores querem'' \cite[p.~3]{The_Myth_of_the_Rational_Voter}. O custo dessa irracionalidade é amplamente socializado, o que desincentiva a revisão crítica de crenças e reforça a estabilidade dos erros cognitivos.

De modo geral, observa-se que o eleitor médio tende a interpretar os mecanismos de mercado com desconfiança, atribuindo-lhes motivações obscuras ou ilegítimas, ainda que sem fundamentação analítica ou respaldo empírico. Tal percepção, embora compreensível sob uma ótica emocional ou intuitiva, reflete padrões cognitivos enviesados que comprometem a avaliação racional de políticas econômicas. Compreender os vieses subjacentes a esse tipo de raciocínio é fundamental para diagnosticar as disfunções do processo democrático e para delimitar os alcances e limitações da deliberação pública em contextos marcados por racionalidade limitada.

\subsection{Vieses Auxiliares: Ideologia, Autoindulgência e o Público Esclarecido}
\label{sec:vieses_auxiliares}

Críticas à hipótese de que os eleitores mantêm crenças sistematicamente equivocadas muitas vezes apontam a suposta falta de neutralidade dos próprios economistas. Uma primeira objeção é que especialistas tenderiam a defender interesses materiais de sua própria classe, reproduzindo inconscientemente as estruturas ideológicas dominantes. Essa crítica parte da ideia de que instituições como a universidade operam como mecanismos de legitimação ideológica, moldando percepções que mantêm o status quo econômico e político \cite{althusser1971ideology, The_Myth_of_the_Rational_Voter}.

Outra objeção frequente alega que a formação econômica tradicional estaria imersa em valores liberais e pressupostos de mercado, comprometendo sua pretensa neutralidade científica. Argumenta-se que, longe de serem análises objetivas, os diagnósticos econômicos carregam elementos retóricos e culturais que favorecem determinadas visões de mundo \cite{mccloskey1998rhetoric, The_Myth_of_the_Rational_Voter}.

Ambas as críticas são relevantes e merecem ser enfrentadas. A melhor maneira de fazê-lo não é apelando à autoridade, mas confrontando as hipóteses com os dados \cite{popperlogic}. Uma estratégia metodológica eficaz consiste em simular um público que possua características socioeconômicas e ideológicas semelhantes às dos economistas, mas sem formação específica em economia. Se esse “público esclarecido” tender a responder como os economistas, isso sugere que a formação técnica exerce papel preponderante na forma como os indivíduos julgam questões econômicas — e não apenas sua renda, ideologia ou status.

A aplicação dessa abordagem mostra que, mesmo controlando para variáveis como renda, escolaridade e posicionamento político, a discordância entre economistas e o público geral persiste — mas desaparece quando se simula um público com o mesmo perfil, exceto pela ausência da formação econômica. Ou seja, as diferenças nas percepções não são explicadas apenas por ideologia ou conveniência, mas sim por conhecimento técnico sistematizado. O público em geral sustenta crenças incompatíveis com o consenso acadêmico não porque é moralmente inferior ou ideologicamente oposto, mas porque toma decisões sob a influência de distorções cognitivas previsíveis — e sem o instrumental técnico para perceber seus próprios erros.

Essa constatação reforça uma das hipóteses centrais deste trabalho: a irracionalidade política não é produto de má-fé ou de má-formação moral, mas de vieses sistemáticos que sobrevivem à escolarização geral e só começam a ser superados por um tipo específico de alfabetização econômica. O contraste entre o público comum, o público esclarecido e os especialistas permite, portanto, isolar o papel da formação econômica como variável de controle, oferecendo um argumento robusto contra explicações reducionistas que atribuem a discordância à ideologia ou ao elitismo.

\section{O Custo da Ignorância} % (Economia da Informação e Racionalidade Limitada)

% Discutir os impactos da desinformação e da racionalidade limitada no processo eleitoral e na formulação de políticas públicas.

A qualidade das decisões políticas em uma democracia depende, ao menos em parte, do nível de informação dos eleitores. No entanto, a literatura em economia política aponta que os cidadãos, em geral, demonstram baixo engajamento com a busca por conhecimento relevante sobre política e economia, o que compromete a racionalidade esperada de seu comportamento eleitoral. Esse fenômeno é abordado por \citeonline{downs1957economic} por meio do conceito de ignorância racional: dado que o custo de se informar adequadamente é alto e o impacto de um único voto é estatisticamente insignificante, os eleitores têm poucos incentivos para adquirir conhecimento político de qualidade.

A decisão de permanecer desinformado, nesse contexto, não é um erro de julgamento, mas uma escolha racional diante de incentivos inadequados. Essa formulação, embora coerente com a lógica econômica tradicional, revela limitações quando confrontada com evidências empíricas mais recentes. Estudos como os de \citeonline{The_Myth_of_the_Rational_Voter} indicam que os eleitores não apenas ignoram informações relevantes, mas também sustentam ativamente crenças falsas, mesmo quando confrontados com dados em contrário. A irracionalidade, nesse caso, não é um subproduto da ignorância passiva, mas um comportamento ativo de resistência à revisão de crenças.

Além disso, como argumenta \citeonline{hayek_knowledge_use}, a informação necessária para tomar decisões políticas eficazes é dispersa, tácita e situada nos agentes que vivenciam realidades econômicas específicas. A tentativa de centralizar tal conhecimento por meio de instituições políticas enfrenta limites intransponíveis, o que compromete a eficácia de políticas públicas baseadas em modelos excessivamente agregados ou descontextualizados. A ordem espontânea gerada pelo mercado — guiada por sinais de preços que condensam informação — é, nesse sentido, superior à coordenação política em muitos aspectos, justamente porque respeita a dispersão irredutível do conhecimento.

\citeonline{von1949human} enfatiza que, embora a racionalidade absoluta seja inatingível, ela deve ser preservada como um ideal regulativo. A ciência econômica, nesse sentido, deve se orientar pela busca contínua de maior coerência lógica e empírica, mesmo reconhecendo os limites cognitivos dos agentes. O problema central, portanto, não reside apenas na ausência de informação, mas na estrutura institucional que desincentiva a busca por verdade e promove a indulgência em crenças infundadas. Conforme argumenta Caplan, o problema da democracia não é apenas ignorância; para ele, ``os eleitores são piores que ignorantes: são, em uma palavra, irracionais — e votam de acordo'' \cite[p.~2]{The_Myth_of_the_Rational_Voter}.

O custo dessa ignorância deliberada transcende o plano individual e se manifesta em escolhas coletivas disfuncionais. Políticas ineficazes, protecionistas ou fiscalmente irresponsáveis são frequentemente resultado de crenças equivocadas, amplificadas por incentivos políticos de curto prazo. Quando o custo do erro é difuso e suas consequências são socializadas, torna-se racional para o indivíduo permanecer irracional.

Em última instância, o problema não é que o eleitor esteja mal informado; é que ele está racionalmente mal informado. Esse paradoxo, situado na interseção entre psicologia, teoria econômica e ciência política, desafia os modelos normativos de democracia e impõe limites práticos à qualidade das decisões públicas em contextos de racionalidade limitada.

\section{Do Sofista ao Populista} %  Como as Ideias Econômicas se Espalham (História do Pensamento Econômico e Propagação de Crenças)

% Explorar como ideias econômicas se propagam, influenciadas por interesses políticos, mídia e cultura.

A persistência de ideias econômicas equivocadas, mesmo diante de abundantes evidências contrárias, exige uma análise que vá além dos modelos tradicionais de aprendizagem racional. Muitos equívocos não decorrem apenas de desconhecimento, mas da forma como certas crenças são formadas, disseminadas e preservadas. A história do pensamento econômico ajuda a entender como erros conceituais ganham legitimidade social e fundamentam políticas públicas baseadas em argumentos intuitivos, morais ou retóricos — ainda que sem sustentação teórica rigorosa.

Esta seção propõe uma analogia entre formas históricas de persuasão política. O título ``Do Sofista ao Populista'' não deve ser lido como juízo moral, mas como recurso heurístico para ilustrar como argumentos emocionalmente sedutores, mas logicamente frágeis, ganham força no debate público. Trata-se de uma tradição crítica que busca evidenciar padrões recorrentes de difusão de crenças econômicas infundadas.

No século XIX, Bastiat já alertava para a ``miopia econômica'' induzida pela retórica política, distinguindo ``o que se vê e o que não se vê'' \cite{bastiat1859sofismas}. Efeitos imediatos de políticas como o protecionismo são politicamente valorizados, enquanto seus custos difusos e de longo prazo são ignorados ou ocultados. Para Bastiat, esse tipo de raciocínio constitui um sofisma: aparentemente plausível, mas embutido em falácias.

Autores como \citeonline{mark_history} e \citeonline{hart2019bastiat} defendem o uso da história das ideias como instrumento crítico. Para Blaug, ao abandonar essa dimensão, a economia perde sua capacidade autocrítica e se torna vulnerável à repetição de erros superados. A análise histórica, portanto, é uma ferramenta para entender como ideias se transformam em crenças e como estas moldam políticas.

Essa transformação é mediada por interesses políticos, cultura e estruturas de poder. \citeonline{schumpeter1976capitalism} observou que o dinamismo do capitalismo — fundado na inovação e substituição constante — gera ressentimento social e apelos nostálgicos por estabilidade. Esse sentimento alimenta discursos populistas que rejeitam os fundamentos da concorrência e da globalização, prometendo uma volta ilusória ao passado.

Nesse sentido, o populismo contemporâneo pode ser visto como uma reedição moderna do sofismo: ambos manipulam a linguagem, simplificam problemas complexos e oferecem soluções intuitivas, porém equivocadas. Se antes o sofista usava a retórica nas ágoras, hoje o populista usa algoritmos e redes sociais. Ambos priorizam apelo simbólico em detrimento da lógica e da evidência.

Mecanismos culturais e midiáticos também reforçam certas narrativas e marginalizam discursos dissonantes. Como nota \citeonline{franco2022cartas}, o economista que busca comunicar ideias impopulares enfrenta não apenas resistência cognitiva, mas também um ambiente institucional hostil à complexidade. O senso comum, moldado por intuição, moralismo e ideologia, opõe-se à crítica técnica, ainda que esta seja sólida.

Robert Higgs contribui com a ideia de ``consentimento ideológico tácito'' \cite{higgs1987crisis}, explicando como crenças populares, mesmo ineficientes, se transformam em políticas institucionais. Compreender esse processo requer atenção às estruturas culturais e institucionais que moldam a relação entre crença e política.

Assim, o estudo da difusão de ideias econômicas exige uma abordagem histórica, cultural e institucional. A racionalidade do eleitor — limitada por vieses e desinformação — encontra na linguagem política e na cultura midiática um terreno fértil para a reprodução de erros. A democracia, nesse quadro, não apenas reflete crenças populares, mas as amplifica — convertendo intuições falaciosas em consenso legislativo.


\section{Preferência por Crenças e Resistência ao Conhecimento}

A manutenção de crenças econômicas disfuncionais, mesmo frente a evidências disponíveis, não pode ser explicada apenas por ignorância racional \cite{downs1957economic}. Diversos estudos sugerem que indivíduos mantêm crenças políticas com base em fatores emocionais, identitários e sociais, distantes de critérios epistêmicos. Trata-se de uma resistência ativa ao conhecimento — racional no curto prazo individual, mas irracional no coletivo.

Caplan argumenta que crenças políticas funcionam como bens de consumo simbólico: geram utilidade subjetiva independentemente de sua veracidade \cite{The_Myth_of_the_Rational_Voter}. Como o custo da crença é socializado e os benefícios emocionais são internos, os incentivos favorecem ideias confortáveis, ainda que falsas.

Kahneman e Tversky descrevem vieses como o da confirmação e ancoragem, que dificultam a revisão de crenças mesmo diante de novas evidências \cite{kahneman2011thinking}. Some-se a isso o \textit{backfire effect}, descrito por Nyhan e Reifler, segundo o qual argumentos contrários reforçam a crença original em contextos polarizados \cite{nyhan2010when}.

A teoria da cognição cultural mostra que indivíduos tendem a filtrar informações de acordo com os valores predominantes em seus grupos sociais, priorizando a coesão identitária em detrimento da precisão factual \cite{kahan2012polarization}. Esse padrão se intensifica em contextos polarizados, nos quais os cidadãos selecionam narrativas que reforçam suas convicções prévias, mesmo quando estas entram em conflito com dados objetivos. A estrutura informacional contemporânea, marcada por algoritmos e bolhas epistêmicas, favorece o isolamento cognitivo e a evasão da dissonância, dificultando o confronto com perspectivas divergentes \cite{sunstein2017republic}.

Esse tipo de resistência ao conhecimento não é exclusividade da era digital. Pensadores clássicos já advertiam contra a rigidez das crenças não fundamentadas. Locke, por exemplo, associava o excesso de certeza a posicionamentos motivados mais por afeto do que por evidência racional \cite{locke2014ensaio}. De forma mais incisiva, a filosofia objetivista descreve o \textit{blanking out} como a recusa deliberada de reconhecer aquilo que contradiz uma crença internalizada — “não ser cego, mas recusar-se a ver” \cite[p.~869]{rand2012revolta}. Essa cegueira voluntária opera como um escudo cognitivo que protege o indivíduo do desconforto gerado pela inconsistência entre suas crenças e a realidade.

Além dos fatores individuais, há elementos institucionais. Algoritmos de redes sociais e tribalismo político criam bolhas epistêmicas que dificultam o contato com visões divergentes. Zaller aponta que a recepção de argumentos depende da disposição prévia do eleitor, tornando o debate público mais reforçador do que crítico \cite{zaller1992nature}.

Em síntese, a resistência ao conhecimento é racional no nível individual, mas prejudica a racionalidade coletiva. Isso limita a eficácia das políticas baseadas em evidência, pois o eleitor médio age não como cientista, mas como consumidor de narrativas. Para que suas crenças políticas sejam tratadas como hipóteses legítimas, elas deveriam estar sujeitas à refutação — o que raramente ocorre no jogo político.

