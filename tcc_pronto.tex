%% abtex2-modelo-trabalho-academico.tex, v-1.9.6 laurocesar
%% Copyright 2012-2016 by abnTeX2 group at http://www.abntex.net.br/ 
%%
%% This work may be distributed and/or modified under the
%% conditions of the LaTeX Project Public License, either version 1.3
%% of this license or (at your option) any later version.
%% The latest version of this license is in
%%   http://www.latex-project.org/lppl.txt
%% and version 1.3 or later is part of all distributions of LaTeX
%% version 2005/12/01 or later.
%%
%% This work has the L PPL maintenance status `maintained'.
%% 
%% The Current Maintainer of this work is the abnTeX2 team, led
%% by Lauro César Araujo. Further information are available on 
%% http://www.abntex.net.br/
%%
%% This work consists of the files abntex2-modelo-trabalho-academico.tex,
%% abntex2-modelo-include-comandos and abntex2-modelo-references.bib
% ------------------------------------------------------------------------
% ------------------------------------------------------------------------
% abnTeX2: Modelo de Trabalho Academico (tese de doutorado, dissertacao de
% mestrado e trabalhos monograficos em geral) em conformidade com 
% ABNT NBR 14724:2011: Informacao e documentacao - Trabalhos academicos -
% Apresentacao
% ------------------------------------------------------------------------
% ------------------------------------------------------------------------
% Personalização para o modelo Udesc 2024 9. ed. revisada e modificada
% Manual_13_05_2024_17175220258266_12510.pdf acesso em: 13/08/2024
% Autor: Felipe Joel Zimann (felipezimann@hotmail.com)
% Data: 02/12/2020 v1.0
% Data: 13/02/2021 v1.0.1 alterado tamanho numeração da página para 10pt
% Data: 13/08/2024 v1.0.2 alterado para citação (Autor, Ano) ao invés de (AUTOR, Ano) conforme ABNT NBR 10520:2023
% ------------------------------------------------------------------------
% ------------------------------------------------------------------------

\documentclass[
	12pt,					% tamanho da fonte
	openright,				% capítulos começam em pág ímpar (insere página vazia caso preciso)
	oneside,				% para impressão em recto e verso (twoside). Oposto a (oneside)
	a4paper,				% tamanho do papel. 
	chapter=TITLE,			% títulos de capítulos convertidos em letras maiúsculas
	section=TITLE,			% títulos de seções convertidos em letras maiúsculas
	sumario=abnt-6027-2012,
	english,				% idioma adicional para hifenização
	brazil,					% o último idioma é o principal do documento
	fleqn,					% equações alinhadas a esquerda (UDESC/CCT)+
	]{abntex2}

% ----------------------------------------------------------
% Pacotes básicos 
% ----------------------------------------------------------
\usepackage{amsmath}							% Pacote matemático
\usepackage{amssymb}							% Pacote matemático
\usepackage{amsfonts}							% Pacote matemático
%\usepackage{lmodern}							% Usa a fonte Latin Modern		
\usepackage{mathptmx} 							% Usa a fonte Times New Roman	 (UDESC/CCT)
\usepackage[T1]{fontenc}						% Selecao de codigos de fonte.
\usepackage[utf8]{inputenc}						% Codificacao do documento (conversão automática dos acentos)
\usepackage{lastpage}							% Usado pela Ficha catalográfica
\usepackage{indentfirst}						% Indenta o primeiro parágrafo de cada seção.
\usepackage[dvipsnames,table]{xcolor}			% Controle das cores
\usepackage{graphicx}							% Inclusão de gráficos
\usepackage{microtype} 							% para melhorias de justificação
\usepackage{lipsum}								% para geração de dummy text
\usepackage[brazilian,hyperpageref]{backref}	% Paginas com as citações na bibl
\usepackage[alf,abnt-emphasize=bf,abnt-full-initials=yes]{abntex2cite}
\usepackage{booktabs}
\usepackage{url} % se ainda não estiver ativo
\citeoption{abnt-url-package=url}

%\usepackage[num]{abntex2cite}					% Citações padrão ABNT numérica
\usepackage{adjustbox}							% Pacote de ajuste de boxes
\usepackage{subcaption}							% Inclusão de Subfiguras e sublegendas		
\usepackage{enumitem}							% Personalização de listas
\usepackage{siunitx}							% Grandezas e unidades
\usepackage[section]{placeins}					% Manter as figuras delimitadas na respectiva seção com a opção [section]
\usepackage{multirow}							% Multi colunas nas tabelas
\usepackage{array,tabularx} 					% Pacotes de tabelas
\usepackage{rotating}							% Rotacionar figuras e tabelas
\usepackage{xfrac}								% Fazer frações n/d em linha
\usepackage{bm}									% Negrito em modo matemático
\usepackage{xstring}							% Manipulação de strings
\usepackage{pgfplots}							% Pacote de Gráficos
\usepackage{tikz}								% Pacote de Figuras
\usepackage[american, cuteinductors,smartlabels, fulldiode, siunitx, americanvoltages, oldvoltagedirection, smartlabels]{circuitikz}						% Pacote de circuitos elétricos
\usepackage{chemformula}						% Pacote para fórmulas químicas
\usepackage{chngcntr}							% Pacte usado para deixar numeração de equações sequencial (UDESC/CCT)
\usepackage{caption}
\usepackage{makecell} % Add this line at the beginning of the document
\usepackage{array}
\usepackage{ragged2e}
\usepackage{pdflscape}
\usepackage{threeparttable}
\usepackage{threeparttablex}
\usepackage{ltablex}
\usepackage{csvsimple}    % lê CSV
\usepackage{booktabs}     % linhas bonitas
\usepackage{longtable}    % tabelas que quebram página
% \usepackage{float}
\counterwithout{equation}{chapter}
% fonte: https://latex.org/forum/viewtopic.php?t=15392

% Comando para deixar numeração das equações contínua (1), (2), (3)... ao invés de organizar por capítulos (1.1)(1.2)... (2.1)(2.2)
%\renewcommand{\theequation}{\arabic{equation}}

%\numberwithin{equation}{section}

\usepackage[utf8]{inputenc}
\usepackage[T1]{fontenc}
\DeclareUnicodeCharacter{2212}{\ensuremath{-}} % “−” → sinal de menos
\DeclareUnicodeCharacter{2013}{--}             % “–” → en-dash

% depois de babel[brazilian] e hyperref
\addto\extrasbrazilian{%
  \renewcommand{\sectionautorefname}{Seção}%
  \renewcommand{\subsectionautorefname}{Seção}%
  \renewcommand{\appendixname}{Apêndice}%
  \renewcommand{\tableautorefname}{Tabela}%
  \renewcommand{\figureautorefname}{Figura}%
}

\newcommand{\notafig}{\newline Fonte: elaboração do autor com base nos modelos da \autoref{sec:modelo-estatistico} (contrafactual da \autoref{sec:publico_esclarecido})}%


% Cabecalho cabeçalho somente com numeração de página 10pt
\makepagestyle{PagNumReduzida}
\makeevenhead{PagNumReduzida}{\ABNTEXfontereduzida\thepage}{}{}
\makeoddhead{PagNumReduzida}{}{}{\ABNTEXfontereduzida\thepage}
%fonte: https://github.com/abntex/abntex2/wiki/HowToCustomizarCabecalhoRodape
%fonte: Manual memoir seção 7.3 pg. 111 pdf http://linorg.usp.br/CTAN/macros/latex/contrib/memoir/memman.pdf 

% Personalização das opções das listas
\setlist[itemize]{leftmargin=\parindent}

% Citação online --- MODIFICAR ---
\newcommand{\citeshort}[1]{\citeauthoronline{#1}~(\citeyear{#1})}

\newcommand{\me}[1]{Elaborado pelo autor (#1).}

% Configuração do pgfplots
\pgfplotsset{compat=newest} %compat=1.14
\pgfplotsset{plot coordinates/math parser=false} 
\newlength\figureheight 
\newlength\figurewidth 

% Libraries do TiKz
\usetikzlibrary{quotes,angles,arrows}
\usetikzlibrary{through,calc,math}
\usetikzlibrary{graphs,backgrounds,fit}
\usetikzlibrary{shapes,positioning,patterns,shadows}
\usetikzlibrary{decorations.pathreplacing}
\usetikzlibrary{shapes.geometric}
\usetikzlibrary{arrows.meta}
\usetikzlibrary{external}

%\tikzexternalize[]
%\tikzexternalenable
%\tikzexternalize
%\tikzexternaldisable
%\tikzset{external/force remake}
%\tikzexternalize[shell escape=-enable-write18]

% Configurações do CircuiTiKz
\ctikzset{bipoles/thickness=1}
%\ctikzset{bipoles/length=1.2cm}
\ctikzset{monopoles/ground/width/.initial=.2}
\ctikzset{bipoles/resistor/height=0.25}
\ctikzset{bipoles/resistor/width=0.6}
\ctikzset{bipoles/capacitor/height=0.5}
\ctikzset{bipoles/capacitor/width=0.15}
\ctikzset{bipoles/generic/height=0.25}
\ctikzset{bipoles/generic/width=0.6}
%\ctikzset{bipoles/capacitor polar/length=0.5}
%\ctikzset{bipoles/diode/height=.375}
%\ctikzset{bipoles/diode/width=.3}
%\ctikzset{tripoles/thyristor/height=.8}
%\ctikzset{tripoles/thyristor/width=1}
\ctikzset{bipoles/vsourcesin/height=.5}
\ctikzset{bipoles/vsourcesin/width=.5}
\ctikzset{bipoles/cvsourceam/height=.6}
\ctikzset{bipoles/cvsourceam/width=.6}
%\ctikzset{tripoles/european controlled voltage source/width=.4}

\tikzstyle{every node}=[font=\footnotesize]
\tikzstyle{every path}=[line width=0.25pt,line cap=round,line join=round]
%\tikzstyle{every path}=[line cap=round,line join=round]


% Definição de cores MATLAB
\definecolor{matlab_blue}{rgb}	{         0,    0.4470,    0.7410}
\definecolor{matlab_orange}{rgb}{    0.8500,    0.3250,    0.0980}
\definecolor{matlab_yellow}{rgb}{    0.9290,    0.6940,    0.1250}
\definecolor{matlab_violet}{rgb}{    0.4940,    0.1840,    0.5560}
\definecolor{matlab_green}{rgb}	{	 0.4660,    0.6740,    0.1880}
\definecolor{matlab_lblue}{rgb}	{    0.3010,    0.7450,    0.9330}
\definecolor{matlab_red}{rgb}	{    0.6350,    0.0780,    0.1840}

% Personalização das legendas
\usepackage[format = plain, %hang
			justification = centering,
			labelsep = endash,
			singlelinecheck = false,
			skip = 6pt,
			listformat = simple]{caption}	

% Personalização das unidades
\sisetup{output-decimal-marker = {,}}
\sisetup{exponent-product = \cdot}
\sisetup{tight-spacing=true}
\sisetup{group-digits = false}

% Personalizações de tipo de colunas de tabelas
\newcolumntype{L}[1]{>{\raggedright\let\newline\\\arraybackslash\hspace{0pt}}m{#1}}
\newcolumntype{C}[1]{>{\centering\let\newline\\\arraybackslash\hspace{0pt}}m{#1}}
\newcolumntype{R}[1]{>{\raggedleft\let\newline\\\arraybackslash\hspace{0pt}}m{#1}}

% Personalizações de cores da UDESC
\definecolor{CapaAmareloUDESC}{RGB}{243,186,83}		% Especializacao
\definecolor{CapaVerdeUDESC}{RGB}{0,112,52}			% Mestrado
\definecolor{CapaVermelhoUDESC}{RGB}{171,35,21}		% Doutorado
\definecolor{CapaAzulUDESC}{RGB}{38,54,118} 		% Pós-Doutorado

% CONFIGURAÇÕES DE PACOTES
% Configurações do pacote backref
% Usado sem a opção hyperpageref de backref
\renewcommand{\backrefpagesname}{}
% \renewcommand{\backrefpagesname}{Citado na(s) página(s):~}
% Texto padrão antes do número das páginas
\renewcommand{\backref}{}
% Define os textos da citação
% \renewcommand*{\backrefalt}[4]{
% 	\ifcase #1 %
% 	Nenhuma citação no texto.%
% 	\or
% 	Citado na página #2.%
% 	\else
% 	Citado #1 vezes nas páginas #2.%
% 	\fi}%

\renewcommand*{\backrefalt}[4]{}%

% alterando o aspecto da cor azul
%\definecolor{blue}{RGB}{41,5,195}

% informações do PDF
\makeatletter
\hypersetup{
	%pagebackref=true,
	pdftitle={\@title}, 
	pdfauthor={\@author},
	pdfsubject={\imprimirpreambulo},
	pdfcreator={LaTeX with abnTeX2},
	pdfkeywords={abnt}{latex}{abntex}{abntex2}{trabalho academico}, 
	colorlinks=true,       		% false: boxed links; true: colored links
	linkcolor=black,          	% color of internal links
	citecolor=black,        	% color of links to bibliography
	filecolor=black,      		% color of file links
	urlcolor=black,
	bookmarksdepth=4
}
\makeatother


\makeatletter
\newcommand{\includetikz}[1]{%
	\tikzsetnextfilename{#1}%
	\input{#1.tex}%
}
\makeatother


% ---
% Possibilita criação de Quadros e Lista de quadros.
% Ver https://github.com/abntex/abntex2/issues/176
%
\newcommand{\quadroname}{Quadro}
\newcommand{\listofquadrosname}{Lista de quadros}

\newfloat[chapter]{quadro}{loq}{\quadroname}
\newlistof{listofquadros}{loq}{\listofquadrosname}
\newlistentry{quadro}{loq}{0}

% configurações para atender às regras da ABNT
\setfloatadjustment{quadro}{\centering}
\counterwithout{quadro}{chapter}
\renewcommand{\cftquadroname}{\quadroname\space} 
\renewcommand*{\cftquadroaftersnum}{\hfill--\hfill}

\setfloatlocations{quadro}{hbtp} % Ver https://github.com/abntex/abntex2/issues/176
% ---


% Espaçamento depois do título
\setlength{\afterchapskip}{0.7\baselineskip}
% O tamanho do parágrafo é dado por:
\setlength{\parindent}{1.25cm}
% Controle do espaçamento entre um parágrafo e outro:
\setlength{\parskip}{0.0cm}  % tente também \onelineskip
%\SingleSpacing % Espaçamento simples 
\OnehalfSpacing % Espaçamento 1,5 (UDESC/CCT)
%\DoubleSpacing	% Espaçamento duplo

% ---
% Margens - NBR 14724/2011 - 5.1 Formato
% ---
\setlrmarginsandblock{3cm}{2cm}{*}
\setulmarginsandblock{3cm}{2cm}{*}
\checkandfixthelayout[fixed]
% ---


% To use externalize consider
%https://tex.stackexchange.com/questions/182783/tikzexternalize-not-compatible-with-miktex-2-9-abntex2-package
%Lauro Cesar digged into the problem until he came with a solution for me to test. And it Works!
%
%According to this link:
%
%The package calc changed the commands \setcounter and friends to be fragile. So you have to make them robust. The example below uses etoolbox with \robustify:
%
\usepackage{etoolbox}
\robustify\setcounter
\robustify\addtocounter
\robustify\setlength
\robustify\addtolength


%% How to silence memoir class warning against the use of caption package?
%% https://tex.stackexchange.com/questions/391993/how-to-silence-memoir-class-warning-against-the-use-of-caption-package
%\usepackage{silence}
%\WarningFilter*{memoir}{You are using the caption package with the memoir class}
%\WarningFilter*{Class memoir Warning}{You are using the caption package with the memoir class}

% --------------------------------------------------------
% INICIO DAS CUSTOMIZACOES PARA A UDESC
% --------------------------------------------------------

% --------------------------------------------------------
% Fontes padroes de part, chapter, section, subsection e subsubsection
% --------------------------------------------------------
% --- Chapter ---
\renewcommand{\ABNTEXchapterfont}{\fontseries{b}} %\bfseries
\renewcommand{\ABNTEXchapterfontsize}{\normalsize}
% --- Part ---
\renewcommand{\ABNTEXpartfont}{\ABNTEXchapterfont}
\renewcommand{\ABNTEXpartfontsize}{\LARGE}
% --- Section ---
\renewcommand{\ABNTEXsectionfont}{\normalfont}
\renewcommand{\ABNTEXsectionfontsize}{\normalsize}
% --- SubSection ---
\renewcommand{\ABNTEXsubsectionfont}{\fontseries{b}} %\bfseries
\renewcommand{\ABNTEXsubsectionfontsize}{\normalsize}
% --- SubSubSection ---
\renewcommand{\ABNTEXsubsubsectionfont}{\itshape}
\renewcommand{\ABNTEXsubsubsectionfontsize}{\normalsize}

\renewcommand{\ABNTEXsubsubsubsectionfont}{\normalfont}
\renewcommand{\ABNTEXsubsubsubsectionfontsize}{\normalsize}
% ---

% --------------------------------------------------------
% Fontes das entradas do sumario
% --------------------------------------------------------

\renewcommand{\cftpartfont}{\ABNTEXpartfont\selectfont}
\renewcommand{\cftpartpagefont}{\normalsize\selectfont}

\renewcommand{\cftchapterfont}{\ABNTEXchapterfont\selectfont}
\renewcommand{\cftchapterpagefont}{\normalsize\selectfont}

\renewcommand{\cftsectionfont}{\ABNTEXsectionfont\selectfont}
\renewcommand{\cftsectionpagefont}{\normalsize\selectfont}

\renewcommand{\cftsubsectionfont}{\ABNTEXsubsectionfont\selectfont}
\renewcommand{\cftsubsectionpagefont}{\normalsize\selectfont}

\renewcommand{\cftsubsubsectionfont}{\normalfont\itshape\selectfont}
\renewcommand{\cftsubsubsectionpagefont}{\normalsize\selectfont}

\renewcommand{\cftparagraphfont}{\normalfont\selectfont}
\renewcommand{\cftparagraphpagefont}{\normalsize\selectfont}

% --------------------------------------------------------
% Usando os pacotes hyperref, uppercase... 
% Para deixar a section do toc uppercase precisa de:
% --------------------------------------------------------
\usepackage{textcase}

\makeatletter

\let\oldcontentsline\contentsline
\def\contentsline#1#2{%
	\expandafter\ifx\csname l@#1\endcsname\l@section
	\expandafter\@firstoftwo
	\else
	\expandafter\@secondoftwo
	\fi
	{%
		\oldcontentsline{#1}{\MakeTextUppercase{#2}}%
	}{%
		\oldcontentsline{#1}{#2}%
	}%
}
\makeatother

% --------------------------------------------------------
% Renomenando as entradas de APÊNDICES E ANEXOS
% --------------------------------------------------------

\renewcommand{\apendicesname}{AP\^ENDICES}
\renewcommand{\anexosname}{ANEXOS}


% Manipulação de Strings
%\RequirePackage{xstring}

% Comando para inverter sobrenome e nome
\newcommand{\invertname}[1]{%
	\StrBehind{#1}{{}}, \StrBefore{#1}{{}}%
}%


% --------------------------------------------------------
% Alterando os estilos de Caption e Fonte
% --------------------------------------------------------
\makeatletter
% Define o comando \fonte que respeita as configurações de caption do memoir ou do caption
\renewcommand{\fonte}[2][\fontename]{%
	\M@gettitle{#2}%
	\memlegendinfo{#2}%
	\par
	\begingroup
	\@parboxrestore
	\if@minipage
	\@setminipage
	\fi
	\ABNTEXfontereduzida
	\configureseparator
	\captiondelim{\ABNTEXcaptionfontedelim}
	\@makecaption{#1}{\ignorespaces #2}\par
	\endgroup}


\captionstyle[\raggedright]{\raggedright}

\makeatother

\setlength{\cftbeforechapterskip}{0pt plus 0pt}
\renewcommand*{\insertchapterspace}{}

\newlength{\mylen}	% New length to use with spacing
\setlength{\mylen}{1pt}

\setlength{\cftbeforechapterskip}{\mylen}
\setlength{\cftbeforesectionskip}{\mylen}
\setlength{\cftbeforesubsectionskip}{\mylen}
\setlength{\cftbeforesubsubsectionskip}{\mylen}
\setlength{\cftbeforesubsubsubsectionskip}{\mylen}


% ---
% Ajuste das listas de abreviaturas e siglas ; e símbolos [Personalizada para UDESC com espaçamento 1,5]
% ---

% ---
% Redefinição da Lista de abreviaturas e siglas [Personalizada para UDESC com espaçamento 1,5]
\renewenvironment{siglas}{%
	\pretextualchapter{\listadesiglasname}
	\begin{symbols} 
		\setlength{\itemsep}{0pt}	% Ajuste para Espaçamento 1,5 (UDESC/CCT)
	}{% 
	\end{symbols}
	\cleardoublepage
}
% ---

% ---
% Redefinição da Lista de símbolos [Personalizada para UDESC com espaçamento 1,5]
\renewenvironment{simbolos}{%
	\pretextualchapter{\listadesimbolosname}
	\begin{symbols}
		\setlength{\itemsep}{0pt}	% Ajuste para Espaçamento 1,5 (UDESC/CCT)
	}{%
	\end{symbols}
	\cleardoublepage
}
% ---


% ---
% Remocao dos simbolos de < > das urls, ver manual pacote url pg 6 item 6
% https://github.com/abntex/biblatex-abnt/issues/16
\def\UrlLeft{}
\def\UrlRight{}
% ---

% ---
% FIM DAS CUSTOMIZACOES PARA A  Universidade do Estado de Santa Catarina - UDESC/CCT
% ---





	% Incliu pacotes básicos 

% -----------------------------------------------------------------
% Você pode adicionar seus pacotes a partir desta linha;
% -----------------------------------------------------------------

%\usepackage[showframe,pass]{geometry}
%\usepackage[11,12]{pagesel}

% -----------------------------------------------------------------
% Informações de dados para CAPA e FOLHA DE ROSTO
% -----------------------------------------------------------------
\titulo{Econometria Comportamental na Política Pública: O Impacto dos Vieses Cognitivos nas Escolhas Econômicas e Políticas}%

\autor{Bruno Francisco {}Schaden}%
\orientador{Marianne Zwiling {}Stampe}%
% \coorientador{Arnold Alois {}Schwarzenegger}%

% ATENÇÃO: O símbolo {} indica o sobrenome para a ficha catalográfica.
% Exemplo: Sherlock Holmes {}da Silva para sobrenomes compostos;
% Exemplo: Arnold Alois {}Schwarzenegger para sobrenome simples.

\instituicao{Universidade do Estado de Santa Catarina, Centro de Ciências da Administração e Socioeconômicas, Graduação em Ciências Econômicas}%

%\tipotrabalho{Tese (Doutorado)}
\tipotrabalho{Dissertação (Graduação)}

%\preambulo{Tese apresentada ao Programa de Pós--Graduação em Engenharia Elétrica do Centro de Ciências Tecnológicas da Universidade do Estado de Santa Catarina, como requisito parcial para a obtenção do grau de Doutor em Engenharia Elétrica.}

\preambulo{Trabalho de Conclusão de Curso apresentado ao Curso de Graduação em Ciências Econômicas do Centro de Ciências da Administração e Socioeconômicas da Universidade do Estado de Santa Catarina, como requisito parcial para a obtenção do grau de Bacharel em Ciências Econômicas.}

\local{Florianópolis}%

\data{\the\year}%
% ---

% compila o indice
\makeindex

% -----------------------------------------------------------------
% Início do documento
% -----------------------------------------------------------------
\begin{document}

\selectlanguage{brazil}
\frenchspacing  % Retira espaço extra obsoleto entre as frases.

% -----------------------------------------------------------------
% ELEMENTOS PRÉ-TEXTUAIS
% -----------------------------------------------------------------
\pretextual

% Você pode comentar os elementos que não deseja em seu trabalho;

% A capa pode ser escolhida dentro do arquivo Capa.tex (TCC, Master, Doc, ...)
% ---
% Capa
% ---


% --------------------------------------------------------
% Capa Padrão
% --------------------------------------------------------
\renewcommand{\imprimircapa}{%
	\begin{capa}%
		\center

		{\fontseries{b}\selectfont\MakeTextUppercase{UNIVERSIDADE DO ESTADO DE SANTA CATARINA -- UDESC}}
		
		{\fontseries{b}\selectfont\MakeTextUppercase{CENTRO DE CIÊNCIAS DA ADMINISTRAÇÃO E SOCIOECONÔMICAS -- ESAG  }}
		
		{\fontseries{b}\selectfont\MakeTextUppercase{GRADUAÇÃO -- CIÊNCIAS ECONÔMICAS  }}
		
		\vfill
		
		{\fontseries{b}\selectfont\MakeTextUppercase{\normalsize\imprimirautor}}
		
		\vfill
		\begin{center}
			{\fontseries{b}\selectfont\MakeTextUppercase{\imprimirtitulo}}
		\end{center}
		\vfill
		
		\vfill
		
		{\fontseries{b}\selectfont\MakeTextUppercase{\imprimirlocal}}
		\par
		{\fontseries{b}\selectfont \imprimirdata}
		\vspace*{1cm}
	\end{capa}
}



\imprimircapa				% Capa padrão

					% Elemento Obrigatório
% ---
% Folha de rosto
% ---








% --------------------------------------------------------
% folha de rosto 
% --------------------------------------------------------

\makeatletter

\renewcommand{\folhaderostocontent}{
	\begin{center}
		
		{\fontseries{b}\selectfont\MakeTextUppercase{\imprimirautor}}
		
		\vfill
		
		\begin{center}
			{\fontseries{b}\selectfont\MakeTextUppercase{\imprimirtitulo}}
		\end{center}
	
		\vspace*{1.5cm}

		\abntex@ifnotempty{\imprimirpreambulo}{%
			\hspace{.45\textwidth}
			{\begin{minipage}{.5\textwidth}
					\SingleSpacing
					\imprimirpreambulo\par
					\vspace*{4pt}
					{\imprimirorientadorRotulo~\imprimirorientador\par}
					\abntex@ifnotempty{\imprimircoorientador}{%
						{\imprimircoorientadorRotulo~\imprimircoorientador}%
					}%
			\end{minipage}}%
		}%
	
		
		\vfill
		
	{\fontseries{b}\selectfont\MakeTextUppercase{\imprimirlocal}}
	\par
	{\fontseries{b}\selectfont \imprimirdata}
	\vspace*{1cm}
	\end{center}
}


% (o * indica que haverá a ficha bibliográfica)
% ---
\imprimirfolhaderosto*
% ---


			% Elemento Obrigatório
% Caso não utilize a Ficha Catalográfica entre na folha de rosto e retire o * de dentro do arquivo FolhadeRosto
% 
% ---
% Inserir a ficha bibliografica
% ---

% Isto é um exemplo de Ficha Catalográfica, ou ``Dados internacionais de
% catalogação-na-publicação''. Você pode utilizar este modelo como referência. 
% Porém, provavelmente a biblioteca da sua universidade lhe fornecerá um PDF
% com a ficha catalográfica definitiva após a defesa do trabalho. Quando estiver
% com o documento, salve-o como PDF no diretório do seu projeto e substitua todo
% o conteúdo de implementação deste arquivo pelo comando abaixo:



% \begin{fichacatalografica}
%     \includepdf{fig_ficha_catalografica.pdf}
% \end{fichacatalografica}


%	\setlength{\parindent}{0cm}
%	\setlength{\parskip}{0pt}
\begin{fichacatalografica}
	%\sffamily
	%\rmfamily
	%\ttfamily 
	\hbadness=10000
	\vspace*{\fill}					% Posição vertical
	\begin{center}					% Minipage Centralizado
		Para gerar a ficha catalográfica de teses e \\
		dissertações acessar o link:  \\
		https://www.udesc.br/bu/manuais/ficha

		\vspace*{8pt}

		%	\begin{minipage}[c]{8cm}
		%	\centering \sffamily
		%	 Ficha catalográfica elaborada pelo(a) autor(a), com auxílio do programa de geração automática da Biblioteca Setorial do CCT/UDESC
		%	\end{minipage}
		\fbox{\begin{minipage}[c]{13.5cm}		% Largura
				\flushright
				{\begin{minipage}[c]{10.5cm}		% Largura
						\vspace{1.25cm}
						%\footnotesize
						\setlength{\parindent}{1.5em}
						\noindent \invertname{\imprimirautor} \par
						\imprimirtitulo{ }/{ }\imprimirautor. -- \imprimirlocal, \imprimirdata .\par
						\pageref{LastPage} p. : il. \par
						\vspace{1.5em}
						\imprimirorientadorRotulo~\imprimirorientador.\par
						\imprimircoorientadorRotulo~\imprimircoorientador.\par
						\imprimirtipotrabalho~--~\imprimirinstituicao, \imprimirlocal, \imprimirdata.\par
						\vspace{1.5em}
						1. Vieses de julgamento.
						2. Economia política comportamental.
						3. Crenças econômicas.
						4. Escolhas políticas.
						5. Educação econômica.
						I. \invertname{\imprimirorientador}.
						% II. \invertname{\imprimircoorientador}.
						II. \imprimirinstituicao.
						III. Título. %
						\vspace{1.25cm}	%		
					\end{minipage}%
				}% 
				\hspace{10mm}
			\end{minipage}}%

		\vspace*{0.5cm}

	\end{center}
\end{fichacatalografica}


%\begin{fichacatalografica}
%	\sffamily
%	\vspace*{\fill}					% Posição vertical
%	\begin{center}					% Minipage Centralizado
%	\fbox{\begin{minipage}[c][8cm]{13.5cm}		% Largura
%	\small
%	\imprimirautor
%	%Sobrenome, Nome do autor
%	
%	\hspace{0.5cm} \imprimirtitulo  / \imprimirautor. --
%	\imprimirlocal, \imprimirdata-
%	
%	\hspace{0.5cm} \pageref{LastPage} p. : il. (algumas color.) ; 30 cm.\\
%	
%	\hspace{0.5cm} \imprimirorientadorRotulo~\imprimirorientador\\
%	
%	\hspace{0.5cm}
%	\parbox[t]{\textwidth}{\imprimirtipotrabalho~--~\imprimirinstituicao,
%	\imprimirdata.}\\
%	
%	\hspace{0.5cm}
%		1. Palavra-chave1.
%		2. Palavra-chave2.
%		3. Palavra-chave3.
% 		4. Palavra-chave4.
%		5. Palavra-chave5.
%		I. Orientador.
%		II. Universidade xxx.
%		III. Faculdade de xxx.
%		IV. Título 			
%	\end{minipage}}
%	\end{center}
%\end{fichacatalografica}
% ---

	% Elemento Obrigatório (Verso da Folha)
% 
% ---
% Inserir errata
% ---
\begin{errata}
Elemento opcional. 

Exemplo:

\vspace{\onelineskip}

Sobrenome, Prenome do Autor. Título de obra: subtítulo (se houver). Ano de depósito. Tipo do trabalho (grau e curso) - Vinculação acadêmica, local de apresentação/defesa, data.

\begin{table}[htb]
\center
\begin{tabular}{|p{2.4cm}|p{2cm}|p{3cm}|p{3cm}|}
  \hline
   \textbf{Folha} & \textbf{Linha}  & \textbf{Onde se lê}  & \textbf{Leia-se}  \\
    \hline
    1 & 10 & auto-conclavo & autoconclavo\\
   \hline
\end{tabular}
\end{table}

\end{errata}
% ---				% Elemento Opcional

% ---
% Inserir folha de aprovação
% ---

% Isto é um exemplo de Folha de aprovação, elemento obrigatório da NBR
% 14724/2011 (seção 4.2.1.3). Você pode utilizar este modelo até a aprovação
% do trabalho. Após isso, substitua todo o conteúdo deste arquivo por uma
% imagem da página assinada pela banca com o comando abaixo:
%
% \includepdf{folhadeaprovacao_final.pdf}
%
\begin{folhadeaprovacao}



	\begin{center}
		{\fontseries{b}\selectfont\MakeTextUppercase{\normalsize\imprimirautor}}
	\end{center}
    \vfill
    
	\vfill
	\begin{center}
		{\fontseries{b}\selectfont\MakeTextUppercase{\imprimirtitulo}}
	\end{center}
	\vfill

    
\abntex@ifnotempty{\imprimirpreambulo}{%
	\hspace{.45\textwidth}
	{\begin{minipage}{.5\textwidth}
			\SingleSpacing
			\imprimirpreambulo\par
			\vspace*{4pt}
			{\imprimirorientadorRotulo~\imprimirorientador\par}
			\abntex@ifnotempty{\imprimircoorientador}{%
				{\imprimircoorientadorRotulo~\imprimircoorientador}%
			}%
	\end{minipage}}%
}%


\vfill
        
	 \begin{center}
	 	
    	{\fontseries{b}\selectfont BANCA EXAMINADORA: }
    	\vspace*{1.75cm}
    
		Prof. Marianne Zwiling Stampe, Dra. \par
		Universidade do Estado de Santa Catarina
	 \end{center}
	
    {Membros:} 
    
	\begin{center}
		\vspace*{1.25cm}
		Nome do Orientador e Titulação \par
		Nome da Instituição
		
		\vspace*{1.25cm}
		Nome do Orientador e Titulação \par
		Nome da Instituição
		
		\vspace*{1.25cm}
		Nome do Orientador e Titulação \par
		Nome da Instituição

	
	\end{center}
    
    \vspace*{\fill}  
    \begin{center}
    {\imprimirlocal, 01 de maio de \imprimirdata}
	\end{center}
    \vspace*{0.25cm}  
\end{folhadeaprovacao}
% ---




%\textbf{	{Orientador: \vspace{-16pt} }
%	\assinatura{\textbf{Prof. \imprimirorientador , Dr.} \\ Univ. XXX} 
%	{Coorientador: \vspace{-16pt}}   
%	\assinatura{\textbf{Prof. \imprimircoorientador , Dr.} \\ Univ. XXX}
%	
%	{Membros: \vspace{-16pt} } 
%	
%	% --- Exemplo de assinaturas em sequência ---       
%	\setlength{\ABNTEXsignwidth}{8.5cm}
%	
%	\assinatura{\textbf{Prof. Professor, Dr.} \\ Univ. XXX}
%	\assinatura{\textbf{Prof. Professor, Dr.} \\ Univ. XXX}
%	\assinatura{\textbf{Prof. Professor, Dr.} \\ Univ. XXX}
%	
%	% --- Exemplo de assinaturas lado a lado ---
%	\setlength{\ABNTEXsignwidth}{7.5cm}
	%
	%    \noindent\hfill\assinatura*{\textbf{Prof. Professor, Dr.} \\ Univ. XXX}%
	%    \hfill%
	%    \assinatura*{\textbf{Prof. Professor, Dr.} \\ Univ. XXX}%
	%    \hfill
	%    
	%    \noindent\hfill\assinatura*{\textbf{Prof. Professor, Dr.} \\ Univ. XXX}%
	%    \hfill%
	%    \assinatura*{\textbf{Prof. Professor, Dr.} \\ Univ. XXX}%
	%    \hfill}		% Elemento Obrigatório
% ---
% Dedicatória
% ---
\begin{dedicatoria}
   \vspace*{\fill}
%   \begin{flushright}
%   \noindent
%	Este trabalho é dedicado às crianças adultas que,\\
%	quando pequenas, sonharam em se tornar cientistas. 
%   \end{flushright}

{%
	\noindent\hspace{.5\textwidth}
	{\begin{minipage}{.5\textwidth}
			\begin{flushleft}
				Dedico este trabalho à minha querida família e aos meus amigos preciosos que, nas minhas estradas tortas, foram retas de cuidado, correção amorosa e silêncio fecundo. Quando saí do tom e quase perdi o rumo, vocês — sem saber — deram-me mão, riso e presença; cada gesto foi o fio que me puxou de volta ao centro, onde tudo encontra sentido. Tarde percebi o que sempre esteve diante de mim; a graça, no entanto, não tardou em me acolher. Por isso, se hoje entrego algo, é porque antes recebi, dom maior que as minhas forças e os meus méritos. Nada aqui é apenas meu: tudo foi sustentado pelo cuidado divino, que, por vocês, nunca desistiu de mim. Vivo para Aquele que dá sentido a tudo; que esta pequena obra, colhida entre esforço, lágrimas e graça, seja mais uma pedra no altar de uma vida que desejo inteira diante d’Ele — com a alegria mansa de quem reencontrou o lar.
			\end{flushleft}
	\end{minipage}}%
\vspace*{3cm}
}%

\end{dedicatoria}
% ---
			% Elemento Opcional
% ---
% Agradecimentos
% ---
\begin{agradecimentos}
Agradeço ao meu orientador por aceitar conduzir o meu trabalho de pesquisa.
A todos os meus professores do curso de da Universidade do Estado de Santa Catarina – Udesc pela excelência da qualidade técnica de cada um.

Aos meus pais que sempre estiveram ao meu lado me apoiando ao longo de toda a minha trajetória. Sou grato à minha família pelo apoio que sempre me deram durante toda a minha vida.

Como disse Snoop Dog: ``Eu quero me agradecer por acreditar em mim mesmo, quero me agradecer por todo esse trabalho duro. Quero me agradecer por não tirar folgas. Quero me agradecer por nunca desistir. Quero me agradecer por ser generoso e sempre dar mais do que recebo. Quero me agradecer por tentar sempre fazer mais o certo do que o errado. Quero me agradecer por ser eu mesmo o tempo inteiro''.

Deixo um agradecimento especial ao meu orientador pelo incentivo e pela dedicação do seu escasso tempo ao meu projeto de pesquisa.


\end{agradecimentos}
% ---		% Elemento Opcional
% ---
% Epígrafe
% ---
\begin{epigrafe}
	\vspace*{\fill}
	%	\begin{flushright}
	%		\textit{``Eu não falhei, encontrei 10 mil soluções que não davam certo.'' (EDISON, [19--])}
	%	\end{flushright}
	{%
		\noindent\hspace{.5\textwidth}
		{\begin{minipage}{.5\textwidth}
				\begin{flushright}
					
					``Estaremos em breve num mundo em que um homem poderá ser vaiado por dizer que dois e dois são quatro; [...] e enforcarão um homem por enfurecer a turba com a notícia de que a grama é verde.'' \newline(Chesterton, 1926)
					
				\end{flushright}
			\end{minipage}}%
		\vspace*{3cm}
	}%
\end{epigrafe}
% ---
				% Elemento Opcional
% ---
% RESUMOS
% ---

% resumo em português
\setlength{\absparsep}{18pt} % ajusta o espaçamento dos parágrafos do resumo
\begin{resumo}
    Este trabalho investiga a influência dos vieses cognitivos na tomada de decisão política e econômica da população. Partindo da premissa de que as crenças econômicas dos eleitores são frequentemente enviesadas, resultando em escolhas subótimas, a pesquisa analisa como a interação entre o Estado e a sociedade civil pode intensificar ou mitigar tais vieses. Além disso, considera-se o papel das teorias econômicas e da disseminação do conhecimento na formação dessas crenças. Com uma abordagem interdisciplinar, a investigação se insere na economia política comportamental, integrando conceitos da economia, ciência política e psicologia comportamental. A metodologia empregada inclui revisão bibliográfica e análise empírica baseada em modelos econométricos, com ênfase na modelagem Logit. Os resultados esperados visam oferecer subsídios para o aprimoramento da educação econômica e a formulação de políticas públicas mais informadas e eficazes.

 \textbf{Palavras-chave}: Vieses de julgamento. Economia política comportamental. Crenças econômicas. Escolhas políticas. Educação econômica.
\end{resumo}
				% Elemento Obrigatório
% ---
% Abstract
% ---

% resumo em inglês
\begin{resumo}[Abstract]
 \begin{otherlanguage*}{english}
  This study analyzes how voters’ cognitive biases influence economic policy-making, leading to suboptimal choices that may undermine economic and institutional development. The research compares the United States and Brazil, drawing on data from the \textit{Survey of Americans and Economists on the Economy} (SAEE) and its replication in Brazil. The theoretical framework falls within Behavioral Political Economy, integrating concepts from economics, political science, and psychology to explain why voters persist in biased beliefs even when confronted with contradictory information. The methodology employs empirical analysis and econometric modeling (\textit{Logit}) to assess the extent of these biases and their implications. The expected results contribute to the debate on improving economic education and reducing the impact of systematic irrationality in democracy.

   \textbf{Keywords}: Cognitive biases, Behavioral Political Economy, Political choices, Economic education, Systematic irrationality.
 \end{otherlanguage*}
\end{resumo}
				% Elemento Obrigatório

% ---
% inserir lista de ilustrações
% ---
\pdfbookmark[0]{\listfigurename}{lof}
\listoffigures*
\cleardoublepage
% ---

% ---
% inserir lista de quadros
% ---
%\pdfbookmark[0]{\listofquadrosname}{loq}
%\listofquadros*
%\cleardoublepage
% ---


% ---
% inserir lista de tabelas
% ---
\pdfbookmark[0]{\listtablename}{lot}
\listoftables*
\cleardoublepage
% ---

% ---
% inserir lista de abreviaturas e siglas
% ---
\begin{siglas}
	\item[SAEE] \textit{Survey of Americans and Economists on the Economy}
	\item[KFF] \textit{Kaiser Family Foundation}
	\item[Logit] Modelo Econômico de Regressão que segue a Distribuição Logística 
\end{siglas}
% ---

% ---
% inserir lista de símbolos
% ---


% \begin{simbolos}
%   \item[@] Arroba
%   \item[\%] Porcento
%   \item[$^\circ$C] Graus Celsius
%   \item[Ca] Cálcio
% \end{simbolos}

% % ---
				% Elemento Opcional
% ---
% inserir o sumario
% ---
\pdfbookmark[0]{\contentsname}{toc}
\tableofcontents*
\cleardoublepage
% ---
				% Elemento Obrigatório

% -----------------------------------------------------------------
% ELEMENTOS TEXTUAIS
% -----------------------------------------------------------------
\textual

\pagestyle{PagNumReduzida}						% Comando para cabeçalho somente com numeração de página 10pt
\aliaspagestyle{chapter}{PagNumReduzida}		% Deixar numeração da primeira página com tamanho igual ao resto da numeração
% ref.: https://groups.google.com/g/abntex2/c/CP7g8ZMgi-c/m/KjfEnn5b9a4J


% ---- Mantenha está estrutura, assim você deixa o trabalho mais organizado -------



\chapter{Introdução} % O Labirinto das Decisões: Como Julgamos e Escolhemos?
% Adicionar aqui uma parte principal sobre qual a contribuição principal do tcc, que é sobre o desenvolvimento do campo da "economia politica comportamental", que é nova e ainda não foi amplamente explorada. 
% A ideia é que a economia comportamental, que já é um campo estabelecido, pode ser aplicada à política, e que isso pode ser uma nova forma de entender a política, que é mais realista e menos idealista do que a economia política tradicional.



\section{A Racionalidade Coletiva} %(Tema e Problema de Pesquisa)




\section{Por que a Realidade Econômica É Distorcida pelo Eleitor?} %(Justificativa e Relevância do Estudo)
% deve enfatizar que o problema não é ignorância aleatória, mas crenças sistematicamente erradas, o que gera políticas ruins mesmo quando os eleitores estão bem informados.


\section{O Que Precisamos Descobrir} % Hipóteses





\section{Os Filtros da Percepção Econômica} % Objetivos




\subsection{Objetivo Geral} % Objetivo Geral




\subsection{Objetivos Específicos} % Objetivos Específicos





\chapter{Teorias e Evidências Sobre a (I)Racionalidade Humana} % Revisão de Literatura




\section{Entre Adam Smith e Kahneman} % Economia Comportamental vs. Escolha Racional 





\section{Como os Vieses Moldeiam as Escolhas Políticas} % (Vieses Cognitivos e Política)

\section{O Custo da Ignorância} % (Economia da Informação e Racionalidade Limitada)


\section{Do Sofista ao Populista} %  Como as Ideias Econômicas se Espalham (História do Pensamento Econômico e Propagação de Crenças)



\section{Preferência por Crenças e Resistência ao Conhecimento}







\chapter{Metodologia} % Metodologia

% Apresentar os métodos usados na pesquisa empírica.

\section{Definição das Variáveis e Modelagem} % Variáveis e Modelagem

% Explicar quais variáveis foram analisadas e como foram operacionalizadas.

\section{Técnicas de Análise Empírica} % Análise Empírica

% Descrever os métodos estatísticos usados, como a modelagem econométrica (Logit), e a replicação da Survey of Americans and Economists on the Economy (SAEE) no Brasil.

\section{Quando os Números Discordam do Senso Comum} % Resultados das Estimações

% Apresentar os resultados das estimações, destacando discrepâncias entre as percepções populares e os dados econômicos objetivos.

\section{O Eleitor é um Consumidor de Ideias Ruins? } %(Interpretação dos Resultados e Implicações)
% Pode incluir um subtópico específico sobre o custo social da irracionalidade do eleitor, mostrando que, diferentemente do mercado, onde decisões ruins afetam apenas o indivíduo, no voto as decisões ruins afetam a coletividade.

% Discutir as implicações dos resultados, enfatizando que, diferente do mercado, onde decisões ruins afetam apenas o indivíduo, no voto as decisões ruins afetam toda a coletividade.



\chapter{O Que Fazer Quando a Verdade Perde na Urna?} % Conclusão e discussão

% Discutir soluções para mitigar o impacto da irracionalidade dos eleitores.

\section{Limitações das Intervenções Educacionais} % explicando se há evidências empíricas de que a educação econômica pode reduzir vieses no voto. Poderia falar sobre o que o nassim taleb fala no livro "antifrágil" que o principal problema é o excesso de controle e a sobrevalorização da educação formal, onde deveriamos valorizar mais a prática e a experiência.

% Examinar se a educação econômica pode reduzir vieses e discutir a visão de Nassim Taleb sobre o excesso de controle e a sobrevalorização da educação formal.

\section{Educação Econômica e Tomada de Decisão} % Educação Econômica e Tomada de Decisão

% Explorar como o ensino econômico pode ser mais eficaz para diminuir os vieses cognitivos.

\section{Como Melhorar as Escolhas Coletivas} % Implicações para Políticas Públicas

% Analisar possíveis reformas institucionais e medidas para aumentar a qualidade das decisões políticas.

\section{As Perguntas que Ainda Precisamos Responder} % Limitações e Sugestões para Futuras Pesquisas

% Identificar as limitações da pesquisa e sugerir direções para estudos futuros.


% -----------------------------------------------------------------
% ELEMENTOS PÓS-TEXTUAIS
% -----------------------------------------------------------------
\postextual

% Você pode comentar os elementos que não deseja em seu trabalho;

% Referências bibliográficas

%Notar que os autores continuam sendo transpostos em maiúsculas, como preconiza a ABNT NBR 6023:2002. 
%Se, no entanto, não desejar seguir esta regra,
%crie um novo arquivo .bst (por exemplo, novoestilo.bst)a partir do estilo
%usado (abntex2-cite-alf ou abntex2-cite-num) e retire todas as expressões
%"u" change.case$, lembrando-se de indicar o novo arquivo como estilo, por exemplo, 
%\bibliographystyle{novoestilo}, e colocar o arquivo criado na mesma
%pasta em que está compilando o documento.

% Arquivo alterado para citação (Autor, Ano) ao invés de (AUTOR, Ano) conforme ABNT NBR 10520:2023
\bibliographystyle{abntex2-alf_revNBR2023.bst}	

\bibliography{abntex2-ref_UDESC_2020}	% Elemento Obrigatório

% ----------------------------------------------------------
% Glossário
% ----------------------------------------------------------

%Consulte o manual da classe abntex2 para orientações sobre o glossário.

%\glossary




% ----------------------------------------------------------
% Glossário (Formatado Manualmente)
% ----------------------------------------------------------

\chapter*{GLOSSÁRIO}
\addcontentsline{toc}{chapter}{GLOSSÁRIO}

{ \setlength{\parindent}{0pt} % ambiente sem indentação

\textbf{Amostragem não probabilística}: Procedimento de seleção de respondentes sem sorteio aleatório; adequado a testes de hipótese sob severidade, mas com generalização populacional limitada. 

\textbf{Análise contrafactual}: Estimativa de como seria a resposta de um indivíduo sob uma condição hipotética (mantidas as demais características constantes), usando parâmetros de um modelo ajustado.

\textbf{Bola de neve (amostragem)}: Técnica de recrutamento em que participantes indicam novos respondentes, útil para ampliar o alcance da coleta.

\textbf{Coeficiente $\boldsymbol\beta$}: Parâmetro de regressão que quantifica a associação entre uma covariável e a probabilidade (ou categoria) de resposta prevista pelo modelo.

\textbf{Dummy (variável indicadora)}: Variável binária ($0/1$) que representa a presença/ausência de uma característica (por ex., formação em Economia), incluída como regressora no modelo.

\textbf{Escala Likert}: Escala ordinal de resposta (ex.: 3 a 5 pontos) usada para medir concordância/discordância com afirmações.

\textbf{Erros-padrão robustos}: Estimativas de incerteza ajustadas para heteroscedasticidade, tornando a inferência menos sensível a violações de variância constante.

\textbf{Hipótese de chances proporcionais (paralelismo)}: Suposição do logit ordenado de que o efeito das covariáveis é constante entre limiares/categorias; quando plausível, facilita comparabilidade entre itens.

\textbf{Instrumento de pesquisa (questionário)}: Conjunto estruturado de questões aplicado digitalmente, revisado por especialistas e pré-testado para clareza, tempo de resposta e validade de conteúdo.

\textbf{Limiares ($\boldsymbol{\tau_j}$)}: Parâmetros do logit ordenado que definem as fronteiras entre categorias da variável dependente.

\textbf{Logit (modelo)}: Regressão para variáveis dependentes binárias, que modela a probabilidade de um evento via função logística.

\textbf{Logit ordenado}: Extensão do logit para variáveis dependentes ordinais (ex.: Likert), com estrutura de limiares e, usualmente, hipótese de chances proporcionais.

\textbf{Princípios-ponte}: Conjunto de regras que conecta teoria e dados no desenho empírico (operacionalização, codificação/direção esperada e sinal teórico dos efeitos).

\textbf{Público esclarecido (contrafactual)}: Referência simulada que prevê como economistas responderiam “como se fossem leigos” (sem formação econômica), mantendo constantes as demais variáveis — usada para isolar o papel do conhecimento técnico.

\textbf{Racionalidade limitada}: Ideia de que decisões são tomadas sob informação imperfeita e capacidade cognitiva restrita; no contexto político, ajuda a explicar crenças persistentes e vieses.

\textbf{Validade externa}: Alcance da generalização dos resultados além da amostra estudada; com amostragem não probabilística, a inferência é positiva/explicativa e não descritiva da população.

\textbf{Vieses cognitivos (quatro vieses)}: Padrões sistemáticos de julgamento identificados na literatura (antimercado, antiestrangeiro, antitrabalho e pessimista) que afetam percepções econômicas e preferências por políticas.

} % fim ambiente sem indentação



				% Elemento Opcional

% ----------------------------------------------------------
% Apêndices
% ----------------------------------------------------------

% ---
% Inicia os apêndices
% ---
\begin{apendicesenv}

% Imprime uma página indicando o início dos apêndices
%\partapendices



% ----------------------------------------------------------
\chapter{Quadro Síntese das Abordagens sobre (I)Racionalidade Humana e Política}
% ----------------------------------------------------------

\label{apendice:quadro_iracionalidade}

\begin{quadro}[htbp]
\caption{Quadro Síntese – (I)Racionalidade Humana e Política: Principais Abordagens}

\begin{longtable}{p{0.15\textwidth} p{0.23\textwidth} p{0.34\textwidth} p{0.20\textwidth}}
\toprule
\textbf{Autor/Obra (Ano)} & \textbf{Tópico/Conceito Central} & \textbf{Principais Contribuições e Evidências} & \textbf{Relação com o TCC} \\
\midrule
\endfirsthead
\multicolumn{4}{c}{\textbf{Quadro – Continuação}} \\
\toprule
\textbf{Autor/Obra (Ano)} & \textbf{Tópico/Conceito Central} & \textbf{Principais Contribuições e Evidências} & \textbf{Relação com o TCC} \\
\midrule
\endhead

Adam Smith (1759; 1776) & Moralidade e racionalidade econômica & Integra interesse próprio e normas morais; reconhece limites do modelo racional estrito. & Critica o homo economicus e destaca a ética na decisão pública. \\
Herbert Simon (1955) & Racionalidade Limitada & Mostra que decisões são feitas com informação e tempo limitados; conceito de “satisficing”. & Base para analisar limites cognitivos em escolhas políticas. \\
Kahneman \& Tversky (1974, 2011) & Heurísticas e Vieses Cognitivos & Demonstram vieses como representatividade e ancoragem, fundamentais para economia comportamental. & Explica previsibilidade dos desvios do eleitor médio. \\
Anthony Downs (1957) & Ignorância Racional & Eleitor não busca informação pois custo supera o benefício do voto. & Fundamenta debate sobre baixa informação e democracia. \\
Bryan Caplan (2007) & Irracionalidade Racional & Eleitores mantêm crenças falsas por motivações emocionais, não apenas ignorância. & Mostra padrões sistemáticos de viés com impacto em políticas. \\
Hayek (1945) & Limitação do conhecimento & Defende dispersão do conhecimento, inviabilizando decisões centralizadas. & Sustenta crítica à ilusão de controle racional em políticas. \\
Bhagwati, Sowell, Bastiat & Viés antimercado e antiestrangeiro & Abordam recorrência de protecionismo, desconfiança do livre comércio e argumentos populares falhos. & Ampliam análise sobre vieses e políticas populistas. \\
Nyhan, Kahan, Sunstein, Rossini et al. & Resistência à revisão de crenças & Mostram que vieses de confirmação e bolhas informacionais dificultam revisão de crenças e promovem polarização. & Explicam a persistência da desinformação política. \\
\bottomrule
\end{longtable}
\end{quadro}

\chapter{Epílogo Metafísico: Entre a Evidência e a Esperança \\ \smallskip \textit{(Reflexão Filosófica para Além do Empírico)}}

\noindent
\textbf{Nota de método.} \textit{A partir deste ponto, movo-me do domínio do empiricamente testável para a esfera da reflexão filosófica e institucional. Como Popper recomendaria, assumo aqui apenas conjecturas abertas ao debate, inspiradas — e não determinadas — pelos achados empíricos deste trabalho.}

\vspace{1em}

A política moderna consagrou na democracia a liberdade das massas como princípio supremo. Mas que liberdade é essa? A de fazer tudo o que se quer, mesmo sem saber por quê? Ou a de resistir ao próprio desejo, quando tudo grita por rendição? Talvez liberdade seja, afinal, a capacidade de fazer não o que se quer, mas o que se deve — mesmo (ou sobretudo) quando não se deseja. E o dever, aqui, não nasce de comando externo, mas da gravidade interna de quem aprendeu a discernir.

A democracia, em sua forma atual, opera como o teatro onde desejos transitam com legitimidade garantida. A cada eleição, reencena-se a soberania do querer sobre o dever, do imediato sobre o necessário, do impulso sobre o juízo. O sistema, tal como está, recompensa quem responde rápido, não quem pensa devagar. E há algo de estrutural nisso: o calendário democrático segue o tempo do relógio — mas a justiça segue o tempo da consciência. Como ensinou Santo Agostinho, “há um tempo exterior, que corre; e há um tempo interior, que pondera”. A política deveria aspirar a este último.

É nesse palco que se manifesta o paradoxo já desvelado por Caplan: mesmo eleitores bem-intencionados erram. Erram não por ignorância ocasional, mas por uma racionalidade limitada, sistematicamente enviesada, que prefere o conforto da ilusão ao rigor da verdade. E se, como advertiu Hayek, o conhecimento social está sempre disperso e fragmentado, todo sistema que aposta na onisciência da vontade coletiva naufraga entre o caos e a tirania.

Assim, quando um eleitor escolhe, não apenas escolhe: julga. E esse juízo, por mais livre que pareça, carrega uma cadeia invisível de valores, memórias, convicções e desejos. A liberdade, portanto, não é um ponto de partida; é o último degrau de uma escada moral. Só é verdadeiramente livre quem sabe por que deve escolher o que escolhe — e está disposto a não querer o que não deve.

Se a liberdade for reduzida à soma dos desejos, o resultado será sempre um sistema vulnerável à manipulação, à irracionalidade e à moral líquida. Não há mercado perfeito de ideias, nem há “engenharia institucional” capaz de garantir, por decreto, que a liberdade coincida com o bem comum. Como Acemoglu e Robinson demonstram, a liberdade nasce no estreito corredor entre o excesso de poder e a ausência de ordem: não é dádiva, mas conquista precária, resultado de uma tensão viva entre Estado e sociedade.

Votar é fácil; escolher bem é difícil. Não basta, portanto, melhorar os eleitores. É preciso repensar as engrenagens do sistema que transforma preferências em leis. Um sistema político que se submete inteiramente às oscilações das paixões populares perde sua capacidade de guardar aquilo que não muda: a justiça, o bem, a verdade. Precisamos, pois, de instituições que operem não apenas no tempo do poder, mas no tempo do juízo. E esse tempo não é democrático nem autoritário — é simplesmente humano.

Talvez o caminho esteja em resgatar uma arquitetura política onde a liberdade não seja apenas protegida, mas formada; onde a decisão coletiva seja amparada por raízes mais fundas que as modas da hora; onde o bem comum não seja o resultado do grito da maioria, mas o fruto do silêncio que pensa. Pois, se algo há — e há — é a necessidade de juízo. Porque o nada, isto é, o puro desejo, não institui ordem. Só o ser — aquilo que permanece, que resiste ao fluxo — pode fundar liberdade duradoura.

Este trabalho não pretende esgotar essas possibilidades. Limita-se a constatar, com base empírica e fundamento teórico, que o problema não está apenas nos votos, mas naquilo que os votos ignoram, no que, como diria Bastiat, “o que se vê e o que não se vê”. É preciso olhar para o que não se vê, analisando as evidências com cuidado. E que, para que a verdade tenha alguma chance, será preciso mais do que pedagogia: será preciso estrutura, limite e tempo.

Há decisões que não cabem em ciclos de quatro anos. Há verdades que não se revelam em sondagens. Há justiças que não se alcançam por aclamação. E há liberdades que só florescem quando já não estamos mais olhando para nós mesmos — mas para o bem que nos excede.

A política, quando é séria, não é desejo. É juízo. E o juízo não se apressa. Espera o tempo certo — o tempo do discernimento — para decidir com liberdade verdadeira. Talvez esse tempo ainda venha. Ou talvez já esteja vindo, silencioso, em meio às ruínas de um mundo que confundiu liberdade com vontade e nos deixou uma moral relativa com a qual lidar. Seja como for, uma coisa é certa: não será a pedagogia que nos salvará, mas a arquitetura moral que soubermos legar — e que quisermos deixar — para as próximas gerações.

Por fim, não pretendo impor juízos, mas apenas propor reflexão. Que o dissenso, longe de nos afastar, seja ocasião de busca conjunta por aquilo que resiste ao tempo e nos chama para além de nós mesmos. Não encerro com respostas definitivas, mas com a confissão do limite: a verdade é como a água pura que tentamos reter entre as mãos — ela escorre por entre os dedos, mas ainda assim refresca, orienta e purifica. Talvez, como intuiu Jung, só possamos nos aproximar do valor mais alto reconhecendo aquilo que, em nosso íntimo, ocupa o lugar supremo, fonte de todo sentido. E se toda busca sincera se inclina, por fim, diante desse Bem maior que nos transcende e nos funda, resta-nos apenas o papel de aprendizes: atentos ao mistério, pacientes com o tempo, humildes diante do real.

\begin{flushright}
\textit{
“Ninguém alcança a verdade senão aquele\
que ousa caminhar entre dúvidas,\
mas não abandona a esperança.\
Pois só na humildade diante do Mistério\
a razão encontra repouso e o coração, paz.”\
— inspirado em Santo Agostinho, Confissões
}
\end{flushright}




\end{apendicesenv}
% ---				% Elemento Opcional

% ----------------------------------------------------------
% Anexos
% ----------------------------------------------------------
%
% ---
% Inicia os anexos
% ---
\begin{anexosenv}

	% Imprime uma página indicando o início dos anexos
	%\partanexos

	% ---
	\chapter{Variáveis de controle analisadas por Caplan}

	\begin{longtable}{|>{\raggedright\arraybackslash}p{4cm}
        |>{\raggedright\arraybackslash}p{8cm}
        |>{\raggedright\arraybackslash}p{4cm}|}
\caption{Variáveis de Controle e Codificação (Caplan, 2002)}
\label{tab:caplan_controls} \\
\hline
\textbf{Variável} & \textbf{Pergunta} & \textbf{Codificação} \\
\hline
\endfirsthead

\hline
\textbf{Variável} & \textbf{Pergunta} & \textbf{Codificação} \\
\hline
\endhead

\hline
\endfoot

\hline
\endlastfoot

\textbf{Econ} & – & 1 se economista, 0 caso contrário \\
\hline
\textbf{Black} & Qual é a sua raça? Branco, negro, asiático ou outra? & 1 se negro, 0 caso contrário \\
\hline
\textbf{Asian} & Idem & 1 se asiático, 0 caso contrário \\
\hline
\textbf{Othrace} & Idem & 1 se outra raça, 0 caso contrário \\
\hline
\textbf{Age} & – & 1996 - ano de nascimento \\
\hline
\textbf{Male} & – & 1 se homem, 0 caso contrário \\
\hline
\textbf{Jobsecurity} & Qual o seu nível de preocupação com a possibilidade de perder o emprego no próximo ano? &
3 = “Nada preocupado”; 2 = “Pouco preocupado”; 1 = “Um pouco preocupado”; 0 = “Muito preocupado” \\
\hline
\textbf{Yourlast5} & Nos últimos 5 anos, a renda da sua família cresceu mais rápido, igual ou mais devagar que o custo de vida? &
2 = “Cresceu”; 1 = “Igual”; 0 = “Caiu” \\
\hline
\textbf{Yournext5} & Nos próximos 5 anos, sua renda crescerá mais rápido, igual ou mais devagar que o custo de vida? &
2 = “Crescerá mais rápido”; 1 = “Igual”; 0 = “Mais devagar” \\
\hline
\textbf{Income} & Qual a renda total anual da sua família (antes dos impostos)? &
1 = Até \$10.000; 2 = \$10.001–\$19.999; 3 = \$20.000–\$24.999; 4 = \$25.000–\$29.999; 5 = \$30.000–\$39.999;
6 = \$40.000–\$49.999; 7 = \$50.000–\$74.999; 8 = \$75.000–\$99.999; 9 = \$100.000 ou mais \\
\hline
\textbf{Dem} & Você se considera democrata? & 1 se sim, 0 caso contrário \\
\hline
\textbf{Rep} & Você se considera republicano? & 1 se sim, 0 caso contrário \\
\hline
\textbf{Indep} & Você se considera independente? & 1 se sim, 0 caso contrário \\
\hline
\textbf{Othparty} & Você pertence a outro partido? & 1 se sim, 0 caso contrário \\
\hline
\textbf{Ideology} & Como você se classifica ideologicamente? &
-2 = “Muito liberal”; -1 = “Liberal”; 0 = “Moderado”; 1 = “Conservador”; 2 = “Muito conservador”; 3 = “Não pensa nesses termos” \\
\hline
\textbf{Othideol} & Dummy para ideologia indefinida & 1 se Ideology = 3, 0 caso contrário \\
\hline
\textbf{Education} & Qual o maior nível de escolaridade que você concluiu? &
1 = Fundamental (1ª–8ª); 2 = Ensino médio incompleto; 3 = Ensino médio completo;
4 = Técnico/profissionalizante; 5 = Superior incompleto; 6 = Superior completo; 7 = Pós-graduação \\
\end{longtable}

\vspace{2mm}
\noindent\textbf{Fonte:} Elaborado pelo autor com base em \citeonline{Systematically_Biased_Beliefs_about_Economics}.
	% ---



\end{anexosenv}
				% Elemento Opcional

%%---------------------------------------------------------------------
%% INDICE REMISSIVO
%%---------------------------------------------------------------------

%\phantompart
%\printindex

%---------------------------------------------------------------------

%%---------------------------------------------------------------------
%% INDICE REMISSIVO (Formatado Manualmente)
%%---------------------------------------------------------------------

\chapter*{ÍNDICE}
\addcontentsline{toc}{chapter}{ÍNDICE}

{ \setlength{\parindent}{0pt}  % ambiente sem indentação
	
Andesito, 22, 50, 73

Argila, 52, 75, 121

Basalto, 25, 230, 235

	
	
	
	
} % fim ambiente sem indentação


		% Elemento Opcional



\end{document}

% -----------------------------------------------------------------
% Fim do Documento
% -----------------------------------------------------------------	