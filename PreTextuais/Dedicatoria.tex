% ---
% Dedicatória
% ---
\begin{dedicatoria}
   \vspace*{\fill}
%   \begin{flushright}
%   \noindent
%	Este trabalho é dedicado às crianças adultas que,\\
%	quando pequenas, sonharam em se tornar cientistas. 
%   \end{flushright}

{%
	\noindent\hspace{.5\textwidth}
	{\begin{minipage}{.5\textwidth}
			\begin{flushleft}
				Dedico este trabalho à minha querida família e aos meus amigos preciosos que, nas minhas estradas tortas, foram retas de cuidado, correção amorosa e silêncio fecundo. Quando saí do tom e quase perdi o rumo, vocês — sem saber — deram-me mão, riso e presença; cada gesto foi o fio que me puxou de volta ao centro, onde tudo encontra sentido. Tarde percebi o que sempre esteve diante de mim; a graça, no entanto, não tardou em me acolher. Por isso, se hoje entrego algo, é porque antes recebi, dom maior que as minhas forças e os meus méritos. Nada aqui é apenas meu: tudo foi sustentado pelo cuidado divino, que, por vocês, nunca desistiu de mim. Vivo para Aquele que dá sentido a tudo; que esta pequena obra, colhida entre esforço, lágrimas e graça, seja mais uma pedra no altar de uma vida que desejo inteira diante d’Ele — com a alegria mansa de quem reencontrou o lar.
			\end{flushleft}
	\end{minipage}}%
\vspace*{3cm}
}%

\end{dedicatoria}
% ---
