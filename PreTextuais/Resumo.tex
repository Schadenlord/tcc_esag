% ---
% RESUMOS
% ---

% resumo em português
\setlength{\absparsep}{18pt} % ajusta o espaçamento dos parágrafos do resumo
\begin{resumo}
    Este trabalho investiga como vieses de julgamento moldam crenças econômicas e, por consequência, preferências por políticas públicas no Brasil. Partindo do paradoxo da racionalidade coletiva — entre o que a evidência recomenda e o que a esperança deseja —, adapta-se a SAEE ao contexto brasileiro e operacionaliza-se um contrafactual de “público esclarecido” (não-economistas com perfil socioeconômico/ideológico próximo ao de economistas), a fim de isolar o papel do conhecimento técnico frente à ideologia. O desenho empírico combina survey digital (amostragem não probabilística) e modelos logit binário/ordenado com controles demográficos e ideológicos, mantendo especificações comparáveis para todos os itens. Os resultados apontam padrões consistentes com quatro famílias de viés (antimercado, antiestrangeiro, antitrabalho e pessimismo) e indicam que maior literacia econômica está associada a respostas mais próximas ao consenso técnico, embora efeitos ideológicos permaneçam substantivos em temas moralizados e distributivos. O estudo contribui ao: (i) propor um arcabouço mensurável e comparável para crenças econômicas; (ii) explicitar critérios de refutabilidade e testes severos; (iii) discutir limites de validade externa e robustez; e (iv) derivar implicações para educação econômica e desenho institucional. Ao confrontar “dogmas e dados”, os achados sugerem que corrigir crenças requer mais do que informação: exige arranjos que favoreçam a aprendizagem em vez de apenas a afirmação identitária.

 \textbf{Palavras-chave}: economia política comportamental; vieses cognitivos; crenças econômicas; logit ordenado; público esclarecido; educação econômica; racionalidade coletiva.
\end{resumo}
