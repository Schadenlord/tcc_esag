% ---
% RESUMOS
% ---

% resumo em português
\setlength{\absparsep}{18pt} % ajusta o espaçamento dos parágrafos do resumo
\begin{resumo}
    Este trabalho analisa como os vieses cognitivos dos eleitores influenciam a formulação de políticas econômicas, resultando em escolhas subótimas que podem comprometer o desenvolvimento econômico e institucional. A pesquisa se baseia na comparação entre os Estados Unidos e o Brasil, explorando dados da \textit{Survey of Americans and Economists on the Economy} (SAEE) e sua replicação no Brasil. A abordagem teórica se insere na Economia Política Comportamental, integrando conceitos de economia, ciência política e psicologia para explicar por que eleitores persistem em crenças enviesadas, mesmo quando há informações disponíveis que contradizem suas opiniões. A metodologia emprega análise empírica e modelagem econométrica (\textit{Logit}) para avaliar a magnitude desses vieses e suas implicações. Os resultados esperados contribuem para o debate sobre como melhorar a educação econômica e reduzir o impacto da irracionalidade sistemática na democracia.

 \textbf{Palavras-chave}: Vieses cognitivos, Economia Política Comportamental, Escolhas políticas, Educação econômica, Irracionalidade sistemática.
\end{resumo}
