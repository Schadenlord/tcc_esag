% ---
% RESUMOS
% ---

% resumo em português
\setlength{\absparsep}{18pt} % ajusta o espaçamento dos parágrafos do resumo
\begin{resumo}
    Este trabalho investiga a influência dos vieses cognitivos na tomada de decisão política e econômica da população. Partindo da premissa de que as crenças econômicas dos eleitores são frequentemente enviesadas, resultando em escolhas subótimas, a pesquisa analisa como a interação entre o Estado e a sociedade civil pode intensificar ou mitigar tais vieses. Além disso, considera-se o papel das teorias econômicas e da disseminação do conhecimento na formação dessas crenças. Com uma abordagem interdisciplinar, a investigação se insere na economia política comportamental, integrando conceitos da economia, ciência política e psicologia comportamental. A metodologia empregada inclui revisão bibliográfica e análise empírica baseada em modelos econométricos, com ênfase na modelagem Logit. Os resultados esperados visam oferecer subsídios para o aprimoramento da educação econômica e a formulação de políticas públicas mais informadas e eficazes.

 \textbf{Palavras-chave}: Vieses de julgamento. Economia política comportamental. Crenças econômicas. Escolhas políticas. Educação econômica.
\end{resumo}
