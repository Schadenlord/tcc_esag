% ---
% RESUMOS
% ---

% resumo em português
\setlength{\absparsep}{18pt} % ajusta o espaçamento dos parágrafos do resumo
\begin{resumo}
    Este trabalho investiga por que o eleitor comum, mesmo bem-intencionado, adota crenças econômicas sistematicamente divergentes do consenso técnico, examinando a presença e a intensidade de vieses cognitivos em percepções sobre temas macro e microeconômicos no Brasil. O objetivo é medir a distância entre o público geral e um “público contrafactual” mais esclarecido, bem como identificar fatores associados a essa distância (formação, ideologia, renda, etc.). O desenho metodológico combina (i) adaptação da Survey of Americans and Economists on the Economy (SAEE) ao contexto brasileiro e (ii) estimação de modelos logit binário e ordenado com controles demográficos, socioeconômicos e ideológicos, de modo a isolar os efeitos marginais de literacia econômica e outras covariáveis sobre a probabilidade de respostas enviesadas. As análises foram conduzidas em Python, com reprodutibilidade assegurada. Os resultados indicam padrões consistentes de divergência entre o público geral e o contrafactual, compatíveis com vieses clássicos descritos pela literatura (antimercado, antiestrangeiro, pró-trabalho/pessimismo), e sugerem que maior literacia econômica e contato com conteúdos formais da disciplina estão associados, em média, a menor propensão a crenças incorretas ou exageradas; ao mesmo tempo, a orientação ideológica mantém efeitos independentes em questões de política pública. O estudo discute implicações para desenho de políticas e comunicação econômica, ressalta limitações de amostragem e validade externa, e propõe agenda de testes adicionais e replicação em amostras probabilísticas.

 \textbf{Palavras-chave}: economia comportamental; vieses cognitivos; opinião pública; literacia econômica; políticas públicas.
\end{resumo}
