% ---
% Agradecimentos
% ---
\begin{agradecimentos}
    Primeiramente, gostaria de agradecer à minha família, que sempre foi um pilar de apoio e paciência durante toda essa jornada. Mesmo quando minhas opiniões sobre economia pareciam mais filosóficas e divergentes do que realmente informadas, nunca me faltaram palavras de incentivo e uma boa dose de compreensão (em silêncio, como quem diz: "Vamos ver até onde isso vai dar"). Sem vocês, não teria chegado até aqui.

    Aos meus professores, que não só orientaram meu caminho, mas também estiveram ao meu lado nas horas mais difíceis e que viram minha necessidade quando tudo parecia não fazer sentido. Agradeço por suas valiosas orientações, pelas boas risadas que demos na sala do GEA quando eu estava prestes a surtar e vinha buscar uma conversa, mas principalmente, pela paciência em ver minhas perguntas repetidas e minha tendência a complicar tudo de maneiras muito inovadoras. Sem vocês, este trabalho seria apenas uma ideia vaga de um estudante perdido, em crise intelectual que talvez não tivesse chegado a lugar algum sem o seu sincero apoio.
    
    Aos meus amigos, que me apoiaram nos momentos em que mais precisei, especialmente nas vezes em que estive à beira de largar tudo e virar um eremita na floresta mais próxima. Agradeço por aturarem minhas longas e acaloradas discussões de política e economia, por rirem das piadas sem graça e por se envolverem comigo em debates intermináveis sobre as coisas mais absurdas e inúteis nas nossas terças de pizza. Sem vocês, a vida teria sido muito mais monótona e sem sentido. Foram fundamentais para manter minha sanidade intacta (ou quase).

    A Deus, fonte de toda sabedoria, que nestes últimos tempos me reencontrou e me sustentou nos momentos em que nem a mais sofisticada teoria econômica conseguia explicar minha falta de esperança diante das crises – acadêmicas e existenciais. Obrigado por me lembrar que, apesar de todas as estatísticas e hipóteses, é a Tua providência que rege o mundo – e que, no fim das contas, confiar em Ti sempre foi a melhor escolha. Se cheguei até aqui sem perder a fé (e com apenas alguns lapsos de sanidade), sei que foi pura graça Tua.

    Por fim, agradeço a todos que, mesmo quando minhas opiniões foram um tanto ácidas e contraditórias (certamente merecendo uma ou duas correções de rumo), nunca deixaram de me apoiar, orientar e ajudar a encontrar o caminho que hoje sigo – e que, espero, me levará a lugares melhores. A vocês, que me fizeram perceber que, apesar das contradições, a persistência e a vontade de aprender sempre foram mais fortes, meu mais sincero “muito obrigado”.

    E, apesar do tom dramático destes agradecimentos, sempre fui muito feliz neste curso e levo para a vida cada aprendizado, conversa e, principalmente, cada um de vocês. Se você leu isto e lembrou de algum momento comigo, este agradecimento é especial a você, que me apoiou, muitas vezes sem nem perceber. Ainda que este TCC esteja entregue e o diploma em mãos, espero que nossa história não termine aqui. Se alguma memória o tocou ou uma lágrima teimou em cair, saiba que sua presença foi essencial na minha jornada. Felizmente, fiz algumas boas escolhas ao longo deste percurso – e estar cercado por pessoas como vocês certamente foi uma delas.


\end{agradecimentos}
% ---