% ---
% Agradecimentos
% ---
\begin{agradecimentos}
    Primeiramente, gostaria de agradecer à minha família, que sempre foi um pilar de apoio e paciência durante toda essa jornada. Mesmo quando minhas opiniões sobre economia pareciam mais filosóficas do que realmente informadas, nunca me faltaram palavras de incentivo e uma boa dose de compreensão (mesmo que, às vezes, em silêncio, como quem diz: "Vamos ver até onde isso vai dar..."). Sem vocês, não teria chegado até aqui.

    Aos meus professores, que não só orientaram meu caminho acadêmico, mas também estiveram ao meu lado nas horas mais difíceis, quando tudo parecia não fazer sentido. Agradeço por suas valiosas orientações e, principalmente, pela paciência em ver minhas perguntas repetidas e minha tendência a complicar tudo de maneiras muito inovadoras. Sem vocês, este trabalho seria apenas uma ideia vaga de um estudante em crise intelectual.
    
    Aos meus amigos, que me apoiaram quando eu mais precisei, especialmente nos dias em que estava à beira de deixar tudo para trás e virar um eremita na floresta mais próxima. Agradeço também por aturarem minhas longas e acaloradas discussões sobre política e economia, mesmo que no final a gente nunca tenha chegado a um consenso. O apoio de vocês foi fundamental, até mesmo para manter minha sanidade intacta.

    A Deus, fonte de toda sabedoria, que me sustentou nos momentos em que nem a mais elaborada teoria econômica conseguia explicar minha falta de esperança diante das crises (acadêmicas e existenciais). Obrigado por me lembrar que, apesar de todas as estatísticas e hipóteses, é a Tua providência que rege o mundo – e que, no fim das contas, confiar mais em Ti do 5\% dos modelos econométricos sempre foi a melhor escolha. Se cheguei até aqui sem perder a fé (e com apenas alguns lapsos de sanidade), sei que foi pura graça.
    
    Por fim, quero agradecer a todos aqueles que, mesmo quando minhas opiniões foram um pouco ácidas e contraditórias (com certeza merecendo pelo menos uma ou duas correções de rumo), nunca deixaram de me apoiar, me orientar e, de alguma forma, me ajudar a encontrar o caminho. Àqueles que me ajudaram a perceber que, mesmo com todas as minhas contradições, a persistência e a vontade de aprender sempre foram mais fortes.

    A todos vocês, o meu mais sincero “muito obrigado” – não só pelo apoio, mas também por me lembrar de que, no fim das contas, a vida é feita de escolhas. E, apesar do tom dramático destes agradecimentos, sempre fui muito feliz neste curso e levo para a vida cada aprendizado, cada conversa e, principalmente, cada um de vocês. Se você, leitor deste TCC, leu isto e lembrou de algum momento comigo, esse agradecimento é especial a você, que me apoiou de maneiras diversas, muitas vezes sem nem saber. Felizmente, consegui fazer algumas boas escolhas ao longo deste percurso – e estar cercado por pessoas como vocês certamente foi uma delas.


\end{agradecimentos}
% ---