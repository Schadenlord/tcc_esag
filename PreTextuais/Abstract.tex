% ---
% Abstract
% ---

% resumo em inglês
\begin{resumo}[Abstract]
 \begin{otherlanguage*}{english}
  % Este trabalho investiga a influência dos vieses cognitivos na tomada de decisão política e econômica da população. Partindo da premissa de que as crenças econômicas dos eleitores são frequentemente enviesadas, resultando em escolhas subótimas, a pesquisa analisa como a interação entre o Estado e a sociedade civil pode intensificar ou mitigar tais vieses. Além disso, considera-se o papel das teorias econômicas e da disseminação do conhecimento na formação dessas crenças. Com uma abordagem interdisciplinar, a investigação se insere na economia política comportamental, integrando conceitos da economia, ciência política e psicologia comportamental. A metodologia empregada inclui revisão bibliográfica e análise empírica baseada em modelos econométricos, com ênfase na modelagem Logit. Os resultados esperados visam oferecer subsídios para o aprimoramento da educação econômica e a formulação de políticas públicas mais informadas e eficazes.

  This paper investigates the influence of cognitive biases on the population's political and economic decision-making. Based on the premise that voters' economic beliefs are often biased, resulting in suboptimal choices, the research analyzes how the interaction between the State and civil society can intensify or mitigate such biases. In addition, the role of economic theories and the dissemination of knowledge in the formation of these beliefs is considered. With an interdisciplinary approach, the research is part of behavioral political economy, integrating concepts from economics, political science and behavioral psychology. The methodology used includes a literature review and empirical analysis based on econometric models, with an emphasis on Logit modeling. The expected results aim to provide support for the improvement of economic education and the formulation of more informed and effective public policies.

   \textbf{Keywords}: Judgment biases. Behavioral political economy. Economic beliefs. Political choices. Economic education.
 \end{otherlanguage*}
\end{resumo}
