% ---
% Abstract
% ---

% resumo em inglês
\begin{resumo}[Abstract]
 \begin{otherlanguage*}{english}
  This study examines how judgment biases shape economic beliefs and, in turn, preferences for public policies in Brazil. Framing the paradox of collective rationality—between what evidence recommends and what hope desires—we adapt the SAEE to the Brazilian context and operationalize a contrafactual “informed public” (non-economists whose socioeconomic/ideological profile approximates that of economists) to disentangle technical knowledge from ideology. The empirical design combines a digital survey (non-probabilistic sampling) with binary/ordered logit models including demographic and ideological controls, maintaining comparable specifications across items. Results reveal patterns consistent with four bias families (anti-market, anti-foreigner, anti-work, and pessimism) and show that higher economic literacy is associated with responses closer to expert consensus, although ideological effects remain substantive on moralized and distributive topics. The study contributes by: (i) proposing a measurable, comparable framework for economic beliefs; (ii) specifying refutability criteria and severe tests; (iii) discussing limits to external validity and robustness; and (iv) deriving implications for economic education and institutional design. Confronting “dogmas and data,” the findings suggest that correcting beliefs requires more than information alone: it calls for arrangements that foster learning rather than mere identity affirmation.

   \textbf{Keywords}: behavioral political economy; cognitive biases; economic beliefs; ordered logit; informed public; economic literacy; collective rationality.
 \end{otherlanguage*}
\end{resumo}
