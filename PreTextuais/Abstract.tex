% ---
% Abstract
% ---

% resumo em inglês
\begin{resumo}[Abstract]
 \begin{otherlanguage*}{english}
  This study analyzes how voters’ cognitive biases influence economic policy-making, leading to suboptimal choices that may undermine economic and institutional development. The research compares the United States and Brazil, drawing on data from the \textit{Survey of Americans and Economists on the Economy} (SAEE) and its replication in Brazil. The theoretical framework falls within Behavioral Political Economy, integrating concepts from economics, political science, and psychology to explain why voters persist in biased beliefs even when confronted with contradictory information. The methodology employs empirical analysis and econometric modeling (\textit{Logit}) to assess the extent of these biases and their implications. The expected results contribute to the debate on improving economic education and reducing the impact of systematic irrationality in democracy.

   \textbf{Keywords}: Cognitive biases, Behavioral Political Economy, Political choices, Economic education, Systematic irrationality.
 \end{otherlanguage*}
\end{resumo}
