% ---
% Abstract
% ---

% resumo em inglês
\begin{resumo}[Abstract]
 \begin{otherlanguage*}{english}
  This thesis investigates why well-intentioned voters systematically hold economic beliefs that diverge from expert consensus, by measuring the presence and intensity of cognitive biases in Brazilian public perceptions on macro- and micro-economic issues. The goal is to quantify the gap between the general public and a more informed “counterfactual public,” and to identify factors associated with that gap (education, ideology, income, etc.). The research design combines (i) an adaptation of the Survey of Americans and Economists on the Economy (SAEE) to the Brazilian context and (ii) estimation of binary and ordered logit models with demographic, socioeconomic, and ideological controls, in order to isolate the marginal effects of economic literacy and other covariates on the likelihood of biased responses. Analyses were conducted in Python with full reproducibility. Findings show persistent divergences between the general and counterfactual publics consistent with well-documented biases (anti-market, anti-foreign, pro-labor/pessimism). Higher economic literacy and formal exposure to economics are, on average, associated with a lower propensity to hold inaccurate or overstated beliefs, while ideological orientation retains independent effects on public-policy views. The study discusses implications for policy design and economic communication, notes sampling and external-validity limitations, and proposes an agenda for further testing and replication using probabilistic samples.

   \textbf{Keywords}: behavioral economics; cognitive biases; public opinion; economic literacy; public policy.
 \end{otherlanguage*}
\end{resumo}
