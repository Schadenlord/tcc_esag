\documentclass[
	article,
	12pt,
	oneside,
	a4paper,
	english,
	brazil,
	sumario=tradicional
]{abntex2}

% Pacotes fundamentais
\usepackage{times}
\usepackage{amssymb} % For \checkmark command
\usepackage[T1]{fontenc}
\usepackage[utf8]{inputenc}
\usepackage{indentfirst}
\usepackage{color}
\usepackage{graphicx}
\usepackage{amsmath}
\usepackage{gensymb}
\DeclareRobustCommand{\perthousand}{%
  \ifmmode
	\text{\textperthousand}%
  \else
	\textperthousand
  \fi
}
\DeclareRobustCommand{\micro}{%
  \ifmmode
	\text{\textmu}%
  \else
	\textmu
  \fi
}
\usepackage{lscape}
\usepackage{pgfplots}
\pgfplotsset{compat=1.18}
\usepackage[table,xcdraw]{xcolor}
\usepackage{float}
\usepackage{microtype}
\usepackage{multirow}
\usepackage[normalem]{ulem}
\usepackage{rotating}
\usepackage{booktabs}
\usepackage[alf,abnt-etal-cite=3,abnt-etal-list=0]{abntex2cite}
\usepackage{hyperref}
\usepackage{tabularx} % For tabularx environment
\usepackage{caption} % For \captionsetup
\usepackage{makecell}
\usepackage{rotating}
\usepackage{pgfplots}
\usepackage{pgfplotstable}


% Início do documento

\title{Aula de Metodologia Científica}


\begin{document}

\chapter{Pergunta de Pesquisa}



\textbf{Como os vieses cognitivos dos eleitores moldam a formulação de políticas econômicas no Brasil e nos EUA, e quais são as consequências para o desenvolvimento econômico e institucional de longo prazo?}


\textbf{Criterios que essa pergunta atende:}
\begin{itemize}
    \item Enfatiza o papel ativo dos vieses na escolha de políticas ruins.
    \item Explora a comparação Brasil x EUA, o que dá um caráter mais robusto e contribui academicamente.
    \item Foca nas consequências institucionais e de desenvolvimento econômico, indo além de apenas descrever os vieses.
    \item Conecta-se com a economia comportamental e a ciência política.
\end{itemize}










\newpage
\chapter{Objetivos}



\section{Objetivo Geral}

\textbf{Analisar como os vieses cognitivos dos eleitores impactam a formulação de políticas econômicas no Brasil e nos EUA, e avaliar as implicações para o desenvolvimento econômico e a qualidade das instituições democráticas.}

\section{Objetivos Específicos}

\begin{itemize}
    \item Identificar os principais vieses cognitivos que afetam a percepção econômica dos eleitores no Brasil e nos EUA, com base nos dados da SAEE e sua replicação no Brasil.
    \item Investigar como esses vieses influenciam a aceitação de determinadas políticas econômicas e regulatórias.
    \item Avaliar a relação entre a irracionalidade sistemática dos eleitores e a ineficiência das políticas públicas adotadas.
    \item Comparar as diferenças institucionais entre Brasil e EUA que podem amplificar ou mitigar os efeitos desses vieses.
    \item Propor formas de reduzir a influência dos vieses cognitivos na formulação de políticas públicas e melhorar a qualidade do debate econômico.
\end{itemize}



\newpage
\chapter{Pontos que faremos agora:}

\begin{itemize}
  \item Definir uma boa metodologia de análise, detalhando como faremos a revisão da literatura, a coleta e análise dos dados.
  \item Obter uma escrita que não seja excessivamente opinativa ou informal.
  \item Conectar melhor os objetivos à análise empírica, garantindo coerência metodológica.
  \item Reforçar a importância da comparação Brasil x EUA e a relevância dos resultados para a literatura existente.
\end{itemize}







\newpage

% ELEMENTOS PÓS-TEXTUAIS
\renewcommand{\refname}{Referências}
\bibliographystyle{abntex2-alf}
\bibliography{todas}

\end{document}
