\documentclass[
	article,
	12pt,
	oneside,
	a4paper,
	english,
	brazil,
	sumario=tradicional
]{abntex2}

% Pacotes fundamentais
\usepackage{times}
\usepackage{amssymb} % For \checkmark command
\usepackage[T1]{fontenc}
\usepackage[utf8]{inputenc}
\usepackage{indentfirst}
\usepackage{color}
\usepackage{graphicx}
\usepackage{amsmath}
\usepackage{gensymb}
\DeclareRobustCommand{\perthousand}{%
  \ifmmode
	\text{\textperthousand}%
  \else
	\textperthousand
  \fi
}
\DeclareRobustCommand{\micro}{%
  \ifmmode
	\text{\textmu}%
  \else
	\textmu
  \fi
}
\usepackage{lscape}
\usepackage{pgfplots}
\pgfplotsset{compat=1.18}
\usepackage[table,xcdraw]{xcolor}
\usepackage{float}
\usepackage{microtype}
\usepackage{multirow}
\usepackage[normalem]{ulem}
\usepackage{rotating}
\usepackage{booktabs}
\usepackage[alf,abnt-etal-cite=3,abnt-etal-list=0]{abntex2cite}
\usepackage{hyperref}
\usepackage{tabularx} % For tabularx environment
\usepackage{caption} % For \captionsetup
\usepackage{makecell}
\usepackage{rotating}
\usepackage{pgfplots}
\usepackage{pgfplotstable}


% Início do documento

\title{Aula de Metodologia Científica}


\begin{document}

\chapter{Pergunta de Pesquisa}



\textbf{Como os vieses cognitivos dos eleitores moldam a formulação de políticas econômicas no Brasil e nos EUA, e quais são as consequências para o desenvolvimento econômico e institucional de longo prazo?}

\newpage
\chapter{Objetivos}

\section{Objetivo Geral}

\textbf{Investigar como os vieses cognitivos moldam a percepção econômica dos eleitores e contribuem para a formulação de políticas públicas ineficazes.}

\section{Objetivos Específicos}

\begin{itemize}
  \item Identificar os principais vieses econômicos presentes no eleitorado e suas origens psicológicas e históricas.
  \item Analisar o impacto desses vieses na formulação de políticas públicas e no funcionamento da democracia.
  \item Comparar a percepção econômica da população com a dos especialistas, utilizando pesquisas como a \textit{Survey of Americans and Economists on the Economy} (SAEE) e sua replicação no Brasil.
  \item Avaliar possíveis estratégias para mitigar os efeitos desses vieses, como educação econômica e reformas institucionais.
\end{itemize}

\newpage
\chapter{Pontos que faremos agora:}
\begin{itemize}
  \item Adicionar uma seção sobre alternativas para mitigar os problemas identificados, como a exploração do papel das instituições e possíveis reformas institucionais, como constitucionalismo econômico, limites ao poder da maioria e formas alternativas de representação política.
  \item Reformular o título, considerando uma opção mais técnica e menos normativa. Uma sugestão seria: "Vieses Cognitivos e a Formulação de Políticas Econômicas: Uma Análise Comportamental".
  \item Inserir uma seção discutindo as implicações políticas e possíveis soluções para os problemas descritos no estudo. Explore intervenções institucionais ou educacionais que poderiam mitigar os vieses cognitivos.
  \item Expandir a revisão bibliográfica, incluindo autores da área de teoria da escolha pública (Public Choice), como James Buchanan e Gordon Tullock. Referencie também "The Calculus of Consent".
  \item Explorar o impacto da mídia e da comunicação política na formação das crenças econômicas e como esses elementos podem perpetuar os vieses cognitivos.
  \item Justificar melhor a escolha do método Logit, explicando por que ele é a melhor opção para a análise proposta.
  \item Controlar variáveis culturais e institucionais, explicando como as diferenças entre os contextos político e educacional do Brasil e dos EUA serão tratadas na análise.
  \item Adicionar controle de variáveis de confusão, como nível de escolaridade, exposição à mídia e participação política ativa.
  \item Definir uma boa metodologia de análise, detalhando como será feita a revisão da literatura, coleta e análise dos dados.
  \item Justificar melhor a metodologia, especialmente no controle de variáveis culturais.
  \item Refinar os objetivos específicos, tornando-os mais concretos e operacionais. Defina claramente como cada um será investigado.
  \item Definir critérios objetivos para classificar políticas públicas como ineficazes e como será medido o impacto dos vieses.
  \item Conectar melhor os objetivos à análise empírica, garantindo coerência metodológica.
  \item Definir um critério claro de refutação para a hipótese, como: se não houver correlação estatisticamente significativa entre vieses e apoio a políticas ineficazes, a hipótese será revisada.
  \item Incorporar a teoria da escolha pública, considerando o papel dos incentivos políticos e institucionais. Isso fortalecerá a tese e a tornará mais falseável.
  \item Aplicar o critério de refutabilidade de Popper, deixando claro quais resultados empíricos poderiam refutar a hipótese.
  \item Reformular algumas seções que ainda não foram desenvolvidas e garantir que a discussão sobre soluções não fique de fora.
  \item Revisar a escrita, tornando-a mais formal e evitando um tom excessivamente opinativo ou informal.
  \item Reforçar a importância da comparação Brasil x EUA, destacando a relevância dos resultados para a literatura existente.
\end{itemize}
\newpage

% ELEMENTOS PÓS-TEXTUAIS
\renewcommand{\refname}{Referências}
\bibliographystyle{abntex2-alf}
\bibliography{todas}

\end{document}
